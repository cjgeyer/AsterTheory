
\DeclareMathOperator{\pr}{pr}
\DeclareMathOperator{\var}{var}
\DeclareMathOperator{\cov}{cov}
\DeclareMathOperator{\logit}{logit}
\DeclareMathOperator{\card}{card}

\newcommand{\inner}[1]{\langle #1 \rangle}
\newcommand{\set}[1]{\{\, #1 \,\}}
\newcommand{\bigset}[1]{\left\{\, #1 \,\right\}}

\newcommand{\fatdot}{\,\cdot\,}

\newcommand{\nats}{\mathbb{N}}
\newcommand{\real}{\mathbb{R}}

\newcommand{\opand}{\mathbin{\rm and}}
\newcommand{\opor}{\mathbin{\rm or}}
\newcommand{\ifandonlyif}{\mathrel{\rm if\ and\ only\ if}}

\newcommand{\code}[1]{\texttt{#1}}

\newcommand{\myline}{\relbar\joinrel\relbar}

\let\emptyset=\varnothing

\newlength{\foo}
\settowidth{\foo}{=}
\newcommand{\andsoforth}{\mathrel{\makebox[\foo]{\vdots}}}

\newcommand{\REVISED}{\begin{center} \LARGE REVISED DOWN TO HERE \end{center}}
\newcommand{\MOVED}[1][equation]{\begin{center} [#1 moved] \end{center}}

\newtheorem{theorem}{Theorem}[chapter]
\newtheorem{corollary}[theorem]{Corollary}
\newtheorem{lemma}[theorem]{Lemma}

% for indexing
\newcommand{\seeunder}[2]{\emph{see under} #1}

