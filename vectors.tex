
\chapter{Vector Spaces and Differentiation}
\label{app:vectors}

\section{Vector Spaces}

\section{Dual Spaces}

\section{Derivatives as Linear Functions}

\REVISED

The method of gradient descent dates back at least to Cauchy, so is
nearly 200 years old.  It has lately become very popular in machine learning.

In advanced math, all mathematics from theoretical linear algebra up,
including real and functional analysis, abstract algebra, differential
geometry, and any courses more advanced than them, derivatives are
linear functions rather than scalars or vectors
\citet[Chapters~XIII and~XXII]{lang}.  In particular, if $f$ is a real-valued
function on a vector space $V$, then the derivative $f'(x)$ at the point $x$
is a linear function $V \to \real$ and hence is an element of the dual vector
space $V^{\textstyle *}$ to $V$.  The derivative, if it exists is in this
special case, the linear function satisfying
$$
   f(x + h) = f(x) + f'(x)(h) + \norm{h} o(h)
$$
which is the Taylor series up to terms linear in $h$.
This appears to make the definition of derivative dependent on the norm.
If $V$ is a finite-dimensional vector space (the subject of real analysis),
then it does not actually depend on the norm because all norms are equivalent
\citep[Corollary~3.14 of Chapter~II]{lang}.
If $V$ is an infinite-dimensional topological vector space
(the subject of functional analysis), then the definition of derivative
actually does depend on the norm.

The point of all of this is that the gradient vector lies in
$V^{\textstyle *}$ which is not the same vector space as the domain of
the function $f$ that we are differentiating and trying to minimize.
So it does not give a direction in $V$ to go.
In higher mathematics, the method of gradient descent does not make any
sense.  It is literally meaningless.

We have already mentioned or at least alluded to this view of vector spaces
and duals several times.  In our basic definition of exponential families
(Section~1.15) above we said that the angle brackets notation
$\inner{\fatdot, \fatdot}$ is a bilinear form that places the
(finite-dimensional) vector space where the canonical statistic vector
takes values and the
(finite-dimensional) vector space where the canonical parameter vector
takes values in duality.  That is $V$ and $V^{\textstyle *}$ for some
finite-dimensional vector space $V$.

In the fundamental equation giving means and variances of the canonical
statistic vector of an exponential family as derivatives of the cumulant
function \eqref{eq:cumulant-one} and \eqref{eq:cumulant-two} and the use
of these in defining the mean value parameterization and Fisher information
(Section~\ref{sec:mean-value-parameterization} above) if we were going to
be really fussy about using abstract mathematics correctly, we would have
$V$ is the vector space where the canonical statistic takes values,
$V^{\textstyle *}$ is the vector space where the canonical parameter
takes values,
the mean value parameter $\mu = c'(\varphi)$ is a linear function on
$V^{\textstyle *}$ hence an element of $V^{\textstyle {*}{*}}$.
But for finite-dimensional $V$, there is the so-called
\emph{natural isomorphism} $V \to V^{\textstyle {*}{*}}$ that takes the
point $y$ to the linear functional $\inner{y, \fatdot}$.
This allows us to consider $\mu$ to lie in the same vector space $V$
where $y$ takes values.  It also allows us to consider the map
$c' : \varphi \mapsto \mu$ as a nonlinear function $V^{\textstyle *} \to V$.
Hence the derivative of this function $c''$ must be a linear function
$V^{\textstyle *} \to V$.  So Fisher information is this linear function.

That might have been a TL;DR (too long, didn't read).  The only point we
want to stress here is that $\mu$ and $\varphi$ live in dual vector spaces.
They are not in any way comparable.  It is wrongheaded (not in accordance
with abstract math) to think of them as comparable.  This is because
while $V$ and $V^{\textstyle *}$ have the same dimension
\citep[if finite-dimensional][Theorem~2 of Section~15]{halmos-vector-spaces}
there is no unique way to match them up.  It is a mistake to think of them
as the same vector space, as equal to $\real^d$ for some $d$.
If you want to consider them as elements of $\real^d$, then any isomorphisms
(invertible linear transformations) $V \to \real^d$
and $V^{\textstyle *} \to \real^d$ will do.  This is related to the
nonuniqueness of canonical statistics and canonical parameters mentioned
in Section~\ref{sec:define-expfam} above.

