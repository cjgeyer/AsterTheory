
\chapter*{Index of Notation}

\noindent
\begin{raggedright}
\noindent
\begin{longtable}{lp{4.4in}}
$\real$ & the real number system \\
$\nats$ & the natural number system, which starts at zero \\
$N$ & the set of nodes of the full aster graph, including initial nodes \\
$J$ & the set of non-initial nodes of the full aster graph \\
$\mathcal{G}$ & the family of dependence groups, a partition of $J$ \\
$q$ & the set-to-index predecessor function, which maps $\mathcal{G} \to N$ \\
$p$ & the index-to-index predecessor function, which maps $J \to N$ \\
$\real^A$ & the set of all functions $A \to \real$ considered as
    a vector space \\
$y_j$ & a component of the vector $y$: if $y \in \real^A$ and $j \in A$,
    then $y_j$ is the value of the function $y$ at argument $j$ \\
$y_A$ & a subvector of the vector $y$: if $y \in \real^B$ and $A \subset B$,
    then $y_A$ is the restriction of the function $y$ to the set $A$ \\
$\pr(y)$ & unconditional distribution of the random vector $y$ described
    somehow (Section~\ref{sec:factorization}) \\
$\pr(y_A \mid y_B)$ & conditional distribution of the random vector $y_A$
    described somehow (Section~\ref{sec:factorization}
    and equation~\eqref{eq:before-and-after} and the surrounding discussion) \\
$\theta$ & conditional canonical parameter vector of the saturated model \\
$\varphi$ & unconditional canonical parameter vector of the saturated model \\
$\xi$ & conditional mean value parameter vector of the saturated model \\
$\mu$ & unconditional mean value parameter vector of the saturated model \\
$\beta$ & unconditional canonical parameter vector of a canonical
    affine submodel \\
$\tau$ & unconditional mean value parameter vector of a canonical
    affine submodel \\
$a$ & offset vector \\
$M$ & model matrix \\
$f^n$ & the function $f$ composed with itself $n$ times;
    $f^0$ being the identity function, and $f^1 = f$ \\
$\succ$ & transitive closure of the predecessor relation \\
$\succeq$ & reflexive transitive closure of the predecessor relation \\
$\prec$ & transitive closure of the successor relation \\
$\preceq$ & reflexive transitive closure of the successor relation \\
$\inner{\fatdot, \fatdot}$ & bilinear form placing dual vector spaces
    in duality \\
\end{longtable}
\end{raggedright}
