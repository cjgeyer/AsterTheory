
\chapter{Introduction}
\label{ch:introduction}

\section{Background}

Aster models \citep*{aster1,aster2,reaster} are parametric statistical models
specifically designed for life history analysis.  They are exponential family
models that generalize generalized linear models (GLM) that are also
exponential family models (for example, logistic regression and
Poisson regression with log link) in two ways
\begin{itemize}
\item in GLM components of the response vector are
    necessarily conditionally independent given covariate data
    but in aster models they need not be, and
\item in GLM the conditional distributions of components of the
    response vector given covariate data all come from the same family
    but in aster models they need not.
\end{itemize}
As generalizations of GLM, aster models are also regression models.
They model the conditional distribution of the response vector given
covariate data.  The marginal distribution of covariate data is not
modeled.

In life history analysis, the data are about survival and reproduction
of biological organisms.  Thus aster models also generalize discrete time
survival analysis (aster models model not only survival but also
what happens conditional on survival).
Aster models unify many disparate kinds of life history analysis that have
appeared in the biological literature: comparison of Darwinian fitness between
various groups \citep{aster1,aster2}, estimation of fitness landscapes
(\citealp{lande-arnold}; \citealp{aster2,aster3}, \citealp*{aster-hornworm}),
Leslie matrix analysis
\citep{caswell}, life table analysis in demography \citep{goodman},
and estimation of population growth rates
\citep{fisher,lenski-service,aster2,aster-hornworm}.
Aster models also generalize zero-inflated Poisson regression \citep{lambert},
negative binomial regression (overdispersed Poisson regression),
and zero-inflated negative binomial regression.

Aster models are a special case of graphical models \citep{lauritzen}.
In particular, they are statistical models for which the joint distribution
of the response vector factorizes completely as a product of marginal and
conditional distributions (equation~\eqref{eq:factorize} below).
This makes aster models a special case of chain graph models
\citep[Sections~2.1.1 and~3.2.3]{lauritzen}.
Aster models also have the predecessor-is-sample-size property
%%%%%%%%%% NEED FORWARD REFERENCE %%%%%%%%%%
that makes the joint distribution of the response vector an exponential
family.  This property can be seen to generalize unnamed properties
of survival analysis, life-table analysis, and Leslie matrix analysis.

\section{Software}
\label{sec:software}

Currently, all software for aster models is written in the R statistical
computing language \citep{r-core}.  There are two CRAN
(\url{cran.r-project.org}) packages, \code{aster} \citep{aster-package} and
\code{aster2} \citep{aster2-package}.

Both R and these packages can be installed in minutes on any computer,
so any user can get started with aster models in almost no time.
R package \code{aster} is the most complete.
It does everything except dependence groups
%%%%%%%%%% NEED FORWARD REFERENCE %%%%%%%%%%
and limiting conditional models.
%%%%%%%%%% NEED FORWARD REFERENCE %%%%%%%%%%

R package \code{aster2} is the very incomplete.
It does do dependence groups and limiting conditional models, but everything
else is either missing or much harder to use than in R package \code{aster}.

So any aster model that can be done with R package \code{aster} should
be done with that package.

\section{Vectors and Subvectors}

We adopt a notation from \citet{lauritzen} for subvectors, but fuss about it
more.  As in set theory \citep[Section~8]{halmos-set-theory}, if $A$ and $B$
are sets, then $A^B$ denotes the set of all functions $B \to A$.
In particular, if $J$ is a finite set, then we let $\real^J$ denote
the set of all functions $J \to \real$.  This set can also be considered
a finite-dimensional vector space.  That functions $J \to V$ where $J$ is
any set and $V$ is a field or a vector space can be considered
vectors is the reason the study of infinite-dimensional topological vector
spaces is called functional analysis.

Another way of looking at this distinction is that the usual view of
finite-dimensional vector spaces is that they are $\real^d$ for some
natural number $d$, which is tantamount to insisting that the index
set for vectors in this space must be the set $\{1, \ldots, d\}$.
Here we are saying the index set can be any finite set $J$.

Even though we consider vectors to be functions, we write evaluation
of these functions $y_j$ so it looks like usual notation.  We even say
that $y_j$ is a component of the vector $y$ rather than the value of
the function $y$ at the point $j$.  But behind the scenes our vectors
are also functions, and we could write $y(j)$ instead of $y_j$.

We need a notation for subvectors of a vector.  If $y$ is an element
of $\real^J$ and $A \subset J$, then we let $y_A$ denote the restriction
of $y$ to the set $A$.
As such, it is an element of the vector space $\real^A$.
Like all functions, it knows its domain and codomain.
It knows it is a function $A \to \real$.
So it knows its components are $y_j$, $j \in A$.
And these are also the components of $y$ for $j \in A$.

If we were to insist that all vectors, including subvectors, have
index sets $\{1, \ldots, k\}$ for some natural number $k$.  Then we could
not distinguish different subvectors of the same length, or at least could
not without ugly and cumbersome extra decoration of the notation.

Our notation does have the drawback that we have only the convention
that lower case letters denote elements of sets and upper case letters
denote sets, which is why $y_j$ is clearly a component of a vector or subvector
(the value of a function at the index $j$) and $y_A$ is a subvector
(so is still a function, not the value of a function).
We also consider any subscript notation that clearly denotes a set
as indicating a subvector, for example, $y_{\{1, 3, 5\}}$ or $y_{\{j\}}$ or
$y_{\set{ j \in J : j \prec i }}$.

\section{Regression Notation}

Strictly speaking, in regression theory, every probability and expectation
is conditional on covariate data, at least on the part of the covariate data
that is considered random rather than fixed by the design of the experiment.
Thus to be hyperpedantic, we should always write
\begin{gather*}
   E(y_A \mid \text{the part of covariate data that is random})
   \\
   \Pr(y_A \in B \mid \text{the part of covariate data that is random})
\end{gather*}
rather than $E(y_A)$ or $\Pr(y_A \in B)$.  But, like most regression books,
we will not do this.  The dependence of probabilities and expectations on
covariates is usually not made explicit in the notation.

This is especially important in aster models when components of the response
vector depend on the values of other components, so we frequently write
\begin{gather*}
   E(y_A \mid y_j)
   \\
   \Pr(y_A \in B \mid y_j)
\end{gather*}
and the like.  And we do not want this dependence confused with dependence
on covariate data.

When necessary for clarity, as in the discussion of fitness landscapes,
which are regression functions,
%%%%%%%%%% NEED FORWARD REFERENCE %%%%%%%%%%
we can explicitly denote the dependence on covariate data in conditional
probabilities and expectations.

\section{Factorization}
\label{sec:factorization}

If $J$ is the index set of the response vector $y$ of an aster model,
then there is a partition $\mathcal{G}$ of $J$
and a function $q : \mathcal{G} \to N$, where $N \supset J$, such that
the joint distribution of $y$ factorizes as
\begin{equation} \label{eq:factorize}
   \pr(y) = \prod_{G \in \mathcal{G}} \pr(y_G \mid y_{q(G)})
\end{equation}

This differs from the usual notation for chain graph models
\citep[equation~3.23]{lauritzen} in several ways.
\begin{itemize}
\item In \eqref{eq:factorize} $q(G)$ is an index.
      In (3.23) in \citet{lauritzen} the analog of $q(G)$ is an index set.

\item In \eqref{eq:factorize} $q(G)$ takes values in $N$ which is larger
      than the index set $J$ of all random variables.
      In (3.23) in \citet{lauritzen} the analog of $q(G)$ takes values that
      are subsets of the index set $J$ of all random variables.

\item In \eqref{eq:factorize} every distribution on the right-hand side
      appears to be conditional.
      This also holds in (3.23) in \citet{lauritzen}.

\item In \eqref{eq:factorize} we can tell when what appears to be a conditional
      distribution is actually marginal when $q(G) \notin J$, so $y_{q(G)}$
      does not denote a component of the response vector and hence is
      considered a constant random variable (and conditioning on a constant
      random variable or random vector is the same as not conditioning).
      In (3.23) in \citet{lauritzen} we can tell when what appears to be
      a conditional distribution is actually marginal when the analog of
      $q(G)$ is the empty set (and conditioning on the empty set of
      random variables is the same as not conditioning).
\end{itemize}

If we were going to follow the convention in graphical model theory that
$y$ is the vector of all the $y_j$, we would have to rewrite our factorization
as
$$
   \pr(y) = \prod_{G \in \mathcal{G}} \pr(y_G \mid y_{q(G)})
   \prod_{j \in N \setminus J} \pr(y_j)
$$
and consider $y$ to have index set $N$ rather than $J$.
But since each $y_j$ in the second product on the right-hand side is
a constant random variable, we always have $\pr(y_j) = 1$ when $y_j$
is the only value that variable can have,
and the second product disappears giving us \eqref{eq:factorize}.

In \eqref{eq:factorize} we have been deliberately vague about what $\pr$ is
supposed, to mean, since there are many ways to specify probability
distributions and any of them will do.
\begin{itemize}
\item If $y$ is a discrete random vector,
      then $\pr$ could denote probability mass functions.
\item If $y$ is a continuous random vector,
      then $\pr$ could denote probability density functions.
\item If $y$ is a partly discrete and partly continuous continuous
      random vector (either some components discrete and some components
      continuous or some components a mixture of discrete and continuous)
      then $\pr$ could denote probability mass-density functions.
\item No matter what, $\pr$ could denote cumulative distribution functions.
\item No matter what, $\pr$ could denote probability measures.
\end{itemize}
In any of these cases the multiplication indicated in \eqref{eq:factorize}
is actual multiplication of real-valued thingummies.

Now, translating the usual notation of graphical models into our notation,
the condition for \eqref{eq:factorize} to be a valid factorization, that is,
that what appear to be conditional distributions on the right-hand side could
be derived from the joint distribution by the usual operations of probability
theory, is as follows.
\begin{theorem} \label{th:factorize}
The factorization \eqref{eq:factorize} is valid if and only if
the partition $\mathcal{G}$ can be totally ordered
by some total ordering $<$ such that $q(G) \in H$ implies $G < H$.
\end{theorem}
\begin{proof}
A valid factorization factors joint equals conditional times marginal
$$
   \pr(y) = \pr(y_{G_1} \mid y_{N \setminus G_1}) \pr(y_{N \setminus G_1})
$$
The marginal on the right-hand side can then be considered a joint to be
factored further
$$
   \pr(y)
   =
   \pr(y_{G_1} \mid y_{N \setminus G_1})
   \pr(y_{G_2} \mid y_{N \setminus (G_1 \cup G_2)})
   \pr(y_{N \setminus (G_1 \cup G_2)})
$$
and again and again giving
\begin{equation} \label{eq:factorize-general}
   \pr(y)
   =
   \pr(y_{N \setminus \bigcup_{j = 1}^k G_j})
   \prod_{i = 1}^k
   \pr(y_{G_i} \mid y_{N \setminus \bigcup_{j = 1}^i G_j})
\end{equation}
and the only condition that is required to make \eqref{eq:factorize-general}
valid is that the index sets $G_i$ are disjoint.  This is the only operation
in classical (non-measure-theoretic) probability theory that factorizes
probability distributions.  A factorization is valid if and only if it
has the form \eqref{eq:factorize-general}.

When we match up \eqref{eq:factorize} and \eqref{eq:factorize-general}
we see that the $G_i$ must be the elements of $\mathcal{G}$ so the two
products are the same.  For the conditional distributions to
match up we must have $\pr(y_{G_i} \mid y_{N \setminus \bigcup_{j = 1}^i G_j})$
in \eqref{eq:factorize-general} can actually be written as
$\pr(y_{G_i} \mid y_{q(G_i)})$, that is,
\begin{itemize}
\item this conditional distribution actually depends only on the single
    variable $y_{q(G_i)}$ not on the rest of the variables that are components
    of $y_{N \setminus \bigcup_{j = 1}^i G_j}$ and
\item $q(G_i) \in N \setminus \bigcup_{j = 1}^i G_j$, that is, either
    $q(G_i) \in G_j$ for some $j > i$ or $q(G_i)$ is an initial node
    ($q(G_i) \notin G_j$ for any $j$).  In either case,
    $q(G_i) \in G_j$ implies $i < j$.  Thus we have the condition of
    the theorem: $G_i < G_j$ if and only if $i < j$.
\end{itemize}
Finally, we must match up the marginal term on the right-hand side of
\eqref{eq:factorize-general}.  It matches nothing in \eqref{eq:factorize},
which is the same as saying it must be equal to one, which is they same as
saying $y_{N \setminus J}$ where $J = \bigcup \mathcal{G}$ is a constant
random vector.
\end{proof}

The total order asserted to exist by the theorem need not be unique and
usually is not unique.  In the case of an aster graph that has no arrows
or lines, the components of the response vector are stochastically independent,
each element of $\mathcal{G}$ is a singleton set and
$$
   \pr(y) = \prod_{G \in \mathcal{G}} \pr(y_G)
$$
and it is clear that any total order on $\mathcal{G}$ will do in this case.

We can find such a total order using the algorithm called topological sort
\citep[Section~6.6]{aho-et-al} Using R function \code{tsort} in R package
\code{pooh}
\citep{pooh-package} for each $G \in \mathcal{G}$ such that there exists
a (necessarily unique) $H \in \mathcal{G}$ such that $q(G) \in H$
let $G$ be a component of the vector \code{from} that is an argument to this
function and
let $H$ be the corresponding component of the vector \code{to} that is another
argument to this function and
neither vector has any other components.  Then invoking this function with
these \code{from} and \code{to} arguments and \code{domain} argument that is
$\mathcal{G}$ strung out in a vector in any order
will determine a (not necessarily unique) total order that agrees with
Theorem~\ref{th:factorize}.

Current code in R packages \code{aster} and \code{aster2} does not
actually use the topological sort algorithm but rather forces the user
to input the data so that the numerical order of the components of the
response vector is the total order, that is, considering the index set
of the response vector to be $\{1, \ldots, n\}$ for some integer $n$
current code requires $q(G) < j$ for any $j \in G$.
It is up to the user to present the data in this way.  The computer is no help.
But we could make the computer figure this out in future versions of the
software.

\section{Further Factorization}

In \citet{lauritzen} chain graph factorizations like our \eqref{eq:factorize}
and his equation (3.23) can be further factorized, his equation (3.24).
But in aster model theory, we shall never be interested in such further
factorizations (even in cases where they are possible) and never use any
notation that allows for them.  So we will never have an analog of equation
(3.24) in \citet{lauritzen}.  For us, factorization is \eqref{eq:factorize}.

\section{Graphs}

Each factorization goes with a graph \citep[Section~3.2.3]{lauritzen}.
The nodes of the graph are either the elements of $N$ or the components
of $y$ corresponding to these elements ($y_j$ for $j \in N$).
There is a directed edge, also called an \emph{arrow}
$q(G) \longrightarrow j$ (or if one prefers $y_{q(G)} \longrightarrow y_j$)
for every $G \in \mathcal{G}$ and every $j \in G$.
There is an undirected edge, also called a \emph{line}
$j \myline k$ (or if one prefers $y_j \myline y_k$)
for every $G \in \mathcal{G}$ and every $j, k \in G$ such that $j \neq k$.

As we have just seen, the function $q$ determines the graph
(the function $q$ knows its domain $\mathcal{G}$).
Conversely, the graph determines the function $q$.
\begin{itemize}
\item The set $J = \bigcup \mathcal{G}$ is the set of nodes of the graph
    that have incoming arrows (as we shall see, these nodes are called
    non-initial).
\item The elements of $\mathcal{G}$ are the maximal connected components
    of the graph of lines having nodeset $J$.
    This includes any singleton sets of $J$ that have no incoming lines.
\item The graph of $q$ is determined by the arrows: $(G, q(G))$ is an
    argument-value pair whenever there is an arrow $j \longrightarrow k$
    with $j = q(G)$ and $k \in G$.
\end{itemize}

Thus we can reason with with graphs or with $q$ functions (which we will
soon learn to call \emph{predecessor functions}).
%%%%%%%%%% NEED FORWARD REFERENCE %%%%%%%%%%
Graphs can be helpful, but we do not have to use them.

\section{Graphical Terminology}
\label{sec:graphical-terminology}

In aster theory, we say
\begin{itemize}
\item a node is \emph{initial} if it has no incoming arrows
    or lines (when thinking about the graph) or if it is not an element
    of $J = \bigcup \mathcal{G}$ (when thinking about the function $q$),
\item a node is \emph{terminal} if it has no outgoing arrows
    (it may have outgoing lines and will have outgoing lines if it is
    an element of an element of $\mathcal{G}$ that is not a singleton set),
\item if there is an arrow $j \longrightarrow k$, then we say that $j$
    is the \emph{predecessor} of $k$ (or $y_j$ is the predecessor of $y_k$),
\item and, conversely, that $k$ is a \emph{successor} of $j$
    (or $y_k$ is the successor of $y_j$).
\end{itemize}
In mainstream graphical model theory, a different terminology is more widely
used \citep{lauritzen} root = initial, leaf = terminal, parent = predecessor,
child = successor.  We do not use this terminology in aster model theory
because it can cause serious confusion in biological applications.

As a general policy, we eschew all terminology based on biological analogies
when there is an available alternative (even when that alternative is less
popular).

In any aster graph every node has at most one predecessor and all nodes in
the same $G \in \mathcal{G}$ must have the same predecessor (because $q$
is a function that takes elements of $\mathcal{G}$ as arguments).

In mainstream graph theory, a chain graph with only arrows (no lines) having
the at-most-one-predecessor property is called a \emph{tree} and a disjoint
union of trees is called a \emph{forest}, but we do not use this terminology
either (avoiding serious confusion when the application involves data on
real trees in real forests).  It is enough to say that aster graphs
have the at-most-one-predecessor property.

In mainstream graph theory, there is a term \emph{ancestor} that means
predecessor, or predecessor of predecessor,
or predecessor of predecessor of predecessor,
or predecessor of predecessor of predecessor of predecessor,
or the same with arbitrarily many repetitions of ``predecessor of.''
And there is a converse term \emph{descendant}, that is, $i$ is an ancestor
of $j$ if and only if $j$ is a descendant of $i$.

In aster model theory we avoid these terms too (avoiding confusion when
the application involves real biological organisms with real biological
ancestors and real biological descendants).  If we need the concepts,
then we use the long-winded descriptions
predecessor of predecessor of predecessor and so forth or
successor of successor of successor and so forth.
Fortunately, we rarely need these concepts.
And when we do need these concepts we can avoid the cumbersome verbiage
by using mathematical notation introduced in the next section.

Finally, we need a term for $\mathcal{G}$ and its elements.
The terminology we have been using in our writings about aster models is
elements of $\mathcal{G}$ are \emph{dependence groups}.
The mainstream graphical models terminology \citep{lauritzen} is
\emph{chain components}.  Both have two words and four syllables.
Neither is very elegant.  We don't like the ``chain'' terminology because
we are not using general chain graph theory (aster models are very special
chain graphs).  Our term \emph{dependence group} is not great, but we haven't
thought of a better term.

\section{The Other Predecessor Function}

It is useful to have not only the set-to-index predecessor function $q$
defined in Section~\ref{sec:factorization} above but also the index-to-index
predecessor function $p$ defined as follows
$$
   p(j) = k \ifandonlyif j \in G \in \mathcal{G} \opand q(G) = k.
$$
Clearly, $q$ determines $p$.
The converse is not true because $p$ knows nothing about dependence groups.
But $p$ and $\mathcal{G}$ together determine $q$.

\section{The Transitive Closure of the Predecessor Relation}
\label{sec:closure}

The \emph{predecessor relation} on $N$ is the function $p$ thought
of as a relation, that is, as the set
$$
   \set{ (j, p(j)) : j \in J }
$$
of its argument-value pairs.

Its transitive closure is the smallest transitive relation $R$ containing it.
As with most relations, we prefer denoting this relation by infix notation:
saying $j \succ k$ rather than $(j, k) \in R$, that is, $j \succ k$ means
one of the following
\begin{align*}
   k & = p(j)
   \\
   k & = p(p(j))
   \\
   k & = p(p(p(j)))
   \\
   & \andsoforth
\end{align*}
holds (an arbitrary number of applications of $p$ are allowed).

\begin{theorem} \label{th:transitive-closure}
The transitive closure of the predecessor relation is a strict partial order.
\end{theorem}
\begin{proof}
Define $p^0(j) = j$, $p^1(j) = p(j)$, and $p^n(j) = p(p^{n - 1}(j))$ for
$n > 1$.  Then $j \succ k$ implies $j \in G$ for some $G \in \mathcal{G}$ and
$k = p^n(q(G))$ for some natural number $n$.

If $k \in H$ for some $H \in \mathcal{G}$,
then we have $G < H$ in the total ordering
that Theorem~\ref{th:factorize} uses.
Hence we cannot also have $k \succ j$ because that would imply $G < H$
and $H < G$ contradicting $<$ being a strict total order.

If $k \notin H$ for any $H \in \mathcal{G}$ then $k$ has no predecessor
($k$ is initial) and we cannot have $m \succ k$ for any node $m$.

In either of the preceding cases we never have $k \succ j$ and $j \succ k$.
Since $\succ$ is a transitive relation by definition, it is
a strict partial order \citep[Section~14]{halmos-set-theory}
\end{proof}
\begin{corollary} \label{cor:compatible}
The transitive closure of the predecessor relation is compatible with
the total order on the family of dependence groups defined
in Theorem~\ref{th:factorize} in the sense that
$j \in G \in \mathcal{G}$ and $k \in H \in \mathcal{G}$ and $j \succ k$
implies $G < H$.
\end{corollary}

The non-strict counterpart of this relation
is the reflexive transitive closure of the predecessor relation,
which is denoted $\succeq$.
We have $j \succeq k$ if and only if $j \succ k$ or $j = k$.

The inverse of a relation $R$ considered as a set of argument-value pairs
reverses the order in the pairs, that is $(k, j) \in R^{- 1}$ if and only
if $(j, k) \in R$.
As usual, we denote the inverse of a relation by turning its infix notation
around: $\prec$ is the inverse of $\succ$ and $\preceq$ is the inverse
of $\succeq$.

The inverse of the predecessor relation is the successor relation,
so $\prec$ is the transitive closure of the successor relation
and $\preceq$ is the reflexive transitive closure of the successor relation.

The choice of whether the transitive closure of the predecessor relation
is denoted $\succ$ or $\prec$ is arbitrary.  Either choice works so long
as one keeps straight which is which.  Our choice is influenced by an
arbitrary choice in the source code for R package \texttt{aster}.  When
the predecessor function is encoded (as the argument \texttt{pred} to the
R function \texttt{aster}) it is required that predecessors have lower indices
than successors (come before them in the \texttt{pred} vector).  Thus we
want to think of predecessors as ``less than'' successors in some sense.
Hence our decision to make $p(j) \prec j$.

In graphical model theory,
$\succ$ is called the ancestor relation,
$\prec$ the descendant relation,
$\succeq$ the ancestor-or-self relation, and
$\preceq$ the descendant-or-self relation.
But, as stated in Section~\ref{sec:graphical-terminology} above,
our policy is to avoid these terms
to avoid confusion in biological applications.
If we need words rather than symbols, we have to use the long winded ones:
``reflexive transitive closure of the predecessor relation'' and so forth.

\section{Markov Properties}

Markov properties of graphical models are considered a fundamental part
of the theory \citep[Section~3.2]{lauritzen}.  They are much less important
for aster models, so unimportant that the literature does not mention them.
So perhaps most readers will want to skip this section.  Nevertheless,
perhaps these ideas might find some future use.  So we do them.

\begin{theorem} \label{th:markov-one}
For any dependence group $G$, define
$$
   A = \set{ j \in J : (\exists k \in G)(j \succeq k) }
$$
and
$$
   \mathcal{H} = \set{ H \in \mathcal{G} : H \subset A }.
$$
Then
\begin{equation} \label{eq:markov-one}
   \pr(y_{A} \mid y_{J \setminus A})
   =
   \prod_{H \in \mathcal{H}} \pr(y_H \mid y_{q(H)}),
\end{equation}
and $y_A$ and $y_{J \setminus A}$ are conditionally
independent given $y_{q(G)}$.
\end{theorem}
Elements of $A$ are elements of $G$ and their successors,
successors of successors, successors of successors of successors, and so forth.
\begin{proof}
First, $\bigcup \mathcal{H} = A$ because every $j \in A$ is contained in
some $H \in \mathcal{G}$, and either $H = G$ or $q(H) \in A$, and in either
case $H \subset A$.

Using the total order on $\mathcal{G}$ guaranteed to exist
by Theorem~\ref{th:factorize}, enumerate the elements of $H$
to agree with the total order on $\mathcal{G}$, that is,
$H_1 < H_2 < \cdots < H_k$ and $\mathcal{H}$ has $k$ elements.

Now calculate the marginal distribution of $y_{J \setminus A}$ by
summing-integrating out (sum if discrete, integrate if continuous)
$y_{H_1}$, $y_{H_2}$, $\ldots,$ $y_{H_k}$
from the joint distribution \eqref{eq:factorize} in that order.
Since the elements of $H_1$ have no successors, $y_{H_1}$ appears only in
the term $\pr(y_{H_1} \mid y_{q(H_1)})$, so when we sum-integrate we get one.
Now sum-integrate out $y_{H_2}$.  Since the elements of $y_{H_2}$ have no
successors in the variables that are left (they may have had successors in
$y_{H_1}$ but those variables are now gone), we get one again.  And so forth.
When all of the $y_{H_i}$ have been summed-integrated out (in the order we
enumerated them), we get one for every sum-integral.  That leaves
\begin{equation} \label{eq:markov-one-marginal}
   \pr(y_{J \setminus A})
   =
   \prod_{G \in \mathcal{G} \setminus \mathcal{H}} \pr(y_G \mid y_{q(G)}).
\end{equation}
Dividing the joint distribution \eqref{eq:factorize} by the marginal
distribution \eqref{eq:markov-one-marginal}, gives
the conditional distribution \eqref{eq:markov-one}.

The last statement of the theorem follows immediately from
\eqref{eq:markov-one}.  If $U$, $V$, and $W$ are random vectors
and the conditional distribution of $U$ given $V$ and $W$ does not depend
on $W$, then
\begin{align*}
   \pr(U, W \mid V)
   & =
   \frac{\pr(U, V, W)}{\pr(V)}
   \\
   & =
   \frac{\pr(U \mid V, W) \pr(V, W)}{\pr(V)}
   \\
   & =
   \frac{\pr(U \mid V) \pr(V, W)}{\pr(V)}
   \\
   & =
   \pr(U \mid V) \pr(W \mid V)
\end{align*}
which says $U$ and $W$ are conditionally independent given $V$.
Moreover, $U$, $V$, and $W$ are conditionally independent given $V$,
because a constant random vector is independent of any random vector,
even itself, and conditioning on $V$ is like treating $V$ as if it
were constant.
\end{proof}

\begin{theorem} \label{th:markov-two}
Let $\mathcal{H}$ be any subset of $\mathcal{G}$.  Then the random vectors
$y_H$, $H \in \mathcal{H}$ are conditionally independent given
the random scalars $y_{q(H)}$, $H \in \mathcal{H}$.
\end{theorem}
In particular, if $G_1$ and $G_2$ are distinct elements of $\mathcal{G}$,
then $y_{G_1}$ is conditionally independent of $y_{G_2}$
given $y_{q(G_1)}$ and $y_{q(G_2)}$.

This theorem does not say that the components of these random vectors are
conditionally independent.  The components of $y_G$ are dependent given
$y_{q(G)}$.  That is the whole point of dependence groups.
\begin{proof}
As in the preceding proof, use the total order on $\mathcal{G}$ guaranteed
to exist by Theorem~\ref{th:factorize}.
Enumerate $\mathcal{G}$ as $G_1 < G_2 < \cdots < G_n$.
Then $H_j = G_{i_j}$ for $j = 1$, $\ldots,$ $m$,
where $1 \le i_1 < i_2 < \cdots < i_m \le n$.

Also as in the preceding proof, we integrate out $y_G$ one at a time
in order skipping when $G \in \mathcal{H}$ and also not integrating out
any $y_{q(H)}$ for $H \in \mathcal{H}$.

We start by sum-integrating out $y_{G_1}$ if $G_1 \neq H_1$ obtaining
$$
   \pr(y_{G_2 \cup \cdots \cup G_n})
   =
   \prod_{i = 2}^n \pr(y_{G_i} \mid y_{q(G_i)}).
$$
and keep going repeating this again and again obtaining
\begin{align*}
   \pr(y_{\set{G \in \mathcal{G} : G \ge H_1}})
   & =
   \prod_{\substack{G \in \mathcal{G} \\ G \ge H_1}}
   \pr(y_G \mid y_{q(G)})
   \\
   & =
   \pr(y_{H_1} \mid y_{q(H_1)})
   \prod_{\substack{G \in \mathcal{G} \\ G > H_1}}
   \pr(y_G \mid y_{q(G)})
\end{align*}
(if $G_1 = H_1$ we haven't done anything yet and this is just the same
factorization as \eqref{eq:factorize} in different notation).

Now we have to be careful with our notation.  Define
$$
   Q = \set{ q(H) : H \in \mathcal{H} }
$$
we need to not sum-integrate out any components of $y_Q$.

If $G_{i_1 + 1} \neq H_2$, then we want to sum-integrate out
$y_{G_{i_1 + 1} \setminus Q}$ obtaining
\begin{multline*}
   \pr(y_{H_1 \cup \{q(H_1)\} \cup \set{G \in \mathcal{G} : G > G_{i_1 + 1}}})
   \\
   =
   \pr(y_{H_1} \mid y_{q(H_1)})
   \pr(y_{G_{i_1 + 1} \cap Q} \mid y_{q(G_{i_1 + 1})})
   \prod_{j = i_1 + 2}^n
   \pr(y_{G_i} \mid y_{q(G_i)}).
\end{multline*}
Since $G_{i_1 + 1} \cap Q$ may be the empty set, we need to say what
$y_\emptyset$ means.  This is the subvector of length zero that is really
the only function $\emptyset \to \real$, which is the empty function
(which has the empty graph).  Hence $y_\emptyset$ has only one possible
value (the empty function) and must be a constant random vector.
Thus $\pr(y_\emptyset \mid y_j) = 1$ regardless of what $y_j$ is when
$y_\emptyset$ has the only possible value it can have.

Continuing this process, we obtain
\begin{multline*}
   \pr(y_{H_1 \cup \{q(H_1)\} \cup \{ G_{i_2}, \ldots, G_n \}})
   \\
   =
   \pr(y_{H_1} \mid y_{q(H_1)})
   \prod_{j = i_1 + 1}^{i_2 - 1}
   \pr(y_{G_j \cap Q} \mid y_{q(G_j)})
   \prod_{j = i_2}^n
   \pr(y_{G_i} \mid y_{q(G_i)}).
\end{multline*}
And we can now see how this process continues
$$
   \pr(y_{H_1 \cup H_2 \cup \cdots \cup H_m \cup Q})
   =
   \prod_{H \in \mathcal{H}}
   \pr(y_H \mid y_{q(H)})
   \prod_{G \in \mathcal{G} \setminus \mathcal{H}}
   \pr(y_{G \cap Q} \mid y_{q(G)})
$$
And now integrating out $y_{H \setminus Q}$ in order gives
\begin{align*}
   \pr(y_Q)
   & =
   \prod_{H \in \mathcal{H}}
   \pr(y_{H \cap Q} \mid y_{q(H)})
   \prod_{G \in \mathcal{G} \setminus \mathcal{H}}
   \pr(y_{G \cap Q} \mid y_{q(G)})
\end{align*}
(Note that every $j \in Q$ appears ``in front of the bar'' in
exactly one of these conditional probabilities because $\mathcal{G}$
is a partition.)
So
\begin{align*}
   \pr(y_{H_1 \cup H_2 \cup \cdots \cup H_m} \mid y_Q)
   & =
   \prod_{H \in \mathcal{H}}
   \frac{ \pr(y_H \mid y_{q(H)}) }{ \pr(y_{H \cap Q} \mid y_{q(H)}) }
   \\
   & =
   \prod_{H \in \mathcal{H}}
   \frac{ \pr(y_{H \cup \{q(H)\}}) }{ \pr(y_{(H \cap Q) \cup \{q(H)\}}) }
   \\
   & =
   \prod_{H \in \mathcal{H}}
   \pr( y_{H \setminus Q} \mid y_{Q} )
   \\
   & =
   \prod_{H \in \mathcal{H}}
   \pr( y_H \mid y_{Q} )
\end{align*}
(the last step being that variables repeated ``in front of'' and ``behind''
the bar in a conditional probability are treated as constant).
And reading from end to end gives the assertion of the theorem.
\end{proof}

It is not clear (to me) that this theorem is much use,
but the following corollary is very important.
\begin{corollary} \label{cor:independent}
If in Theorem~\ref{th:markov-two}
$$
   \mathcal{H} = \set{ G \in \mathcal{G} : q(G) \notin J },
$$
then the random vectors
$y_H$, $H \in \mathcal{H}$ are stochastically independent.
\end{corollary}
\begin{proof}
This is because the random variables $y_{q(H)}$ for $H \in \mathcal{H}$
are actually constant random variables.  Hence conditioning on them
has no effect.  In the notation of the proof of Theorem~\ref{th:markov-two}
$\pr( y_H \mid y_{Q} ) = \pr( y_H )$.
\end{proof}

\section{Two Kinds of Aster Graphs}

The graphs for aster models are often very large with thousands or tens of
thousands of nodes, but usually they are composed of isomorphic subgraphs.
So drawing one of these isomorphic subgraphs is enough.
If you've seen one, you've seen them all.
(Graphs are isomorphic if a drawing of one can be laid on a drawing of the
other with everything --- nodes, lines, and arrows --- matching up.)

An aster graph need not be composed of all isomorphic subgraphs,
but the only published example of that is, as far as I know,
\citet{aster-hornworm}.

To distinguish these two kinds of graphs, we call the aster graph described
in the preceding section the \emph{full aster graph} (we consider the ``full''
redundant but the emphasis may help avoid confusion).

Certain subgraphs of the full aster graph, we then call graphs
for ``individuals'' (in scare quotes for reasons to be explained presently).
These are easier to recognize than describe.

Current aster software (Section~\ref{sec:software} above) forces
$q(G) \neq q(H)$ whenever $G \neq H$ and $q(G)$ and $q(H)$ are initial nodes.
In this case, the graph for an ``individual'' (in scare quotes)
consists of the subgraph consisting of one initial node and all of its
successors or successors of successors or successors of successors
of successors and so forth with arbitrarily many repetitions
of ``successors of'' and all of the arrows and lines in the full graph
connecting these nodes.  (This is where the term ``descendant'' in its
graph-theoretic sense would
come in handy if we allowed ourselves to use it.  The graph for an
``individual'' consists of one initial node, all of its descendant nodes,
and all of the lines and arrows going between these nodes.  But once we
have the idea of the graph for an ``individual'' we no longer need the
term ``descendant.'')

But aster theory as described so far does not force this convention.
If $y_j = 1$, for all initial nodes $j$, which is the case with most
(but not all) aster applications, then it would do no harm if all initial
nodes were fused into one initial node.  That would invalidate nothing but
the way we just described graphs for ``individuals'' (in scare quotes).

Thus we have to be a bit more careful.  If $G$ is a dependence group whose
predecessor $q(G)$ is initial, then the graph for the ``individual''
(in scare quotes) containing $G$ consists of $q(G)$, the nodes in $G$
and their successors or successors of successors or successors of successors
of successors and so forth with arbitrarily many repetitions
of ``successors of'' and all of the arrows and lines in the full graph
connecting these nodes.  (And it would make this definition a little shorter
if we allowed ourselves to use the word ``descendant'' in its graph-theoretic
sense.)

There are two reasons why the scare quotes.
\begin{itemize}
\item In life history
analysis, the graph for an ``individual'' ideally goes one or more times
around the life cycle (exactly).  Thus it may involve data not only for
one biological individual but also for its offspring and perhaps offspring
of offspring (if the experiment goes twice around the life cycle) or even
perhaps more remote descendants (where here ``descendants'' means real
biological descendants, not the graphical models idea of descendants).
\item If the value of the constant $y_j$ at the initial node of the
graph for an ``individual'' is greater than one, then the data for this
``individual'' is actually cumulative data for $y_j$ real biological
individuals.
\end{itemize}

%%%%%%%%%% NEED FORWARD REFERENCE to example graphs %%%%%%%%%%

If one does not like our terminology of ``individual'' in scare quotes,
our advice is to just explain what data the graph is for.  It may actually
be for a biological individual, for a biological individual
and its offspring, or $n$ biological individuals.  Just say what it is.

Or we could use the following characterization.
\begin{theorem}
Subgraphs for ``individuals'' are maximal stochastically independent
subvectors of the response vector.
\end{theorem}
\begin{proof}
Define
$$
   \mathcal{H} = \set{ G \in \mathcal{G} : q(G) \notin J }
$$
and for $H \in \mathcal{H}$
$$
   A_H = \set{ j \in J : (\exists k \in H)(j \succeq k) }.
$$
The elements of $\mathcal{H}$ are disjoint because the elements of
$\mathcal{G}$ are disjoint.
The elements of $\set{A_H : H \in \mathcal{H}}$ are disjoint
because, if $i \in A_{H_1} \cap A_{H_2}$, then $p^{n_1}(i) \in H_1$
and $p^{n_2}(i) \in H_2$ for some natural numbers $n_1$ and $n_2$
using the notation defined in the proof of Theorem~\ref{th:transitive-closure}.
But then $n_1 \le n_2$ implies $A_{H_1} \subset A_{H_2}$ hence $H_1 = H_2$,
hence $A_{H_1} = A_{H_2}$ and the same with 1 and 2 swapped.
Furthermore $\set{A_H : H \in \mathcal{H}}$ is a partition of $J$ because
$J$ is a finite set (aster graphs are not allowed to be infinite).
Hence for every $j \in J$ there exists a natural number $n$ such that
$p^n(j)$ is an initial node.


Furthermore each $G \in \mathcal{G}$ is contained some $A_H$.
If $G \in \mathcal{H}$, then $G \subset A_G$.

Now \eqref{eq:factorize} implies
$$
   \pr(y)
   =
   \prod_{G \in \mathcal{G}} \pr(y_G \mid y_{q(G)})
$$
\end{proof}

