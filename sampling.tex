
\chapter{Subsampling} \label{chap:subsampling}

\section{Introduction}

In aster models the ideal is to actually measure
all components of fitness and make them nodes in the graphical model.
Sometimes, however, it is just too much work to count all of some component
of fitness, for example, all seeds produced by a plant.

For a part of an aster graph
$$
\begin{CD}
   \cdots @>>> y_\text{this} @>>> y_\text{that} @>>> \cdots
\end{CD}
$$
suppose that $y_\text{that}$ is typically too large to count
(by available methods, in available time).

The conditional distribution of $y_\text{that}$ given $y_\text{this}$ is
the sum of $y_\text{this}$
independent and identically distributed (IID) random variables.
This is the predecessor-is-sample-size property
(Section~\ref{sec:piss} above).
In short, $y_\text{that}$ is a random ``sample''
from some ``population'' and the sample size is $y_\text{this}$, where the scare
quotes are to indicate that ``sample'' and ``population'' don't refer
to an actual sample and population but are just a way of discussing
probability that is common in introductory statistics books, which
take finite population sampling (actually taking a random sample from
a known finite population) as an analogy for all applications of probability
theory; any IID set of random variables is called a ``sample'' from
a ``population'' whether or not that makes literal sense.

The obvious solution to our problem is to ``subsample'' the ``sample.''
Take a random sample of the things $y_\text{this}$ counts,
and for that subsample count how many of whatever component
of fitness $y_\text{that}$ counts (this proposal will be generalized
in Section~\ref{sec:generalize} below).  In this case,
\citet[p.~E43]{aster2} proposed to simply insert an extra arrow in the
graph to represent the subsampling process.
Taking a random subsample is a Bernoulli process (flip a ``biased coin''
to decide for each of the $y_\text{this}$ things whether it goes in the
subsample).
So this is a Bernoulli arrow, but we mark it specially as a subsampling arrow
$$
\begin{CD}
   \cdots @>>> y_\text{this} @>\text{samp}>> y_\text{this-sub}
   @>>> y_\text{that-sub} @>>> \cdots .
\end{CD}
$$
Here $y_\text{this}$ is the same variable it was before (the observed count for
``this'' fitness component), $y_\text{this-sub}$ is the subsample size
(the subset of the $y_\text{this}$ things that go in the subsample),
and $y_\text{that-sub}$ is the observed count for ``that'' fitness component
for the subsample).  We no longer observe $y_\text{that}$, which
is what we would have observed if we had not subsampled.  Not having to
count $y_\text{that}$ was the whole point of the subsampling.

\citet{aster2} further proposed to treat aster models with subsampling
just like any other aster model.  This suggestion was backed
up by Section~8 of a supporting technical report \citep{tr661}.
That technical report, however, notes
% that because of the way the aster transform, the map from
%the conditional canonical parameter vector $\theta$ to the unconditional
%canonical parameter vector $\varphi$ \citep[eq.~(5)]{aster1}, works,
this suggestion is not quite the right thing.  It says
\begin{quote}
The somewhat odd thing about this proposal is that the parameter
$p$ [the subsampling probability] is \emph{known} and is a \emph{conditional}
mean value parameter, but we intend to use an \emph{unconditional} aster model
and treat the [corresponding] unconditional canonical parameter as
\emph{unknown} [emphasis in the original].
\end{quote}
and devotes the rest of its Section~8 to a simulation study that shows
that, although, not quite the right thing, it does well enough.

\citet*[Appendix~S1]{stanton-geddes-et-al} show how to, in effect, remove
the effect of subsampling when producing point estimates and confidence
intervals for expected fitness, completing something left undone by
\citet{aster2} and the accompanying technical report.

Here we give a new proposal that does the right thing with subsampling
arrows and hence supersedes all earlier proposals.
We have to realize that aster models with subsampling are no longer
regular full exponential families, so they no longer satisfy the original
rationale for aster models.

With subsampling there are two models we have to consider: the model with
subsampling, which reflects the experiment actually done, and the same model
with subsampling removed, which reflects biology of the organisms being
studied.  We give both of these models the same parameterization (subsampling
arrows in the aster graph have no unknown parameters because the subsampling
probabilities are known).  Once this fundamental realization is made,
everything else about aster models with subsampling follows from well
known likelihood theory.

\section{Subsampling}

\subsection{Graphs With and Without Subsampling}
\label{sec:with-and-without}

We already have one two-way classification of aster graphs: the full aster
graph and graphs for ``individuals'' in scare quotes
(Section~\ref{sec:scare-quotes} above).  Now we introduce a different
two-way classification: with and without subsampling.  Together these give
us a four-way classification.

When using aster models with subsampling, we are also interested in the graph
and the corresponding aster model if subsampling had not been done.
The graph and model with subsampling represent the experiment actually done.
The graph and model without subsampling represent the biology.
We must refer to both in our discussion.  For example, we use the graph
and model with subsampling to estimate parameters, but we use the graph
and model without subsampling to predict biological properties of
organisms from these estimates.

Understanding the correspondence between the two graphs is helped by
referring to the following picture, which is part of an aster graph
with subsampling.
\begin{equation} \label{graph:with}
\begin{CD}
   \cdots @>>> y_i @>>> y_j @>\text{samp}>> y_k @>>> y_m @>>> \cdots
\end{CD}
\end{equation}
Only one subsampling arrow is shown (labeled ``samp'').  The other arrows
are non-subsampling.  Nodes at the head of subsampling arrows (here $y_k$)
are called \emph{subsampling nodes}.  Thus $y_k$ is the only subsampling node
among the nodes shown.
Here $y_j$ is the count of some
sort of thing (flowers, seeds, etc.)\ of actual individuals in the experiment,
and $y_k$ is the count of the same thing for a random subsample of those
individuals who are
carried forward to later stages of the experiment.
Thus $y_j$ and $y_k$ are both measurements of
the \emph{same} component of fitness.
The relationship between
$y_j$ and $y_k$ is artificial (done by the experimenters) and has nothing
to do with biology.

The corresponding graph without subsampling is formed by removing the
subsampling arrow and subsampling node and pasting together the graph
so no break is formed
\begin{equation} \label{graph:without}
\begin{CD}
   \cdots @>>> y_i @>>> y_j @>>> y_m @>>> \cdots
\end{CD}
\end{equation}
This graph corresponds to the experiment that would have been done if there
were no subsampling.

Here is a more complicated example that illustrates that sometimes
it may be necessary to have subsampling arrows following each other.
\begin{equation} \label{graph:with-with}
\begin{tikzcd}[row sep=tiny]
   \hphantom{y_0} & \hphantom{y_1} & \hphantom{y_2} & y_3 \arrow{r} & y_5
   \\
   y_0 \arrow{r} & y_1 \arrow{r}{\text{samp}}
   & y_2 \arrow{ru}{\text{samp}} \arrow{rd}
   \\
   \hphantom{y_0} & \hphantom{y_1} & \hphantom{y_2} & y_4 \arrow{r} & y_6
\end{tikzcd}
\end{equation}
This is the graph for one ``individual'' assuming all ``individuals'' have
isomorphic subgraphs.
Here is the corresponding subgraph when we remove the subsampling arrows.
\begin{equation} \label{graph:with-with-stripped}
\begin{tikzcd}[row sep=tiny]
   \hphantom{y_0} & \hphantom{y_1} & y_5
   \\
   y_0 \arrow{r} & y_1 \arrow{ru} \arrow{rd}
   \\
   \hphantom{y_0} & \hphantom{y_1} & y_4 \arrow{r} & y_6
\end{tikzcd}
\end{equation}

%%%%% NEED FORWARD REFERENCE to theory about how to go (in general)
%%%%% from with to without

\subsection{Notation for Graphs With and Without Subsampling}
\label{sec:with-and-without-notation}

We are going to use mathematical notation that distinguishes analogous
concepts for the full graphs with and without subsampling by decorating
notation for the former with stars.

\subsubsection{Sets of Nodes}

The set of nodes of the full aster graph with subsampling is denoted $N^*$.
The set of non-subsampling nodes in $N^*$ is denoted $N$.  This is the set
of nodes of the full graph without subsampling.

The set of non-initial nodes in $N^*$ is denoted $J^*$.
The set of non-initial nodes in $N$ is denoted $J$.
($J = N \cap J^*$.)

\subsubsection{Families of Sets of Nodes}

The set of dependence groups
(Section~\ref{sec:factorization} above)
for the aster model with subsampling is denoted $\mathcal{G}^*$.
The set of dependence groups
for the aster model without subsampling is denoted $\mathcal{G}$.

Subsampling nodes are always dependence groups by themselves.
No dependence group with more than one node can have any subsampling nodes.
Thus the set of all subsampling nodes is $J^* \setminus J$.

\subsubsection{Predecessor Functions}

We already have two kinds of predecessor functions:
set-to-index predecessor functions denoted $q$
(Section~\ref{sec:factorization} above) and
index-to-index predecessor functions denoted $p$
(Section~\ref{sec:other} above).
Now we have another two-way classification.
We will have these functions with stars to denote with subsampling
and without stars to denote without subsampling.

For example, in \eqref{graph:with} we have $p^*(m) = k$ but $p(m) = j$.
The latter agrees with \eqref{graph:without}, as it must.

For another example, in \eqref{graph:with-with} we have
\begin{align*}
   p^*(6) & = 4
   \\
   p^*(5) & = 3
   \\
   p^*(4) & = 2
   \\
   p^*(3) & = 2
   \\
   p^*(2) & = 1
   \\
   p^*(1) & = 0
\end{align*}
but
\begin{align*}
   p(6) & = 4
   \\
   p(5) & = 1
   \\
   p(4) & = 1
   \\
   p(1) & = 0
\end{align*}
The latter agrees with \eqref{graph:with-with-stripped}, as it must.

We have not given examples of graphs with subsampling and dependence
groups, but the set-to-index predecessor functions work the same way
\begin{alignat*}{2}
   p(j) & = q(G), & \qquad & j \in G \in \mathcal{G}
   \\
   p^*(j) & = q^*(G), & & j \in G \in \mathcal{G}^*
\end{alignat*}

\subsubsection{Partial Orders}

In Section~\ref{sec:closure} above we introduced partial orders on the
node set of the graph that tell us about predecessors, predecessors of
predecessors, and so forth.  Unlike the case with the other mathematical
objects just discussed, we do not need starred and unstarred versions
for these.

As in Section~\ref{sec:closure} above, let $\succ$ denote the transitive
closure of the predecessor relation on $N^*$.  Then the transitive
closure of the predecessor relation on $N$ is just the same relation
as on $N^*$ but restricted to $N$, that is $j \succ k$ in $N$ if and only
$j \succ k$ when $j$ and $k$ are considered elements of $N^*$.

And similarly for $\succeq$, $\prec$, and $\preceq$.

\subsubsection{Going from One to the Other}

The user specifies the graph with subsampling.  The computer should figure
out the corresponding graph without subsampling (so no mistakes about that
are made).

For this we use the notation $p^k$ for $k$-fold composition of a function
with itself from dynamical systems theory that is also explained in
Section~\ref{sec:closure} above.

Suppose we are trying to determine $p(j)$ for $j \in J$.
Define
$$
   m = \min \set{ k > 0 : (p^*)^k(j) \in N }
$$
Then $p(j) = (p^*)^m(j)$.

In words, we go back in the graph with subsampling looking at the predecessor,
predecessor of predecessor, and so forth until we find one that is not a
subsampling node, and that is the predecessor in the graph without subsampling.

\subsection{Models With and Without Subsampling}
\label{sec:models-with-and-without}

Now that we have the relationship between the graph with and without
subsampling, and hence the factorization (Section~\ref{sec:factorization}
above) with and without subsampling we need to consider statistical models
that go with these factorizations.  Models without subsampling are those
already described (Chapter~\ref{ch:introduction} above).

So we have to describe models with subsampling here.
There are two key ideas.
\begin{itemize}
\item The conditional distributions for subsampling arrows are considered
\emph{known}.
Thus they have \emph{no} unknown parameter values to estimate.
\item The conditional distributions for non-subsampling arrows should be
the same for the models with and without subsampling.  They should have
the same statistical models parameterized in the same way
(this statement is imprecise; it depends on which parameters
we are talking about).
\end{itemize}

\subsubsection{Data}

First we introduce new notation for data (not used above) to distinguish
the models with and without subsampling.  Let $y^*$ denote the response
vector for the model with subsampling and $y$ the response
vector for the model without subsampling.

\subsubsection{Factorization}

Then the model with subsampling factorizes as
\begin{equation} \label{eq:factorization-with}
   f^*(y^*) = \prod_{G \in \mathcal{G}^*} f^*_G(y^*_G \mid y^*_{q^*(G)})
\end{equation}
and the model without subsampling factorizes as
\begin{equation} \label{eq:factorization-without}
   f(y) = \prod_{G \in \mathcal{G}} f_G(y_G \mid y_{q(G)}).
\end{equation}
Now our first principle above says that for $G \notin \mathcal{G}$
$f^*_G(\fatdot \mid \fatdot)$ has no parameters,
and our second principle above says that for $G \in \mathcal{G}$
$$
   f^*_{G, \theta_G}(\fatdot \mid \fatdot)
   =
   f_{G, \theta_G}(\fatdot \mid \fatdot)
$$
that is, these are the same conditional distributions (the same functions
of the variables on each side of the vertical bar) for the same parameter
values.

\subsubsection{Parameterization}

This also makes clear what parameters we are talking about.
The \emph{conditional canonical parameter vector} $\theta$
(Section~\ref{sec:aster-transform} above)
is the same for the models with and without subsampling.
But we also want the aster transform to be the same
(also Section~\ref{sec:aster-transform} above), so
the \emph{unconditional canonical parameter vector} $\varphi$ will also be
the same for the models with and without subsampling.

\subsubsection{Mean Value Parameters}

The \emph{conditional mean value parameter vector} $\xi$ will differ
for the models with and without subsampling simply because the model
with subsampling has more arrows.  The components of $\xi$ will be the
same for the same arrows
(Sections~\ref{sec:conditional-and-unconditional-mean-values}
and~\ref{sec:aster-mean-value-parameters} above), that is, defining
\begin{align*}
   \xi_j & = E(y_j \mid y_{p(j)} = 1)
   \\
   \xi^*_j & = E(y_j^* \mid y^*_{p^*(j)} = 1)
\end{align*}
we have
$$
   \xi_j = \xi^*_j = \frac{\partial c_G(\theta_G)}{\partial \theta_j}
$$
whenever $j$ is not a subsampling node and $j \in G \in \mathcal{G}$.
But when $j$ is a subsampling node there is no $\xi_j$ since there is
no node $j$ in the model without subsampling, and $\xi^*_j$ is not a function
of the parameters of the model ($\theta$ and $\varphi$); it is not an unknown
parameter but rather a known constant (this will be revisited in the
next section).

Thus when we define
\begin{align*}
   \mu & = E(y)
   \\
   \mu^* & = E(y^*)
\end{align*}
we will still have the relationship between $\mu$ and $\xi$ discussed
in Section~\ref{sec:mu-and-xi} above.
And we will have the analogous equations with stars for the relationship
between $\mu^*$ and $\xi^*$ but the relations between $\mu$ and $\mu^*$
will be very complicated and depend on the whole aster graph.
But everything in this section depends on subsampling having the
predecessor-is-sample-size property, which we drop in the next section.
So nothing in this section holds for general aster models with subsampling.

\subsection{Generalizing Our Notion of Subsampling} \label{sec:generalize}

So far, we have been assuming the conditional distributions for the
subsampling arrows obey the predecessor-is-sample-size principle.  This
means the conditional distribution for sample size one is Bernoulli,
so subsampling arrows are Bernoulli arrows (but ones whose conditional
canonical parameters are fixed and known rather than unknown parameters
to be estimated).  And it means the conditional distribution of $y_j^*$
given $y_{p^*(j)}^*$ is binomial (when $j$ is a subsampling node).
Let $\xi_j^*$ be the usual parameter for the binomial distribution,
which is also the mean value parameter for the Bernoulli distribution.
Then, defining $\mu^* = E(y^*)$, we have
the analog of \eqref{eq:mu-and-xi} with stars
\begin{equation} \label{eq:mu-and-xi-star}
   \mu_j^* = \xi_j^* \mu_{p(j)}^*, \qquad j \in J^*.
\end{equation}
and all of the consequences \eqref{eq:mu-and-xi} found in
Section~\ref{sec:mu-and-xi} above, except with stars in the appropriate places.

But when we drop the predecessor-is-sample-size principle for subsampling
arrows \eqref{eq:mu-and-xi-star} no longer holds, and there is no longer
any notion of $\xi^*_j$ for such arrows analogous to non-subsampling
arrows.  We still do have conditional mean values,
$$
   E\bigl\{ y_j^* \mid y_{p^*(j)}^*\bigr\}
$$
but these do not need to satisfy
$$
   E\bigl\{ y_j^* \mid y_{p^*(j)}^*\bigr\}
   =
   \xi_j^* y_{p^*(j)}^*
$$
for any constant $\xi_j^*$ when $j$ is a subsampling node that does not
obey the predecessor-is-sample-size principle, but rather are arbitrary
functions of $y_{p^*(j)}^*$.

This means that conditional mean value parameters
for subsampling arrows make no sense for subsampling arrows that are not
Bernoulli (do not obey the predecessor-is-sample-size principle).
Since many biologists use forms of subsampling that are not Bernoulli
(not simple random sample), we do not want to enforce the Bernoulli
assumption.

All we assume about subsampling distributions is that they are known,
having no unknown parameters to estimate.
We make no other assumptions about them.

We do, of course, have an unconditional mean value parameter vector $\mu^*$
with components
\begin{equation} \label{eq:mu-star}
   \mu^*_j = E(Y^*_j), \qquad j \in J^*
\end{equation}
but since the computer knows nothing about the subsampling distributions
(they can be any distributions), the computer will be unable to compute
them.  If users want to use $\mu^*$ somehow, they will have to provide
it themselves.

Thus $\xi^*$ is undefined, in general, and $\mu^*$, although defined,
is no longer anything the computer can deal with.

Fortunately $\xi^*$ and $\mu^*$ are unbiological.  So users will mostly,
perhaps always, only be interested in $\xi$ and $\mu$, which the computer
can deal with.

\subsection{Log Likelihood}
\label{sec:logl}

When there is no subsampling, the saturated aster model log likelihood
for $\theta$ is given by \eqref{eq:logl-aster-theta} which can be rewritten
\begin{equation} \label{eq:logl}
   l(\theta)
   =
   \sum_{j \in J} y_j \theta_j
   - \sum_{G \in \mathcal{G}} y_{q(G)} c_G(\theta_G).
\end{equation}

\begin{theorem} \label{th:logl-theta}
The log likelihood for a model with subsampling is given by
\begin{equation} \label{eq:logl-with}
   l(\theta)
   =
   \sum_{j \in J} y^*_j \theta_j
   - \sum_{G \in \mathcal{G}} y^*_{q^*(G)} c_G(\theta_G).
\end{equation}
\end{theorem}
In \eqref{eq:logl} and \eqref{eq:logl-with}
$c_G$ is the cumulant function for the exponential family
for dependence group $G$.
\begin{proof}
To be clear, we write down the log likelihood for all the data
using the notation of Section~\ref{sec:generalize} above
$$
   l(\theta^*)
   =
   \sum_{j \in J} y_j^* \theta_j
   - \sum_{G \in \mathcal{G}} y_{q^*(G)}^* c_G(\theta_G)
   +
   \sum_{j \in J^* \setminus J} \log f_j(y_j^* \mid y_{p^*(j)}^*)
$$
and we are allowed to drop terms that do not contain unknown parameters from the
log likelihood because this makes no difference to either frequentist or
Bayesian inference.  This gives \eqref{eq:logl-with}.
\end{proof}

The cumulant function $c_G$ satisfies
\begin{equation} \label{eq:theta-to-xi}
\begin{split}
   \frac{\partial c_G(\theta_G)}{\partial \theta_j}
   & =
   E_\theta(y_j \mid y_{q(G)} = 1)
   \\
   & =
   E_\theta(y^*_j \mid y_{q^*(G)} = 1)
   \\
   & =
   \xi_j
   \\
   & =
   \xi^*_j,
   \qquad j \in G \in \mathcal{G},
\end{split}
\end{equation}
and
\begin{equation} \label{eq:gamma-pre}
\begin{split}
   \frac{\partial^2 c_G(\theta_G)}{\partial \theta_j \partial \theta_k}
   & =
   \cov_\theta(y_j, y_k \mid y_{q(G)} = 1)
   \\
   & =
   \cov_\theta(y^*_j, y^*_k \mid y^*_{q^*(G)} = 1),
   \qquad j, k \in G \in \mathcal{G},
\end{split}
\end{equation}
and these derivatives are zero if $j \notin G$ or (in the latter)
$k \notin G$.

If the conditioning event in these equations has probability
zero (which happens in actual aster models if the family in question is
$k$-truncated with $k > 0$, \citealp{aster2}, have an example), then the
conditional expectations are not well defined, but we still have
\begin{equation} \label{eq:xi-from-c}
   \frac{\partial c_G(\theta_G)}{\partial \theta_j}
   =
   \xi_j
   =
   \xi^*_j
\end{equation}
with $\xi_j$ being defined by the more long winded and careful
definition given in Section~\ref{sec:xi} above when the conditioning
event in \eqref{eq:theta-to-xi} has probability zero.

The reason for the equality of starred and unstarred quantities is that
we want the model without subsampling to be the same as the model
with subsampling when the subsampling arrows are removed as explained
in Section~\ref{sec:with-and-without} above and in this section.
The distribution of $y_G$ given $y_{q(G)} = n$ is the same as
the distribution of $y^*_G$ given $y^*_{q^*(G)} = n$.

There is a similar adjustment to be made for the conditional covariances above
when their conditioning events have probability zero.
We still have that $y_G$ is the sum of $y_{q(G)}$ IID random vectors
and $\partial^2 c_G(\theta_G) / \partial \theta_j \partial \theta_k$
is the unconditional covariance of the $j$ and $k$ components of one
of those random vectors.

\subsection{Aster Transform}
\label{sec:aster-transform-subsampling}

We have the aster transform and inverse aster transform
(Section~\ref{sec:aster-transform}) and these hold for models with
subsampling because they are the same (we assume) as for models
without subsampling.  And these determine $\mu$ and $\xi$ as discussed
in the preceding section.  And we are generally uninterested in
$\mu^*$ and $\xi^*$ as discussed in the preceding section.  Thus
we only have parameters without stars.  We have aster graphs with
stars and aster data with stars, but not parameters.

\subsection{Canonical Affine Submodels}

As is the situation without subsampling, we are interested in
canonical affine submodels (Section~\ref{sec:canonical-affine-submodels}
above) when we have subsampling.  And we want them to be the same models
with the same parameterizations with and without subsampling.  Thus they
are the same as in Section~\ref{sec:canonical-affine-submodels} above.
Unconditional canonical affine submodels make $\varphi$ an affine function
of the submodel parameters $\beta$.
Conditional canonical affine submodels make $\theta$ an affine function
of the submodel parameters $\beta$.

\subsubsection{Conditional}

With subsampling, a conditional aster model still has a concave log likelihood
for the reasons discussed in Section~\ref{sec:conditional-aster-model-mle}
above.  The saturated model log likelihood \eqref{eq:logl} is a sum
of linear and concave functions of $\theta$.  Therefore the submodel
log likelihood is the composition of a concave function and an affine
function, which is again a convex function.  This means the MLE will
be easily found by the computer (by any algorithm that always checks
that it goes uphill on the likelihood in every iteration).  It does
not mean that conditional aster models have any other properties of
regular full exponential families (but this is true with or without
subsampling, as was discussed in Chapter~\ref{ch:introduction}).

\subsubsection{Unconditional}

With subsampling, an unconditional aster model is no longer an exponential
family and does not have any exponential family properties.  All it has
is the usual asymptotics of maximum likelihood.
%%%%% NEED FORWARD REFERENCE to veryfying that

\subsection{Differentiating the Aster Transform and Its Inverse}
\label{sec:aster-transform-deriv}

Here we follow Section~A.2 of the technical report \citet{aster1-tr} which
backs up the paper \citet{aster1}.  Our notation is different from their
notation, our notation here is what we have used for aster models since
\citet{aster-philosophical}.

\subsubsection{Aster Transform}

Let $\Delta \theta_j$ denote an infinitesimal increment of $\theta_j$
and similarly for $\Delta \varphi_j$.  Then
differentiating \eqref{eq:aster-transform} and using \eqref{eq:xi-from-c} and
the chain rule gives
\begin{equation} \label{eq:aster-transform-deriv}
\begin{split}
   \Delta \varphi_j
   & =
   \Delta \theta_j
   -
   \sum_{\substack{G \in \mathcal{G} \\ q(G) = j}} \sum_{k \in G}
   \xi_k \Delta \theta_k
   \\
   & =
   \Delta \theta_j
   -
   \sum_{\substack{k \in J \\ p(k) = j}}
   \xi_k \Delta \theta_k
\end{split}
\end{equation}

In language that does not refer to infinitesimals and using the
sophisticated view that derivatives are linear transformations
(\citealp[Definition~8.9]{browder}; \citealp[p.~334]{lang}),
 the derivative of the
aster transform is the linear transformation that maps
the vector $\Delta \theta$ having components $\Delta \theta_j$ to
the vector $\Delta \varphi$ having components $\Delta \varphi_j$.

We can think of this linear transformation as being represented by the matrix
of partial derivatives, which can be read off \eqref{eq:aster-transform-deriv},
\begin{equation} \label{eq:aster-transform-deriv-partial}
   \frac{\partial \varphi_j}{\partial \theta_k}
   =
   \begin{cases}
   1, & j = k \\
   - \xi_k, & j = p(k) \\
   0, & \text{otherwise}
   \end{cases}
\end{equation}

\subsubsection{Inverse Aster Transform}
\label{sec:inverse-aster-transform-deriv}

By the inverse function theorem, the derivative of the inverse is the
inverse of the derivative (considered as a linear transformation), assuming
it exists, which it does if the derivative is invertible
\cite[p.~361--363]{lang}.  We will prove the derivative is invertible by
inverting it.

As discussed in Section~\ref{sec:aster-transform} above,
the formula \eqref{eq:inverse-aster-transform}
gives an inductive definition that works when nodes of the graph are visited
in any order that visits successors before predecessors.

The same is true of the derivative of the inverse aster transform.
Moving a term from one side of \eqref{eq:aster-transform-deriv}
to the other gives
\begin{equation} \label{eq:inverse-aster-transform-deriv}
   \Delta \theta_j
   =
   \Delta \varphi_j
   +
   \sum_{\substack{k \in J \\ p(k) = j}} \xi_k \Delta \theta_k.
\end{equation}
When $\Delta \theta_j$ on the left-hand side is computed,
all of the $\Delta \theta_k$ on the
right-hand side will already have been computed (when we visit successors
before predecessors).

As before, the derivative of the inverse
aster transform is the linear transformation that maps
the vector $\Delta \varphi$ having components $\Delta \varphi_j$ to
the vector $\Delta \theta$ having components $\Delta \theta_j$.

Because of the nature of inductive definitions, the analog
of \eqref{eq:aster-transform-deriv-partial} for the inverse transform
is a bit more complicated.
To help with it we introduce the following notation and conventions.
Let $\bone(\fatdot)$ denote the function that maps logical formulas
to numbers, mapping false formulas to zero and true formulas to one,
and define empty sums (those having no terms) to be equal to zero
(the identity for addition)
and empty products (those having no terms) to be equal to one
(the identity for multiplication).

\begin{theorem} \label{th:inverse-aster-transform-deriv-partial}
Partial derivatives of the inverse aster transform are given by
\begin{equation} \label{eq:inverse-aster-transform-deriv-partial}
   \frac{\partial \theta_j}{\partial \varphi_k}
   =
   \bone(j \preceq k)
   \prod_{\substack{i \in J \\ j \prec i \preceq k}} \xi_i.
\end{equation}
\end{theorem}
In \eqref{eq:inverse-aster-transform-deriv-partial} the product is empty
when $j = k$.
(The product is also empty when $j \succ k$ but then we have 
$\bone(j \preceq k) = 0$ so it does not matter what the value of the product
is.)
\begin{proof}
What is to be shown is that \eqref{eq:inverse-aster-transform-deriv-partial}
agrees with \eqref{eq:inverse-aster-transform-deriv} in the case
$\Delta \varphi$ has only one nonzero component, say $\Delta \varphi_n$.
In this case \eqref{eq:inverse-aster-transform-deriv} says
\begin{alignat*}{2}
   \Delta \theta_j & = 0, & \qquad & j \not\preceq n
   \\
   \Delta \theta_j
   & =
   \Delta \varphi_n, & & j = n
   \\
   \Delta \theta_j
   & =
   \sum_{\substack{k \in J \\ p(k) = j}} \xi_k \Delta \theta_k,
   & & j \prec n
\end{alignat*}
or
\begin{alignat*}{2}
   \frac{\partial \theta_j}{\partial \varphi_n} & = 0,
   & \qquad & j \not\preceq n
   \\
   \frac{\partial \theta_j}{\partial \varphi_n} & = 1, & & j = n
   \\
   \frac{\partial \theta_j}{\partial \varphi_n}
   & =
   \sum_{\substack{k \in J \\ p(k) = j}} \xi_k 
   \frac{\partial \theta_k}{\partial \varphi_n},
   & & j \prec n
\end{alignat*}
Clearly the first two lines agree with
\eqref{eq:inverse-aster-transform-deriv-partial}.
That leaves only the third line to check.
We note in this line that the partial derivative on the right-hand side
is zero unless $j \prec k \preceq n$ and that there is hence exactly one
term in the sum.
Thus the third line just above agrees with
\eqref{eq:inverse-aster-transform-deriv-partial} by mathematical induction.
\end{proof}

\subsection{Log Likelihood Derivatives}
\label{sec:log-likelihood-derivatives}

\subsubsection{First Derivatives With Respect To $\boldsymbol{\theta}$}
\label{sec:wrt-theta}

Applying \eqref{eq:xi-from-c} and \eqref{eq:theta-to-xi}
to \eqref{eq:logl-with}, we obtain
\begin{equation} \label{eq:logl-deriv-wrt-theta}
   \frac{\partial l(\theta)}{\partial \theta_j}
   =
   y^*_j - y^*_{p^*(j)} \xi_j,
   \qquad j \in J.
\end{equation}
These are the first derivatives of the log likelihood for a saturated model
with subsampling with respect to components of $\theta$.

Notice that, as always, there is the curious mix of starred and unstarred
thingummies.  The data and the predecessor function have stars because this
is for models with subsampling.  The parameters do not have stars because
we insist that models have the same parameters with and without subsampling.
The index set for \eqref{eq:logl-deriv-wrt-theta} is $J$ because that is
the index set for $\theta$.

\subsubsection{First Derivatives for CAM}
\label{sec:logl-cam-deriv}

In conditional aster models (CAM) with model equation
\eqref{eq:affine-conditional}
we have $\partial \theta_j / \partial \beta_k = m_{j k}$
where $M$ has components $m_{j k}$.
Thus, by \eqref{eq:logl-deriv-wrt-theta} and the chain rule
$$
   \frac{\partial l(\beta)}{\partial \beta_k}
   =
   \sum_{j \in J}
   \frac{\partial l(\theta)}{\partial \theta_j}
   \frac{\partial \theta_j}{\partial \beta_k}
   =
   \sum_{j \in J} ( y^*_j - y^*_{p^*(j)} \xi_j ) m_{j k}
$$
MLE are derived by setting these equal to zero, considering the $\xi$'s
as functions of $\beta$, and solving for $\beta$.

\subsubsection{First Derivatives With Respect To $\boldsymbol{\varphi}$}
\label{sec:wrt-phi}

\begin{theorem} \label{th:partial-like-wrt-phi}
For $k \in J$ define
\begin{equation} \label{eq:infimum}
   n = \inf \set{ j \in J : j \preceq k }.
\end{equation}
Then
\begin{equation} \label{eq:partial-like-wrt-phi}
   \frac{\partial l(\varphi)}{\partial \varphi_k}
   =
   y^*_k
   -
   y^*_{p^*(n)}
   \left(
   \prod_{\substack{i \in J \\ n \preceq i \preceq k}} \xi_i
   \right)
   +
   \sum_{\substack{m \in J \\ n \prec m \preceq k}}
   \left[ y^*_{p(m)} - y^*_{p^*(m)} \right]
   \left(
   \prod_{\substack{i \in J \\ m \preceq i \preceq k}} \xi_i
   \right)
\end{equation}
\end{theorem}
The infimum in \eqref{eq:infimum} means $j \preceq k$ implies $n \preceq j$.
The at-most-one-predecessor property
(Section~\ref{sec:with-and-without-notation} above)
makes the set $\set{ j \in J : j \preceq k }$ totally ordered by $\preceq$.
This set is also nonempty (it contains $k$)
and finite (aster graphs are finite).
Hence the infimum in \eqref{eq:infimum} is always well defined.

The conventions that
empty sums are zero and empty products are one (established in
Section~\ref{sec:inverse-aster-transform-deriv} above) are still in force.

If a node $p^*(m)$ is not a subsampling node, then $p(m) = p^*(m)$.
Hence, if there is no subsampling,
all terms $y^*_{p(m)} - y^*_{p^*(m)}$ in \eqref{eq:partial-like-wrt-phi}
are zero, and \eqref{eq:partial-like-wrt-phi} reduces to
$$
   \frac{\partial l(\varphi)}{\partial \varphi_k}
   =
   y_k - y_{p(n)}
   \prod_{\substack{i \in J \\ n \preceq i \preceq k}} \xi_i
   =
   y_k - \mu_k
$$
by \eqref{eq:xi-mu-prod}, and this
agrees with previous aster theory \citep[Section~3.2]{aster1}.

Node $p(n)$ is the unique initial node of the full aster graph satisfying
$p(n) \prec k$.  If $p^*(n) \neq p(n)$, then $p^*(n)$ is a subsampling node.

\begin{proof}
\begin{align*}
   \frac{\partial l(\varphi)}{\partial \varphi_k}
   & =
   \sum_{j \in J}
   \frac{\partial l(\theta)}{\partial \theta_j}
   \frac{\partial \theta_j}{\partial \varphi_k}
   \\
   & =
   \sum_{\substack{j \in J \\ j \preceq k}}
   ( y^*_j - y^*_{p^*(j)} \xi_j )
   \prod_{\substack{i \in J \\ j \prec i \preceq k}} \xi_i
   \\
   & =
   \left(
   \sum_{\substack{j \in J \\ j \preceq k}}
   y^*_j
   \prod_{\substack{i \in J \\ j \prec i \preceq k}} \xi_i
   \right)
   -
   \left(
   \sum_{\substack{j \in J \\ j \preceq k}}
   y^*_{p^*(j)}
   \prod_{\substack{i \in J \\ j \preceq i \preceq k}} \xi_i
   \right)
   \\
   & =
   y^*_k
   -
   y^*_{p^*(n)}
   \left(
   \prod_{\substack{i \in J \\ n \preceq i \preceq k}} \xi_i
   \right)
   \\
   & \qquad
   +
   \left(
   \sum_{\substack{j \in J \\ n \preceq j \prec k}}
   y^*_j
   \prod_{\substack{i \in J \\ j \prec i \preceq k}} \xi_i
   \right)
   -
   \left(
   \sum_{\substack{m \in J \\ n \prec m \preceq k}}
   y^*_{p^*(m)}
   \prod_{\substack{i \in J \\ m \preceq i \preceq k}} \xi_i
   \right)
   \\
   & =
   y^*_k
   -
   y^*_{p^*(n)}
   \left(
   \prod_{\substack{i \in J \\ n \preceq i \preceq k}} \xi_i
   \right)
   \\
   & \qquad
   +
   \left(
   \sum_{\substack{m \in J \\ n \prec m \preceq k}}
   y^*_{p(m)}
   \prod_{\substack{i \in J \\ m \preceq i \preceq k}} \xi_i
   \right)
   -
   \left(
   \sum_{\substack{m \in J \\ n \prec m \preceq k}}
   y^*_{p^*(m)}
   \prod_{\substack{i \in J \\ m \preceq i \preceq k}} \xi_i
   \right)
   \\
   & =
   y^*_k
   -
   y^*_{p^*(n)}
   \left(
   \prod_{\substack{i \in J \\ n \preceq i \preceq k}} \xi_i
   \right)
   +
   \sum_{\substack{m \in J \\ n \prec m \preceq k}}
   \left[ y^*_{p(m)} - y^*_{p^*(m)} \right]
   \left(
   \prod_{\substack{i \in J \\ m \preceq i \preceq k}} \xi_i
   \right)
\end{align*}
where the first equality is the chain rule,
the second equality is \eqref{eq:inverse-aster-transform-deriv-partial}
and \eqref{eq:logl-deriv-wrt-theta}, and the rest are just algebra.
\end{proof}

\subsubsection{First Derivatives for UAM}
\label{sec:logl-uam-deriv}

Hence, for unconditional aster models (UAM),
$$
   \frac{\partial l(\beta)}{\partial \beta_n}
   =
   \sum_{k \in J}
   \frac{\partial l(\varphi)}{\partial \varphi_k}
   m_{k n}.
$$

\subsubsection{Second Derivatives With Respect To $\boldsymbol{\theta}$}
\label{sec:wrt-theta-theta}

Because the R function \texttt{mlogl} in R package \texttt{aster} calculates
\emph{minus} the log likelihood and its first and second derivatives, we do
the same.
Negating and differentiating \eqref{eq:logl-deriv-wrt-theta} gives
\begin{equation} \label{eq:logl-deriv-wrt-theta-theta}
   - \frac{\partial^2 l(\theta)}{\partial \theta_j \partial \theta_k}
   =
   y^*_{p^*(j)} \gamma_{j k},
   \qquad j, k \in J,
\end{equation}
where
\begin{equation} \label{eq:gamma}
   \gamma_{j k} = \begin{cases}
   \partial^2 c_G(\theta_G) / \partial \theta_j \partial \theta_k,
   & j, k \in G \in \mathcal{G}
   \\
   0, & \text{otherwise} \end{cases}
\end{equation}
See \eqref{eq:gamma-pre} above for more on $\gamma_{j k}$.

\subsubsection{Hessian for CAM}
\label{sec:mlogl-hessian-cam}

Second derivative matrices are commonly called Hessian matrices in optimization
theory.  If the matrix having components \eqref{eq:logl-deriv-wrt-theta-theta}
is denoted $H(\theta)$, then the Hessian for $\beta$ is given by
\begin{equation} \label{eq:hessian-beta}
   H(\beta) = M^T H(\theta) M, \qquad \text{when $\theta = a + M \beta$},
\end{equation}
where, as usual, $a$ is the offset vector and $M$ is the model matrix.
This is because $M$ is the Jacobian matrix of the parameter transformation
$\beta \to \theta$ in a CAM, and \eqref{eq:hessian-beta} is just the chain
rule.

\subsubsection{Hessian for UAM}
\label{sec:mlogl-hessian-uam}

If we follow the preceding section in denoting the Hessian for $\beta$
by $H(\beta)$, the Hessian for $\theta$ by $H(\theta)$, and the Hessian
for $\varphi$ by $H(\varphi)$, then
\begin{equation} \label{eq:hessian-beta-uam}
   H(\beta) = M^T H(\varphi) M, \qquad \text{when $\varphi = a + M \beta$},
\end{equation}
and the argument is exactly the same as in the preceding section.

\subsection{Second Derivatives With Respect To $\boldsymbol{\varphi}$}
\label{sec:wrt-phi-phi}

That leaves us with having to derive $H(\varphi)$, which is complicated.
Before differentiating \eqref{eq:partial-like-wrt-phi} again, it will
simplify computations if we recognize that the terms in big round brackets
are ``parameterized common sub-expressions'' (one has $n$ where the other
has $m$).  Using \eqref{eq:inverse-aster-transform-deriv-partial}, we can
give one of these another notation
\begin{equation} \label{eq:product-subexpression}
   \frac{\partial \theta_{p(m)}}{\partial \varphi_k}
   =
   \prod_{\substack{i \in J \\ m \preceq i \preceq k}} \xi_i,
\end{equation}
but when we change $m$ to $n$ in \eqref{eq:product-subexpression}
to get a notation for the other
term in big round brackets in \eqref{eq:partial-like-wrt-phi} this
does not work, because $p(n)$ is an initial node so there is no
parameter $\theta_{p(n)}$.
This would work, however, if we imagine that the aster graph we are
working with is part of a larger aster graph in which $p(n)$ is not
initial and not a subsampling node.
If we work under this fiction, we should get correct mathematics.

\begin{lemma} \label{lem:kumquat}
\begin{equation} \label{eq:kumquat}
   \frac{\partial^2 \theta_{p(m)}}{\partial \varphi_j \partial \varphi_k}
   =
   \sum_{\substack{r \in J \\ p(m) \prec r \preceq k}}
   \sum_{\substack{s \in J \\ p(m) \prec s \preceq j}}
   \gamma_{r s}
   \prod_{\substack{t \in J \\ p(m) \prec t \prec r}} \xi_t
   \prod_{\substack{u \in J \\ s \prec u \preceq j}} \xi_u
   \prod_{\substack{v \in J \\ r \prec v \preceq k}} \xi_v
\end{equation}
When all dependence groups are singletons, as with R package {\tt aster},
this specializes to
\begin{equation} \label{eq:kumquat-no-groups}
   \frac{\partial^2 \theta_{p(m)}}{\partial \varphi_j \partial \varphi_k}
   =
   \sum_{\substack{r \in J \\ p(m) \prec r \preceq k \\ p(m) \prec r \preceq j}}
   \gamma_{r r}
   \prod_{\substack{t \in J \\ p(m) \prec t \prec r}} \xi_t
   \prod_{\substack{u \in J \\ r \prec u \preceq j}} \xi_u
   \prod_{\substack{v \in J \\ r \prec v \preceq k}} \xi_v
\end{equation}
\end{lemma}

\begin{lemma} \label{lem:quince}
If $m \prec r$ and $\gamma_{r s} \neq 0$, then $m \prec s$.
\end{lemma}
\begin{proof}
If $r = s$ the assertion is trivial.
Otherwise, $\gamma_{r s} \neq 0$ implies that $r$ and $s$ are in the
same dependence group, say $G$, and none of the arrows for this dependence
group are subsampling arrows.  It follows that $p(r) = p(s) = q(G)$.
So $m \prec r$ implies $m \preceq q(G)$ implies $m \prec s$.
\end{proof}

\begin{proof}[Proof of Lemma~\ref{lem:kumquat}]
\begin{align*}
   \frac{\partial^2 \theta_{p(m)}}{\partial \varphi_j \partial \varphi_k}
   & =
   \frac{\partial}{\partial \varphi_j}
   \prod_{\substack{i \in J \\ m \preceq i \preceq k}} \xi_i
   \\
   & =
   \sum_{\substack{r \in J \\ m \preceq r \preceq k}}
   \frac{\partial \xi_r}{\partial \varphi_j}
   \prod_{\substack{i \in J \\ m \preceq i \preceq k \\ i \neq r}} \xi_i
   \\
   & =
   \sum_{\substack{r \in J \\ m \preceq r \preceq k}}
   \sum_{s \in J}
   \frac{\partial \xi_r}{\partial \theta_s}
   \frac{\partial \theta_s}{\partial \varphi_j}
   \prod_{\substack{i \in J \\ m \preceq i \preceq k \\ i \neq r}} \xi_i
   \\
   & =
   \sum_{\substack{r \in J \\ m \preceq r \preceq k}}
   \sum_{\substack{s \in J \\ s \preceq j}}
   \gamma_{r s}
   \prod_{\substack{u \in J \\ s \prec u \preceq j}} \xi_u
   \prod_{\substack{i \in J \\ m \preceq i \preceq k \\ i \neq r}} \xi_i
   \\
   & =
   \sum_{\substack{r \in J \\ m \preceq r \preceq k}}
   \sum_{\substack{s \in J \\ s \preceq j}}
   \gamma_{r s}
   \prod_{\substack{u \in J \\ s \prec u \preceq j}} \xi_u
   \prod_{\substack{t \in J \\ m \preceq t \prec r}} \xi_t
   \prod_{\substack{v \in J \\ r \prec v \preceq k}} \xi_v
   \\
   & =
   \sum_{\substack{r \in J \\ p(m) \prec r \preceq k}}
   \sum_{\substack{s \in J \\ s \preceq j}}
   \gamma_{r s}
   \prod_{\substack{u \in J \\ s \prec u \preceq j}} \xi_u
   \prod_{\substack{t \in J \\ p(m) \prec t \prec r}} \xi_t
   \prod_{\substack{v \in J \\ r \prec v \preceq k}} \xi_v
\end{align*}
where the first equality is \eqref{eq:product-subexpression},
the second equality is the product rule,
the third equality is the chain rule,
the fourth equality is \eqref{eq:xi-from-c}, \eqref{eq:gamma}, and
\eqref{eq:inverse-aster-transform-deriv-partial},
the fifth equality just splits one product into two products.
and the last equality is $m \preceq r$ if and only if $p(m) \prec r$
and similarly for $t$.  Finally we use Lemma~\ref{lem:quince} to get
$p(m) \prec r$ and $\gamma_{r s} \neq 0$ implies $p(m) \prec s$.

Going from \eqref{eq:kumquat} to \eqref{eq:kumquat-no-groups} is just that,
when all dependence groups are singletons, $\gamma_{r s} \neq 0$ implies
$r = s$.
\end{proof}

\begin{theorem} \label{th:hessian-uam}
Define $n$ by \eqref{eq:infimum}.
For an unconditional aster model with subsampling,
the $j, k$ component of $H(\varphi)$ is
\begin{equation} \label{eq:hessian-groups}
   - \frac{\partial^2 l(\varphi)}{\partial \varphi_j \partial \varphi_k}
   =
   y^*_{p^*(n)}
   \frac{\partial^2 \theta_{p(n)}}{\partial \varphi_j \partial \varphi_k}
   -
   \sum_{\substack{m \in J \\ n \prec m \preceq k}}
   \left[ y^*_{p(m)} - y^*_{p^*(m)} \right]
   \frac{\partial^2 \theta_{p(m)}}{\partial \varphi_j \partial \varphi_k}
\end{equation}
where the second partial derivatives of $\theta$ with respect to $\varphi$
are given by \eqref{eq:kumquat} or \eqref{eq:kumquat-no-groups}
and where these equations are to be used regardless of whether $p(n)$
actually indexes a parameter.
\end{theorem}
\begin{proof}
Immediate from \eqref{eq:partial-like-wrt-phi}.
\end{proof}
As was remarked after Theorem~\ref{th:partial-like-wrt-phi},
a node $p^*(m)$ is not a subsampling node if and only if $p(m) = p^*(m)$,
in which case the term containing $y^*_{p(m)} - y^*_{p^*(m)}$ is exactly zero.

Since cumulant functions are infinitely differentiable,
so is the aster transform, the inverse aster transform,
and aster log likelihoods.  Thus formulas \eqref{eq:kumquat},
\eqref{eq:kumquat-no-groups} and \eqref{eq:hessian-groups}
must be equal when $j$ and $k$ are interchanged.
We have written them in a form so that this is almost obvious.
The only non-obvious spot is when we interchange $j$ and $k$
in \eqref{eq:kumquat} we (in effect) change $p(m) \prec t \prec r$
into $p(m) \prec t \prec s$, but this agrees with Lemma~\ref{lem:quince}:
$t \prec r$ and $\gamma_{r s} \neq 0$ implies $t \prec s$.

\subsection{Fisher Information}
\label{sec:fisher}

\subsubsection{Change of Parameter}

We begin with a theorem about change of Fisher information under smooth change
of parameter that is for general likelihood inference under the ``usual
regularity conditions.''  It has nothing in particular to do with
exponential families.  The ``usual regularity conditions'' hold for all
regular exponential families including curved exponential families.
So they hold for all models in this article.   But they hold for many
other models too.

The only regularity condition we need is the so-called Bartlett identities,
which are usually derived by differentiating the integral of the probability
densities twice with respect to the parameters
\begin{align}
   E_\theta\{ \nabla l(\theta) \} & = 0
   \label{eq:bartlett-general-one}
   \\
   \var_\theta\{ \nabla l(\theta) \} & = - E_\theta \{ \nabla^2 l(\theta) \}
   \label{eq:bartlett-general-two}
\end{align}
where $\nabla l(\theta)$ denotes the vector of first partial derivatives
of the log likelihood, $\nabla^2 l(\theta)$ denotes the matrix of second
partial derivatives of the log likelihood,
and $\var$ denotes the variance operator that produces a variance matrix
(also called variance-covariance matrix, covariance matrix,
and dispersion matrix).

Expected Fisher information is either side of \eqref{eq:bartlett-general-two}.
Observed Fisher information is $- \nabla^2 l(\theta)$.

The theorem is about what happens under a change of parameter
$\theta = g(\psi)$.  In short, the theorem says Fisher information
transforms like a tensor.
If expected Fisher information for $\theta$ is denoted $I(\theta)$,
and observed Fisher information for $\theta$ is denoted $J(\theta)$
(note that the latter is a random quantity despite there being no
explicit indication of this), and the Jacobian of the transformation
is $B(\psi) = \nabla g(\psi)$, then
\begin{equation} \label{eq:transform-like-tensor-expected}
   I(\psi) = B(\psi)^T I(\theta) B(\psi), \qquad \text{when $\theta = g(\psi)$}.
\end{equation}
This is reminiscent of and ultimately derives from the formula for change
of variance under a linear transformation.  If $y = B x$, where $x$ and $y$
are random vectors and $B$ is a known matrix, then
\begin{equation} \label{eq:transform-like-tensor-variance}
   \var(y) = B \var(x) B^T.
\end{equation}
The analogous result for observed Fisher information is a bit trickier.
Changing $I$ to $J$ in \eqref{eq:transform-like-tensor-expected} gives
a statement that is, in general, false.  But it is true when MLE are
plugged in
\begin{equation} \label{eq:transform-like-tensor-observed}
   J(\hat{\psi}) = B(\hat{\psi})^T J(\hat{\theta}) B(\hat{\psi}),
   \qquad \text{when $\hat{\theta} = g(\hat{\psi})$}.
\end{equation}
When $\hat{\theta} = g(\hat{\psi})$ and $\hat{\psi}$ is an MLE, then
$\hat{\theta}$ is also an MLE by by invariance of MLE under
parameter transformation.
\begin{theorem}
Assume \eqref{eq:bartlett-general-one} and \eqref{eq:bartlett-general-two}.
Then statement \eqref{eq:transform-like-tensor-expected} is correct.
If $\hat{\theta} = g(\hat{\psi})$ is a zero of the first derivative of
the log likelihood for $\theta$,
then statement \eqref{eq:transform-like-tensor-observed} is correct.
The change of parameter function $g$ must be injective and differentiable
but need not be surjective.
\end{theorem}
\begin{proof}
By the chain rule
$$
   \nabla l(\psi)
   = \nabla l(\theta) B(\psi), \qquad \text{when $\theta = g(\psi)$}.
$$
Take the variance of both sides and
use \eqref{eq:transform-like-tensor-variance}
to obtain \eqref{eq:transform-like-tensor-expected}.

The situation is more complicated with second derivatives
$$
   - \frac{\partial^2 l(\psi)}{\partial \psi_i \partial \psi_j}
   =
   -
   \left(
   \sum_k \sum_m
   \frac{\partial^2 l(\theta)}{\partial \theta_k \partial \theta_m}
   \frac{\partial \theta_k}{\partial \psi_i}
   \frac{\partial \theta_m}{\partial \psi_j}
   \right)
   -
   \sum_k
   \frac{\partial l(\theta)}{\partial \theta_k}
   \frac{\partial^2 \theta_k}{\partial \psi_i \partial \psi_j}.
$$
When MLE are plugged in,
$$
   \left. \frac{\partial l(\theta)}{\partial \theta_k}
   \right|_{\theta = \hat{\theta}} = 0
$$
so this gives \eqref{eq:transform-like-tensor-observed}.
\end{proof}

\subsubsection{Fisher Information for $\boldsymbol{\theta}$}
\label{sec:I-J-theta}

Returning to aster models with subsampling, from
\eqref{eq:logl-deriv-wrt-theta-theta} we get:
\begin{itemize}
\item observed Fisher information for $\theta$ is the matrix having
components $y^*_{p^*(j)} \gamma_{j k}$ and
\item
expected Fisher information for $\theta$ is the matrix having
components $\mu^*_{p^*(j)} \gamma_{j k}$.
\end{itemize}
Note that, since expected Fisher information depends on
components of $\mu^*$, which cannot, in general, be calculated
by the computer,
use of expected Fisher information depends on those subsampling probability
distributions,
whereas the log likelihood, its derivatives, and observed Fisher information
do not depend on them.

Thus, in general, we can only use observed Fisher information.

\subsubsection{Fisher Information for $\boldsymbol{\beta}$ for CAM}
\label{sec:I-J-cam}

If $I(\theta)$ is expected Fisher information for $\theta$ derived
in the preceding section, then expected Fisher information for $\beta$ is
$$
   I(\beta) = M^T I(\theta) M, \qquad \text{when $\theta = a + M \beta$},
$$
because $M$ is the derivative of $\theta$ with respect to $\beta$.
We also have the analogous relationship for observed Fisher information
when MLE are plugged in
$$
   J(\hat{\beta})
   =
   M^T J(\hat{\theta}) M,
   \qquad \text{when $\hat{\theta} = a + M \hat{\beta}$}.
$$

\subsubsection{Fisher Information for $\boldsymbol{\beta}$ for UAM}
\label{sec:I-J-uam}

As in the preceding section, let $I(\theta)$ and $J(\theta)$ be expected
and observed Fisher information for $\theta$ for the saturated model with
subsampling, which were derived in Section~\ref{sec:I-J-theta} above.
Now let $a(\varphi)$ denote the inverse aster transform described in
Section~\ref{sec:aster-transform} above, and let $A(\varphi)$ denote
its derivative,
whose components are given by \eqref{eq:inverse-aster-transform-deriv-partial}.
Now the model equation is \eqref{eq:affine-unconditional}
so $M$ is the derivative of $\varphi$ with respect to $\beta$.

So now expected Fisher information for $\beta$ is
$$
   I(\beta) = M^T A(\varphi)^T I(\theta) A(\varphi) M,
   \qquad \text{when $\varphi = a + M \beta$ and $\theta = a(\varphi)$},
$$
and the analogous relationship for observed Fisher information
when MLE are plugged in is
$$
   J(\hat{\beta})
   =
   M^T A(\hat{\varphi})^T I(\hat{\theta}) A(\hat{\varphi}) M,
   \qquad \text{when $\hat{\varphi} = a + M \hat{\beta}$
   and $\hat{\theta} = a(\hat{\varphi})$}.
$$

\subsection{Prediction}
\label{sec:predict}

Prediction for all six parameterizations of UAM discussed in
Section \ref{sec:revisited} above can be handled by the \texttt{aster}
and \texttt{aster.formula} methods of the R generic function \texttt{predict}
(that is, the functions \texttt{predict.aster}
and \texttt{predict.aster.formula} considered as non-generic functions).
One calls these functions by the name \texttt{predict} but must look up the
help page with \texttt{help(predict.aster)}.
These functions are in R package \texttt{aster}.
(A referee for \citet{aster1} complained that these are not predictions
but rather parameter transformations and we agreed, but users expect to use
the R generic function \texttt{predict} to do this job.)

Since version 1.0 of the package, these functions have a new optional argument
\texttt{is.always.parameter = TRUE} which makes them ``predict'' $\xi$
rather than the vector having components $E(y_j \mid y_{p(j)})$, which
is not a parameter.
If we always use this option, this function can make predictions for all
parameters in the chain $\beta \to \varphi \to \theta \to \xi \to \mu$,
and if we add $\tau = M^T \mu$, which we can do ourselves (given $\mu$
calculated by \texttt{predict}), we have all six parameters.

As parameters, none of these depend on the response vector, although they
do depend on covariates because the model matrix depends on covariates.
Thus if the R function \texttt{predict} is provided an object of class
\texttt{"aster"} or \texttt{"aster.formula"}
that has the MLE $\hat{\beta}$ for the aster model with subsampling
as its \texttt{coefficients} component, and corresponding expected and observed
Fisher information matrices as its \texttt{fisher} and \texttt{hessian}
components but every other component as if subsampling had not been done,
then \texttt{predict} will operate to make predictions without subsampling.

We also want the \texttt{deviance} component of our object of
class \texttt{"aster"} or \texttt{"aster.formula"} containing the result
of fitting the model with subsampling to be reflect the subsampling
(be minus twice the maximized log likelihood) so that the R generic
function \texttt{anova} does the right thing with these objects.

\subsection{Parameter Transformation}
\label{sec:parameter-transformation}

The methods for the R generic function \texttt{predict} in R package
\texttt{aster} cannot do all of the parameter
transformations discussed in Sections~\ref{sec:plethora} and~\ref{sec:revisited}
above.  A function \texttt{astertransform} was added to this package, but
it only does transformations from $\theta$ and $\varphi$ to any of
$\theta$, $\varphi$, $\xi$, or $\mu$.

R package \texttt{aster2} does do all of these parameter transformations.
The function \texttt{transformSaturated} does any of the transformations
from $\theta$, $\varphi$, $\xi$, or $\mu$ to any of these.
The function \texttt{transformConditional} does any of the transformations
for a CAM without subsampling from $\beta$
to $\theta$, $\varphi$, $\xi$, or $\mu$ (but not vice versa).
The function \texttt{transformUnconditional} does any of the transformations
from $\beta$ or $\tau$ (the only parameters that are unconstrained)
to $\beta$, $\theta$, $\varphi$, $\xi$, $\mu$, or $\tau$.

Both \texttt{transformSaturated} and \texttt{transformUnconditional} are
able to transform from mean value to canonical parameters.

Calculating the parameter transformations
$\xi \to \theta$ or $\mu \to \varphi$ for saturated aster models without
subsampling and $\tau \to \beta$ for UAM without subsampling is
equivalent to doing maximum likelihood
with data replaced by unconditional mean value parameters.
Theorem~\ref{th:mean-value-inversion} covers all of these cases.

