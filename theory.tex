
\documentclass[11pt,oneside]{book}

\usepackage{algorithm}
\usepackage{amscd}
\usepackage{amsfonts}
\usepackage{amsmath}
\usepackage{amssymb}
\usepackage{amsthm}
\usepackage{bbold}
\usepackage{ifthen}
\usepackage{indentfirst}
\usepackage[utf8]{inputenc}
\usepackage{longtable}
\usepackage{makeidx}
\usepackage{natbib}
\usepackage{rotating}
\usepackage{tikz-cd}
\usepackage{url}

\bibliographystyle{charlie}

\setcounter{tocdepth}{2}


\DeclareMathOperator{\pr}{pr}
\DeclareMathOperator{\var}{var}
\DeclareMathOperator{\cov}{cov}
\DeclareMathOperator{\logit}{logit}
\DeclareMathOperator{\card}{card}

\newcommand{\inner}[1]{\langle #1 \rangle}
\newcommand{\set}[1]{\{\, #1 \,\}}
\newcommand{\bigset}[1]{\left\{\, #1 \,\right\}}

\newcommand{\fatdot}{\,\cdot\,}

\newcommand{\nats}{\mathbb{N}}
\newcommand{\real}{\mathbb{R}}

\newcommand{\opand}{\mathbin{\rm and}}
\newcommand{\opor}{\mathbin{\rm or}}
\newcommand{\ifandonlyif}{\mathrel{\rm if\ and\ only\ if}}

\newcommand{\code}[1]{\texttt{#1}}

\newcommand{\myline}{\relbar\joinrel\relbar}

\let\emptyset=\varnothing

\newlength{\foo}
\settowidth{\foo}{=}
\newcommand{\andsoforth}{\mathrel{\makebox[\foo]{\vdots}}}

\newcommand{\REVISED}{\begin{center} \LARGE REVISED DOWN TO HERE \end{center}}
\newcommand{\MOVED}[1][equation]{\begin{center} [#1 moved] \end{center}}

\newtheorem{theorem}{Theorem}
\newtheorem{corollary}[theorem]{Corollary}
\newtheorem{lemma}[theorem]{Lemma}

% for indexing
\newcommand{\seeunder}[2]{\emph{see under} #1}



\makeindex

\begin{document}

\title{Theory of Aster Models}

\author{Charles J. Geyer}

\maketitle

\frontmatter

\chapter{Preface}

This is a book about the theory of aster models.  Its main intended
readers are implementers of aster software.  So far, that is just the
author.  The reason for the book is to go into all the gory details
so the software can do the Right Thing.

Other readers may also find this book useful in having all of the theory
of aster models in one place and presented with consistent notation.
But they may have to skip a lot of material they don't want to read.

\chapter{License}

This work is licensed under a Creative Commons Attribution-ShareAlike 4.0
International License
\begin{trivlist}
\item
\url{http://creativecommons.org/licenses/by-sa/4.0/}
\end{trivlist}
\index{Creative Commons}
\index{license}

\LaTeX\ source for this work can be found in the git repository
\begin{trivlist}
\item
\url{https://github.com/cjgeyer/AsterTheory}
\end{trivlist}
\index{Github}

Since this book has no edition numbers or version numbers (so far),
one should cite it by the sha1 hash of its Git commit.

\tableofcontents

\mainmatter


\chapter{Introduction}
\label{ch:introduction}

\section{Background}
\label{sec:background}

Aster models \citep*{aster1,aster2,reaster} \index{aster model}
are parametric statistical models
specifically designed for life history analysis.  They are exponential family
models that generalize generalized linear models (GLM) that are also
exponential family models (for example, logistic regression and
Poisson regression with log link) in two ways
\begin{itemize}
\item in GLM components of the response vector are
    necessarily conditionally independent given covariate data
    but in aster models they need not be, and
\item in GLM the conditional distributions of components of the
    response vector given covariate data all come from the same family
    but in aster models they need not.
\end{itemize}
As generalizations of GLM, aster models are also regression models.
They model the conditional distribution of the response vector given
covariate data.  The marginal distribution of covariate data is not
modeled.

In life history analysis, \index{life history analysis}
the data are about survival and reproduction
of biological organisms.  Thus aster models also generalize discrete time
survival analysis (aster models model not only survival but also
what happens conditional on survival).
Aster models unify many disparate kinds of life history analysis that have
appeared in the biological literature: comparison of Darwinian fitness between
various groups \citep{aster1,aster2}, estimation of fitness landscapes
(\citealp{lande-arnold}; \citealp{aster2,aster3}, \citealp*{aster-hornworm}),
Leslie matrix analysis
\citep{caswell}, life table analysis in demography \citep{goodman},
and estimation of population growth rates
\citep{fisher,lenski-service,aster2,aster-hornworm}.
Aster models also generalize zero-inflated Poisson regression \citep{lambert},
negative binomial regression (overdispersed Poisson regression),
and zero-inflated negative binomial regression.

Aster models are a special case of graphical models \citep{lauritzen}.
In particular, they are statistical models for which the joint distribution
of the response vector factorizes completely as a product of marginal and
conditional distributions (equation~\eqref{eq:factorize} below).
This makes aster models a special case of chain graph models
\citep[Sections~2.1.1 and~3.2.3]{lauritzen}.
Aster models also have the predecessor-is-sample-size property
(Section~\ref{sec:piss} below)
that makes the joint distribution of the response vector an exponential
family.  This property can be seen to generalize unnamed properties
of survival analysis, life-table analysis, Leslie matrix analysis,
and population growth rate analysis (Section~\ref{sec:mu-and-xi} below).

\section{Software}
\label{sec:software}

Currently, all software for aster models is written in the R statistical
computing language \citep{r-core}.  There are two CRAN
(\url{cran.r-project.org}) packages, \code{aster} \citep{aster-package} and
\code{aster2} \citep{aster2-package}.
\index{R package!aster@\code{aster}}
\index{R package!aster2@\code{aster2}}
Both R and these packages can be installed in minutes on any computer,
so any user can get started with aster models in almost no time.

R package \code{aster} is the most complete.
It does everything except dependence groups
%%%%%%%%%% NEED FORWARD REFERENCE %%%%%%%%%%
and limiting conditional models.
%%%%%%%%%% NEED FORWARD REFERENCE %%%%%%%%%%

R package \code{aster2} is the very incomplete.
It does do dependence groups and limiting conditional models, but everything
else is either missing or much harder to use than in R package \code{aster}.

So any aster model that can be done with R package \code{aster} should
be done with that package.

\section{Summary}

Aster models combine three ideas
\begin{itemize}
\item factorization of a joint distribution into a product of marginals
    and conditionals (Section~\ref{sec:factorization} below),
\item the predecessor is sample size property (Section~\ref{sec:piss} below),
    which says the conditioning variables in the conditional distributions
    in the factorization act like sample sizes, and
\item the exponential family property (Section~\ref{sec:aster-expfam} below),
    which says the conditional distributions in the factorization are
    exponential families of distributions,
\end{itemize}
into one big idea.  Together the make the joint distribution of the
response vector also an exponential family.  And this makes aster models
as well-behaved as generalized linear models or log-linear models for
categorical data analysis, even though they are far more more complicated.

\section{Vectors and Subvectors}
\label{sec:subvector}

We adopt a notation from \citet{lauritzen} for subvectors, but fuss about it
more.

As in set theory \citep[Section~8]{halmos-set-theory}, if $A$ and $B$
are sets, then $A^B$ denotes the set of all functions $B \to A$.
In particular, if $J$ is a finite set, then we let $\real^J$ denote
the set of all functions $J \to \real$.  This set can also be considered
a finite-dimensional vector space.  That functions $J \to V$ where $J$ is
any set and $V$ is a field or a vector space can be considered
vectors is the reason the study of infinite-dimensional topological vector
spaces is called functional analysis.

Another way of looking at this distinction is that the usual view of
finite-dimensional vector spaces is that they are $\real^d$ for some
natural number $d$, which is tantamount to insisting that the index
set for vectors in this space must be the set $\{1, \ldots, d\}$.
Here we are saying the index set can be any finite set $J$.

Even though we consider vectors to be functions, we write evaluation
of these functions $y_j$ so it looks like usual notation.  We even say
that $y_j$ is a component of the vector $y$ rather than the value of
the function $y$ at the point $j$.  But behind the scenes our vectors
are also functions, and we could write $y(j)$ instead of $y_j$.

We need a notation for subvectors of a vector.  If $y$ is an element
of $\real^J$ and $A \subset J$, then we let $y_A$ denote the restriction
of $y$ to the set $A$.
As such, it is an element of the vector space $\real^A$.
Like all functions, it knows its domain and codomain.
It knows it is a function $A \to \real$.
So it knows its components are $y_j$, $j \in A$.
And these are also the components of $y$ for $j \in A$.
Since the components of $y_A$ are a subset of the components of $y$,
we say $y_A$ is a \emph{subvector} of $y$.
\index{subvector}

If we were to insist that all vectors, including subvectors, have
index sets $\{1, \ldots, k\}$ for some natural number $k$.  Then we could
not distinguish different subvectors of the same length, or at least could
not without ugly and cumbersome extra decoration of the notation.
(It is hard to explain how elegant this notation is with simple examples,
but a perusal of Appendix~\ref{app:markov} will show this notation is
extremely powerful, and that appendix would be much longer and more confusing
if we had to use conventional notation with indices going from 1 to $k$.)

Our notation does have the drawback that we have only the convention
that lower case letters denote elements of sets and upper case letters
denote sets, which is why $y_j$ is clearly a component of a vector or subvector
(the value of a function at the index $j$) and $y_A$ is a subvector
(so is still a function, not the value of a function).
We also consider any subscript notation that clearly denotes a set
as indicating a subvector, for example, $y_{\{1, 3, 5\}}$ or $y_{\{j\}}$ or
$y_{\set{ j \in J : j \prec i }}$.

\section{Regression Notation}

Strictly speaking, in regression theory, every probability and expectation
is conditional on covariate data, at least on the part of the covariate data
that is considered random rather than fixed by the design of the experiment.
Thus to be hyperpedantic, we should always write
\begin{gather*}
   E(y_A \mid \text{the part of covariate data that is random})
   \\
   \Pr(y_A \in B \mid \text{the part of covariate data that is random})
\end{gather*}
rather than $E(y_A)$ or $\Pr(y_A \in B)$.  But, like most regression books,
we will not do this.  The dependence of probabilities and expectations on
covariates is usually not made explicit in the notation.

This is especially important in aster models when components of the response
vector depend on the values of other components, so we frequently write
\begin{gather*}
   E(y_A \mid y_j)
   \\
   \pr(y_A \mid y_j)
\end{gather*}
and the like.  And we do not want this dependence confused with dependence
on covariate data.

When necessary for clarity, as in the discussion of fitness landscapes,
which are regression functions,
%%%%%%%%%% NEED FORWARD REFERENCE %%%%%%%%%%
we can explicitly denote the dependence on covariate data in conditional
probabilities and expectations.

\section{Factorization}
\label{sec:factorization}

If $J$ is the index set of the response vector $y$ of an aster model,
then there is a partition $\mathcal{G}$ of $J$
and a function $q : \mathcal{G} \to N$, where $N \supset J$, such that
\index{aster model!property!factorization}
\index{factorization|seeunder{aster model}}
the joint distribution of $y$ factorizes as
\begin{equation} \label{eq:factorize}
   \pr(y) = \prod_{G \in \mathcal{G}} \pr(y_G \mid y_{q(G)})
\end{equation}

In this factorization, each component $y_j$ of the response vector $y$
appears exactly once ``in front of the bar'' in a conditional on
the right-hand side (because $\mathcal{G}$ is a partition of $J$ so
each $j \in J$ is in exactly one $G \in \mathcal{G}$).
So every component of $y$ is treated as random (the joint distribution
of $y$ is modeled).
Random variables $y_{q(G)}$ that appear ``behind the bar'' in a conditional
on the right-hand side may or may not be elements of $y$.  They are not
if $q(G) \notin J$.  The distribution of such random variables is not
modeled by \eqref{eq:factorize}.  So they are treated as constant random
variables.

We say \eqref{eq:factorize} is \emph{valid} if what are denoted as
conditional distributions on the right-hand side agree with the conditional
distributions derived from the left-hand side (the joint distribution) by
the usual operations of probability theory.
\begin{theorem} \label{th:factorize}
The factorization \eqref{eq:factorize} is valid if and only if
the partition $\mathcal{G}$ can be totally ordered
by some total ordering $<$ such that $q(G) \in H$ implies $G < H$.
\end{theorem}
A proof of this theorem is straightforward and given
in Appendix~\ref{app:factorize}.
It could also be derived from the discussion of chain graph models
in \citet[equation~3.23]{lauritzen}.

In \eqref{eq:factorize} we have been deliberately vague about what $\pr$ is
supposed to mean, since there are many ways to specify probability
distributions and any of them will do.
\begin{itemize}
\item If $y$ is a discrete random vector,
      then $\pr$ could denote probability mass functions.
\item If $y$ is a continuous random vector,
      then $\pr$ could denote probability density functions.
\item If $y$ is a partly discrete and partly continuous continuous
      random vector (either some components discrete and some components
      continuous or some components a mixture of discrete and continuous)
      then $\pr$ could denote probability mass-density functions.
\item No matter what, $\pr$ could denote cumulative distribution functions.
\item No matter what, $\pr$ could denote probability measures.
\end{itemize}
In any of these cases the multiplication indicated in \eqref{eq:factorize}
is actual multiplication of real-valued thingummies.

\subsection{Topological Sort}

The total order asserted to exist by the theorem need not be unique and
usually is not unique.
We can find such a total order using the algorithm called topological sort
\index{topological sort}
\citep[Section~6.6]{aho-et-al}.

Using R function \code{tsort} in R package \code{pooh}
\citep{pooh-package}
\index{R package!pooh@\code{pooh}}
for each $G \in \mathcal{G}$ such that there exists
a (necessarily unique) $H \in \mathcal{G}$ such that $q(G) \in H$
let $G$ be a component of the vector \code{from} that is an argument to
\code{tsort} and
let $H$ be the corresponding component of the vector \code{to} that is another
argument to \code{tsort} and
neither vector has any other components.  Then invoking \code{tsort} with
these \code{from} and \code{to} arguments and \code{domain} argument that is
$\mathcal{G}$ strung out in a vector in any order
will determine a (not necessarily unique) total order that agrees with
Theorem~\ref{th:factorize}.  If the user has made a mistake and incorrectly
specified the $q$ function so there is no total order that satisfies
Theorem~\ref{th:factorize}, then \code{tsort} will give an error.

Current code in R packages \code{aster} and \code{aster2} does not
actually use the topological sort algorithm but rather forces the user
to input the data so that the numerical order of the components of the
response vector is the total order, that is, considering the index set
of the response vector to be $\{1, \ldots, n\}$ for some integer $n$,
current code requires $q(G) < j$ for any $j \in G$.
It is up to the user to input the data in this way.  The computer is no help.

But we could make the computer figure this out in future versions of the
software.

\subsection{Further Factorization}
\label{sec:further-factorize}

In \citet{lauritzen} chain graph factorizations like our \eqref{eq:factorize}
and his equation (3.23) can be further factorized, his equation (3.24).
But in aster model theory, we shall never be interested in such further
factorizations (even in cases where they are possible) and never use any
notation that allows for them.  So we will never have an analog of equation
(3.24) in \citet{lauritzen}.  For us, factorization is \eqref{eq:factorize}.

\section{Graphs}

Each factorization goes with a graph \citep[Section~3.2.3]{lauritzen}.
\index{aster graph}
\index{graph|see{aster graph}}
The nodes of the graph are either the elements of $N$ or the components
of $y$ corresponding to these elements ($y_j$ for $j \in N$).
There is a directed edge, also called an \emph{arrow},
\index{node}
\index{edge}
\index{arrow}
\index{line}
$q(G) \longrightarrow j$ (or if one prefers $y_{q(G)} \longrightarrow y_j$)
for every $G \in \mathcal{G}$ and every $j \in G$.
There is an undirected edge, also called a \emph{line},
$j \myline k$ (or if one prefers $y_j \myline y_k$)
for every $G \in \mathcal{G}$ and every $j, k \in G$ such that $j \neq k$.

As we have just seen, the function $q$ determines the graph
(the function $q$ knows its domain $\mathcal{G}$).
Conversely, the graph determines the function $q$.
\begin{itemize}
\item The set $J = \bigcup \mathcal{G}$ is the set of nodes of the graph
    that have incoming arrows (as we shall see, these nodes are called
    non-initial).
\item The elements of $\mathcal{G}$ are the maximal connected components
    of the graph of lines having node set $J$.
    (The graph of lines is the graph obtained by keeping all the nodes
    and lines but removing all the arrows).
    \index{aster graph!of lines}
    This includes any singleton sets of $J$ that have no incoming lines.
\item The graph of $q$ is determined by the arrows: $(G, q(G))$ is an
    argument-value pair whenever there is an arrow $j \longrightarrow k$
    with $j = q(G)$ and $k \in G$.
\end{itemize}

Thus we can reason with with graphs or with $q$ functions (which we will
soon learn to call \emph{predecessor functions}, Section~\ref{sec:other} below).
Graphs can be helpful, but we do not have to use them.

\subsection{Exception}
\label{sec:exception-dependence-group-lines}

In theory, as stated above, there is a line between every pair of distinct
elements of every dependence group and no other lines.

In practice, this leads to annoying and unnecessary clutter.
Because we never further factorize dependence groups
(Section~\ref{sec:further-factorize} above), we can find the dependence
groups from the graph if we only include enough lines so that each
dependence group is a connected subgraph of the graph of lines
(again, this is the graph obtained by keeping all the nodes and lines but
removing all the arrows).
\index{aster graph!of lines}

This exception is illustrated in graph \eqref{gr:multi} below
where only two lines
rather than three are used to connect the nodes of each
dependence group of size three.

\section{Graphical Terminology}
\label{sec:graphical-terminology}

In aster theory, we say
\begin{itemize}
\item a node is \emph{initial} if it has no incoming arrows
\index{node!initial}
\index{initial node|seeunder{node}}
    or lines (when thinking about the graph) or if it is not an element
    of $J = \bigcup \mathcal{G}$ (when thinking about the function $q$),
\item a node is \emph{terminal} if it has no outgoing arrows
\index{node!terminal}
\index{terminal node|seeunder{node}}
    (it may have outgoing lines and will have outgoing lines if it is
    an element of an element of $\mathcal{G}$ that is not a singleton set)
    (when thinking about the graph) or if it is not an element
    of $\set{ q(G) : G \in \mathcal{G} }$
    (when thinking about the function $q$),
\item if there is an arrow $j \longrightarrow k$, then we say that $j$
    is the \emph{predecessor} of $k$ (or $y_j$ is the predecessor of $y_k$),
\index{node!predecessor}
\index{predecessor node|seeunder{node}}
\item and, conversely, that $k$ is a \emph{successor} of $j$
    (or $y_k$ is the successor of $y_j$).
\index{node!successor}
\index{successor node|seeunder{node}}
\end{itemize}

In mainstream graphical model theory, a different terminology is more widely
used \citep{lauritzen} root = initial, leaf = terminal, parent = predecessor,
child = successor.  We do not use this terminology in aster model theory
because it can cause serious confusion in biological applications.

As a general policy, we eschew all terminology based on biological analogies
when there is an available alternative (even when that alternative is less
popular).

In any aster graph every node has at most one predecessor and all nodes in
\index{aster model!property!at most one predecessor}
the same $G \in \mathcal{G}$ must have the same predecessor (because $q$
is a function that takes elements of $\mathcal{G}$ as arguments).

In mainstream graph theory, a chain graph with only arrows (no lines) having
the at-most-one-predecessor property is called a \emph{forest} and its maximal
connected components are called \emph{trees}, but we do not use this terminology
either (avoiding serious confusion when the application involves data on
real trees in real forests).  It is enough to say that aster graphs
have the at-most-one-predecessor property.

In mainstream graph theory, there is a term \emph{ancestor} that means
predecessor, or predecessor of predecessor,
or predecessor of predecessor of predecessor,
or predecessor of predecessor of predecessor of predecessor,
or the same with arbitrarily many repetitions of ``predecessor of.''
And there is a converse term \emph{descendant}, that is, $i$ is an ancestor
of $j$ if and only if $j$ is a descendant of $i$.

In aster model theory we avoid these terms too (avoiding confusion when
the application involves real biological organisms with real biological
ancestors and real biological descendants).  If we need the concepts,
then we use the long-winded descriptions
predecessor of predecessor of predecessor and so forth or
successor of successor of successor and so forth.
Fortunately, we rarely need these concepts.
And when we do need these concepts we can avoid the cumbersome verbiage
by using mathematical notation introduced in Section~\ref{sec:closure}
below.

Finally, we need a term for $\mathcal{G}$ and its elements.
The terminology we have been using in our writings about aster models is
elements of $\mathcal{G}$ are \emph{dependence groups}.
\index{dependence group}
The mainstream graphical models terminology \citep{lauritzen} is
\emph{chain components}.  Both have two words and four syllables.
Neither is very elegant.  We don't like the ``chain'' terminology because
we are not using general chain graph theory (aster models are very special
chain graphs).  Our term \emph{dependence group} is not great, but we haven't
thought of a better term.

\subsection{Exception}
\label{sec:exception-root}

R package \code{aster} uses ``root'' node for initial node.
\index{node!root}
\index{root node|seeunder{node}}
We hadn't completely thought through the terminology when that package
was written, and we have kept this inconsistency for reasons of backward
compatibility.

\section{Two Kinds of Aster Graphs}
\label{sec:scare-quotes}

The graphs for aster models are often very large with thousands or tens of
thousands of nodes, but usually they are composed of isomorphic subgraphs.
So drawing one of these isomorphic subgraphs is enough.
If you've seen one, you've seen them all.
(Graphs are isomorphic if a drawing of one can be laid on a drawing of the
other with everything --- nodes, lines, and arrows --- matching up.)

An aster graph need not be composed of all isomorphic subgraphs,
but the only published example of that is, as far as I know,
\citet{aster-hornworm}.

To distinguish these two kinds of graphs, we call the aster graph described
in the preceding section the \emph{full aster graph} (we consider the ``full''
redundant but the emphasis may help avoid confusion).
\index{aster graph!full}

Certain subgraphs of the full aster graph, we then call graphs
for ``individuals'' (in scare quotes for reasons to be explained presently).
These are easier to recognize than describe.

Current aster software (Section~\ref{sec:software} above) forces
$q(G) \neq q(H)$ whenever $G \neq H$ and $q(G)$ and $q(H)$ are initial nodes.
In this case, the graph for an ``individual'' (in scare quotes)
consists of the subgraph consisting of one initial node and all of its
successors or successors of successors or successors of successors
of successors and so forth with arbitrarily many repetitions
of ``successors of'' and all of the arrows and lines in the full graph
connecting these nodes.
\index{aster graph!for ``individual''}
(This is where the term ``descendant'' in its graph-theoretic sense would
come in handy if we allowed ourselves to use it.  The graph for an
``individual'' consists of one initial node, all of its descendant nodes,
and all of the lines and arrows going between these nodes.  But once we
have the idea of the graph for an ``individual'' we no longer need the
term ``descendant.'')

But aster theory as described so far does not force this convention.
If $y_j = 1$, for all initial nodes $j$, which is the case with most
(but not all) aster applications, then it would do no harm if all initial
nodes were fused into one initial node.  That would invalidate nothing but
the way we just described graphs for ``individuals'' (in scare quotes).

Thus we have to be a bit more careful.  If $G$ is a dependence group whose
predecessor $q(G)$ is initial, then the graph for the ``individual''
(in scare quotes) containing $G$ consists of $q(G)$, the nodes in $G$
and their successors or successors of successors or successors of successors
of successors and so forth with arbitrarily many repetitions
of ``successors of'' and all of the arrows and lines in the full graph
connecting these nodes.
\index{aster graph!for ``individual''}
(And it would make this definition a little shorter
if we allowed ourselves to use the word ``descendant'' in its graph-theoretic
sense.)

There are two reasons why the scare quotes.
\begin{itemize}
\item In life history
analysis, the graph for an ``individual'' ideally goes one or more times
around the life cycle (exactly).  Thus it may involve data not only for
one biological individual but also for its offspring and perhaps offspring
of offspring (if the experiment goes twice around the life cycle) or even
perhaps more remote descendants (where here ``descendants'' means real
biological descendants, not the graphical models idea of descendants).
\item If the value of the constant $y_j$ at the initial node of the
graph for an ``individual'' is greater than one, then the data for this
``individual'' is actually cumulative data for $y_j$ real biological
individuals and perhaps their real biological descendants.
\end{itemize}

%%%%%%%%%% NEED FORWARD REFERENCE to example graphs %%%%%%%%%%
%%%%%%%%%% maybe ?????

If one does not like our terminology of ``individual'' in scare quotes,
our advice is to just explain what data the graph is for.  It may actually
be for a biological individual, for a biological individual
and its offspring, or $n$ biological individuals.  Just say what it is.

Or we could use the characterization of Corollary~\ref{cor:markov} in
Appendix~\ref{app:markov}, which says the subgraphs for ``individuals''
are stochastically independent subvectors of the response vector.
\index{aster graph!for ``individual''}

In general, the subgraphs for ``individuals'' are the minimal stochastically
independent subvectors, in the sense that the data for an ``individual'' has
no stochastically independent parts.  But when limiting conditional models
%%%%%%%%%% NEED FORWARD REFERENCE to limiting conditional models %%%%%%%%%%
come into play, this is no longer the case.
Thus independence of data for ``individuals'' is an important property of
aster models, but it does not (in general) characterize
subgraphs for ``individuals.''

\section{The Other Predecessor Function}
\label{sec:other}

It is useful to have not only the set-to-index predecessor function $q$
defined in Section~\ref{sec:factorization} above but also the index-to-index
\index{predecessor function!set-to-index}
\index{predecessor function!index-to-index}
predecessor function $p$ defined as follows
$$
   p(j) = k \ifandonlyif j \in G \in \mathcal{G} \opand q(G) = k.
$$
Clearly, $q$ determines $p$.
The converse is not true because $p$ knows nothing about dependence groups.
But $p$ and $\mathcal{G}$ together determine $q$.

\section{The Transitive Closure of the Predecessor Relation}
\label{sec:closure}

The \emph{predecessor relation} on $N$ is the function $p$ thought
of as a relation, that is, thinking set-theoretically
\citep[Section~7]{halmos-set-theory} as the set
$$
   \set{ (j, p(j)) : j \in J }
$$
of its argument-value pairs.

We need a notation from dynamical systems theory for repeated application
of a function.  If $f$ is any function whose domain and codomain are the same,
then it makes sense to compose $f$ with itself.  Then we let $f^0$ denote
the identity function on the domain of $f$, let $f^1 = f$, $f^2 = f \circ f$,
and, in general, $f^{n + 1} = f^n \circ f$.
So
\begin{align*}
   f^0(x) & = x
   \\
   f^1(x) & = f(x)
   \\
   f^2(x) & = f(f(x))
   \\
   f^3(x) & = f(f(f(x)))
\end{align*}
and so forth.

The transitive closure of the predecessor relation
is the smallest transitive relation $R$ containing it.
\index{predecessor relation!transitive closure}
As with most relations, we prefer denoting this relation by infix notation:
saying $j \succ k$ rather than $(j, k) \in R$, that is, $j \succ k$ means
$k = p^n(j)$ for some positive integer $n$.

\begin{theorem} \label{th:transitive-closure}
Under the conditions of Theorem~\ref{th:factorize},
the transitive closure of the predecessor relation is a strict partial order.
\end{theorem}
\begin{proof}
If $j \succ k$, then $j \in G$ for some $G \in \mathcal{G}$ and
$k = p^n(q(G))$ for some natural number $n$ ($n = 0$ is allowed).

If $k \in H$ for some $H \in \mathcal{G}$,
then we have $G < H$ in the total ordering
that Theorem~\ref{th:factorize} uses.
Hence we cannot also have $k \succ j$ because that would imply $G < H$
and $H < G$ contradicting $<$ being a strict total order.

If $k \notin H$ for any $H \in \mathcal{G}$ then $k$ has no predecessor
($k$ is initial) and we cannot have $m \succ k$ for any node $m$.

In either of the preceding cases we never have $k \succ j$ and $j \succ k$.
Since $\succ$ is a transitive relation by definition, it is
a strict partial order \citep[Section~14]{halmos-set-theory}
\end{proof}
\begin{corollary} \label{cor:compatible}
The transitive closure of the predecessor relation is compatible with
the total order on the family of dependence groups defined
in Theorem~\ref{th:factorize} in the sense that
$j \in G \in \mathcal{G}$ and $k \in H \in \mathcal{G}$ and $j \succ k$
implies $G < H$.
\end{corollary}

The non-strict counterpart of this relation
is the reflexive transitive closure of the predecessor relation,
\index{predecessor relation!reflexive transitive closure}
which is denoted $\succeq$.
We have $j \succeq k$ if and only if $j \succ k$ or $j = k$.

The inverse of a relation $R$ considered as a set of argument-value pairs
reverses the order in the pairs, that is $(k, j) \in R^{- 1}$ if and only
if $(j, k) \in R$.
As usual, we denote the inverse of a relation by turning its infix notation
around: $\prec$ is the inverse of $\succ$ and $\preceq$ is the inverse
of $\succeq$.

The inverse of the predecessor relation is the successor relation,
so $\prec$ is the transitive closure of the successor relation
and $\preceq$ is the reflexive transitive closure of the successor relation.
\index{successor relation!transitive closure}
\index{successor relation!reflexive transitive closure}
\index{transitive closure|seeunder{predecessor relation}}
\index{transitive closure|seeunder{successor relation}}
\index{reflexive closure|seeunder{predecessor relation}}
\index{reflexive closure|seeunder{successor relation}}

The choice of whether the transitive closure of the predecessor relation
is denoted $\succ$ or $\prec$ is arbitrary.  Either choice works so long
as one keeps straight which is which.  Our choice is influenced by an
arbitrary choice in the source code for R package \texttt{aster}.  When
the predecessor function is encoded (as the argument \texttt{pred} to the
R function \texttt{aster}) it is required that predecessors have lower indices
than successors (come before them in the \texttt{pred} vector).  Thus we
want to think of predecessors as ``less than'' successors in some sense.
Hence our decision to make $p(j) \prec j$.

In graphical model theory,
$\succ$ is called the ancestor relation,
$\prec$ the descendant relation,
$\succeq$ the ancestor-or-self relation, and
$\preceq$ the descendant-or-self relation.
But, as stated in Section~\ref{sec:graphical-terminology} above,
our policy is to avoid these terms
to avoid confusion in biological applications.
If we need words rather than symbols, we have to use the long winded ones:
``reflexive transitive closure of the predecessor relation'' and so forth.

\section{Predecessor is Sample Size}
\label{sec:piss}

All aster models have the \emph{predecessor is sample size} property.
This is a very important property that separates them from all other
graphical models.  There is a long history of models that have this
property.  Life table analysis and discrete time survival analysis have it.
So does Leslie matrix analysis \citep{caswell} and other methods
of estimation of population growth rate \citep{fisher,goodman,lenski-service}.
But none of those models were regression models, nor did they have
the generality of aster models in their graphical structure.
They do have the basic relationship of conditional and unconditional means
implied by this property (Section~\ref{sec:mu-and-xi} below)
but nothing else of aster model theory.

For one conditional distribution in the factorization \eqref{eq:factorize},
say for the conditional distribution of $y_G$ given $y_{q(G)}$,
\begin{itemize}
\item conditional on $y_{q(G)} = 0$, the distribution of $y_G$ is concentrated
    at zero (the zero vector having all components equal to zero), 
\item conditional on $y_{q(G)} = 1$, the distribution of $y_G$ is whatever
    this distribution is designated to be, and
\item conditional on $y_{q(G)} = n$ with the $n > 1$, the distribution
    of $y_G$ is the $n$-fold convolution of the distribution for sample
    \index{convolution}
    size one.
\end{itemize}
In short, the conditional distribution of $y_G$ given $y_q(G)$ is the
distribution of the sum of $y_{q(G)}$
independent and identically distributed (IID)
random vectors having whatever the distribution is for sample size one.
(By convention, a sum having zero terms is zero, and a sum having one term
is that term.)
Or, even shorter, the predecessor plays the role of sample size for this
conditional distribution.
Or, shorter still, \emph{predecessor is sample size}.
\index{aster model!property!predecessor is sample size}
\index{predecessor is sample size|seeunder{aster model}}

Note that we name families for dependence groups by the conditional
distribution for sample size one.
This is an unusual practice.  It is not the way families are named for
generalized linear models.
And it can seem unnecessarily mysterious at first sight.

All of our example graphs in Section~\ref{sec:graphs} below
have Bernoulli arrows.  For such an arrow
$$
\begin{CD}
   y_i @>\text{Ber}>> y_j
\end{CD}
$$
why not just say the conditional distribution of $y_j$ given $y_i$ is binomial
with sample size $y_i$ (because the sum of IID Bernoulli is binomial)?
For one thing,
it is not clear what sample size zero means without further explanation.
For another thing, for an arrow
$$
\begin{CD}
   y_i @>\text{0-Poi}>> y_j
\end{CD}
$$
the distribution of the sum of IID zero-truncated Poisson random variables
is not a ``brand name distribution.''  And its
probability mass function has no closed-form expression.
So we could not label this arrow with the name of the conditional distribution
of $y_j$ given $y_i$ because there is no such name.

One consequence of the predecessor-is-sample-size property is that $y_j$
that are predecessors (are at nonterminal nodes) must be
nonnegative-integer-valued random variables.
There is an exception to this requirement that will be discussed in
Section~\ref{sec:infinitely-divisible} below, but that exception has
never been used.

Another consequence of the predecessor-is-sample-size property is
the following section.

\section{Conditional and Unconditional Mean Values}

\subsection{Unconditional}

Let $y$ be the response vector of an aster model.  We define a parameter
vector $\mu = E(y)$.  This is the vector having components
$$
   \mu_j = E(y_j), \qquad j \in J,
$$
where, as usual, $J$ is the index set of the response vector (the set
of non-initial nodes of the full aster graph).

This is called the \emph{unconditional mean value parameter vector}.
\index{parameter vector!mean value!unconditional}
This name is getting us a little bit ahead of ourselves.
At this point, we don't even know these means exist.
(We will eventually find out they do exist.)
%%%%%%%%%% NEED FORWARD REFERENCE to regular full and moments %%%%%%%%%%
And, at this point, we don't know that means parameterize aster models,
since we haven't yet even completely specified what the distribution
of an aster model is.  We know the fundamental factorization
\eqref{eq:factorize}, and we know each of those factors obeys the
predecessor-is-sample-size property, but we don't yet know anything more.
(We will eventually find out means do parameterize aster models.)
%%%%%%%%%% NEED FORWARD REFERENCE to mean value parameterization %%%%%%%%%%

For now we will just assume these means exist.

\subsection{Conditional}

We define another parameter vector $\xi$ having components
$$
   \xi_j = E(y_j \mid y_{p(j)} = 1), \qquad j \in J,
$$
if this expression makes sense.  It will not make sense when the
conditioning event has probability zero (so the conditional expectation
can be defined arbitrarily).  In that case we have to use a different
definition that does not come with an equation.
The predecessor-is-sample-size property says that $y_j$ is the sum
of $y_{p(j)}$ IID random variables, and we say $\xi_j$ is the mean
of those random variables.

This is called the \emph{conditional mean value parameter vector}.
\index{parameter vector!mean value!conditional}
As in the preceding section, this name is getting us a little bit ahead
of ourselves.
At this point, we don't even know these means exist.
(We will eventually find out they do exist.)
%%%%%%%%%% NEED FORWARD REFERENCE to regular full and moments %%%%%%%%%%
And, at this point, we don't know that means (conditional or unconditional)
parameterize aster models.  But the next section will show the unconditional
means determine conditional means and vice versa.  So if $\mu$ parameterizes,
then so does $\xi$, and vice versa.

\subsection{The Combination of the Two}
\label{sec:mu-and-xi}

It follows from the predecessor-is-sample-size property and linearity of
expectation that
\begin{equation} \label{eq:cond-exp}
   E(y_j \mid y_{p(j)}) = \xi_j y_{p(j)}, \qquad j \in J.
\end{equation}
Then it follows from the iterated expectation axiom of conditional probability
$$
   E(y_j)
   =
   E\{E(y_j \mid y_{p(j)})\}
$$
that
\begin{equation} \label{eq:mu-and-xi}
   \mu_j = \xi_j \mu_{p(j)}, \qquad j \in J.
\end{equation}
This is the fundamental recursive relation that shows (as we examine in
more detail presently) how $\mu$ is determined by $\xi$ and vice versa.

To map from $\xi$ to $\mu$ we use \eqref{eq:mu-and-xi} recursively
\begin{align*}
   \mu_j
   & =
   \xi_j \mu_{p(j)}
   \\
   & =
   \xi_j \xi_{p(j)} \mu_{p(p(j))}
   \\
   & =
   \xi_j \xi_{p(j)} \xi_{p(p(j))} \mu_{p(p(p(j)))}
\end{align*}
and so forth, with as many recursive applications as necessary.  In practice,
the computer traverses the graph in any order that visits predecessors before
successors using \eqref{eq:mu-and-xi} to determine $\mu_j$ as a function of
$\xi$ ($\mu_{p(j)}$ having already been determined when its node was visited
previously).  To get the recursion started, we need the mean values at initial
nodes, which are given by
\begin{equation} \label{eq:xi-mu-initial}
   \mu_j = y_j, \qquad j \in N \setminus J,
\end{equation}
because the mean value of a constant random variable is its constant value.

Using the reflexive transistive closure of the predecessor relation $\preceq$
we can rewrite the above as follows
\begin{equation} \label{eq:xi-mu-prod}
    \mu_j
    =
    \left( \prod_{\substack{i \in J \\ i \preceq j}} \xi_i \right)
    \left( \prod_{\substack{i \in N \setminus J \\ i \preceq j}} \mu_i \right),
    \qquad j \in J,
\end{equation}
where we note that the second product always has exactly one term: there
is always exactly one initial node $i$ such that $i \preceq j$.
We could also rewrite \eqref{eq:xi-mu-prod} as
\begin{equation} \label{eq:xi-mu-prod-too}
    \mu_j
    =
    \left( \prod_{\substack{i \in J \\ i \preceq j}} \xi_i \right)
    \left( \prod_{\substack{i \in N \setminus J \\ i \preceq j}} y_i \right),
    \qquad j \in J,
\end{equation}
by \eqref{eq:xi-mu-initial}.

To map from $\mu$ to $\xi$, rewrite \eqref{eq:mu-and-xi} as
\begin{equation} \label{eq:mu-to-xi}
   \xi_j = \frac{\mu_j}{\mu_{p(j)}}
\end{equation}
but for this to make sense, we must know that $\mu_{p(j)}$ is never zero.

We will eventually find out that $\mu_{p(j)}$ is never zero except in
%%%%%%%%%% NEED FORWARD REFERENCE %%%%%%%%%%
limiting conditional models.  So we do not always have this property.
Thus \eqref{eq:mu-to-xi} makes sense when $\mu_{p(j)}$ is never zero,
but otherwise some components of $\xi$ are not determined by $\mu$.

Conversely, multiplication by zero is not a problem (unlike division by zero),
so \eqref{eq:xi-mu-prod} and \eqref{eq:xi-mu-prod-too} always determine
$\mu$ as a function of $\xi$.

Everything in this section up to this point is an elementary consequence
of the laws of conditional and unconditional expectation and the
predecessor-is-sample-size property.  Consequently,
everything in this section up to this point is also true of all previous
models in survival analysis and demography that have also had this property
cited in Sections~\ref{sec:background} and~\ref{sec:piss} above.

\subsection{Confession}

\Citet{aster1} did not define $\xi$
the way we do here.  Instead they used that Greek letter to denote
\eqref{eq:cond-exp}.
A referee said this definition is dumb.  It makes $\xi$ a function
of both random variables and parameters, so it is not a parameter,
and one shouldn't use Greek letters for things that aren't parameters.
We didn't listen then and managed to get the
paper published overriding this objection.  But now we agree with the referee.

The vector $\xi$ as defined here is an important parameterization of
aster models \citep[this has been realized since][]{aster-philosophical}.

R package \texttt{aster} used the same dumb definition until version
1.0-2 of the package, when a new optional argument \code{is.always.parameter}
was added to the method of R generic function \code{predict} that handles
aster model objects.  And, for reasons of backward compatibility,
the dumb definition is still the default.
One must use the optional argument \code{is.always.parameter = TRUE}
to estimate $\xi$ as defined in this section.

R package \texttt{aster2} and recent papers and technical reports use
the definition presented here (the conditional mean value parameter vector
is $\xi$ if they mention conditional mean value parameters at all).

\section{Some Aster Graphs}
\label{sec:graphs}

The first published aster model \citep{aster1} had this graph
\begin{equation} \label{gr:aster1}
\begin{CD}
   1
   @>\text{Ber}>>
   y_1
   @>\text{Ber}>>
   y_2
   @>\text{Ber}>>
   y_3
   \\
   @.
   @VV\text{Ber}V
   @VV\text{Ber}V
   @VV\text{Ber}V
   \\
   \hphantom{1}
   @.
   y_4
   @.
   y_5
   @.
   y_6
   \\
   @.
   @VV\text{0-Poi}V
   @VV\text{0-Poi}V
   @VV\text{0-Poi}V
  \\
   \hphantom{1}
   @.
   y_7
   @.
   y_8
   @.
   y_9
\end{CD}
\end{equation}
which is for one individual.  There are 570 individuals in the data set,
which is included in the R package \texttt{aster}.  So one can think of the
full aster graph as 570 copies of this graph with the subscripts changed
so the nodes (the $y_j$) are all different.

Because this graph has only arrows, no lines, each node is
a dependence group all by itself.

The individuals are plants of the species \emph{Echinacea angustifolia},
whose common name is narrow-leaved
purple coneflower.  These data were collected by the Echinacea Project
(\url{http://echinaceaproject.org/}), a long-running project funded by
the National Science Foundation (the co-PI's are the second and third authors
of \citet{aster1}).  The way \eqref{gr:aster1} is laid out,
variables in the first column ($y_1$, $y_4$, and $y_7$) are for 2002,
those in the second column are for 2003,
those in the third column are for 2004,
those in the first row ($y_1$, $y_2$, and $y_3$) measure survival
(0 = dead, 1 = alive),
those in the second row indicate flowering
(0 = no flowers, 1 = some flowers),
those in the third row are flower head counts
(actual number of flower heads).

Of course,
the ``rows'' and ``columns'' are not part of the graphical structure.
The only thing that matters is which nodes are connected by which arrows.

Aster graphs can get a lot bigger than \eqref{gr:aster1}.
The Echinacea Project now has data for years since 2004 (which extends
the graph with many more ``columns'') and data for more life history
stages (which extends the graph with more ``rows'').

The node labels (the $y_j$) are random variables, components of the response
vector.  The arrows indicate conditional distributions.
An arrow
\begin{equation} \label{gr:one-arrow-conditional}
\begin{CD}
   y_i @>>> y_j
\end{CD}
\end{equation}
indicates the conditional distribution of $y_j$ given $y_i$.
An arrow
\begin{equation} \label{gr:one-arrow-marginal}
\begin{CD}
   1 @>>> y_i
\end{CD}
\end{equation}
indicates the marginal distribution of $y_i$,
because conditioning on a constant random variable is the same as not
conditioning.

Labels on the arrows name the distribution.
Ber is for Bernoulli (any zero-or-one-valued random variable), and
0-Poi is for zero-truncated Poisson (Poisson conditioned on being nonzero).
This explanation of arrows and their distributions is incomplete and will
be picked up again in Section~\ref{sec:piss}.
%%%%%%%%%% NEED FORWARD REFERENCE to exponential family assumption %%%%%%%%%%

Here is a more complicated aster graph from \citet{aster3}
\begin{equation} \label{gr:aster3}
\begin{CD}
   1
   @>\text{Ber}>>
   y_1
   @>\text{Ber}>>
   y_2
   @>\text{Ber}>>
   y_3
   @>\text{Ber}>>
   y_4
   \\
   @.
   @VV\text{Ber}V
   @VV\text{Ber}V
   @VV\text{Ber}V
   @VV\text{Ber}V
   \\
   \hphantom{1}
   @.
   y_5
   @.
   y_6
   @.
   y_7
   @.
   y_8
   \\
   @.
   @VV\text{0-Poi}V
   @VV\text{0-Poi}V
   @VV\text{0-Poi}V
   @VV\text{0-Poi}V
   \\
   \hphantom{1}
   @.
   y_9
   @.
   y_{10}
   @.
   y_{11}
   @.
   y_{12}
   \\
   @.
   @VV\text{Poi}V
   @VV\text{Poi}V
   @VV\text{Poi}V
   @VV\text{Poi}V
   \\
   \hphantom{1}
   @.
   y_{13}
   @.
   y_{14}
   @.
   y_{15}
   @.
   y_{16}
   \\
   @.
   @VV\text{Ber}V
   @VV\text{Ber}V
   @VV\text{Ber}V
   @VV\text{Ber}V
   \\
   \hphantom{1}
   @.
   y_{17}
   @.
   y_{18}
   @.
   y_{19}
   @.
   y_{20}
\end{CD}
\end{equation}

Again, because this graph has only arrows, no lines, each node is
a dependence group all by itself.

This graph is for simulated data, which \citet{aster3} used because
at the time no data for aster models as complicated as \eqref{gr:aster3} had
been collected by biologists, and it was important to give such
an illustration of the possibilities of aster models.
Like in \eqref{gr:aster1} the ``columns'' in \eqref{gr:aster3} are for
data in successive years.  The first three ``rows'' of \eqref{gr:aster3}
can be taken to be the same as those of \eqref{gr:aster1}: survival,
flowering indicator variables, and flower counts.  The fourth row of
\eqref{gr:aster3} is seed counts, and the fifth row is number of seeds
that germinate (produce new plants).  Of course, since the data
are simulated, the story about these variables is just a story.
It could be told differently, and \citet{aster3} do have a story
where the same graph could be for data about an animal rather than a plant.

This graph did serve as a good example of what was possible.
\citet*[in the on-line appendix]{stanton-geddes-tiffin-shaw}
discuss an aster model with seven life history stages (one of which is
artificial, modeling random sampling in the data collection process, hence
only six are about life history of the organisms) and thus would be
like the graph \eqref{gr:aster3} except with seven ``rows.''
Because the organism in question (\emph{Chamaecrista fasciculata}, common
name partridge pea) was an annual plant, there is only one ``column.''
(As mentioned above, ``rows'' and ``columns'' are not part of the graphical
structure --- the only thing that matters is which nodes are connected
by which arrows --- and \citeauthor{stanton-geddes-tiffin-shaw} actually
laid out their graph in a row.)

Statistically, there are some important differences between these graphs.
Graph \eqref{gr:aster3} has both Poisson (Poi) and zero-truncated Poisson
(0-Poi) arrows and hence illustrates when to use which.
In graph \eqref{gr:aster1} every predecessor node is Bernoulli,
but in graph \eqref{gr:aster3} $y_{13}$ through $y_{16}$ are non-Bernoulli
predecessor nodes.  So \eqref{gr:aster3} shows that predecessor values can
be any nonnegative integer.

The graph \eqref{gr:multi} comes from a still unpublished manuscript for a book
about aster models.  It was the first graph for a model for an animal having
life history stages like an insect's larva, pupa, and adult.  We present this
graph for hypothetical data even though a similar model has been fit to real
data by \citet{aster-hornworm}.  Those data are for the tobacco hornworm
\emph{Manducca sexta}, which is an insect (a moth) that does have these life
history stages.  These data were not collected with the intention of using
an aster model (which were very new when the experiment was done) and so were
not ideal for aster analysis.  Although an aster analysis was done by
\citet{aster-hornworm}, it does not serve as quite as clean an example as
the graph \eqref{gr:multi}.

\begin{equation} \label{gr:multi}
\begin{tikzcd}
  \hphantom{1} & y_1 & y_4 & y_7 & y_{10}
  \\
  1
  \arrow{r}{\mathcal{M}}
  \arrow{rd}{\mathcal{M}}
  \arrow{ru}{\mathcal{M}}
  & y_2
  \arrow[dash]{u}
  \arrow[dash]{d}
  \arrow{r}{\mathcal{M}}
  \arrow{rd}{\mathcal{M}}
  \arrow{ru}{\mathcal{M}}
  & y_5
  \arrow[dash]{u}
  \arrow[dash]{d}
  \arrow{r}{\mathcal{M}}
  \arrow{rd}{\mathcal{M}}
  \arrow{ru}{\mathcal{M}}
  & y_8
  \arrow[dash]{u}
  \arrow[dash]{d}
  \arrow{r}{\mathcal{M}}
  \arrow{rd}{\mathcal{M}}
  \arrow{ru}{\mathcal{M}}
  & y_{11}
  \arrow[dash]{u}
  \arrow[dash]{d}
  \\
  \hphantom{1} & y_3 \arrow{d}{\text{Ber}} & y_6 \arrow{d}{\text{Ber}}
  & y_9 \arrow{d}{\text{Ber}} & y_{12} \arrow{d}{\text{Ber}}
  \\
  \hphantom{1} & y_{13} \arrow{d}{\text{0-Poi}}
  & y_{14} \arrow{d}{\text{0-Poi}}
  & y_{15} \arrow{d}{\text{0-Poi}} & y_{16} \arrow{d}{\text{0-Poi}}
  \\
  \hphantom{1} & y_{17} & y_{18} & y_{19} & y_{20}
\end{tikzcd}
\end{equation}

As always, the constant 1 at the initial node
of the graph indicates that the graph is for one individual.
In addition to the notations Ber = Bernoulli
and 0-Poi = zero-truncated Poisson, which we have already met,
we now also have $\mathcal{M}$ = multinomial.
Lines without arrowheads are ``lines'' connecting nodes in the
same dependence group.  Hence the dependence groups containing more than
one node are $\{1, 2, 3\}$, $\{4, 5, 6\}$, $\{7, 8, 9\}$, and $\{10, 11, 12\}$.
Other nodes are dependence groups all by themselves; $\{j\}$ is a dependence
group for $j \ge 13$.

As stated in Section~\ref{sec:exception-dependence-group-lines} above,
this graph does not follow the theory, which requires a line connecting
each pair of distinct elements of each dependence group and which would
require us to add lines $y_1 \myline y_3$ and $y_4 \myline y_6$
and $y_7 \myline y_9$ and $y_{10} \myline y_{12}$ to the graph.
But, also as stated in Section~\ref{sec:exception-dependence-group-lines}
above, we don't need these lines to infer the dependence groups from the
graph.  The lines $y_1 \myline y_2$ and $y_2 \myline y_3$ that are in the
graph say $y_1$ and $y_3$ are in the same dependence group as $y_2$,
and since there are no other lines to these nodes, they must constitute
a dependence group.  Since there doesn't seem to be any room in
the picture \eqref{gr:multi} for these additional lines required by theory,
we omit them.

Each of the multi-node dependence groups has a conditional multinomial
distribution with, as usual, predecessor as sample size.  Since each
predecessor is zero-or-one-valued, if a predecessor (say $y_2$) is
equal to one, then exactly one of its three successor nodes ($y_4$, $y_5$,
and $y_6$) is equal to one, and, if this predecessor
is equal to zero, then all of its three successor nodes are also equal to zero.
In effect, exactly one of the ``exterior nodes'' of this group of switches
($y_1$, $y_4$, $y_7$, $y_{10}$, $y_{11}$, $y_{12}$, $y_9$, $y_6$, and $y_3$)
is equal to one.  There is one path taken by any particular individual,
from the initial node (marked 1) through these four multinomial dependence
groups.

The intended application for this graph \citep{aster-hornworm} is life history
data for an insect.  As in our graphs without dependence groups,
``columns'' of the graph are for different
times (here days, there years).
Nodes in the top ``row'' of this graph ($y_1$, $y_4$, $y_7$, and $y_{10}$)
indicate death.
Nodes in the second ``row'' of this graph ($y_2$, $y_5$, $y_8$, and $y_{11}$)
indicate the individual is a larva (caterpillar).
Nodes in the third ``row'' of this graph ($y_3$, $y_4$, $y_9$, and $y_{12}$)
indicate the individual is an adult (moth, with wings, flying around trying
to mate).
Nodes in the fourth ``row'' of this graph ($y_{13}$ through $y_{16}$)
indicate the mating success.
Nodes in the bottom ``row'' of this graph ($y_{17}$ through $y_{20}$)
count number of eggs laid.  So this graph is for female individuals.
In \citet{aster-hornworm} the same graph with only the multinomial dependence
groups (nodes 1 through $y_{12}$) is used for male individuals because
the sex of individuals was not determined before they reached adulthood.

So this graph illustrates two important points not seen before.
It is not necessary for every individual to
have the same graph (here females and males have different graphs).
And we have non-singleton dependence groups,
multinomial ``switches'' between different life history stages.

Here is yet another graph illustrating normal dependence groups.
\begin{equation} \label{gr:normal}
\begin{tikzcd}
   1
   \arrow{r}{\text{Ber}}
   & y_1
   \arrow{d}[swap]{\mathcal{N}}
   \arrow[bend left]{dd}{\mathcal{N}}
   \arrow{r}{\text{Ber}}
   & y_2
   \arrow{d}[swap]{\mathcal{N}}
   \arrow[bend left]{dd}{\mathcal{N}}
   \arrow{r}{\text{Ber}}
   & y_3
   \arrow{d}[swap]{\mathcal{N}}
   \arrow[bend left]{dd}{\mathcal{N}}
   \arrow{r}{\text{Ber}}
   & y_4
   \arrow{d}[swap]{\mathcal{N}}
   \arrow[bend left]{dd}{\mathcal{N}}
   \\
   \hphantom{1}
   & y_5 \arrow[dash]{d}
   & y_6 \arrow[dash]{d}
   & y_7 \arrow[dash]{d}
   & y_8 \arrow[dash]{d}
   \\
   \hphantom{1} & y_9 & y_{10} & y_{11} & y_{12}
\end{tikzcd}
\end{equation}
Here the top ``row'' indicates survival.  And the next two rows are
for normally distributed something or other given survival.  Here we
model the normal as two-node dependence groups
$\{ 5, 9 \}$,
$\{ 6, 10 \}$.
$\{ 7, 11 \}$, and
$\{ 8, 12 \}$ because
we do not want to assume variance is known.  As we shall see, this permits
but does not require, modeling variance as a function of covariates.

We see that the aster formalism suggests new possibilities.  In order to
have a two-parameter normal distribution, we need two-node dependence groups.
%%%%%%%%%% NEED FORWARD REFERENCE to two-parameter normal %%%%%%%%%%

\subsection{Not Really Missing Data}

\subsection{Bernoulli versus Multinomial}

Bernoulli arrows and multinomial dependence groups are closely related but
work differently.  Bernoulli is related to multinomial but different.

In \eqref{gr:multi} every arrow labled with an $\mathcal{M}$ is Bernoulli
marginally, but the whole point of dependence groups is that the components
of the response vector in a dependence group are not
conditionally independent given their predecessors, unlike the Bernoulli
arrows labeled Ber in \eqref{gr:multi} or in any of the graphs in this
section (by Lemma~\ref{lem:markov}).

Conversely, if
$$
\begin{CD}
   y_i @>\text{Ber}>> y_j
\end{CD}
$$
is a Bernoulli arrow, then we could replace this with a multinomial dependence
group that does the same thing
$$
\begin{tikzcd}
  \hphantom{1} & y_k
  \\
  1
  \arrow{r}{\mathcal{M}}
  \arrow{ru}{\mathcal{M}}
  & y_j
  \arrow[dash]{u}
\end{tikzcd}
$$
where $k$ is some index that hasn't been used in the rest of the graph.

For example, we could change graph \eqref{gr:aster1} to
\begin{equation} \label{gr:aster1-too}
\begin{tikzcd}
  \hphantom{1} & y_{10} & y_{11} & y_{12}
  \\
  1
  \arrow{r}{\mathcal{M}}
  \arrow{ru}{\mathcal{M}}
  & y_1
  \arrow[dash]{u}
  \arrow{d}{\text{Ber}}
  \arrow{r}{\mathcal{M}}
  \arrow{ru}{\mathcal{M}}
  & y_2
  \arrow[dash]{u}
  \arrow{d}{\text{Ber}}
  \arrow{r}{\mathcal{M}}
  \arrow{ru}{\mathcal{M}}
  & y_3
  \arrow[dash]{u}
  \arrow{d}{\text{Ber}}
  \\
  \hphantom{1} & y_4 \arrow{d}{\text{0-Poi}}
  & y_5 \arrow{d}{\text{0-Poi}}
  & y_6 \arrow{d}{\text{0-Poi}}
  \\
  \hphantom{1} & y_7 & y_8 & y_9
\end{tikzcd}
\end{equation}
and all of the components of the response vector that have the same indices
would have the same interpretation and the same values in the same data
in both graphs \eqref{gr:aster1} and \eqref{gr:aster1-too}.
Then the additional nodes $y_{10}$, $y_{11}$, and $y_{12}$ are determined
by the properties of the multinomial distribution
\begin{align*}
   y_{10} & = 1 - y_1
   \\
   y_{11} & = y_1 (1 - y_2)
   \\
   y_{12} & = y_2 (1 - y_3)
\end{align*}
(this is does not follow from what we have said so far, it also needs
Section~\ref{sec:piss} below).

\section{Exponential Families of Distributions}




\chapter{Completion} \label{chap:completion}

In this chapter we deal with what to do when maximum likelihood estimates
do not exist in the exponential family or aster model we are initially given.
There may, and usually do, exist maximum likelihood estimates in the
\emph{completion} of the family.  It is a bit unclear what we should call
the statistical models studied in this chapter.
\begin{itemize}
\item \citet[Sections~9.3 and~9.4]{barndorff-nielsen} calls this concept
    \emph{completion}.
\item \citet[Chapter~6]{brown} calls this concept
    an \emph{aggregate exponential family} for reasons that will be explained
    presently.
\item \citet[Chapters~2 and~4]{geyer-thesis} calls this concept
    \emph{closure}.
\item \citet{geyer-gdor} calls this concept
    \emph{Barndorff-Nielsen completion}.
\end{itemize}
\citeauthor[personal communication]{brown} pointed out that the eponym chosen
in \citet{geyer-gdor} was not quite correct, since \citet{barndorff-nielsen}
works under more restrictive regularity conditions than \citet{brown}, and
\citet{brown} works under more restrictive regularity conditions than
\citet{geyer-thesis}.  The choice in \citet{geyer-gdor} follows
Stigler's law of eponomy.  At least in this case \citeauthor{barndorff-nielsen}
had the concept first if not in the most generality.
The reason why \citet{geyer-thesis} chose ``closure'' rather than ``completion''
is that when one works under the weakest regularity conditions, the topological
space that is the statistical model being ``completed'' is not metrizable,
hence ``complete'' (every Cauchy sequence converges) doesn't make any sense
(the definition of Cauchy sequence requires a metric).  Thus we have only
the more general topological concept of closure.
We won't fuss about any of this and will continue use Barndorff-Nielsen
completion or just completion.

\section{Binomial Example}

For this simplest example of the phenomenon of interest, we consider the
binomial distribution.  We know from the discussion
in Section~\ref{sec:direction-of-recession} above that the MLE does not exist
when the observed value of the canonical statistic, which for the binomial
distribution is the number of successes, is an extreme value, either as small
as it can be or as large as it can be, in this case either 0 or $n$, where
$n$ is the sample size.

Usually, we think the MLE for the usual parameter $p$, the success probability,
does exist for all data and is $\hat{p} = x / n$.  But when $x = 0$ or $x = n$,
so $\hat{p}$ is zero or one, the MLE for the canonical parameter
$\theta = \logit(p)$
%%%%%%%%%% NEED FORWARD REFERENCE %%%%%%%%%%
does not exist because the domain of the logit function is the open interval
$(0, 1)$ and does not include the endpoints.  Since
\begin{align*}
   \lim_{p \downarrow 0} \logit(p) & = - \infty
   \\
   \lim_{p \uparrow 0} \logit(p) & = \infty
\end{align*}
we could try to identify these endpoints with infinite values of the canonical
parameter, but that is not the way exponential family theory works,
and, as we shall see, it does not generalize to multiparameter problems.

So instead of trying to complete the parameter space, we try to complete the
family of distributions.  These distributions have PMF
$$
   f_p(x) = \binom{n}{x} p^x (1 - p)^{n - x}
$$
and we have
\begin{align*}
   \lim_{p \downarrow 0} f_p(x) & = \begin{cases} 1, & x = 0 \\ 0, & x > 0
   \end{cases}
   \\
   \lim_{p \uparrow 1} f_p(x) & = \begin{cases} 0, & x < n \\ 1, & x = n
   \end{cases}
\end{align*}
so the completion contains the original exponential family we were given
plus two new distributions, the degenerate distribution concentrated at zero
and the degenerate distribution concentrated at $n$.  And these new
distributions are what are usually thought of as the binomial distributions
for $p = 0$ and $p = 1$ (when $p = 0$ no successes are possible so $x = 0$
almost surely; when $p = 1$ no failures are possible so $x = n$ almost surely).

\section{General Exponential Families}

\subsection{Support and Support Function}

Let $C$ denote the \emph{closed convex support} of the exponential family
under discussion.  This is the smallest closed convex set that contains
the canonical statistic vector with probability one.  Hence it is a closed
convex subset of the vector space where the canonical statistic takes values.

The closed convex support always exists because the intersection of closed sets
is closed and the intersection of convex sets is convex and because
finite-dimensional vector spaces are second countable.  Define $C$ to be
the intersection of all closed convex sets that contain the canonical
statistic vector $Y$ with probability one under some distribution in the
family, and hence for all distributions in the family (because all have
the same support).  Then event $Y \notin C$ is the
union of a countable family of open sets having probability zero,
hence $Y \in C$ almost surely.

Let $\sigma_C$ denote the \emph{support function} of $C$, defined by
\begin{equation} \label{eq:support-function}
   \sigma_C(\delta) = \sup_{y \in C} \inner{y, \delta}
\end{equation}
\citep[Section~8.E]{rockafellar-wets}.  The term ``support'' here is
unfortunate in that it is unrelated to the term ``support'' in $C$ being
a support of the canonical statistic vector of the exponential family.
But both terms are well established (``closed convex support'' in exponential
family theory and ``support function'' in convex analysis).

\subsection{Probability Mass-Density Functions}

If \eqref{eq:logl-expfam} is the log likelihood of an exponential family,
the the PMDF of that family must be the exponential of the log likelihood.
In order that we do not get extra terms that do not appear in the log
likelihood and in order to get the right support of the family, we take
the measure with respect to which we calculate densities to be a measure
in the family, say the measure corresponding to canonical parameter vector
$\psi$.  Then the PMDF are
\begin{equation} \label{eq:pmdf-expfam}
   f_\theta(\omega) = e^{\inner{Y(\omega), \theta - \psi} - c(\theta) + c(\psi)}
\end{equation}
where $\omega$ is the complete data (remember that $Y$ is a statistic,
not necessarily the complete data) \citep[Equation~(4)]{geyer-gdor}.

\subsection{Straight Line Limits}

\begin{theorem} \label{th:completion-fundamental}
For a full exponential family having log likelihood \eqref{eq:logl-expfam},
densities \eqref{eq:pmdf-expfam}, canonical statistic vector $Y$,
full canonical parameter space $\Theta$, and closed convex support $C$,
suppose $\delta$ is a direction in the vector space where the canonical
parameter takes values,
\begin{equation} \label{eq:complete-fundamental-hyperplane}
   H_\delta = \set{ y \in \real^J : \inner{y, \delta} = \sigma_C(\delta) },
\end{equation}
then for all $\theta \in \Theta$
\begin{equation} \label{eq:complete-fundamental-limit}
   \lim_{\theta + s \delta} f_{\theta + s \delta}(\omega)
   =
   \begin{cases}
   0, & \inner{Y(\omega), \delta} < \sigma_C(\delta)
   \\
   f_\theta(\omega) / \Pr_\theta(Y \in H_\delta),
   & \inner{Y(\omega), \delta} = \sigma_C(\delta)
   \\
   \infty, & \inner{Y(\omega), \delta} > \sigma_C(\delta)
   \end{cases}
\end{equation}
where the middle term is defined to be $\infty$ in case of divide by zero.
If $\delta$ is not a direction of constancy
and $\Pr_\theta(Y \in H_\delta) > 0$, then the function
$s \mapsto \Pr_{\theta + s \delta}(Y \in H_\delta)$ is continuous,
strictly increasing, and converges to one
as $s \to \infty$.
\end{theorem}
In the two cases ruled out by the precondition of the last sentence
the function $s \mapsto \Pr_{\theta + s \delta}(Y \in H_\delta)$ is
a constant function.  If $\delta$ is a direction of constancy,
then $\Pr_\theta(Y \in H_\delta) = 1$ for all $\theta$.
If $\Pr_\theta(Y \in H_\delta) = 0$ for some $\theta$,
then $\Pr_\theta(Y \in H_\delta) = 0$ for all $\theta$.
\begin{proof}
This is a complication of Theorem~{6} in \citet{geyer-gdor}
that is essentially Theorem~{2.3} in \citet{geyer-thesis}.
However the proof of that Theorem~{2.3} contains some errors,
so a corrected proof is given in the appendix of \citet{geyer-gdor}.
Then the case $\Pr_\theta(Y \in H_\delta) > 0$ is Theorem~{2.6}
in \citet{geyer-thesis},
and the case $\Pr_\theta(Y \in H_\delta) = 0$ follows from Theorem~{2.2}
in \citet{geyer-thesis}.
The last sentence of the theorem statement is in Corollary~{5} and Theorem~{6}
in \citet{geyer-gdor}.
\end{proof}
In case $\Pr_\theta(Y \in H_\delta) > 0$,
we note three things about the limit in \eqref{eq:complete-fundamental-limit}.
\begin{itemize}
\item It is a probability distribution because the set where it is infinite
    has measure zero under the dominating measure $\Pr_\psi$
\item It is a conditional distribution of the original family,
    the conditional distribution of $Y$ given the event $Y \in H_\delta$
    for the parameter vector $\theta$.
\item It is a limit distribution of the original family,
    the limit of the distributions for parameter vectors $\theta + s \delta$
    as $s \to \infty$.  By Scheff\'{e}'s lemma, convergence of PMDF implies
    convergence in total variation of the corresponding probability measures.
\end{itemize}
Note that the distribution under discussion is both a limit distribution
and a conditional distribution.  Thinking of it as just one or the other
is missing something.  It is both.

In case $\Pr_\theta(Y \in H_\delta) = 0$,
we note one thing about the limit in \eqref{eq:complete-fundamental-limit}.
\begin{itemize}
\item It is the zero measure because it is zero on the support of
the dominating measure $\Pr_\psi$.
\end{itemize}

\subsection{Limiting Conditional Models}

We note one thing about the set of all limits
in \eqref{eq:complete-fundamental-limit}
in the case when $\Pr_\theta(Y \in H_\delta) > 0$.
\begin{itemize}
\item They form an exponential family of distributions.  The log likelihood is
\begin{equation} \label{eq:logl-expfam-lcm}
   l_\delta(\theta) =
   \inner{y, \theta} - c(\theta) - \log \Pr\nolimits_\theta(Y \in H_\delta)
\end{equation}
and this is clearly an exponential family with
\begin{itemize}
\item canonical statistic vector $y$,
\item canonical parameter vector $\theta$, and
\item cumulant function given by
\begin{equation} \label{eq:cumfun-lcm}
   c_\delta(\theta) = c(\theta) + \log \Pr\nolimits_\theta(Y \in H_\delta)
\end{equation}
\end{itemize}
\end{itemize}
\citet{geyer-gdor} calls this family the \emph{limiting conditional model}
(LCM).  Of course, there are many LCM, one in each direction, but as we
shall presently see, there is usually only one LCM of interest
in any particular data analysis.

\begin{theorem} \label{th:cumfun-lcm}
Equation \eqref{eq:cumfun-lcm} gives the correct limit of the cumulant function
to make \eqref{eq:logl-expfam-lcm} equal to the limit of \eqref{eq:logl-expfam}
when limits are taken as in Theorem~\ref{th:completion-fundamental}
in the case $\inner{y, \theta} = \sigma_C(\delta)$.
\end{theorem}
\begin{proof}
In symbols, the assertion of the theorem is
$$
   \lim_{s \to \infty} l(\theta + s \delta)
   =
   l_\delta(\theta)
$$
And
\begin{align*}
   \lim_{s \to \infty} l(\theta + s \delta)
   & =
   \lim_{s \to \infty}
   \bigl[ \inner{y, \theta + s \delta} - c(\theta + s \delta) \bigr]
   \\
   & =
   \inner{y, \theta} + \lim_{s \to \infty}
   \bigl[ s \inner{y, \delta} - c(\theta + s \delta) \bigr]
   \\
   & =
   \inner{y, \theta} + \lim_{s \to \infty}
   \bigl[ s \sigma_C(\delta) - c(\theta + s \delta) \bigr]
   \\
   & =
   \inner{y, \theta} - c(\theta) - \log \Pr\nolimits_\theta(Y \in H_\delta)
   \\
   & =
   \inner{y, \theta} - c_\delta(\theta)
\end{align*}
where the fourth equality is Theorem~{2.2} in \citet{geyer-thesis}.
\end{proof}

We know, of course, that cumulant functions can be redefined by adding an
arbitrary constant (the $c(\psi)$ in \eqref{eq:cumfun-expfam}).
As mentioned in Section~\ref{sec:define-expfam} above, we could even redefine
the cumulant function by adding an arbitrary affine function if we were to
accept a different choice of canonical statistic.  But things would get
very confusing if we made different arbitrary choices for the original
exponential family and its limiting conditional models.  Hence, however
the cumulant function of the original exponential family was chosen, we
will always use \eqref{eq:cumfun-lcm} to define cumulant functions
for limiting conditional models.

\subsection{Aggregate Exponential Family}

Denote the LCM in the direction $\delta$ by $\mathcal{P}_\delta$.
When $\Pr_\theta(Y \in H_\delta) = 0$ we say $\mathcal{P}_\delta$ is empty
(there are no limit probability distributions, and we do not want to include
the zero measure in our completion, at least not yet).
Taking limits when $\delta = 0$ does nothing
(because \eqref{eq:support-function} says $\sigma_C(0) = 0$ for any $C$
and this gives $H_\delta = \real^J$
in \eqref{eq:complete-fundamental-hyperplane}).
So $\mathcal{P}_0$ is the exponential family we started with, which we
call the original model (OM) for short.

Then
$$
   \mathcal{P} = \bigcup_{\delta \in \real^J} \mathcal{P}_\delta
$$
is a union of exponential families that contains all straight-line limits.

Under certain regularity conditions used by \citet{barndorff-nielsen},
\citet{brown}, and \citet{geyer-gdor} this union is the completion.
We do not get anything more by taking further straight-line limits
in $\mathcal{P}_\delta$ for the various $\delta$.

But in general \citep[Chapters~2 and~4]{geyer-thesis} we may need to take
further straight-line limits or general (not straight line) limits
to arrive at the completion.

Anyway, one can see why \citet{brown} gave this idea the name aggregate
exponential family.  It is a union (or aggregate) of exponential families.

\subsection{Support and Directions of Recession and Constancy}

\begin{theorem} \label{th:recession-constancy}
A vector $\delta$ is a direction of recession of the log likelihood of
a full exponential family
with closed convex support $C$ and observed value of the canonical statistic
vector $y$ if and only if $\inner{y, \delta} \ge \sigma_C(\delta)$.
If $\delta$ and $- \delta$ are both directions of recession, then $\delta$
is a direction of constancy.
Conversely, if $\delta$ is a direction of constancy and $y \in C$,
then $\delta$ and $- \delta$ are both directions of recession.
\end{theorem}
The condition $y \in C$ is measure-theoretic nonsense.
We have to say $y \in C$ to be measure-theoretically correct, but
if your data fail to satisfy $y \in C$, then something is wrong with your data.
\begin{proof}
The first sentence is Corollary~{2.4.1} in \citet{geyer-thesis}.
If $\delta$ and $- \delta$ are both directions of recession then
$$
   \inner{y, \delta} \ge \sigma_C(\delta) = \sup_{x \in C} \inner{x, \delta}
$$
and
\begin{align*}
   - \inner{y, \delta}
   & =
   \inner{y, - \delta}
   \\
   & \ge
   \sigma_C(- \delta)
   \\
   & =
   \sup_{x \in C} \inner{x, - \delta}
   \\
   & =
   - \inf_{x \in C} \inner{x, \delta}
\end{align*}
or
$$
   \inner{y, \delta} \le \inf_{x \in C} \inner{x, \delta}
$$
so
\begin{equation} \label{eq:recession-constancy-inequalities}
   \inner{y, \delta} \le \inf_{x \in C} \inner{x, \delta}
   \le
   \sup_{x \in C} \inner{x, \delta} \le \inner{y, \delta} 
\end{equation}
hence
\begin{equation} \label{eq:recession-constancy}
   \inner{x, \delta} = \inner{y, \delta}, \qquad x \in C
\end{equation}
hence $\inner{Y, \delta} = \inner{y, \delta}$ almost surely,
and $\delta$ is a direction of constancy.

Conversely, if $\delta$ is a direction of constancy, then $\inner{Y, \delta}$
is constant almost surely.  And if $y \in C$, that constant must
be $\inner{y, \delta}$.  Hence \eqref{eq:recession-constancy} holds.
Hence \eqref{eq:recession-constancy-inequalities} holds.
And we have already seen that \eqref{eq:recession-constancy-inequalities}
is equivalent to both $\delta$ and $- \delta$ being directions of recession.
\end{proof}

\subsection{Curved Line Limits}

Chapter~4 of \citet{geyer-thesis} covers completely general
limits of sequences of distributions in an exponential family of distributions,
these limits being in the sense of convergence of probability mass-density
functions.  Although the section title says ``curved line limits,'' these
limits are just limits of sequences.  We only get a line by connecting
the dots, and that line does not have to be a smooth curve.

Theorems~4.1 through~{4.5} in \citet{geyer-thesis} show that taking general
limits gives no more limits that correspond to probability distributions
than taking iterated straight line limits.  General limits can produce
limits that are subprobability distributions
\citep[Examples~4.2 through~4.4]{geyer-thesis}, but these can never be
maximum likelihood estimates for a full family, because iterated straight
line limits produce the corresponding probability distribution, which must
have higher likelihood.
This shows that we do not need to consider curved line limits, so long
as we limit our attention to full families.
%%%%% Need to worry about subsampling, which are curved exponential family
%%%%% Do not need to worry about conditional aster models, which are also
%%%%%     curved exponential family, because of associated independence
%%%%%     models, which are full exponential family

\section{Unconditional Aster Models}

Unconditional aster models are regular full exponential families.
Thus the theory of the preceding section applies to them.

\begin{theorem} \label{th:dor-arrow}
Suppose $\{j\}$ is a univariate dependence group in an aster graph, and
the one-parameter exponential family of distributions for the arrow
$y_{p(j)} \longrightarrow y_j$ has closed convex support that is an
interval with endpoints $a_j$ and $b_j$ (either of which may be infinite and
which satisfy $a_j \le b_j$ with equality possible, in which case this
distribution is concentrated at one point).  Let $J$
be the set of non-initial nodes of the aster graph,
and let $Y$ denote the response vector and $y$ its observed value.

If $y_j = a_j y_{p(j)}$, then the vector $\eta$ having index set $J$
and coordinates
\begin{equation} \label{eq:dor-lower-bound}
   \eta_i = \begin{cases} -1, & i = j \\ a_j & i = p(j) \\
   0, & \text{otherwise} \end{cases}
\end{equation}
is a direction of recession of the saturated aster model.

Taking the limit in the direction of recession \eqref{eq:dor-lower-bound}
gives the LCM that is the same as the OM except the arrow
$y_{p(j)} \longrightarrow y_j$ has the degenerate family of distributions
concentrated at $a_j$.
This LCM is the OM conditioned on the event $Y_j = a_j Y_{p(j)}$.

If $y_j = b_j y_{p(j)}$, then the vector $\eta$ having index set $J$
and coordinates
\begin{equation} \label{eq:dor-upper-bound}
   \eta_i = \begin{cases} 1, & i = j \\ - b_j & i = p(j) \\
   0, & \text{otherwise} \end{cases}
\end{equation}
is a direction of recession of the saturated aster model.

Taking the limit in the direction of recession \eqref{eq:dor-upper-bound}
gives the LCM that is the same as the OM except the arrow
$y_{p(j)} \longrightarrow y_j$ has the degenerate family of distributions
concentrated at $b_j$.
This LCM is the OM conditioned on the event $Y_j = b_j Y_{p(j)}$.

In case $a_j = b_j$ the vectors \eqref{eq:dor-lower-bound}
and \eqref{eq:dor-upper-bound} are both directions of recession,
and are negatives of each other, hence are both directions of constancy.
But in this case we only need one direction of constancy since one
is a scalar multiple of the other.

In case $y_{p(j)} = 0$  and $-\infty < a_j < b_j < \infty$
the vectors \eqref{eq:dor-lower-bound}
and \eqref{eq:dor-upper-bound} are still both directions of recession,
but are not directions of constancy unless $Y_{p(j)} = 0$ almost surely.
\end{theorem}
\begin{proof}
We have a direction of recession if and only if $\inner{Y - y, \eta} \le 0$
almost surely.
From the definition
of $a_j$ and $b_j$ we know $a_j Y_{p(j)} \le Y_j \le b_j Y_{p(j)}$
almost surely.

In case $y_j = a_j y_{p(j)}$ and $\eta$ is
given by \eqref{eq:dor-lower-bound} we have two cases.
If $p(j)$ is noninitial, then
$$
   \inner{Y - y, \eta} = - (Y_j - y_j) + a_j (Y_{p(j)} - y_{p(j)})
   =
   - Y_j + a_j Y_{p(j)}
$$
and this is indeed less than or equal to zero almost surely by definition
of $a_j$.
If $p(j)$ is initial, then
$$
   \inner{Y - y, \eta} = - (Y_j - y_j)
   =
   - Y_j + a_j y_{p(j)}
   =
   - Y_j + a_j Y_{p(j)}
$$
the last equality being that $Y_{p(j)}$ is a constant random variable
so $Y_{p(j)} = y_{p(j)}$ almost surely,
and this is indeed less than or equal to zero almost surely by definition
of $a_j$.  Thus in either case we have \eqref{eq:dor-lower-bound} is
a direction of recession and
\begin{equation} \label{eq:lower-foo}
   \inner{Y - y, \eta} = - Y_j + a_j Y_{p(j)}
\end{equation}
Taking limits in the direction \eqref{eq:dor-lower-bound} arrives at
the limiting conditional model that conditions
on \eqref{eq:lower-foo} being equal to zero, that is, on the event
$Y_j = a_j Y_{p(j)}$.  By the predecessor-is-sample-size principle
this is the same thing as saying the arrow
$y_{p(j)} \longrightarrow y_j$ has the degenerate family of distributions
concentrated at $a_j$.

The proofs of the assertions about \eqref{eq:dor-upper-bound} are similar.

That the conditions for \eqref{eq:dor-lower-bound}
and \eqref{eq:dor-upper-bound} both hold when $a_j = b_j$ and that
one is then the negative of the other is obvious.
That $\eta$ and $- \eta$ both being
directions of recession implies either is a direction of constancy is
\citet{geyer-gdor} Theorem~3 part (e) and Theorem~1 part (g).

In case $y_{p(j)} = 0$  and $-\infty < a_j < b_j < \infty$
we also have $y_j = 0$ so $y_j = a_j y_{p(j)} = b_j y_{p(j)}$ holds trivially.
Thus we have already proved that both are directions of recession.
In order for \eqref{eq:dor-lower-bound} to be a direction of constancy
we need $Y_j = a_j Y_{p(j)}$ to hold almost surely, but this is false
unless $Y_{p(j)} = 0$ almost surely.
Similarly for \eqref{eq:dor-upper-bound}.
\end{proof}

As we see in the proof, there are two cases.  When $p(j)$ is noninitial
the arrow $y_{p(j)} \longrightarrow y_j$ represents a conditional
distribution and $\eta$ has two nonzero components unless $a_j = 0$ and
$\eta$ is given by \eqref{eq:dor-lower-bound} or $b_j = 0$ and
$\eta$ is given by \eqref{eq:dor-upper-bound}.
When $p(j)$ is initial the arrow $y_{p(j)} \longrightarrow j$ represents,
in effect, a marginal distribution and $\eta$ has one nonzero component.
But the formulas \eqref{eq:dor-lower-bound} and \eqref{eq:dor-upper-bound}
work in either case because the middle case does not occur when
$p(j) \notin J$.

The case $a_j = b_j$ cannot occur in aster models allowed
by R package \code{aster}.  But once we start
taking limits, then they can.  So they are allowed by R package \code{aster2}.

Degenerate distributions concentrated at $a_j$ or $b_j$ are further
discussed in the appropriate section of Appendix~\ref{app:families}
(the details depend on the family the degenerate distribution is derived
from).
R package \code{aster2} implements them.

In the case considered last in the theorem where $y_{p(j)} = 0$ and
\eqref{eq:dor-lower-bound} and \eqref{eq:dor-upper-bound} are both
directions of recession and point in different directions, any nonnegative
combination (linear combination with nonnegative coefficients) of these
two vectors is another direction of recession (any nonnegative combimation
of directions of recession is another direction of recession,
\citealp{geyer-gdor}, Theorem~3).  For example, the vector $\eta$ whose
only nonzero component is $\eta_{p(j)} = a_j - b_j$ is a direction
of recession.

\begin{theorem} \label{th:dor-multinomial}
Suppose $G$ is a multinomial dependence group in an aster graph.
Let $J$ be the set of non-initial nodes of the aster graph,
and let $Y$ denote the response vector and $y$ its observed value.

If $j \in G$ and $y_j = 0$, then the vector $\eta$ having index set $J$
and coordinates
\begin{equation} \label{eq:dor-multinomial}
   \eta_i = \begin{cases} -1, & i = j \\
   0, & \text{otherwise} \end{cases}
\end{equation}
is a direction of recession of the saturated aster model.

This vector is not a direction of constancy
of the saturated aster model unless $Y_{q(G)} = 0$ almost surely.

The vector $\eta$ having index set $J$ and coordinates
\begin{equation} \label{eq:doc-multinomial}
   \eta_i = \begin{cases} -1, & i \in G \\
   +1, & i = q(G) \\
   0, & \text{otherwise} \end{cases}
\end{equation}
is a direction of constancy of the saturated aster model.

Taking the limit in the direction of recession \eqref{eq:dor-multinomial}
gives the LCM that is the same as the OM except the arrow
$y_{p(j)} \longrightarrow y_j$ has the degenerate family of distributions
concentrated at zero, and the conditional distribution for dependence group
$G$ becomes a partially degenerate multinomial distribution that has
$Y_j = 0$ almost surely.
\end{theorem}
\begin{proof}
For \eqref{eq:dor-multinomial} we need to show
that $\inner{Y - y, \eta} \le 0$ almost surely.
This is obvious.
\begin{equation} \label{eq:dor-multinomial-foo}
   \inner{Y - y, \eta} = - (Y_j - y_j)
   =
   - Y_j
\end{equation}
and this is indeed less than or equal to zero almost surely
by definition of the multinomial distribution.

In order for \eqref{eq:dor-multinomial} to be a direction of constancy
we need \eqref{eq:dor-multinomial-foo} to be zero almost surely.
But this is false unless $Y_{q(G)} = 0$ almost surely.

For $\eta$ given by \eqref{eq:doc-multinomial} to be a direction of
constancy we need to show
that $\inner{Y - y, \eta} = 0$ almost surely,
where $Y$ and $y$ are as above.
This too is obvious.
$$
   \inner{Y - y, \eta} = (Y_{q(G)} - y_{q(G)}) - \sum_{j \in G} (Y_j - y_j)
$$
and this is indeed equal to zero almost surely,
by definition of the multinomial distribution.

By Theorem~\ref{th:completion-fundamental}, taking limits in the direction
\eqref{eq:dor-multinomial} results in an LCM that conditions the OM on
the event $Y_j = 0$ almost surely, and this corresponds to
the arrow $y_{p(j)} \longrightarrow y_j$ having the degenerate family
concentrated at zero almost surely.  And this makes the multinomial
distribution of $Y_G$ given $Y_{q(G)}$ partially degenerate.
\end{proof}

As mentioned after the preceeding theorem, any nonnegative combination
of directions of recession is another direction of recession.  This includes
the direction of constancy (any direction of constancy is a direction
of recession, and so is the negative of any direction of constancy).
Hence a vector $\eta$ is a direction of recession described by this theorem
if and only if
\begin{gather*}
    \eta_j < \max_{i \in G} \eta_i \quad \text{implies} \quad y_j = 0
    \\
    \text{$q(G)$ noninitial} \quad \text{implies} \quad 
    \max_{i \in G} \eta_i = - \eta_{q(G)}
\end{gather*}

In case all but one of the components of a multinomial $Y_G$ are zero,
we can apply the theorem repeatedly to get a completely degenerate
multinomial family.  If $j \in G$ and $y_k = 0$ for $k \in G \setminus \{j\}$,
then repeated limits give us the degenerate multinomial family that
conditions on $Y_k = 0$ for $k \in G \setminus \{j\}$, but then we must
also have $Y_j = Y_{q(G)}$ almost surely by definition of the multinomial
distribution.

These partially degenerate multinomial distributions are further
discussed in Section~\ref{app:sec:multinomial} in the appendix.

In case $y_{q(G)} = 0$ but $Y_{q(G)} = 0$ does not hold almost surely,
the theorem says that every $\eta$ whose only nonzero component is
$\eta_j = -1$ is a direction of recession.  Hence any nonnegative
combination of these is a direction of recession.  Hence considering
also \eqref{eq:doc-multinomial} gives a vector $\eta$ having nonzero
components $\eta_G$ in case $q(G)$ is initial and $\eta_{G \cup \{q(G)\}}$
in case $q(G)$ is noninitial, is a direction of recession if and only if
$$
    \text{$q(G)$ noninitial} \quad \text{implies} \quad 
    \max_{i \in G} \eta_i = - \eta_{q(G)}
$$
and this direction of recession is a direction of constancy if and only if
$$
    \eta_i = \eta_j, \qquad i, j \in G
$$

\begin{theorem} \label{th:dor-normal}
Suppose $G = \{j, k\}$ is a normal-location-scale dependence group
in an aster graph.
Let $J$ be the set of non-initial nodes of the aster graph,
and let $Y$ denote the response vector and $y$ its observed value.

If $y_{q(G)} = 1$, then the vector $\eta$ having index set $J$
and coordinates
\begin{equation} \label{eq:dor-normal}
   \eta_i = \begin{cases} 2 y_j, & i = j \\
   -1, & i = k \\
   0, & \text{otherwise} \end{cases}
\end{equation}
is a direction of recession of the saturated aster model.

But this direction of recession does not produce a limiting conditional model
because it corresponds to the case $\Pr_\theta(Y \in H_\delta) = 0$ in
in Theorem~\ref{th:completion-fundamental}.

If $y_{q(G)} \ge 2$, then almost surely there are no directions of recession
for this family.

If $y_{q(G)} = 0$, then every vector of the form \eqref{eq:dor-normal}
is a direction of recession (for all real numbers $y_j$).
These directions of recession are not directions of constancy unless
$Y_{q(G)} = 0$ almost surely.
\end{theorem}
\begin{proof}
The closed curve consisting of points $y_G$ such that $y_k = y_j^2$
supports the conditional distribution of $y_G$ given $y_{q(G)} = 1$.
The function $f : y_j \mapsto y_j^2$ is a convex function.
By the gradient inequality \citep[Theorem~2.13 (b)]{rockafellar-wets}
$$
   f(Y_j) - f(y_j) \ge f'(y_j) (Y_j - y_j)
$$
but this can also be written
$$
   Y_k - y_k \ge 2 y_j (Y_j - y_j)
$$
because $f(y_j) = y_k$ and $f(Y_j) = Y_k$ and $f'(y_j) = 2 y_j$.
And it can also be written
$$
   - \eta_k (Y_k - y_k) \ge \eta_j (Y_j - y_j)
$$
because of the definition of $\eta$ in the theorem statement.
And this says $\inner{Y - y, \eta} \le 0$ so $\eta$ is a direction of
recession.

Since $f$ is strictly convex
\citep[Theorem~2.13 ($\text{a}'$)]{rockafellar-wets},
we have strict inequality in the gradient inequality
\citep[Theorem~2.13 ($\text{b}'$)]{rockafellar-wets} when $Y_j \neq y_j$.
Hence we have \hbox{$\inner{Y - y, \eta} < 0$} almost surely.
But the latter is equivalent to $\Pr_\theta(Y \in H_\delta) = 0$ in
in Theorem~\ref{th:completion-fundamental}.

The (closed) convex support of the family for sample size one is
the set
$$
   C = \set{ y_G : y_k \ge y_j^2 }
$$
The (closed) convex support of the family for sample size $n$ is
is the $n$-fold Minkowski sum of this set, which is $n C$
\citep[Proposition~2.23]{rockafellar-wets}.  Because the distributions
in this family are continuous, the interior of $n C$ actually supports
the family for $n \ge 2$.  Hence (almost surely) it is not possible
to observe data on the boundary of the convex support for $n \ge 2$.

In case $y_{q(g)} = 0$, the convex support of the conditional distribution
of $Y_G$ given $Y_{q(G)}$ is $0 \cdot C = \{ 0 \}$.
Now we have to consider the three-dimensional set of all possible vectors
$Y_{\{j, k, l\}}$ with $l = q(G)$.  Since limits of sequences of normal vectors
are again normal vectors
\citep[Proposition~6.6 and Theorem~6.9]{rockafellar-wets},
normal vectors at the point $0 = (0, 0, 0)$ are those of the form
\eqref{eq:dor-normal} and any nonnegative combinations of such.

In order for such a vector $\eta$ to be a direction of constancy,
we must have $\eta_j Y_j + \eta_k Y_k = 0$ almost surely.
But this is false unless $Y_{q(G)} = 0$ almost surely.
\end{proof}

If $y_{q(G)} \ge 1$, then the theorem gives at most one direction of recession,
and it is not a direction of constancy.
If $y_{q(G)} = 0$, then the theorem gives an infinite number of directions
of recession pointing in different directions.
Our intention with these theorems is to use them to discover generic
directions of recession of (not saturated) unconditional aster models
using repeated linear programming.  But we cannot put an infinite number
of vectors into a linear program.

Moreover, our use of this theorem has to be fundamentally different from
our use of Theorems~\ref{th:dor-arrow} and~\ref{th:dor-multinomial}.
From the latter we discover DOR that lead to LCM in which we find MLE.
From the former we discover DOR that do not lead to LCM, and the MLE
does not exist, and we have to rudely inform users that their models
are no good: you cannot estimate the variance of a normal distribution
from one observation.

Users can avoid these kind of error messages by assuming homoscedastic
errors (just like in linear models).  The way this is done in aster models
is to have the variance node of each normal-location-scale family
have the same parameter.  And the way to do that is to have \code{+ foo}
in the formula, where \code{foo} is the indicator vector of the variance
nodes of all normal-location-scale dependence groups (the $k$ in the theorem)
and no other appearance of \code{foo} in the formula.

Users can have more complicated formulas in which variance differs among
normal-location-scale dependence groups, but then it is the job of the
computer to catch situations in which this leads to directions of recession
described by the theorem.

Fortunately, we can separate these two kinds of problems.
All normal dependence groups must be terminal (so not really
``fortunately'' because this follows from the predecessor-is-sample-size
principle). 
Thus we can trim all of the normal dependence groups off of the aster graph
and still have a possible aster model.
Then we can apply our algorithm (still to be developed in what follows)
to determine whether a GDOR exists and, if so, what the LCM is
(for the model with normal dependence groups trimmed off).

Then we can put the normal dependence groups back, and look for more
directions of recession.  In this, the following theorem helps.

\begin{theorem} \label{th:dor-predecessor-zero}
Suppose $G$ is a dependence group in an aster graph, and $y_{q(G)} = 0$
where $Y$ is the response vector and $y$ its observed value, then the
vector $\eta$ whose only nonzero component is $\eta_{q(G)} = -1$ is a
direction of recession of the saturated aster model.

Taking the limit in this direction gives the LCM that is the same as the
OM except $Y_{q(G)} = 0$ almost surely, and this means the distributions
of all successors, successors of successors, successors of successors of
successors, etc.\ are not identifiable in this LCM.

This direction of recession is not a direction of constancy
unless $Y_{q(G)} = 0$ almost surely.
\end{theorem}
\begin{proof}
We know $q(G)$ is not a terminal node.  It is not an initial node either,
because aster models are required to have nonzero data at initial nodes.
We know from
the predecessor-is-sample-size property that $Y_{q(G)} \ge 0$ is required.
Since $y_{q(G)}$ is at the lower endpoint of the support of $Y_{q(G)}$,
the vector described in the theorem statement is a direction of recession.

If we take limits in this direction, we get the OM conditioned on the event
$Y_{q(G)} = 0$ almost surely, and this implies $Y_j = 0$ almost surely for
all $j \prec q(G)$, where $\prec$ is the transitive closure of the successor
relation.  The cumulant function for this LCM does not depend on any
of the variables $\varphi_j$ for $j \preceq q(G)$.
This can be seen by applying \eqref{eq:cumfun-expfam} to this model.
\begin{align*}
   c(\varphi)
   & =
   c(\varphi^*) + \log\left\{ E_{\varphi^*} \left(
   e^{\inner{Y, \varphi - \varphi^*}} \right) \right\}
   \\
   & =
   c(\varphi^*) + \log\left\{ E_{\varphi^*} \left(
   \prod_{j \in J} e^{Y_j (\varphi_j - \varphi^*_j)}
   \right) \right\}
   \\
   & =
   c(\varphi^*) + \log\left\{ E_{\varphi^*} \left(
   \prod_{\substack{j \in J \\ j \not\preceq q(G)}}
   e^{Y_j (\varphi_j - \varphi^*_j)}
   \right) \right\}
\end{align*}
where $\varphi$ varies and $\varphi^*$ is fixed.
And then \eqref{eq:logl-aster-phi} shows that the log likelihood for the
LCM does not depend on any of these variables either.

This vector is a direction of constancy if and only if $Y_{q(G)} = 0$ almost
surely.
\end{proof}

The vector this theorem finds to be a direction of recession is also found
by theorems preceeding it, but the point of this theorem is that it
applies to any aster model whatsoever, even those having families that
have not been implemented yet.  And the theorem also provides more information
about LCM.

In particular, it tells us that for LCM we found by applying our
(yet to be developed) GDOR algorithm to the model with normal dependence
groups trimmed off, any normal dependence groups having $Y_{q(G)} = 0$
in the LCM can still be ignored: their parameters will be non-identifiable
in the LCM.

\begin{theorem} \label{th:dor-aster}
The set of all directions of recession is a closed convex cone.
The set of all directions of constancy is a vector subspace.
Every direction of constancy is also a direction of recession.
A vector $\delta$ is a direction of constancy if and only if
both $\delta$ and $- \delta$ are directions of recession.

The set of all directions of recession of saturated aster models
having families described by Theorems~\ref{th:dor-arrow},
\ref{th:dor-multinomial}, \ref{th:dor-normal},
and~\ref{th:dor-predecessor-zero}
is the smallest closed convex cone containing all of the directions
of recession described by those theorems.

The set of all directions of constancy of such aster models
is the smallest vector subspace containing all of the directions
of constancy described by those theorems.
\end{theorem}
\begin{proof}
The assertions of the first paragraph are all in Theorems 1 and 3 of
\citet{geyer-gdor} and the discussion surrounding them.

Sections~4.1 and~{4.2} in \citet{geyer-thesis} characterize all possible
limit distributions in an exponential family of distributions.
Theorem~{2.7} in \citet{geyer-thesis} says that all limit distributions
can be obtained by taking iterated straight line limits.
The limit of a product being the product of the limits, when we take a
limit we get a limit in each term of the fundamental factorization of
aster models \eqref{eq:factorize}.
Thus when we have all possible limiting conditional models for each of
the families for each of the dependence groups, we have also gotten
all of the limits for the whole aster model.

Conversely, since the theorems mentioned describe all possible limits
of distributions for dependence groups in the families described by
those theorems, which includes all families currently implemented in
R packages \code{aster} and \code{aster2}, we have discovered all
possible limits.  There are no other directions of recession.
\end{proof}

As the theorem statement only implicitly refers to but the the proof
explicitly says, the theorem does not apply to aster models having
multivariate dependence groups whose families have not been invented yet.
We would need to add theorems about their directions of recession if we
add them to aster models.

Now we need to figure out how to use these theorems when applied to
general unconditional aster models (canonical affine submodels of
the saturated aster model).  The principle is simple.  If $M$ is the
model matrix of a canonical affine submodel, then $\delta$ is a direction
of recession (resp.\ constancy) of that submodel if and only
if $\eta = M \delta$ is a direction of recession (resp.\ constancy)
of the saturated model.

So we revisit the theorems.

In Theorem~\ref{th:dor-arrow} we have either
\eqref{eq:dor-lower-bound} or \eqref{eq:dor-upper-bound} or both or neither
is a direction of recession.  If $a_j = b_j$, then we can take either to
be a direction of constancy and ignore the other.
If the predecessor is zero almost surely (this cannot happen unless the
model we are considering is already an LCM) then any directions of recession
are directions of constancy, but not otherwise.

In Theorem~\ref{th:dor-multinomial} when $y_{q(G)} > 0$ we have one
direction of recession for each $j \in G$ such that $y_j = 0$ and
we also have one direction of constancy for the whole dependence group.
If the predecessor is zero almost surely (this cannot happen unless the
model we are considering is already an LCM) then all directions of recession
are directions of constancy, but not otherwise.

For now we ignore Theorem~\ref{th:dor-normal}.

So that completes our list of directions of recession and constancy.

\begin{theorem} \label{th:dor-aster-explicit}
A vector $\delta$ is a direction of recession
of an unconditional canonical affine submodel
with normal dependence groups trimmed off and
no arrows having degenerate families
if and only if $\eta = M \delta$ has the form
\begin{equation} \label{eq:dor-aster-explicit}
   \eta = \sum_{j \in J_r} e_j \eta_j
\end{equation}
where $J_r$ is the index set for directions of recession
of the saturated model discussed above,
$J_c$ is the subset of $J_r$ indexing directions of constancy, and
the $e_j$ are real numbers satisfying $e_j \ge 0$
for $j \in J_r \setminus J_c$.
\end{theorem}
\begin{proof}
This just makes explicit what Theorem~\ref{th:dor-aster} already says.
\end{proof}

Consider the following linear program having variables $e_j$
for $j \in J_r$ and $\delta_k$ for $k \in K$.
\begin{alignat}{2}
  \text{maximize}   & \ \sum_{j \in J_r \setminus J_c} e_j
  \nonumber
  \\
  \text{subject to} & \ 0 \le e_j \le 1, & \qquad & j \in J_r \setminus J_c
  \label{prog:foo}
  \\
                    & \ M \delta =
  \sum_{j \in J_r} e_j \eta_j
  \nonumber
\end{alignat}
\begin{theorem} \label{th:lin-prog-one}
Linear program \eqref{prog:foo} always has a solution.
Linear program \eqref{prog:foo} has optimal value zero if and only if
there does
not exist a direction of recession that is not a direction of constancy
and the MLE exists in the originally given unconditional aster model.
Otherwise, the optimal value is greater than or equal to one and the
$\delta$ part of the solution is a direction of recession that is not
a direction of constancy.
\end{theorem}
\begin{proof}
The feasible region is nonempty because it always contains the zero vector.
Then solutions exist because the objective function is obviously bounded
on the feasible region.

The optimal value is zero if and only if at the solution $e_j = 0$ for
$j \in J_r \setminus J_c$, in which case the $\delta$ part of the solution
is a direction of constancy of the submodel and $\eta = M \delta$ is a
direction of constancy of the saturated model.  The assertion about
existence of MLE is then Theorem~{4} in \citet{geyer-gdor}.

If there exists any feasible point such that some $e_j$
for $j \in J_r \setminus J_c$ is nonzero, then we can multiply all components
of $\delta$ and all $e_j$ by a strictly positive constant to make
the largest $e_j$ for $j \in J_r \setminus J_c$ equal to one, in which case
the objective function is greater than or equal to one.
Optimizing then only increases the objective function.
The solution then is clearly a direction of recession that is not a
direction of constancy by Theorem~\ref{th:dor-aster-explicit}.
\end{proof}

We are not done yet because we haven't yet in the terminology of
\citet{geyer-gdor} found a \emph{generic} direction of recession (GDOR).
That will be one that has the maximal number of nonzero $e_j$ for
for $j \in J_r \setminus J_c$.  Since any nonnegative combination of
directions of recession is another direction of recession, we can
seek GDOR by modifying our linear program to find DOR with $e_j > 0$
that we haven't found so far.

Let $J^{*}$ be any nonempty subset of $J_r \setminus J_c$,
and consider the following linear program.
\begin{alignat}{2}
  \text{maximize}   & \ \sum_{j \in J^{*}} e_j
  \nonumber
  \\
  \text{subject to} & \ 0 \le e_j \le 1, & \qquad & j \in J^{*}
  \label{prog:foobar}
  \\
                    & \ 0 \le e_j, & \qquad &
                      j \in (J_r \setminus J_c) \setminus J^{*}
  \nonumber
  \\
                    & \ M \delta = \sum_{j \in J_r} e_j \eta_j
  \nonumber
\end{alignat}

Then we iterate
(Algorithm~\ref{alg:unconditional}\ifthenelse{\equal{\arabic{page}}{\pageref{alg:unconditional}}}{).}
{, page~\pageref{alg:unconditional}).}
\begin{algorithm}
\caption{Find GDOR for Unconditional Aster Model}
\label{alg:unconditional}
\begin{tabbing}
Set $J^{*} = J_r \setminus J_c$\\
Set $J^{{*}{*}} = \emptyset$\\
Set $\gamma = 0$\\
\textbf{repeat} \{\\
\qquad \= Solve the linear program \eqref{prog:foobar}\\
\> \textbf{if} (linear program has no solution) \textbf{error}\\
\> \textbf{if} (optimal value is zero) \textbf{break}\\
\> Set $\delta$ to be the $\delta$ part of the solution of the linear program\\
\> Set $\gamma = \gamma + \delta$\\
\> Set $e$ to be the $e$ part of the solution of the linear program\\
\> Set $J^{**} = J^{**} \cup \set{j \in J^{*} : e_j > 0}$\\
\> Set $J^{*} = J^{*} \setminus J^{{*}{*}}$\\
\> \textbf{if} ($J^{*} = \emptyset$) \textbf{break}\\
\}
\end{tabbing}
\end{algorithm}
\begin{theorem} \label{th:lin-prog-two}
Algorithm~\ref{alg:unconditional} always terminates, and $\gamma$ is
a generic direction of recession unless $\gamma = 0$, in which case
the MLE exists in the originally given unconditional aster model.
\end{theorem}
\begin{proof}
The algorithm must terminate because $J^{*}$ decreases in each iteration.
So we terminate when $J^{*} = \emptyset$ if not before.

Since each $\delta$ found is a direction of recession that is not
a direction of constancy, so is $\gamma$.

The termination condition of optimal value zero or $J^{{*}{*}} = \emptyset$,
proves that $\gamma$ is such that $\eta = M \gamma$ has the most possible
nonzero components $\eta_j$ for $j \in J_r \setminus J_c$.  Hence it
is generic unless $\gamma = 0$ and the algorithm proves the MLE exists
in the OM.
\end{proof}

From now on we only use the LCM corresponding to the GDOR found.
This means every $\eta_j$ for $j \in J^{{*}{*}}$ found by the algorithm
is a direction of constancy of this LCM.  The other DOR remain the same.

Now we add back in all of the normal dependence groups.
Normal-location arrows are covered by Theorem~\ref{th:dor-arrow}
with $a_j = - \infty$ and $b_j = + \infty$.  They can never have
directions of recession.  So that leaves normal-location-scale dependence
groups.

From Theorem~\ref{th:dor-normal} we know that those with $y_{q(G)} \ge 2$
have no directions of recession to add to our problem.
From Theorem~\ref{th:dor-predecessor-zero} we know that
those with $y_{q(G)} = 0$ have non-identifiable parameters in the LCM.
If $G = \{ j, k \}$ then we add a vector whose only nonzero component
is $\eta_j = 1$ to the list of directions of constancy and also a vector
whose only nonzero component
is $\eta_k = 1$ to the list of directions of constancy.
Finally, from Theorem~\ref{th:dor-normal} we know that those
with $y_{q(G)} = 1$ have exactly one direction of recession that is not
a direction of constancy given by \eqref{eq:dor-normal}.
There are no further directions of recession to add to our problem.

So we throw all of this back into linear program \eqref{prog:foo}.
It had better have optimal value zero.
Otherwise users get rude error messages.

\section{Conditional Aster Models}

Saturated aster models are regular full exponential families.
Unconditional canonical affine submodels of aster models are
regular full exponential families.
So the theory in this chapter up to now applies to them.

Conditional canonical affine submodels of aster models are not
regular full exponential families.
As smooth submodels of saturated aster models, they are what the
jargon calls curved exponential families.
But that does not tell us much about existence or non-existence of MLE.
We know that all possible limits have to be limits of distributions
in the saturated model (because it is a submodel of the saturated model).
But when those limits are MLE is something for which there is no general
theory for curved exponential families.

\subsection{Associated Independence Models}

A cheap trick, however, does crack the problem of conditional aster models.
This is the notion of associated independence models
(Section~\ref{sec:conditional-aster-model-mle} above).
For reference, we repeat \eqref{eq:logl-aster-theta-tricky} above
\begin{equation} \label{eq:logl-aster-theta-tricky-duplicate}
   l(\theta)
   =
   \sum_{G \in \mathcal{G}}
   \left[ \inner{y_G, \theta_G} - n_{q(G)} c_G(\theta_G) \right]
\end{equation}
(so this equation now has two equation numbers, one here and one there).

The actual conditional model has (saturated model) log likelihood
that is \eqref{eq:logl-aster-theta-tricky-duplicate}
with $n_{q(G)}$ replaced by $y_{q(G)}$.
The associated independence model (AIM) has (saturated model) log likelihood
that is \eqref{eq:logl-aster-theta-tricky-duplicate}
with $n_{q(G)}$ constant and $y_j$ random.

As Section~\ref{sec:conditional-aster-model-mle} above says,
the AIM makes no sense when considered statistically, probabilistically,
because it pretends that variables that are actually the same
($n_{q(G)}$ and $y_{q(G)}$) are different, and one is constant
and the other random.  But,
as Section~\ref{sec:conditional-aster-model-mle} above also says,
the AIM makes perfect sense when considered numerically, algebraically
when we are considering maximum likelihood estimation.
Then $n_{q(G)}$ and $y_{q(G)}$ are just numbers, fixed at their observed
values, and if we use different notation for the same number in different
parts of the expression, that is OK.

In Section~\ref{sec:conditional-aster-model-mle}
we used the AIM to reach the conclusion that the log likelihood of
a conditional aster model is concave, something that is not generally
true of curved exponential family models.

In this section,
we will use the AIM to completely characterize existence and uniqueness
of MLE for conditional aster models and directions of recession and constancy
of \eqref{eq:logl-aster-theta-tricky-duplicate} and for its canonical
affine submodels (conditional aster models).

What puts the I (for independence) in AIM is that the AIM makes
the $Y_G$ for $G \in \mathcal{G}$ independent random vectors.
This makes the AIM much easier to reason about than the actual conditional
aster model.

So now we repeat the preceding section, \emph{mutatis mutandis} reasoning
about AIM rather than unconditional aster models.

\begin{theorem} \label{th:dor-aim-arrow}
Suppose $\{j\}$ is a univariate dependence group in an AIM, and
the one-parameter exponential family of distributions for the arrow
$n_{p(j)} \longrightarrow y_j$ has closed convex support that is an
interval with endpoints $a_j$ and $b_j$ (either of which may be infinite and
which satisfy $a_j \le b_j$ with equality possible, in which case this
distribution is concentrated at one point).  Let $J$
be the set of non-initial nodes of the aster graph,
let $Y$ denote the response vector and $y$ its observed value,
and let $n$ denote the vector of sample sizes whose components are $n_j$.

If $n_{p(j)} = 0$ or $a_j = b_j$,
then the vector $\eta$ whose only nonzero component
is $\eta_j = 1$ is a direction of constancy
of \eqref{eq:logl-aster-theta-tricky-duplicate}.

If $n_{p(j)} > 0$ and $a_j < b_j$ and $y_j = a_j n_{p(j)}$,
then the vector $\eta$ whose only nonzero component
is $\eta_j = -1$ is a direction of recession
that is not a direction of constancy
of \eqref{eq:logl-aster-theta-tricky-duplicate}.

Taking the limit in this direction of recession
gives the LCM that is the same as the OM except the arrow
$n_{p(j)} \longrightarrow y_j$ has the degenerate family of distributions
concentrated at $a_j$.
This LCM is the OM (of the AIM) conditioned on the event $Y_j = a_j n_{p(j)}$.

If $n_{p(j)} > 0$ and $a_j < b_j$ and $y_j = b_j n_{p(j)}$,
then the vector $\eta$ whose only nonzero component
is $\eta_j = 1$ is a direction of recession
that is not a direction of constancy
of \eqref{eq:logl-aster-theta-tricky-duplicate}.

Taking the limit in this direction of recession
gives the LCM that is the same as the OM except the arrow
$n_{p(j)} \longrightarrow y_j$ has the degenerate family of distributions
concentrated at $b_j$.
This LCM is the OM (of the AIM) conditioned on the event $Y_j = b_j n_{p(j)}$.
\end{theorem}

\begin{theorem} \label{th:dor-aim-multinomial}
Suppose $G$ is a multinomial dependence group in an AIM.
Let $J$ be the set of non-initial nodes of the aster graph,
let $Y$ denote the response vector and $y$ its observed value,
and let $n$ denote the vector of sample sizes whose components are $n_j$.

If $n_{p(j)} = 0$ then the vector $\eta$ whose only nonzero component
is $\eta_j = 1$ is a direction of constancy
of \eqref{eq:logl-aster-theta-tricky-duplicate},
and this is true for each $j \in G$.

If $n_{p(j)} > 0$ and $y_j = 0$,
then the vector $\eta$ whose only nonzero component
is $\eta_j = -1$ is a direction of recession
that is not a direction of constancy
of \eqref{eq:logl-aster-theta-tricky-duplicate},
and this is true for each $j \in G$.

Taking the limit in any of these directions of recession,
say the one having $\eta_j$ nonzero,
gives the LCM that is the same as the OM (of the AIM) except the arrow
$n_{p(j)} \longrightarrow y_j$ has
the degenerate family of distributions
concentrated at zero.

The vector $\eta$ having index set $J$ and coordinates
\begin{equation} \label{eq:doc-aim-multinomial}
   \eta_i = \begin{cases} 1, & i \in G \\
   0, & \text{otherwise} \end{cases}
\end{equation}
is a direction of constancy of
of \eqref{eq:logl-aster-theta-tricky-duplicate}.
\end{theorem}

As in the discussion following Theorem~\ref{th:dor-multinomial}
(which this theorem duplicates \emph{mutatis mutandis}),
we note that any nonnegative combination of directions of recession is
another direction of recession.  Hence the directions of recession
described by this theorem are vectors $\eta$ whose only nonzero components
are in the subvector $\eta_G$ and such a vector is a direction of recession
of \eqref{eq:logl-aster-theta-tricky-duplicate} if
$$
   \eta_j < \max_{i \in G} \eta_i \quad \text{implies} \quad y_j = 0
$$
and such a vector is a direction of constancy
of \eqref{eq:logl-aster-theta-tricky-duplicate} if $n_{q(G)} = 0$ or if
its nonzero components are all the same.

\begin{theorem} \label{th:dor-aim-normal}
Suppose $G = \{j, k\}$ is a normal-location-scale dependence group
in an AIM.
Let $J$ be the set of non-initial nodes of the aster graph,
let $Y$ denote the response vector and $y$ its observed value,
and let $n$ denote the vector of sample sizes whose components are $n_j$.

If $n_{q(G)} = 0$, then any vector $\eta$ whose only nonzero components
are $\eta_j$ and $\eta_k$ is a direction of constancy of
of \eqref{eq:logl-aster-theta-tricky-duplicate}.

If $n_{q(G)} = 1$, then \eqref{eq:dor-normal} is a direction of recession
of \eqref{eq:logl-aster-theta-tricky-duplicate}
that is not a direction of constancy.

But this direction of recession does not produce a limiting conditional model
because it corresponds to the case $\Pr_\theta(Y \in H_\delta) = 0$ in
in Theorem~\ref{th:completion-fundamental}.

If $n_{q(G)} \ge 2$, then, almost surely,
there are no directions of recession for this family.
\end{theorem}

\begin{theorem} \label{th:dor-aim}
In addition to the general properties of directions of recession
and constancy found in Theorem~\ref{th:dor-aster},
the set of all directions of recession of AIM
having families described by Theorems~\ref{th:dor-aim-arrow},
\ref{th:dor-aim-multinomial}, and~\ref{th:dor-aim-normal}
is the smallest closed convex cone containing all of the directions
of recession described by those theorems.

The set of all directions of constancy of such aster models
is the smallest vector subspace containing all of the directions
of constancy described by those theorems.
\end{theorem}

\begin{theorem} \label{th:dor-aim-explicit}
For an AIM with normal dependence groups trimmed off define
\begin{align*}
   J_\text{\normalfont up}
   & =
   \set{ j \in J : y_j = a_j n_{p(j)} }
   \\
   J_\text{\normalfont dn}
   & =
   \set{ j \in J : y_j = b_j n_{p(j)} }
\end{align*}
where these include multinomial dependence groups with the convention
$a_j = 0$ and $b_j = \infty$ for them.
Then a vector $\eta$ is a direction of recession of this model if
\begin{alignat*}{2}
   \eta_j & \le 0, & \qquad &
   j \in J_\text{\normalfont dn} \setminus J_\text{\normalfont up}
   \\
   \eta_j & \ge 0, & \qquad &
   j \in J_\text{\normalfont up} \setminus J_\text{\normalfont dn}
\end{alignat*}
Conversely, any direction of recession of this model satisfies these
conditions if we modify it by subtracting $\max(\eta_G)$ from the elements
of $\eta_G$ for each multinomial dependence group $G$.
\end{theorem}



\chapter{Subsampling} \label{ch:subsampling}

\section{Introduction}

In aster models the ideal is to actually measure
all components of fitness and make them nodes in the graphical model.
Sometimes, however, it is just too much work to count all of some component
of fitness, for example, all seeds produced by a plant.

For a part of an aster graph
$$
\begin{CD}
   \cdots @>>> y_\text{this} @>>> y_\text{that} @>>> \cdots
\end{CD}
$$
suppose that $y_\text{that}$ is typically too large to count
(by available methods, in available time).

The conditional distribution of $y_\text{that}$ given $y_\text{this}$ is
the sum of $y_\text{this}$
independent and identically distributed (IID) random variables.
This is the predecessor-is-sample-size property
(Section~\ref{sec:piss} above).
In short, $y_\text{that}$ is a random ``sample''
from some ``population'' and the sample size is $y_\text{this}$, where the scare
quotes are to indicate that ``sample'' and ``population'' don't refer
to an actual sample and population but are just a way of discussing
probability that is common in introductory statistics books, which
take finite population sampling (actually taking a random sample from
a known finite population) as an analogy for all applications of probability
theory; any IID set of random variables is called a ``sample'' from
a ``population'' whether or not that makes literal sense.

The obvious solution to our problem is to ``subsample'' the ``sample.''
Take a random sample of the things $y_\text{this}$ counts,
and for that subsample count how many of whatever component
of fitness $y_\text{that}$ counts (this proposal will be generalized
in Section~\ref{sec:generalize} below).  In this case,
\citet[p.~E43]{aster2} proposed to simply insert an extra arrow in the
graph to represent the subsampling process.
Taking a random subsample is a Bernoulli process (flip a ``biased coin''
to decide for each of the $y_\text{this}$ things whether it goes in the
subsample).
So this is a Bernoulli arrow, but we mark it specially as a subsampling arrow
$$
\begin{CD}
   \cdots @>>> y_\text{this} @>\text{samp}>> y_\text{this-sub}
   @>>> y_\text{that-sub} @>>> \cdots .
\end{CD}
$$
Here $y_\text{this}$ is the same variable it was before (the observed count for
``this'' fitness component), $y_\text{this-sub}$ is the subsample size
(the subset of the $y_\text{this}$ things that go in the subsample),
and $y_\text{that-sub}$ is the observed count for ``that'' fitness component
for the subsample).  We no longer observe $y_\text{that}$, which
is what we would have observed if we had not subsampled.  Not having to
count $y_\text{that}$ was the whole point of the subsampling.

\citet{aster2} further proposed to treat aster models with subsampling
just like any other aster model.  This suggestion was backed
up by Section~8 of a supporting technical report \citep{tr661}.
That technical report, however, notes
% that because of the way the aster transform, the map from
%the conditional canonical parameter vector $\theta$ to the unconditional
%canonical parameter vector $\varphi$ \citep[eq.~(5)]{aster1}, works,
this suggestion is not quite the right thing.  It says
\begin{quote}
The somewhat odd thing about this proposal is that the parameter
$p$ [the subsampling probability] is \emph{known} and is a \emph{conditional}
mean value parameter, but we intend to use an \emph{unconditional} aster model
and treat the [corresponding] unconditional canonical parameter as
\emph{unknown} [emphasis in the original].
\end{quote}
and devotes the rest of its Section~8 to a simulation study that shows
that, although, not quite the right thing, it does well enough.

\citet*[Appendix~S1]{stanton-geddes-et-al} show how to, in effect, remove
the effect of subsampling when producing point estimates and confidence
intervals for expected fitness, completing something left undone by
\citet{aster2} and the accompanying technical report.

Here we give a new proposal that does the right thing with subsampling
arrows and hence supersedes all earlier proposals.
We have to realize that aster models with subsampling are no longer
regular full exponential families, so they no longer satisfy the original
rationale for aster models.

With subsampling there are two models we have to consider: the model with
subsampling, which reflects the experiment actually done, and the same model
with subsampling removed, which reflects biology of the organisms being
studied.  We give both of these models the same parameterization (subsampling
arrows in the aster graph have no unknown parameters because the subsampling
probabilities are known).  Once this fundamental realization is made,
everything else about aster models with subsampling follows from well
known likelihood theory.

\section{Subsampling}

\subsection{Graphs With and Without Subsampling}
\label{sec:with-and-without}

We already have one two-way classification of aster graphs: the full aster
graph and graphs for ``individuals'' in scare quotes
(Section~\ref{sec:scare-quotes} above).  Now we introduce a different
two-way classification: with and without subsampling.  Together these give
us a four-way classification.

When using aster models with subsampling, we are also interested in the graph
and the corresponding aster model if subsampling had not been done.
The graph and model with subsampling represent the experiment actually done.
The graph and model without subsampling represent the biology.
We must refer to both in our discussion.  For example, we use the graph
and model with subsampling to estimate parameters, but we use the graph
and model without subsampling to predict biological properties of
organisms from these estimates.

Understanding the correspondence between the two graphs is helped by
referring to the following picture, which is part of an aster graph
with subsampling.
\begin{equation} \label{graph:with}
\begin{CD}
   \cdots @>>> y_i @>>> y_j @>\text{samp}>> y_k @>>> y_m @>>> \cdots
\end{CD}
\end{equation}
Only one subsampling arrow is shown (labeled ``samp'').  The other arrows
are non-subsampling.  Nodes at the head of subsampling arrows (here $y_k$)
are called \emph{subsampling nodes}.  Thus $y_k$ is the only subsampling node
among the nodes shown.
Here $y_j$ is the count of some
sort of thing (flowers, seeds, etc.)\ of actual individuals in the experiment,
and $y_k$ is the count of the same thing for a random subsample of those
individuals who are
carried forward to later stages of the experiment.
Thus $y_j$ and $y_k$ are both measurements of
the \emph{same} component of fitness.
The relationship between
$y_j$ and $y_k$ is artificial (done by the experimenters) and has nothing
to do with biology.

The corresponding graph without subsampling is formed by removing the
subsampling arrow and subsampling node and pasting together the graph
so no break is formed
\begin{equation} \label{graph:without}
\begin{CD}
   \cdots @>>> y_i @>>> y_j @>>> y_m @>>> \cdots
\end{CD}
\end{equation}
This graph corresponds to the experiment that would have been done if there
were no subsampling.

Here is a more complicated example that illustrates that sometimes
it may be necessary to have subsampling arrows following each other.
\begin{equation} \label{graph:with-with}
\begin{tikzcd}[row sep=tiny]
   \hphantom{y_0} & \hphantom{y_1} & \hphantom{y_2} & y_3 \arrow{r} & y_5
   \\
   y_0 \arrow{r} & y_1 \arrow{r}{\text{samp}}
   & y_2 \arrow{ru}{\text{samp}} \arrow{rd}
   \\
   \hphantom{y_0} & \hphantom{y_1} & \hphantom{y_2} & y_4 \arrow{r} & y_6
\end{tikzcd}
\end{equation}
This is the graph for one ``individual'' assuming all ``individuals'' have
isomorphic subgraphs.
Here is the corresponding subgraph when we remove the subsampling arrows.
\begin{equation} \label{graph:with-with-stripped}
\begin{tikzcd}[row sep=tiny]
   \hphantom{y_0} & \hphantom{y_1} & y_5
   \\
   y_0 \arrow{r} & y_1 \arrow{ru} \arrow{rd}
   \\
   \hphantom{y_0} & \hphantom{y_1} & y_4 \arrow{r} & y_6
\end{tikzcd}
\end{equation}

%%%%% NEED FORWARD REFERENCE to theory about how to go (in general)
%%%%% from with to without

\subsection{Notation for Graphs With and Without Subsampling}
\label{sec:with-and-without-notation}

We are going to use mathematical notation that distinguishes analogous
concepts for the full graphs with and without subsampling by decorating
notation for the former with stars.

\subsubsection{Sets of Nodes}

The set of nodes of the full aster graph with subsampling is denoted $N^*$.
The set of non-subsampling nodes in $N^*$ is denoted $N$.  This is the set
of nodes of the full graph without subsampling.

The set of non-initial nodes in $N^*$ is denoted $J^*$.
The set of non-initial nodes in $N$ is denoted $J$.
($J = N \cap J^*$.)

\subsubsection{Families of Sets of Nodes}

The set of dependence groups
(Section~\ref{sec:factorization} above)
for the aster model with subsampling is denoted $\mathcal{G}^*$.
The set of dependence groups
for the aster model without subsampling is denoted $\mathcal{G}$.

Subsampling nodes are always dependence groups by themselves.
No dependence group with more than one node can have any subsampling nodes.
Thus the set of all subsampling nodes is $J^* \setminus J$.

\subsubsection{Predecessor Functions}

We already have two kinds of predecessor functions:
set-to-index predecessor functions denoted $q$
(Section~\ref{sec:factorization} above) and
index-to-index predecessor functions denoted $p$
(Section~\ref{sec:other} above).
Now we have another two-way classification.
We will have these functions with stars to denote with subsampling
and without stars to denote without subsampling.

For example, in \eqref{graph:with} we have $p^*(m) = k$ but $p(m) = j$.
The latter agrees with \eqref{graph:without}, as it must.

For another example, in \eqref{graph:with-with} we have
\begin{align*}
   p^*(6) & = 4
   \\
   p^*(5) & = 3
   \\
   p^*(4) & = 2
   \\
   p^*(3) & = 2
   \\
   p^*(2) & = 1
   \\
   p^*(1) & = 0
\end{align*}
but
\begin{align*}
   p(6) & = 4
   \\
   p(5) & = 1
   \\
   p(4) & = 1
   \\
   p(1) & = 0
\end{align*}
The latter agrees with \eqref{graph:with-with-stripped}, as it must.

We have not given examples of graphs with subsampling and dependence
groups, but the set-to-index predecessor functions work the same way
\begin{alignat*}{2}
   p(j) & = q(G), & \qquad & j \in G \in \mathcal{G}
   \\
   p^*(j) & = q^*(G), & & j \in G \in \mathcal{G}^*
\end{alignat*}

\subsubsection{Partial Orders}

In Section~\ref{sec:closure} above we introduced partial orders on the
node set of the graph that tell us about predecessors, predecessors of
predecessors, and so forth.  Unlike the case with the other mathematical
objects just discussed, we do not need starred and unstarred versions
for these.

As in Section~\ref{sec:closure} above, let $\succ$ denote the transitive
closure of the predecessor relation on $N^*$.  Then the transitive
closure of the predecessor relation on $N$ is just the same relation
as on $N^*$ but restricted to $N$, that is $j \succ k$ in $N$ if and only
$j \succ k$ when $j$ and $k$ are considered elements of $N^*$.

And similarly for $\succeq$, $\prec$, and $\preceq$.

\subsubsection{Going from One to the Other}

The user specifies the graph with subsampling.  The computer should figure
out the corresponding graph without subsampling (so no mistakes about that
are made).

For this we use the notation $p^k$ for $k$-fold composition of a function
with itself from dynamical systems theory that is also explained in
Section~\ref{sec:closure} above.

Suppose we are trying to determine $p(j)$ for $j \in J$.
Define
$$
   m = \min \set{ k > 0 : (p^*)^k(j) \in N }
$$
Then $p(j) = (p^*)^m(j)$.

In words, we go back in the graph with subsampling looking at the predecessor,
predecessor of predecessor, and so forth until we find one that is not a
subsampling node, and that is the predecessor in the graph without subsampling.

\subsection{Models With and Without Subsampling}
\label{sec:models-with-and-without}

Now that we have the relationship between the graph with and without
subsampling, and hence the factorization (Section~\ref{sec:factorization}
above) with and without subsampling we need to consider statistical models
that go with these factorizations.  Models without subsampling are those
already described (Chapter~\ref{ch:introduction} above).

So we have to describe models with subsampling here.
There are two key ideas.
\begin{itemize}
\item The conditional distributions for subsampling arrows are considered
\emph{known}.
Thus they have \emph{no} unknown parameter values to estimate.
\item The conditional distributions for non-subsampling arrows should be
the same for the models with and without subsampling.  They should have
the same statistical models parameterized in the same way
(this statement is imprecise; it depends on which parameters
we are talking about).
\end{itemize}

\subsubsection{Data}

First we introduce new notation for data (not used above) to distinguish
the models with and without subsampling.  Let $y^*$ denote the response
vector for the model with subsampling and $y$ the response
vector for the model without subsampling.

\subsubsection{Factorization}

Then the model with subsampling factorizes as
\begin{equation} \label{eq:factorization-with}
   f^*(y^*) = \prod_{G \in \mathcal{G}^*} f^*_G(y^*_G \mid y^*_{q^*(G)})
\end{equation}
and the model without subsampling factorizes as
\begin{equation} \label{eq:factorization-without}
   f(y) = \prod_{G \in \mathcal{G}} f_G(y_G \mid y_{q(G)}).
\end{equation}
Now our first principle above says that
$f^*_G(\fatdot \mid \fatdot)$ has no parameters,
for $G \notin \mathcal{G}$,
and our second principle above says that
$$
   f^*_{G, \theta_G}(\fatdot \mid \fatdot)
   =
   f_{G, \theta_G}(\fatdot \mid \fatdot),
   \qquad
   G \in \mathcal{G}
$$
that is, these are the same conditional distributions (the same functions
of the variables on each side of the vertical bar) for the same parameter
values.

\subsubsection{Parameterization}

This also makes clear what parameters we are talking about.
The \emph{conditional canonical parameter vector} $\theta$
(Section~\ref{sec:aster-transform} above)
is the same for the models with and without subsampling.
But we also want the aster transform to be the same
(also Section~\ref{sec:aster-transform} above), so
the \emph{unconditional canonical parameter vector} $\varphi$ will also be
the same for the models with and without subsampling.

\subsubsection{Mean Value Parameters}

The \emph{conditional mean value parameter vector} $\xi$ will differ
for the models with and without subsampling simply because the model
with subsampling has more arrows.  The components of $\xi$ will be the
same for the same arrows
(Sections~\ref{sec:conditional-and-unconditional-mean-values}
and~\ref{sec:aster-mean-value-parameters} above), that is, defining
\begin{align*}
   \xi_j & = E(y_j \mid y_{p(j)} = 1)
   \\
   \xi^*_j & = E(y_j^* \mid y^*_{p^*(j)} = 1)
\end{align*}
we have
$$
   \xi_j = \xi^*_j = \frac{\partial c_G(\theta_G)}{\partial \theta_j}
$$
whenever $j$ is not a subsampling node and $j \in G \in \mathcal{G}$.
But when $j$ is a subsampling node there is no $\xi_j$ since there is
no node $j$ in the model without subsampling, and $\xi^*_j$ is not a function
of the parameters of the model ($\theta$ and $\varphi$); it is not an unknown
parameter but rather a known constant (this will be revisited in the
next section).

Thus when we define
\begin{align*}
   \mu & = E(y)
   \\
   \mu^* & = E(y^*)
\end{align*}
we will still have the relationship between $\mu$ and $\xi$ discussed
in Section~\ref{sec:mu-and-xi} above.
And we will have the analogous equations with stars for the relationship
between $\mu^*$ and $\xi^*$ but the relations between $\mu$ and $\mu^*$
will be very complicated and depend on the whole aster graph.
But everything in this section depends on subsampling having the
predecessor-is-sample-size property, which we drop in the next section.
So nothing in this section holds for general aster models with subsampling.

\subsection{Generalizing Our Notion of Subsampling} \label{sec:generalize}

So far, we have been assuming the conditional distributions for the
subsampling arrows obey the predecessor-is-sample-size principle.  This
means the conditional distribution for sample size one is Bernoulli,
so subsampling arrows are Bernoulli arrows (but ones whose conditional
canonical parameters are fixed and known rather than unknown parameters
to be estimated).  And it means the conditional distribution of $y_j^*$
given $y_{p^*(j)}^*$ is binomial (when $j$ is a subsampling node).
Let $\xi_j^*$ be the usual parameter for the binomial distribution,
which is also the mean value parameter for the Bernoulli distribution.
Then, defining $\mu^* = E(y^*)$, we have
the analog of \eqref{eq:mu-and-xi} with stars
\begin{equation} \label{eq:mu-and-xi-star}
   \mu_j^* = \xi_j^* \mu_{p(j)}^*, \qquad j \in J^*.
\end{equation}
and all of the consequences \eqref{eq:mu-and-xi} found in
Section~\ref{sec:mu-and-xi} above, except with stars in the appropriate places.

But when we drop the predecessor-is-sample-size principle for subsampling
arrows \eqref{eq:mu-and-xi-star} no longer holds, and there is no longer
any notion of $\xi^*_j$ for such arrows analogous to non-subsampling
arrows.  We still do have conditional mean values,
$$
   E\bigl\{ y_j^* \mid y_{p^*(j)}^*\bigr\}
$$
but these do not need to satisfy
$$
   E\bigl\{ y_j^* \mid y_{p^*(j)}^*\bigr\}
   =
   \xi_j^* y_{p^*(j)}^*
$$
for any constant $\xi_j^*$ when $j$ is a subsampling node that does not
obey the predecessor-is-sample-size principle, but rather are arbitrary
functions of $y_{p^*(j)}^*$.

This means that conditional mean value parameters
for subsampling arrows make no sense for subsampling arrows that are not
Bernoulli (do not obey the predecessor-is-sample-size principle).
Since many biologists use forms of subsampling that are not Bernoulli
(not simple random sample), we do not want to enforce the Bernoulli
assumption.

All we assume about subsampling distributions is that they are known,
having no unknown parameters to estimate.
We make no other assumptions about them.

We do, of course, have an unconditional mean value parameter vector $\mu^*$
with components
\begin{equation} \label{eq:mu-star}
   \mu^*_j = E(Y^*_j), \qquad j \in J^*
\end{equation}
but since the computer knows nothing about the subsampling distributions
(they can be any distributions), the computer will be unable to compute
them.  If users want to use $\mu^*$ somehow, they will have to provide
it themselves.

Thus $\xi^*$ is undefined, in general, and $\mu^*$, although defined,
is no longer anything the computer can deal with.

Fortunately $\xi^*$ and $\mu^*$ are unbiological.  So users will mostly,
perhaps always, only be interested in $\xi$ and $\mu$, which the computer
can deal with.

\subsection{Log Likelihood}
\label{sec:logl}

When there is no subsampling, the saturated aster model log likelihood
for $\theta$ is given by \eqref{eq:logl-aster-theta} which can be rewritten
\begin{equation} \label{eq:logl}
   l(\theta)
   =
   \sum_{j \in J} y_j \theta_j
   - \sum_{G \in \mathcal{G}} y_{q(G)} c_G(\theta_G).
\end{equation}

\begin{theorem} \label{th:logl-theta}
The log likelihood for a model with subsampling is given by
\begin{equation} \label{eq:logl-with}
   l(\theta)
   =
   \sum_{j \in J} y^*_j \theta_j
   - \sum_{G \in \mathcal{G}} y^*_{q^*(G)} c_G(\theta_G).
\end{equation}
\end{theorem}
In \eqref{eq:logl} and \eqref{eq:logl-with}
$c_G$ is the cumulant function for the exponential family
for dependence group $G$.
\begin{proof}
To be clear, we write down the log likelihood for all the data
using the notation of Section~\ref{sec:generalize} above
$$
   l(\theta^*)
   =
   \sum_{j \in J} y_j^* \theta_j
   - \sum_{G \in \mathcal{G}} y_{q^*(G)}^* c_G(\theta_G)
   +
   \sum_{j \in J^* \setminus J} \log f_j(y_j^* \mid y_{p^*(j)}^*)
$$
and we are allowed to drop terms that do not contain unknown parameters from the
log likelihood because this makes no difference to either frequentist or
Bayesian inference.  This gives \eqref{eq:logl-with}.
\end{proof}

The cumulant function $c_G$ satisfies
\begin{equation} \label{eq:theta-to-xi}
\begin{split}
   \frac{\partial c_G(\theta_G)}{\partial \theta_j}
   & =
   E_\theta(y_j \mid y_{q(G)} = 1)
   \\
   & =
   E_\theta(y^*_j \mid y_{q^*(G)} = 1)
   \\
   & =
   \xi_j
   \\
   & =
   \xi^*_j,
   \qquad j \in G \in \mathcal{G},
\end{split}
\end{equation}
and
\begin{equation} \label{eq:gamma-pre}
\begin{split}
   \frac{\partial^2 c_G(\theta_G)}{\partial \theta_j \partial \theta_k}
   & =
   \cov_\theta(y_j, y_k \mid y_{q(G)} = 1)
   \\
   & =
   \cov_\theta(y^*_j, y^*_k \mid y^*_{q^*(G)} = 1),
   \qquad j, k \in G \in \mathcal{G},
\end{split}
\end{equation}
and these derivatives are zero if $j \notin G$ or (in the latter)
$k \notin G$.

If the conditioning event in these equations has probability
zero (which happens in actual aster models if the family in question is
$k$-truncated with $k > 0$, \citealp{aster2}, have an example), then the
conditional expectations are not well defined, but we still have
\begin{equation} \label{eq:xi-from-c}
   \frac{\partial c_G(\theta_G)}{\partial \theta_j}
   =
   \xi_j
   =
   \xi^*_j
\end{equation}
with $\xi_j$ being defined by the more long winded and careful
definition given in Section~\ref{sec:xi} above when the conditioning
event in \eqref{eq:theta-to-xi} has probability zero.

The reason for the equality of starred and unstarred quantities is that
we want the model without subsampling to be the same as the model
with subsampling when the subsampling arrows are removed as explained
in Section~\ref{sec:with-and-without} above and in this section.
The distribution of $y_G$ given $y_{q(G)} = n$ is the same as
the distribution of $y^*_G$ given $y^*_{q^*(G)} = n$.

There is a similar adjustment to be made for the conditional covariances above
when their conditioning events have probability zero.
We still have that $y_G$ is the sum of $y_{q(G)}$ IID random vectors
and $\partial^2 c_G(\theta_G) / \partial \theta_j \partial \theta_k$
is the unconditional covariance of the $j$ and $k$ components of one
of those random vectors.

\subsection{Aster Transform}
\label{sec:aster-transform-subsampling}

We have the aster transform and inverse aster transform
(Section~\ref{sec:aster-transform}) and these hold for models with
subsampling because they are the same (we assume) as for models
without subsampling.  And these determine $\mu$ and $\xi$ as discussed
in the preceding section.  And we are generally uninterested in
$\mu^*$ and $\xi^*$ as discussed in the preceding section.  Thus
we only have parameters without stars.  We have aster graphs with
stars and aster data with stars, but not parameters.

\subsection{Canonical Affine Submodels}

As is the situation without subsampling, we are interested in
canonical affine submodels (Section~\ref{sec:canonical-affine-submodels}
above) when we have subsampling.  And we want them to be the same models
with the same parameterizations with and without subsampling.  Thus they
are the same as in Section~\ref{sec:canonical-affine-submodels} above.
Unconditional canonical affine submodels make $\varphi$ an affine function
of the submodel parameters $\beta$.
Conditional canonical affine submodels make $\theta$ an affine function
of the submodel parameters $\beta$.

\subsubsection{Conditional}

With subsampling, a conditional aster model still has a concave log likelihood
for the reasons discussed in Section~\ref{sec:conditional-aster-model-mle}
above.  The saturated model log likelihood \eqref{eq:logl} is a sum
of linear and concave functions of $\theta$.  Therefore the submodel
log likelihood is the composition of a concave function and an affine
function, which is again a convex function.  This means the MLE will
be easily found by the computer (by any algorithm that always checks
that it goes uphill on the likelihood in every iteration).  It does
not mean that conditional aster models have any other properties of
regular full exponential families (but this is true with or without
subsampling, as was discussed in Chapter~\ref{ch:introduction}).

\subsubsection{Unconditional}

With subsampling, an unconditional aster model is no longer an exponential
family and does not have any full exponential family properties.
It is, of course, still a curved exponential family, which gives it
the usual asymptotics of maximum likelihood.
%%%%% NEED FORWARD REFERENCE to veryfying that

\subsection{Differentiating the Aster Transform and Its Inverse}
\label{sec:aster-transform-deriv}

Here we follow Section~A.2 of the technical report \citet{aster1-tr} which
backs up the paper \citet{aster1}.  Our notation is different from their
notation, our notation here is what we have used for aster models since
\citet{aster-philosophical}.

\subsubsection{Aster Transform}

Let $\Delta \theta_j$ denote an infinitesimal increment of $\theta_j$
and similarly for $\Delta \varphi_j$.  Then
differentiating \eqref{eq:aster-transform} and using \eqref{eq:xi-from-c} and
the chain rule gives
\begin{equation} \label{eq:aster-transform-deriv}
\begin{split}
   \Delta \varphi_j
   & =
   \Delta \theta_j
   -
   \sum_{\substack{G \in \mathcal{G} \\ q(G) = j}} \sum_{k \in G}
   \xi_k \Delta \theta_k
   \\
   & =
   \Delta \theta_j
   -
   \sum_{\substack{k \in J \\ p(k) = j}}
   \xi_k \Delta \theta_k
\end{split}
\end{equation}

In language that does not refer to infinitesimals and using the
sophisticated view that derivatives are linear transformations
(\citealp[Definition~8.9]{browder}; \citealp[p.~334]{lang}),
 the derivative of the
aster transform is the linear transformation that maps
the vector $\Delta \theta$ having components $\Delta \theta_j$ to
the vector $\Delta \varphi$ having components $\Delta \varphi_j$.

We can think of this linear transformation as being represented by the matrix
of partial derivatives, which can be read off \eqref{eq:aster-transform-deriv},
\begin{equation} \label{eq:aster-transform-deriv-partial}
   \frac{\partial \varphi_j}{\partial \theta_k}
   =
   \begin{cases}
   1, & j = k \\
   - \xi_k, & j = p(k) \\
   0, & \text{otherwise}
   \end{cases}
\end{equation}

\subsubsection{Inverse Aster Transform}
\label{sec:inverse-aster-transform-deriv}

By the inverse function theorem, the derivative of the inverse is the
inverse of the derivative (considered as a linear transformation), assuming
it exists, which it does if the derivative is invertible
\cite[p.~361--363]{lang}.  We will prove the derivative is invertible by
inverting it.

As discussed in Section~\ref{sec:aster-transform} above,
the formula \eqref{eq:inverse-aster-transform}
gives an inductive definition that works when nodes of the graph are visited
in any order that visits successors before predecessors.

The same is true of the derivative of the inverse aster transform.
Moving a term from one side of \eqref{eq:aster-transform-deriv}
to the other gives
\begin{equation} \label{eq:inverse-aster-transform-deriv}
   \Delta \theta_j
   =
   \Delta \varphi_j
   +
   \sum_{\substack{k \in J \\ p(k) = j}} \xi_k \Delta \theta_k.
\end{equation}
When $\Delta \theta_j$ on the left-hand side is computed,
all of the $\Delta \theta_k$ on the
right-hand side will already have been computed (when we visit successors
before predecessors).

As before, the derivative of the inverse
aster transform is the linear transformation that maps
the vector $\Delta \varphi$ having components $\Delta \varphi_j$ to
the vector $\Delta \theta$ having components $\Delta \theta_j$.

Because of the nature of inductive definitions, the analog
of \eqref{eq:aster-transform-deriv-partial} for the inverse transform
is a bit more complicated.
To help with it we introduce the following notation and conventions.
Let $\bone(\fatdot)$ denote the function that maps logical formulas
to numbers, mapping false formulas to zero and true formulas to one,
and define empty sums (those having no terms) to be equal to zero
(the identity for addition)
and empty products (those having no terms) to be equal to one
(the identity for multiplication).

\begin{theorem} \label{th:inverse-aster-transform-deriv-partial}
Partial derivatives of the inverse aster transform are given by
\begin{equation} \label{eq:inverse-aster-transform-deriv-partial}
   \frac{\partial \theta_j}{\partial \varphi_k}
   =
   \bone(j \preceq k)
   \prod_{\substack{i \in J \\ j \prec i \preceq k}} \xi_i.
\end{equation}
\end{theorem}
In \eqref{eq:inverse-aster-transform-deriv-partial} the product is empty
when $j = k$.
(The product is also empty when $j \succ k$ but then we have 
$\bone(j \preceq k) = 0$ so it does not matter what the value of the product
is.)
\begin{proof}
What is to be shown is that \eqref{eq:inverse-aster-transform-deriv-partial}
agrees with \eqref{eq:inverse-aster-transform-deriv} in the case
$\Delta \varphi$ has only one nonzero component, say $\Delta \varphi_n$.
In this case \eqref{eq:inverse-aster-transform-deriv} says
\begin{alignat*}{2}
   \Delta \theta_j & = 0, & \qquad & j \not\preceq n
   \\
   \Delta \theta_j
   & =
   \Delta \varphi_n, & & j = n
   \\
   \Delta \theta_j
   & =
   \sum_{\substack{k \in J \\ p(k) = j}} \xi_k \Delta \theta_k,
   & & j \prec n
\end{alignat*}
or
\begin{alignat*}{2}
   \frac{\partial \theta_j}{\partial \varphi_n} & = 0,
   & \qquad & j \not\preceq n
   \\
   \frac{\partial \theta_j}{\partial \varphi_n} & = 1, & & j = n
   \\
   \frac{\partial \theta_j}{\partial \varphi_n}
   & =
   \sum_{\substack{k \in J \\ p(k) = j}} \xi_k 
   \frac{\partial \theta_k}{\partial \varphi_n},
   & & j \prec n
\end{alignat*}
Clearly the first two lines agree with
\eqref{eq:inverse-aster-transform-deriv-partial}.
That leaves only the third line to check.
We note in this line that the partial derivative on the right-hand side
is zero unless $j \prec k \preceq n$ and that there is hence exactly one
term in the sum.
Thus the third line just above agrees with
\eqref{eq:inverse-aster-transform-deriv-partial} by mathematical induction.
\end{proof}

\subsection{Log Likelihood Derivatives}
\label{sec:log-likelihood-derivatives}

\subsubsection{First Derivatives With Respect To $\boldsymbol{\theta}$}
\label{sec:wrt-theta}

Applying \eqref{eq:xi-from-c} and \eqref{eq:theta-to-xi}
to \eqref{eq:logl-with}, we obtain
\begin{equation} \label{eq:logl-deriv-wrt-theta}
   \frac{\partial l(\theta)}{\partial \theta_j}
   =
   y^*_j - y^*_{p^*(j)} \xi_j,
   \qquad j \in J.
\end{equation}
These are the first derivatives of the log likelihood for a saturated model
with subsampling with respect to components of $\theta$.

Notice that, as always, there is the curious mix of starred and unstarred
thingummies.  The data and the predecessor function have stars because this
is for models with subsampling.  The parameters do not have stars because
we insist that models have the same parameters with and without subsampling.
The index set for \eqref{eq:logl-deriv-wrt-theta} is $J$ because that is
the index set for $\theta$.

\subsubsection{First Derivatives for CAM}
\label{sec:logl-cam-deriv}

In conditional aster models (CAM) with model equation
\eqref{eq:affine-conditional}
we have $\partial \theta_j / \partial \beta_k = m_{j k}$
where $M$ has components $m_{j k}$.
Thus, by \eqref{eq:logl-deriv-wrt-theta} and the chain rule
$$
   \frac{\partial l(\beta)}{\partial \beta_k}
   =
   \sum_{j \in J}
   \frac{\partial l(\theta)}{\partial \theta_j}
   \frac{\partial \theta_j}{\partial \beta_k}
   =
   \sum_{j \in J} ( y^*_j - y^*_{p^*(j)} \xi_j ) m_{j k}
$$
MLE are derived by setting these equal to zero, considering the $\xi$'s
as functions of $\beta$, and solving for $\beta$.

\subsubsection{First Derivatives With Respect To $\boldsymbol{\varphi}$}
\label{sec:wrt-phi}

\begin{theorem} \label{th:partial-like-wrt-phi}
For $k \in J$ define
\begin{equation} \label{eq:infimum}
   n = \inf \set{ j \in J : j \preceq k }.
\end{equation}
Then
\begin{equation} \label{eq:partial-like-wrt-phi}
   \frac{\partial l(\varphi)}{\partial \varphi_k}
   =
   y^*_k
   -
   y^*_{p^*(n)}
   \left(
   \prod_{\substack{i \in J \\ n \preceq i \preceq k}} \xi_i
   \right)
   +
   \sum_{\substack{m \in J \\ n \prec m \preceq k}}
   \left[ y^*_{p(m)} - y^*_{p^*(m)} \right]
   \left(
   \prod_{\substack{i \in J \\ m \preceq i \preceq k}} \xi_i
   \right)
\end{equation}
\end{theorem}
The infimum in \eqref{eq:infimum} means $j \preceq k$ implies $n \preceq j$.
The at-most-one-predecessor property
(Section~\ref{sec:with-and-without-notation} above)
makes the set $\set{ j \in J : j \preceq k }$ totally ordered by $\preceq$.
This set is also nonempty (it contains $k$)
and finite (aster graphs are finite).
Hence the infimum in \eqref{eq:infimum} is always well defined.

The conventions that
empty sums are zero and empty products are one (established in
Section~\ref{sec:inverse-aster-transform-deriv} above) are still in force.

If a node $p^*(m)$ is not a subsampling node, then $p(m) = p^*(m)$.
Hence, if there is no subsampling,
all terms $y^*_{p(m)} - y^*_{p^*(m)}$ in \eqref{eq:partial-like-wrt-phi}
are zero, and \eqref{eq:partial-like-wrt-phi} reduces to
$$
   \frac{\partial l(\varphi)}{\partial \varphi_k}
   =
   y_k - y_{p(n)}
   \prod_{\substack{i \in J \\ n \preceq i \preceq k}} \xi_i
   =
   y_k - \mu_k
$$
by \eqref{eq:xi-mu-prod}, and this
agrees with previous aster theory \citep[Section~3.2]{aster1}.

Node $p(n)$ is the unique initial node of the full aster graph satisfying
$p(n) \prec k$.  If $p^*(n) \neq p(n)$, then $p^*(n)$ is a subsampling node.

\begin{proof}
\begin{align*}
   \frac{\partial l(\varphi)}{\partial \varphi_k}
   & =
   \sum_{j \in J}
   \frac{\partial l(\theta)}{\partial \theta_j}
   \frac{\partial \theta_j}{\partial \varphi_k}
   \\
   & =
   \sum_{\substack{j \in J \\ j \preceq k}}
   ( y^*_j - y^*_{p^*(j)} \xi_j )
   \prod_{\substack{i \in J \\ j \prec i \preceq k}} \xi_i
   \\
   & =
   \left(
   \sum_{\substack{j \in J \\ j \preceq k}}
   y^*_j
   \prod_{\substack{i \in J \\ j \prec i \preceq k}} \xi_i
   \right)
   -
   \left(
   \sum_{\substack{j \in J \\ j \preceq k}}
   y^*_{p^*(j)}
   \prod_{\substack{i \in J \\ j \preceq i \preceq k}} \xi_i
   \right)
   \\
   & =
   y^*_k
   -
   y^*_{p^*(n)}
   \left(
   \prod_{\substack{i \in J \\ n \preceq i \preceq k}} \xi_i
   \right)
   \\
   & \qquad
   +
   \left(
   \sum_{\substack{j \in J \\ n \preceq j \prec k}}
   y^*_j
   \prod_{\substack{i \in J \\ j \prec i \preceq k}} \xi_i
   \right)
   -
   \left(
   \sum_{\substack{m \in J \\ n \prec m \preceq k}}
   y^*_{p^*(m)}
   \prod_{\substack{i \in J \\ m \preceq i \preceq k}} \xi_i
   \right)
   \\
   & =
   y^*_k
   -
   y^*_{p^*(n)}
   \left(
   \prod_{\substack{i \in J \\ n \preceq i \preceq k}} \xi_i
   \right)
   \\
   & \qquad
   +
   \left(
   \sum_{\substack{m \in J \\ n \prec m \preceq k}}
   y^*_{p(m)}
   \prod_{\substack{i \in J \\ m \preceq i \preceq k}} \xi_i
   \right)
   -
   \left(
   \sum_{\substack{m \in J \\ n \prec m \preceq k}}
   y^*_{p^*(m)}
   \prod_{\substack{i \in J \\ m \preceq i \preceq k}} \xi_i
   \right)
   \\
   & =
   y^*_k
   -
   y^*_{p^*(n)}
   \left(
   \prod_{\substack{i \in J \\ n \preceq i \preceq k}} \xi_i
   \right)
   +
   \sum_{\substack{m \in J \\ n \prec m \preceq k}}
   \left[ y^*_{p(m)} - y^*_{p^*(m)} \right]
   \left(
   \prod_{\substack{i \in J \\ m \preceq i \preceq k}} \xi_i
   \right)
\end{align*}
where the first equality is the chain rule,
the second equality is \eqref{eq:inverse-aster-transform-deriv-partial}
and \eqref{eq:logl-deriv-wrt-theta}, and the rest are just algebra.
\end{proof}

\subsubsection{First Derivatives for UAM}
\label{sec:logl-uam-deriv}

Hence, for unconditional aster models (UAM),
$$
   \frac{\partial l(\beta)}{\partial \beta_n}
   =
   \sum_{k \in J}
   \frac{\partial l(\varphi)}{\partial \varphi_k}
   m_{k n}.
$$

\subsubsection{Second Derivatives With Respect To $\boldsymbol{\theta}$}
\label{sec:wrt-theta-theta}

Because the R function \texttt{mlogl} in R package \texttt{aster} calculates
\emph{minus} the log likelihood and its first and second derivatives, we do
the same.
Negating and differentiating \eqref{eq:logl-deriv-wrt-theta} gives
\begin{equation} \label{eq:logl-deriv-wrt-theta-theta}
   - \frac{\partial^2 l(\theta)}{\partial \theta_j \partial \theta_k}
   =
   y^*_{p^*(j)} \gamma_{j k},
   \qquad j, k \in J,
\end{equation}
where
\begin{equation} \label{eq:gamma}
   \gamma_{j k} = \begin{cases}
   \partial^2 c_G(\theta_G) / \partial \theta_j \partial \theta_k,
   & j, k \in G \in \mathcal{G}
   \\
   0, & \text{otherwise} \end{cases}
\end{equation}
See \eqref{eq:gamma-pre} above for more on $\gamma_{j k}$.

\subsubsection{Hessian for CAM}
\label{sec:mlogl-hessian-cam}

Second derivative matrices are commonly called Hessian matrices in optimization
theory.  If the matrix having components \eqref{eq:logl-deriv-wrt-theta-theta}
is denoted $H(\theta)$, then the Hessian for $\beta$ is given by
\begin{equation} \label{eq:hessian-beta}
   H(\beta) = M^T H(\theta) M, \qquad \text{when $\theta = a + M \beta$},
\end{equation}
where, as usual, $a$ is the offset vector and $M$ is the model matrix.
This is because $M$ is the Jacobian matrix of the parameter transformation
$\beta \to \theta$ in a CAM, and \eqref{eq:hessian-beta} is just the chain
rule.

\subsubsection{Hessian for UAM}
\label{sec:mlogl-hessian-uam}

If we follow the preceding section in denoting the Hessian for $\beta$
by $H(\beta)$, the Hessian for $\theta$ by $H(\theta)$, and the Hessian
for $\varphi$ by $H(\varphi)$, then
\begin{equation} \label{eq:hessian-beta-uam}
   H(\beta) = M^T H(\varphi) M, \qquad \text{when $\varphi = a + M \beta$},
\end{equation}
and the argument is exactly the same as in the preceding section.

\subsection{Second Derivatives With Respect To $\boldsymbol{\varphi}$}
\label{sec:wrt-phi-phi}

That leaves us with having to derive $H(\varphi)$, which is complicated.
Before differentiating \eqref{eq:partial-like-wrt-phi} again, it will
simplify computations if we recognize that the terms in big round brackets
are ``parameterized common sub-expressions'' (one has $n$ where the other
has $m$).  Using \eqref{eq:inverse-aster-transform-deriv-partial}, we can
give one of these another notation
\begin{equation} \label{eq:product-subexpression}
   \frac{\partial \theta_{p(m)}}{\partial \varphi_k}
   =
   \prod_{\substack{i \in J \\ m \preceq i \preceq k}} \xi_i,
\end{equation}
but when we change $m$ to $n$ in \eqref{eq:product-subexpression}
to get a notation for the other
term in big round brackets in \eqref{eq:partial-like-wrt-phi} this
does not work, because $p(n)$ is an initial node so there is no
parameter $\theta_{p(n)}$.
This would work, however, if we imagine that the aster graph we are
working with is part of a larger aster graph in which $p(n)$ is not
initial and not a subsampling node.
If we work under this fiction, we should get correct mathematics.

\begin{lemma} \label{lem:kumquat}
\begin{equation} \label{eq:kumquat}
   \frac{\partial^2 \theta_{p(m)}}{\partial \varphi_j \partial \varphi_k}
   =
   \sum_{\substack{r \in J \\ p(m) \prec r \preceq k}}
   \sum_{\substack{s \in J \\ p(m) \prec s \preceq j}}
   \gamma_{r s}
   \prod_{\substack{t \in J \\ p(m) \prec t \prec r}} \xi_t
   \prod_{\substack{u \in J \\ s \prec u \preceq j}} \xi_u
   \prod_{\substack{v \in J \\ r \prec v \preceq k}} \xi_v
\end{equation}
When all dependence groups are singletons, as with R package {\tt aster},
this specializes to
\begin{equation} \label{eq:kumquat-no-groups}
   \frac{\partial^2 \theta_{p(m)}}{\partial \varphi_j \partial \varphi_k}
   =
   \sum_{\substack{r \in J \\ p(m) \prec r \preceq k \\ p(m) \prec r \preceq j}}
   \gamma_{r r}
   \prod_{\substack{t \in J \\ p(m) \prec t \prec r}} \xi_t
   \prod_{\substack{u \in J \\ r \prec u \preceq j}} \xi_u
   \prod_{\substack{v \in J \\ r \prec v \preceq k}} \xi_v
\end{equation}
\end{lemma}

\begin{lemma} \label{lem:quince}
If $m \prec r$ and $\gamma_{r s} \neq 0$, then $m \prec s$.
\end{lemma}
\begin{proof}
If $r = s$ the assertion is trivial.
Otherwise, $\gamma_{r s} \neq 0$ implies that $r$ and $s$ are in the
same dependence group, say $G$, and none of the arrows for this dependence
group are subsampling arrows.  It follows that $p(r) = p(s) = q(G)$.
So $m \prec r$ implies $m \preceq q(G)$ implies $m \prec s$.
\end{proof}

\begin{proof}[Proof of Lemma~\ref{lem:kumquat}]
\begin{align*}
   \frac{\partial^2 \theta_{p(m)}}{\partial \varphi_j \partial \varphi_k}
   & =
   \frac{\partial}{\partial \varphi_j}
   \prod_{\substack{i \in J \\ m \preceq i \preceq k}} \xi_i
   \\
   & =
   \sum_{\substack{r \in J \\ m \preceq r \preceq k}}
   \frac{\partial \xi_r}{\partial \varphi_j}
   \prod_{\substack{i \in J \\ m \preceq i \preceq k \\ i \neq r}} \xi_i
   \\
   & =
   \sum_{\substack{r \in J \\ m \preceq r \preceq k}}
   \sum_{s \in J}
   \frac{\partial \xi_r}{\partial \theta_s}
   \frac{\partial \theta_s}{\partial \varphi_j}
   \prod_{\substack{i \in J \\ m \preceq i \preceq k \\ i \neq r}} \xi_i
   \\
   & =
   \sum_{\substack{r \in J \\ m \preceq r \preceq k}}
   \sum_{\substack{s \in J \\ s \preceq j}}
   \gamma_{r s}
   \prod_{\substack{u \in J \\ s \prec u \preceq j}} \xi_u
   \prod_{\substack{i \in J \\ m \preceq i \preceq k \\ i \neq r}} \xi_i
   \\
   & =
   \sum_{\substack{r \in J \\ m \preceq r \preceq k}}
   \sum_{\substack{s \in J \\ s \preceq j}}
   \gamma_{r s}
   \prod_{\substack{u \in J \\ s \prec u \preceq j}} \xi_u
   \prod_{\substack{t \in J \\ m \preceq t \prec r}} \xi_t
   \prod_{\substack{v \in J \\ r \prec v \preceq k}} \xi_v
   \\
   & =
   \sum_{\substack{r \in J \\ p(m) \prec r \preceq k}}
   \sum_{\substack{s \in J \\ s \preceq j}}
   \gamma_{r s}
   \prod_{\substack{u \in J \\ s \prec u \preceq j}} \xi_u
   \prod_{\substack{t \in J \\ p(m) \prec t \prec r}} \xi_t
   \prod_{\substack{v \in J \\ r \prec v \preceq k}} \xi_v
\end{align*}
where the first equality is \eqref{eq:product-subexpression},
the second equality is the product rule,
the third equality is the chain rule,
the fourth equality is \eqref{eq:xi-from-c}, \eqref{eq:gamma}, and
\eqref{eq:inverse-aster-transform-deriv-partial},
the fifth equality just splits one product into two products.
and the last equality is $m \preceq r$ if and only if $p(m) \prec r$
and similarly for $t$.  Finally we use Lemma~\ref{lem:quince} to get
$p(m) \prec r$ and $\gamma_{r s} \neq 0$ implies $p(m) \prec s$.

Going from \eqref{eq:kumquat} to \eqref{eq:kumquat-no-groups} is just that,
when all dependence groups are singletons, $\gamma_{r s} \neq 0$ implies
$r = s$.
\end{proof}

\begin{theorem} \label{th:hessian-uam}
Define $n$ by \eqref{eq:infimum}.
For an unconditional aster model with subsampling,
the $j, k$ component of $H(\varphi)$ is
\begin{equation} \label{eq:hessian-groups}
   - \frac{\partial^2 l(\varphi)}{\partial \varphi_j \partial \varphi_k}
   =
   y^*_{p^*(n)}
   \frac{\partial^2 \theta_{p(n)}}{\partial \varphi_j \partial \varphi_k}
   -
   \sum_{\substack{m \in J \\ n \prec m \preceq k}}
   \left[ y^*_{p(m)} - y^*_{p^*(m)} \right]
   \frac{\partial^2 \theta_{p(m)}}{\partial \varphi_j \partial \varphi_k}
\end{equation}
where the second partial derivatives of $\theta$ with respect to $\varphi$
are given by \eqref{eq:kumquat} or \eqref{eq:kumquat-no-groups}
and where these equations are to be used regardless of whether $p(n)$
actually indexes a parameter.
\end{theorem}
\begin{proof}
Immediate from \eqref{eq:partial-like-wrt-phi}.
\end{proof}
As was remarked after Theorem~\ref{th:partial-like-wrt-phi},
a node $p^*(m)$ is not a subsampling node if and only if $p(m) = p^*(m)$,
in which case the term containing $y^*_{p(m)} - y^*_{p^*(m)}$ is exactly zero.

Since cumulant functions are infinitely differentiable,
so is the aster transform, the inverse aster transform,
and aster log likelihoods.  Thus formulas \eqref{eq:kumquat},
\eqref{eq:kumquat-no-groups} and \eqref{eq:hessian-groups}
must be equal when $j$ and $k$ are interchanged.
We have written them in a form so that this is almost obvious.
The only non-obvious spot is when we interchange $j$ and $k$
in \eqref{eq:kumquat} we (in effect) change $p(m) \prec t \prec r$
into $p(m) \prec t \prec s$, but this agrees with Lemma~\ref{lem:quince}:
$t \prec r$ and $\gamma_{r s} \neq 0$ implies $t \prec s$.

\subsection{Fisher Information}
\label{sec:fisher}

\subsubsection{Change of Parameter}

We begin with a theorem about change of Fisher information under smooth change
of parameter that is for general likelihood inference under the ``usual
regularity conditions.''  It has nothing in particular to do with
exponential families.  The ``usual regularity conditions'' hold for all
regular exponential families including curved exponential families.
So they hold for all models in this article.   But they hold for many
other models too.

The only regularity condition we need is the so-called Bartlett identities,
which are usually derived by differentiating the integral of the probability
densities twice with respect to the parameters
\begin{align}
   E_\theta\{ \nabla l(\theta) \} & = 0
   \label{eq:bartlett-general-one}
   \\
   \var_\theta\{ \nabla l(\theta) \} & = - E_\theta \{ \nabla^2 l(\theta) \}
   \label{eq:bartlett-general-two}
\end{align}
where $\nabla l(\theta)$ denotes the vector of first partial derivatives
of the log likelihood, $\nabla^2 l(\theta)$ denotes the matrix of second
partial derivatives of the log likelihood,
and $\var$ denotes the variance operator that produces a variance matrix
(also called variance-covariance matrix, covariance matrix,
and dispersion matrix).

Expected Fisher information is either side of \eqref{eq:bartlett-general-two}.
Observed Fisher information is $- \nabla^2 l(\theta)$.

The theorem is about what happens under a change of parameter
$\theta = g(\psi)$.  In short, the theorem says Fisher information
transforms like a tensor.
If expected Fisher information for $\theta$ is denoted $I(\theta)$,
and observed Fisher information for $\theta$ is denoted $J(\theta)$
(note that the latter is a random quantity despite there being no
explicit indication of this), and the Jacobian of the transformation
is $B(\psi) = \nabla g(\psi)$, then
\begin{equation} \label{eq:transform-like-tensor-expected}
   I(\psi) = B(\psi)^T I(\theta) B(\psi), \qquad \text{when $\theta = g(\psi)$}.
\end{equation}
This is reminiscent of and ultimately derives from the formula for change
of variance under a linear transformation.  If $y = B x$, where $x$ and $y$
are random vectors and $B$ is a known matrix, then
\begin{equation} \label{eq:transform-like-tensor-variance}
   \var(y) = B \var(x) B^T.
\end{equation}
The analogous result for observed Fisher information is a bit trickier.
Changing $I$ to $J$ in \eqref{eq:transform-like-tensor-expected} gives
a statement that is, in general, false.  But it is true when MLE are
plugged in
\begin{equation} \label{eq:transform-like-tensor-observed}
   J(\hat{\psi}) = B(\hat{\psi})^T J(\hat{\theta}) B(\hat{\psi}),
   \qquad \text{when $\hat{\theta} = g(\hat{\psi})$}.
\end{equation}
When $\hat{\theta} = g(\hat{\psi})$ and $\hat{\psi}$ is an MLE, then
$\hat{\theta}$ is also an MLE by by invariance of MLE under
parameter transformation.
\begin{theorem}
Assume \eqref{eq:bartlett-general-one} and \eqref{eq:bartlett-general-two}.
Then statement \eqref{eq:transform-like-tensor-expected} is correct.
If $\hat{\theta} = g(\hat{\psi})$ is a zero of the first derivative of
the log likelihood for $\theta$,
then statement \eqref{eq:transform-like-tensor-observed} is correct.
The change of parameter function $g$ must be injective and differentiable
but need not be surjective.
\end{theorem}
\begin{proof}
By the chain rule
$$
   \nabla l(\psi)
   = \nabla l(\theta) B(\psi), \qquad \text{when $\theta = g(\psi)$}.
$$
Take the variance of both sides and
use \eqref{eq:transform-like-tensor-variance}
to obtain \eqref{eq:transform-like-tensor-expected}.

The situation is more complicated with second derivatives
$$
   - \frac{\partial^2 l(\psi)}{\partial \psi_i \partial \psi_j}
   =
   -
   \left(
   \sum_k \sum_m
   \frac{\partial^2 l(\theta)}{\partial \theta_k \partial \theta_m}
   \frac{\partial \theta_k}{\partial \psi_i}
   \frac{\partial \theta_m}{\partial \psi_j}
   \right)
   -
   \sum_k
   \frac{\partial l(\theta)}{\partial \theta_k}
   \frac{\partial^2 \theta_k}{\partial \psi_i \partial \psi_j}.
$$
When MLE are plugged in,
$$
   \left. \frac{\partial l(\theta)}{\partial \theta_k}
   \right|_{\theta = \hat{\theta}} = 0
$$
so this gives \eqref{eq:transform-like-tensor-observed}.
\end{proof}

\subsubsection{Fisher Information for $\boldsymbol{\theta}$}
\label{sec:I-J-theta}

Returning to aster models with subsampling, from
\eqref{eq:logl-deriv-wrt-theta-theta} we get:
\begin{itemize}
\item observed Fisher information for $\theta$ is the matrix having
components $y^*_{p^*(j)} \gamma_{j k}$ and
\item
expected Fisher information for $\theta$ is the matrix having
components $\mu^*_{p^*(j)} \gamma_{j k}$.
\end{itemize}
Note that, since expected Fisher information depends on
components of $\mu^*$, which cannot, in general, be calculated
by the computer,
use of expected Fisher information depends on those subsampling probability
distributions,
whereas the log likelihood, its derivatives, and observed Fisher information
do not depend on them.

Thus, in general, we can only use observed Fisher information.

\subsubsection{Fisher Information for $\boldsymbol{\beta}$ for CAM}
\label{sec:I-J-cam}

If $I(\theta)$ is expected Fisher information for $\theta$ derived
in the preceding section, then expected Fisher information for $\beta$ is
$$
   I(\beta) = M^T I(\theta) M, \qquad \text{when $\theta = a + M \beta$},
$$
because $M$ is the derivative of $\theta$ with respect to $\beta$.
We also have the analogous relationship for observed Fisher information
when MLE are plugged in
$$
   J(\hat{\beta})
   =
   M^T J(\hat{\theta}) M,
   \qquad \text{when $\hat{\theta} = a + M \hat{\beta}$}.
$$

\subsubsection{Fisher Information for $\boldsymbol{\beta}$ for UAM}
\label{sec:I-J-uam}

As in the preceding section, let $I(\theta)$ and $J(\theta)$ be expected
and observed Fisher information for $\theta$ for the saturated model with
subsampling, which were derived in Section~\ref{sec:I-J-theta} above.
Now let $a(\varphi)$ denote the inverse aster transform described in
Section~\ref{sec:aster-transform} above, and let $A(\varphi)$ denote
its derivative,
whose components are given by \eqref{eq:inverse-aster-transform-deriv-partial}.
Now the model equation is \eqref{eq:affine-unconditional}
so $M$ is the derivative of $\varphi$ with respect to $\beta$.

So now expected Fisher information for $\beta$ is
$$
   I(\beta) = M^T A(\varphi)^T I(\theta) A(\varphi) M,
   \qquad \text{when $\varphi = a + M \beta$ and $\theta = a(\varphi)$},
$$
and the analogous relationship for observed Fisher information
when MLE are plugged in is
$$
   J(\hat{\beta})
   =
   M^T A(\hat{\varphi})^T I(\hat{\theta}) A(\hat{\varphi}) M,
   \qquad \text{when $\hat{\varphi} = a + M \hat{\beta}$
   and $\hat{\theta} = a(\hat{\varphi})$}.
$$

\subsection{Prediction}
\label{sec:predict}

Prediction for all six parameterizations of UAM discussed in
Section \ref{sec:revisited} above can be handled by the \texttt{aster}
and \texttt{aster.formula} methods of the R generic function \texttt{predict}
(that is, the functions \texttt{predict.aster}
and \texttt{predict.aster.formula} considered as non-generic functions).
One calls these functions by the name \texttt{predict} but must look up the
help page with \texttt{help(predict.aster)}.
These functions are in R package \texttt{aster}.
(A referee for \citet{aster1} complained that these are not predictions
but rather parameter transformations and we agreed, but users expect to use
the R generic function \texttt{predict} to do this job.)

Since version 1.0 of the package, these functions have a new optional argument
\texttt{is.always.parameter = TRUE} which makes them ``predict'' $\xi$
rather than the vector having components $E(y_j \mid y_{p(j)})$, which
is not a parameter.
If we always use this option, this function can make predictions for all
parameters in the chain $\beta \to \varphi \to \theta \to \xi \to \mu$,
and if we add $\tau = M^T \mu$, which we can do ourselves (given $\mu$
calculated by \texttt{predict}), we have all six parameters.

As parameters, none of these depend on the response vector, although they
do depend on covariates because the model matrix depends on covariates.
Thus if the R function \texttt{predict} is provided an object of class
\texttt{"aster"} or \texttt{"aster.formula"}
that has the MLE $\hat{\beta}$ for the aster model with subsampling
as its \texttt{coefficients} component, and corresponding expected and observed
Fisher information matrices as its \texttt{fisher} and \texttt{hessian}
components but every other component as if subsampling had not been done,
then \texttt{predict} will operate to make predictions without subsampling.

We also want the \texttt{deviance} component of our object of
class \texttt{"aster"} or \texttt{"aster.formula"} containing the result
of fitting the model with subsampling to be reflect the subsampling
(be minus twice the maximized log likelihood) so that the R generic
function \texttt{anova} does the right thing with these objects.

\subsection{Parameter Transformation}
\label{sec:parameter-transformation}

The methods for the R generic function \texttt{predict} in R package
\texttt{aster} cannot do all of the parameter
transformations discussed in Sections~\ref{sec:plethora} and~\ref{sec:revisited}
above.  A function \texttt{astertransform} was added to this package, but
it only does transformations from $\theta$ and $\varphi$ to any of
$\theta$, $\varphi$, $\xi$, or $\mu$.

R package \texttt{aster2} does do all of these parameter transformations.
The function \texttt{transformSaturated} does any of the transformations
from $\theta$, $\varphi$, $\xi$, or $\mu$ to any of these.
The function \texttt{transformConditional} does any of the transformations
for a CAM without subsampling from $\beta$
to $\theta$, $\varphi$, $\xi$, or $\mu$ (but not vice versa).
The function \texttt{transformUnconditional} does any of the transformations
from $\beta$ or $\tau$ (the only parameters that are unconstrained)
to $\beta$, $\theta$, $\varphi$, $\xi$, $\mu$, or $\tau$.

Both \texttt{transformSaturated} and \texttt{transformUnconditional} are
able to transform from mean value to canonical parameters.

Calculating the parameter transformations
$\xi \to \theta$ or $\mu \to \varphi$ for saturated aster models without
subsampling and $\tau \to \beta$ for UAM without subsampling is
equivalent to doing maximum likelihood
with data replaced by unconditional mean value parameters.
Theorem~\ref{th:mean-value-inversion} covers all of these cases.




\chapter{Aster Mixed Models} \label{ch:reaster}

\section{Introduction}

In this chapter we cover the theory of aster mixed models (AMM),
a special case of exponential family mixed models (EFMM),
as described by \citet{reaster-tr,reaster} and as implemented by
R function \code{reaster} in R package \code{aster} \citep{aster-package}.

Except we correct one mistake in \citet{reaster} and also one wrong-headedness.
We also add some theory.

\subsection{Log Likelihood}

Let $l$ be the log likelihood for the canonical parameter vector of a
regular full exponential family whose canonical parameter space is a
full vector space.  A saturated aster model having the unconditional
canonical parameterization (Sections~\ref{sec:aster-transform},
\ref{sec:aster-mean-value-parameters}, and~\ref{sec:plethora} above)
is an example.  Following \citet{breslow-clayton}
and \citet*{stiratelli-laird-ware} and the thousands of papers citing them,
\citet{reaster} define an exponential family mixed model (EFMM) to be
a canonical affine submodel (Section~\ref{sec:canonical-affine-submodel}
above) in which some of the fixed effects are converted to random effects.

In more detail, if $\varphi$ is the canonical parameter vector of the
saturated aster model, we write
\begin{equation} \label{eq:reaster-model-equation}
   \varphi = a + M \alpha + Z b
\end{equation}
where
\begin{itemize}
\item $a$ is a known vector (the \emph{offset vector}),
\item $M$ is a known matrix (the \emph{model matrix for fixed effects}),
\item $\alpha$ is a vector of unknown parameters (the vector of
    \emph{fixed effects}),
\item $Z$ is a known matrix (the \emph{model matrix for random effects}), and
\item $b$ is a mean-zero multivariate normal random vector (the vector of
    \emph{random effects}).
\end{itemize}
The variance matrix of $b$ (also called variance-covariance matrix,
Section \ref{sec:mean-variance-cumulant} above), which we denote $D$,
determines the distribution of $b$.  In turn, $D$ is a function of
other unknown parameters.

As always (Section~\ref{sec:regression-notation} above) ``known'' means
not a function of the response vector.  The vector $a$ and the matrices
$M$ and $Z$ may depend on covariates or on the experimental design.
Usually $a$ does not, and usually $M$ and $Z$ do.  They are ``known'' in
the sense that our whole analysis is done conditional on the values of any
covariates that are random, so they are essentially treated as fixed at their
observed values (because that is how conditional probability works).

\citet{reaster} and R function \code{reaster} in R package \code{aster}
assume that $D$ is diagonal and the diagonal elements of $D$ are the unknown
parameters, which are traditionally called \emph{variance components}.
They denote the vector of variance components $\nu$.

In almost all applications there are groups of random effects whose variance
is the same by design, hence their variance is the same variance component.
We keep track of this by defining
\begin{equation} \label{eq:eek}
   E_k = \frac{\partial D}{\partial \nu_k}
\end{equation}
so each $E_k$ is a diagonal matrix whose diagonal elements are either zero
or one, the sum of the $E_k$ is the identity matrix, and the product of the
$E_k$ is the zero matrix.
We can now write
$$
   D = \sum\nolimits_k \nu_k E_k
$$
to show the explict dependence of $D$ on $\nu$.

We no longer think we should make this restriction on the structure of $D$
part of the definition of EFMM.  In other contexts, users sometimes use
correlated random effects.  If we assume AR(1) (auto-regressive order one)
structure, then $D$ can no longer be written as above.  If our random effects
come from quantitative genetics, then we can write $D$ in the form above if
the pedigree of our experimental design has no inbreeding.  But otherwise we
cannot.

Nevertheless, we have not yet implemented more complicated
specification of $D$ in R package \code{aster}.
And we will assume $D$ has the structure described above in the rest of this
chapter.

Minus the the so-called ``complete data log likelihood,''
what the log likelihood would
be if the random effects were observed data rather than unobservable latent
variables, is
\begin{equation} \label{eq:mlogl-complete}
   - l(a + M \alpha + Z b) + \tfrac{1}{2} b^T D^{-1} b
   + \tfrac{1}{2} \log \det(D)
\end{equation}
Recall that $D$ is a function of the unknown parameter vector $\nu$
even though the notation does not explicitly indicate this.
Note that \eqref{eq:mlogl-complete} only makes sense when no variance
components are zero (otherwise $D^{-1}$ does not exist, much more on this
later, starting with Section~\ref{sec:lsc} below).

The likelihood for the random effects model is the conditional expectation
of the complete data likelihood (the exponential of \eqref{eq:mlogl-complete})
with respect to the observed data (the response vector $y$, which is also
not explicitly indicated in the notation but is in the log likelihood $l$,
see \eqref{eq:logl-aster-phi} above).  That is, we exponentiate
\eqref{eq:mlogl-complete} and then integrate out $b$ with respect to its
assumed multivariate normal distribution, and then take the logarithm to get
the log likelihood for the random effects.  This process rarely, if ever,
can be done exactly.

\subsection{Laplace Approximation}

Thus, following \citet{breslow-clayton} and \citet{reaster}
we use the Laplace approximation
of the log likelihood for the random effects model,
\begin{equation} \label{eq:mlogl-pickle}
\begin{split}
   q(\alpha, \nu)
   & =
   - l(a + M \alpha + Z b^*) + \tfrac{1}{2} (b^*)^T D^{-1} b^*
   \\
   & \qquad
   + \tfrac{1}{2} \log \det \bigl[ Z^T W(a + M \alpha + Z b^*) Z D + \text{Id}
   \bigr]
\end{split}
\end{equation}
where
\begin{itemize}
\item $W$ is the Fisher information matrix for the saturated exponential
    family model, $W = - l''(\fatdot) = \nabla^2 c''(\fatdot)$
\item $b^*$ is the minimizer of \eqref{eq:mlogl-complete} with respect to $b$,
    holding $\alpha$ and $\nu$ fixed, and
\item $\text{Id}$ is the identity matrix of the appropriate dimension.
\end{itemize}
So $b^*$ is a function of $\alpha$ and $\nu$ although the notation in
\eqref{eq:mlogl-pickle} does not explicitly indicate this.

Summarizing the implicit dependencies:
\begin{itemize}
\item $l$ is a function of the response vector $y$ as well as its indicated
    arguments,
\item $D$ is a function of $\nu$, and
\item $b^*$ is a function of $\alpha$ and $\nu$.
\end{itemize}
If we need to make these dependencies explicit we could write $l_y$ or
$D(\nu)$ or $b^*(\alpha, \nu)$.

\begin{theorem} \label{th:pickle}
If \eqref{eq:mlogl-complete} considered a function of $b$ for
fixed $\alpha$ and $\nu$ is bounded below, then has a unique minimizer $b^*$.
\end{theorem}
\begin{proof}
The first term in \eqref{eq:mlogl-complete} considered as a function of $b$
for fixed $\alpha$ is a convex function and $b \mapsto b^T D^{-1} b$ is a
strictly convex function with bounded level sets,
hence the sum is a strictly convex function with bounded level sets
(under the assumption that it is bounded below).
Thus the minimizer $b^*$ exists and is unique.
\end{proof}

Then the $\alpha$ and $\nu$ that minimize \eqref{eq:mlogl-pickle} are
our estimates of these parameters.  They are approximate MLE (minimizers
of the Laplace approximation of the log likelihood).  Call them $\hat{\alpha}$
and $\hat{\nu}$.

The condition in Theorem~\ref{th:pickle} that \eqref{eq:mlogl-complete} be
bounded below is not restrictive.
Any aster log likelihood that does not involve
normal location-scale dependence groups is bounded below, and even those
having such dependence groups will be bounded below if the model equation
\eqref{eq:reaster-model-equation} does not send any components of $\varphi$
to infinity that correspond to variance nodes of a normal-location-scale
dependence group (Theorem~\ref{th:dor-normal} above).  And, of course,
$b^T D^{-1} b$ is bounded below by zero.

\subsection{Estimates of Random Effects}

We then have $b^*(\hat{\alpha}, \hat{\nu})$ as ``estimates'' (in scare quotes)
of the random effects.  We say ``in scare quotes'' because, of course, the
random effects are not parameters to estimate.  They are (supposed to be)
random variables, not unknown constants.

Thus we should ``estimate'' something about their conditional distribution
given the observed value of the response vector and the parameter estimates
$\hat{\alpha}$ and $\hat{\nu}$.

Since \eqref{eq:mlogl-complete} is minus a log unnormalized
conditional probability density function of $b$ given $y$, it follows
that $b^*(\alpha, \nu)$ is the mode of that conditional distribution
for parameter values $\alpha$ and $\nu$.
Because of the symmetry of the normal distribution, these are also medians
of this conditional distribution.

When we map these estimates through a nonlinear transformation, such as the
map from canonical to mean value parameters,
they are no longer modes or medians, so it is unclear how to describe these
``estimates'' in scare quotes.

\subsection{A Key Concept}

Following \citet{reaster},
for any positive semidefinite matrix $W$, define
\begin{equation} \label{eq:pee}
   p_W(\alpha, b, \nu) =
   - l(a + M \alpha + Z b) + \tfrac{1}{2} b^T D^{-1} b
   + \tfrac{1}{2} \log \det \bigl[ Z^T W Z D + \text{Id}
   \bigr]
\end{equation}
Considered as functions of $b$ for fixed $\alpha$ and $\nu$
\eqref{eq:mlogl-complete} and \eqref{eq:pee} differ only by a constant,
hence have the same minimizer.

We thus consider the problem of jointly
minimizing $p_W$ as a function of $(\alpha, b, \nu)$.
When $W$ is close to
\begin{equation} \label{eq:w-hat}
   \widehat{W} =
   W\bigl(a + M \hat{\alpha} + Z b^*(\hat{\alpha}, \hat{\nu})\bigr)
\end{equation}
The function
$$
   \inf p_W(\alpha, \fatdot, \nu)
$$
should be close to the Laplace approximation \eqref{eq:mlogl-pickle}.
(This is the conclusion of Theorem~\ref{th:pee-epi} below.)

So the joint minimizer of $p_W$ should be close to the minimizers
of the Laplace approximation \eqref{eq:mlogl-pickle},
when $W$ is close to $\hat{W}$.
(This is implied by Theorems~\ref{th:attouch} and~\ref{th:pee-epi} below.)

\subsection{Basic Optimization Theory}
\label{sec:basic}

In optimization theory, we have the following basic ideas
\citep[Sections~1.A, 1.B, and~1.C]{rockafellar-wets}.
\begin{itemize}
\item We explicitly deal with minimization rather than maximization.  If
    you have a maximization problem, just turn everything upside down.
    To maximize $f$, minimize $- f$.
\item For constrained optimization, we incorporate the constraints into
    the objective function (the function to minimize).
    To minimize $f$ subject to the constraint
    that the minimizer lie in $C$, redefine $f$ to have the value $+\infty$
    off of $C$.  Thus (unless $C$ is empty) the minimizer (if it exists)
    must lie in $C$.
\item We consider only lower semicontinuous (LSC) objective functions.
\end{itemize}
(LSC is defined at the beginning of Section~\ref{sec:lsc} below.)

The point of the first item is conservation of blather.
\citet{rockafellar-wets} is already a long book, more than 700 pages.
If it gratuitously introduced notation for terminology and discussion
of both maximization and minimization, it would be twice as long with
zero additional mathematical content.  After all, the change of point of
view from maximizing $f$ to minimizing $- f$ is truly trivial.

This is why, in this chapter, we are minimizing minus the log likelihood
(or its Laplace approximation) rather than maximizing the log likelihood.
We want to use a lot of optimization theory.

The point of the second item is hard to explain.  Like many places
in mathematics, simple changes can simplify greatly.  Incorporating
constraints into the objective function with the $+\infty$ trick
greatly simplifies (or even makes
possible) most of the theory in \citet{rockafellar-wets}.

Note the interplay of the first two items.  If you are maximizing, then
the objective function is $-\infty$ off the constraint set.

Since we are only minimizing (the first item), we no longer have any need
for continuity.  Continuity gives us existence of minimizers.  Any continuous
function achieves its maximum and minimum on any compact set.  But we don't
care about maxima.  So we replace continuity by LSC.  Any LSC function
achieves its minimum on any compact set.  That is all we need.

Note the interplay of the last two items.  No function that jumps from finite
to infinite at the boundary of the constraint set can be continuous.
But it can be LSC.

Having allowed $+\infty$ as a value of the objective function, we also allow
$-\infty$ as a value for technical reasons.  This means we are working with
the \emph{extended real number system} with its usual arithmetic, order, and
topology \citep[Section~1.E]{rockafellar-wets}.  The only tricky bit is that
there are no natural definitions of zero times infinity or infinity minus
infinity.  \citet{rockafellar-wets} define $0 \cdot \infty = 0$, a convention
familiar from probability theory, and define $\infty - \infty = \infty$ in
the context of minimization (they note that you would
want $\infty - \infty = - \infty$ in the context of maximization, which,
of course, the turn-everything-upside-down principle gives).

Topologically, the extended real number system, denoted $\exreal$ is a compact
set.  So every sequence has a convergent subsequence.  What it means for
objective function values to converge to infinity is just what it means in
calculus.

\subsection{Lower Semicontinuous Regularization}
\label{sec:lsc}

An extended-real-valued function $f$ is LSC at the point $x$ if
$$
   f(x) \le \liminf f(x_n), \qquad \text{whenever $x_n \to x$}
$$
and $f$ is LSC if it is LSC at each point.

To make a function LSC, we use the process of LSC regularization.
For any extended-real-valued function $f$, there is a least LSC function
majorized by $f$, denoted $\closure f$ \citep[Section~1.D]{rockafellar-wets}.

We want to apply LSC regularization to \eqref{eq:pee}.  First we define
\eqref{eq:pee} to have the value $+\infty$ when any component of $\nu$ is
negative.  This agrees with item 2 on our list (no variance can be negative).

Now we apply LSC regularization.  First we note that $b^T D^{-1} b$ is the
sum of terms of the form $h(b_i, \nu_k) = b_i^2 / \nu_k$,
where $\nu_k = \var(b_i)$.  Thus we start with the LSC regularization of $h$.

Again we define $h(b, \nu) = +\infty$ when $\nu < 0$ to enforce
the constraints.  Clearly $h(b, \nu) \ge 0$ for all $b$ and $\nu \neq 0$, and
$$
   \lim_{\substack{b = 0 \\ \nu \searrow 0}} h(b, \nu) = 0
$$
So we define $h(0, 0) = 0$.  Conversely, if $b_n$ is a sequence bounded
away from zero
$$
   \lim_{\nu_n \to 0} h(b_n, \nu_n) = + \infty
$$
Thus the LSC regularization has the form
$$
   h(b, \nu) =
   \begin{cases}
   b^2 / \nu, & \nu > 0
   \\
   0, & \nu = b = 0
   \\
   +\infty & \text{otherwise}
   \end{cases}
$$
\begin{equation} \label{eq:pee-lsc}
\begin{split}
   p_W(\alpha, b, \nu)
   & =
   - l(a + M \alpha + Z b)
   + \tfrac{1}{2} \sum_{(j, k) \in \mathcal{J}} h(b_j, \nu_k)
   \\
   & \qquad
   + \tfrac{1}{2} \log \det \bigl[ Z^T W Z D + \text{Id}
   \bigr]
\end{split}
\end{equation}
where $\mathcal{J}$ is the set of all $(j, k)$ such that $j$ is a possible
index for $b$, $k$ is a possible index for $\nu$ and the variance of $b_j$
is $\nu_k$ by design.

This still leaves us with a problem with the third term in \eqref{eq:pee}
or \eqref{eq:pee-lsc}.
\begin{lemma}
The third term in \eqref{eq:pee-lsc} is nonnegative when all components of
$\nu$ are nonnegative.
\end{lemma}
\begin{proof}
When all components of $\nu$ are nonnegative we can define $D^{1/2}$ to be
the diagonal matrix whose diagonal elements are the (positive) square roots
of the corresponding diagonal elements of $D$.
When all components of $\nu$ are strictly positive we can define $D^{-1/2}$
to be the diagonal matrix whose $k, k$ component is $1 / \sqrt{\nu_k}$.
Clearly, $D^{1/2}$ and $D^{-1/2}$ are inverse matrices.

First assume all components of $\nu$ are positive.
Then, because the determinant
of a product is the product of determinants,
\begin{align*}
   \det \left( Z^T W Z D + \text{Id} \right)
   & =
   \det \left( \bigl[ Z^T W Z D^{1/2} + D^{-1/2} \bigr] D^{1/2} \right)
   \\
   & =
   \det \left( Z^T W Z D^{1/2} + D^{-1/2} \right) \det \left( D^{1/2} \right)
   \\
   & =
   \det \left( D^{1/2} Z^T W Z D^{1/2} + \text{Id} \right)
\end{align*}
Reading from end to end, we have
\begin{equation*}
   \det \left( Z^T W Z D + \text{Id} \right)
   =
   \det \left( D^{1/2} Z^T W Z D^{1/2} + \text{Id} \right)
\end{equation*}
which no longer contains $D^{-1/2}$.  Hence, by continuity, this equation
also holds when all components of $\nu$ are nonnegative.

$W$ is assumed to be symmetric and positive semidefinite by definition,
hence so are $D^{1/2} Z^T W Z D^{1/2}$
and $D^{1/2} Z^T W Z D^{1/2} + \text{Id}$.
The determinant of a symmetric matrix is the product of its eigenvalues.
The eigenvalues of a symmetric positive semidefinite matrix are nonnegative.
The eigenvalues of $D^{1/2} Z^T W Z D^{1/2} + \text{Id}$ are one plus the
corresponding eigenvalues of $D^{1/2} Z^T W Z D^{1/2}$.  Hence the determinant
of $D^{1/2} Z^T W Z D^{1/2} + \text{Id}$ is greater than or equal to one.
Hence the log determinant is greater than or equal to zero.
\end{proof}

It follows from the lemma that the third term in \eqref{eq:pee-lsc} is
finite on an open neighborhood of the set of $(b, \nu)$ such that all
components of $\nu$ are nonnegative and infinitely differentiable on
that neighborhood.  But at some $(b, \nu)$ with some components of $\nu$
sufficiently negative, we may have the determinant negative so the log
determinant is undefined.  We have to separately define the value of
\eqref{eq:pee-lsc} to be $+\infty$ in this case.
(Of course, the principle of incorporating the constraints into the
objective function, requires the value to be $+\infty$ whenever any
component of $\nu$ is negative, so this paragraph is just about being
careful about that.)

\subsection{Epiconvergence}

Epiconvergence is a form of convergence of sequences of functions useful
in optimization \citep[Chapter~7]{rockafellar-wets}.  It has been little
used in statistics, notable uses being \citet{geyer-1994-jrssb} and
\citet{constrained-m-estimation}.

Epiconvergence has multiple characterizations.  The one we shall use is
\citet[Proposition~7.2]{rockafellar-wets} a sequence of extended-real-valued
functions $f_n$ \emph{epiconverges} to an extended-real-valued function $f$
at the point $x$ if
\begin{subequations}
\begin{alignat}{2}
   f(x) & \le \liminf\nolimits_n  f_n(x_n), & \qquad &
   \text{for every sequence $x_n \to x$}
   \label{eq:epiconvergence-every}
   \\
   f(x) & \ge \limsup\nolimits_n  f_n(x_n), & \qquad &
   \text{for some sequence $x_n \to x$}
   \label{eq:epiconvergence-some}
\end{alignat}
\end{subequations}
and $f_n$ \emph{epiconverges} to $f$ if it epiconverges at every $x$.
This is denoted $f_n \eto f$.

If we changed \eqref{eq:epiconvergence-some} so it said ``every'' instead
of ``some'' this would define uniform convergence on compact sets
(also called continuous convergence) \citep[Section~7.C]{rockafellar-wets}.

As it is \eqref{eq:epiconvergence-some} is weaker than what pointwise
convergence says: that \eqref{eq:epiconvergence-some} holds when $x_n = x$
for all $n$.

Thus epiconvergence provides the same guarantee from below as uniform
convergence on compact sets, but a weaker guarantee from above than
pointwise convergence.  As with replacing continous with LSC, this is
just what is needed for minimization.
\begin{theorem} \label{th:attouch}
Suppose $f_n \eto f$ and $x_n \to x$ and $f_n(x_n) - \inf f_n \to 0$.
Then
\begin{itemize}
\item[\normalfont (a)] $f(x) = \inf f$.
\item[\normalfont (b)] $f_n(x_n) \to f(x)$.
\end{itemize}
\end{theorem}
This is Proposition~{3.1} in \citet{constrained-m-estimation} which
mostly comes from Theorem~{1.10} in \citet{attouch}.

It may seem strange that our theorem assumes $x_n \to x$ when we want
that to be a conclusion rather than an assumption.  How one uses the
theorem is that if we assume $x_n$ is a bounded sequence, then it has
convergent subsequences to which the theorem can be applied.  Then
we use the subsequence principle (if every convergent subsequence converges
to the same limit, then the whole sequence converges to that limit) to
conclude $x_n \to x$.  The problem is obtaining boundedness (no escape
to infinity) and uniqueness ($f$ has a unique minimizer).  Neither is
easy.  We will not be able to prove either: that our algorithm (still
to be described) has a bounded sequence of iterates or that our objective
function \eqref{eq:mlogl-pickle} has a unique minimizer.
So Theorem~\ref{th:attouch} is often the best we can do.

\begin{theorem} \label{th:pee-epi}
Suppose $W_n \to W$, then $p_{W_n} \eto p_{W}$.
\end{theorem}
In the theorem statement $W_n \to W$ means
componentwise convergence of matrices.
\begin{proof}
We consider separately the three terms of \eqref{eq:pee-lsc}.
\begin{align*}
   p_1(\alpha, b, \nu)
   & =
   - l(a + M \alpha + Z b)
   \\
   p_2(\alpha, b, \nu)
   & =
   \tfrac{1}{2} \sum_{(j, k) \in \mathcal{J}} h(b_j, \nu_k)
   \\
   p_{3, W}(\alpha, b, \nu)
   & =
   \tfrac{1}{2} \log \det \bigl[ Z^T W Z D + \text{Id} \bigr]
\end{align*}
Note that $p_1$ does not actually depend on $\nu$ or $W$ so we are
considering constant sequences for it as $W_n \to W$.  Since $p_1$
is a continuous function of its arguments, we have continuous convergence
for these constant sequences.

Also note that $p_2$ does not actually depend on $\alpha$ or $W$ so we are
considering constant sequences for it as $W_n \to W$.  Since each term of
its definition is LSC, the sum multiplied by a positive scalar is LSC
\citet[Theorem~1.39]{rockafellar-wets}, that is, $p_2$ is LSC.
Since a constant sequence of functions $f$ epiconverges to $\closure f$,
we have epiconvergence of these constant sequences.

Now $p_{3, W}$ does not actually depend on $\alpha$ or $b$.  And its
defining formula is a continuous function of $\nu$ and $W$ so
$$
   p_{3, W_n}(\nu_n) \to p_{3, W}(\nu), \qquad \text{as $W_n \to W$
   and $\nu_n \to \nu$}
$$
This is continuous convergence $p_{3, W_n} \to p_{3. W}$.

Now we apply (twice) that the sum of an epiconverging sequence
and a continuously converging sequence epiconverges
\citep[Theorem~7.46 (b)]{rockafellar-wets}.
\end{proof}

\subsection{Algorithm}

This gives us an algorithm.
\begin{enumerate}
\item Initialize $W$ to be a positive definite symmetric matrix.
\item Find $(\alpha, b, \nu)$ that minimize $p$ given by \eqref{eq:pee-lsc}.
\item Set $W$ to $- l''(a + M \alpha + Z b)$.
\item If the sequence of iterates $(\alpha, b, \nu, W)$ appears
    to have converged, stop.  Otherwise, go to step 2.
\end{enumerate}

This is cheating a bit in several ways.
\begin{itemize}
\item Theorem~\ref{th:attouch} is about global minimizers, but all the
    optimization algorithms we have access to only find local minimizers
    (p given by \eqref{eq:pee-lsc} is not convex).
\item We have not stated what our convergence criteria are.
\end{itemize}

Nevertheless, Theorems~\ref{th:attouch} and~\ref{th:pee-epi} do say
that if we are finding global minimizers in Step 2 and if the iterates
converge, then they converge to the global minimizer of
\eqref{eq:mlogl-pickle}.

Note that, despite what might appear at first sight.  This is an algorithm
for minimizing the Laplace approximation $q$ given by \eqref{eq:mlogl-pickle}
not for minimizing $p_W$.

\section{Derivatives}

\REVISED

\section{What Happens When Variance Components are Zero?}

\subsection{Square Root Parameterization}

\citet{reaster} suggest a re-parameterization that makes the problem with
zero variance components go away.
The original variables are $(\alpha, b, \nu)$, and
the new variables are $(\alpha, c, \sigma)$, where
\begin{alignat*}{2}
   \nu_k & = \sigma^2_k, & \qquad & \text{for all $k$}
   \\
   b_j & = \sigma_k c_j, & & \text{whenever $\var(b_j) = \nu_k$}
\end{alignat*}
The inverse transformation is not one-to-one.  We allow $\sigma_k$ to be
either square root of $\nu_k$, that is, $\sigma_k = \pm \sqrt{\nu_k}$.
We do this in order that our objective function be continuous and
differentiable at zero.



\appendix


\chapter{The Factorization Theorem}
\label{app:factorize}

\begin{proof}[Proof of Theorem~\ref{th:factorize}]
A valid factorization factors joint equals conditional times marginal
$$
   \pr(y) = \pr(y_{G_1} \mid y_{N \setminus G_1}) \pr(y_{N \setminus G_1})
$$
The marginal on the right-hand side can then be considered a joint to be
factored further
$$
   \pr(y)
   =
   \pr(y_{G_1} \mid y_{N \setminus G_1})
   \pr(y_{G_2} \mid y_{N \setminus (G_1 \cup G_2)})
   \pr(y_{N \setminus (G_1 \cup G_2)})
$$
and again and again giving
\begin{equation} \label{eq:factorize-general}
   \pr(y)
   =
   \pr(y_{N \setminus \bigcup_{j = 1}^k G_j})
   \prod_{i = 1}^k
   \pr(y_{G_i} \mid y_{N \setminus \bigcup_{j = 1}^i G_j})
\end{equation}
and the only condition that is required to make \eqref{eq:factorize-general}
valid is that the index sets $G_i$ are disjoint.  This is the only operation
in classical (non-measure-theoretic) probability theory that factorizes
probability distributions.  A factorization is valid if and only if it
has the form \eqref{eq:factorize-general}.

When we match up \eqref{eq:factorize} and \eqref{eq:factorize-general}
we see that the $G_i$ must be the elements of $\mathcal{G}$ so the two
products are the same.  For the conditional distributions to
match up we must have $\pr(y_{G_i} \mid y_{N \setminus \bigcup_{j = 1}^i G_j})$
in \eqref{eq:factorize-general} can actually be written as
$\pr(y_{G_i} \mid y_{q(G_i)})$, that is,
\begin{itemize}
\item this conditional distribution actually depends only on the single
    variable $y_{q(G_i)}$ not on the rest of the variables that are components
    of $y_{N \setminus \bigcup_{j = 1}^i G_j}$ and
\item $q(G_i) \in N \setminus \bigcup_{j = 1}^i G_j$, that is, either
    $q(G_i) \in G_j$ for some $j > i$ or $q(G_i)$ is an initial node
    ($q(G_i) \notin G_j$ for any $j$).  In either case,
    $q(G_i) \in G_j$ implies $i < j$.  Thus we have the condition of
    the theorem: $G_i < G_j$ if and only if $i < j$.
\end{itemize}
Finally, we must match up the marginal term on the right-hand side of
\eqref{eq:factorize-general}.  It matches nothing in \eqref{eq:factorize},
which is the same as saying it must be equal to one, which is they same as
saying $y_{N \setminus J}$ is a constant random vector,
where $J = \bigcup \mathcal{G}$ as always.
\end{proof}




\chapter{Markov Properties}
\label{app:markov}

Markov properties of graphical models are considered a fundamental part
of the theory \citep[Chapter~3]{lauritzen}.  They are much less important
for aster models, so unimportant that the literature on aster models
does not mention them.
So perhaps most readers will want to skip this appendix.  Nevertheless,
perhaps these ideas might find some future use.  So we do them.

A Markov property is a conditional independence relation derived from a
\index{aster model!property!Markov}
\index{Markov property|seeunder{aster model}}
graph (or for aster models from the fundamental factorization
\eqref{eq:factorize}).
There are many more Markov properties than we bother to prove here.

\begin{lemma} \label{lem:markov}
Let $\mathcal{H}$ be any subset of $\mathcal{G}$.  Then the random vectors
$y_H$, $H \in \mathcal{H}$ are conditionally independent given
the random scalars $y_{q(H)}$, $H \in \mathcal{H}$.
\end{lemma}

Note that some $y_j$ can possibly appear among some $H \in \mathcal{H}$
and in some $y_{q(H)}$, $H \in \mathcal{H}$ so we have to say what that means.
Conditioning on a random variable is the same as treating it as constant,
and a constant random variable is independent of any random variables
including itself.  Thus for any sets $A$ and $B$, we have
\begin{equation} \label{eq:before-and-after}
   \pr(y_A \mid y_B)
   =
   \pr(y_{A \setminus B} \mid y_B).
\end{equation}

In \eqref{eq:before-and-after}, the case $A \setminus B = \emptyset$
is possible, in which case $y_\emptyset$ is the constant random vector
discussed in Section~\ref{sec:subvector} above.
Thus $\pr(y_\emptyset \mid y_B) = 1$ regardless of what $y_B$ is.

This lemma does not say that the components of the random vectors $y_H$ are
conditionally independent.  The components of $y_G$ are dependent given
$y_{q(G)}$ for any $G$.  That is the whole point of dependence groups.

\begin{proof}
Use the total order on $\mathcal{G}$ guaranteed
to exist by Theorem~\ref{th:factorize}
to enumerate $\mathcal{G}$ as $G_1 < G_2 < \cdots < G_n$.
Then $H_j = G_{i_j}$ for $j = 1$, $\ldots,$ $m$,
where $1 \le i_1 < i_2 < \cdots < i_m \le n$.

We integrate out $y_G$ one at a time
in order skipping when $G \in \mathcal{H}$ and also not integrating out
any $y_{q(H)}$ for $H \in \mathcal{H}$.

We start by sum-integrating out $y_{G_1}$ if $G_1 \neq H_1$ obtaining
$$
   \pr(y_{G_2 \cup \cdots \cup G_n})
   =
   \prod_{i = 2}^n \pr(y_{G_i} \mid y_{q(G_i)}).
$$
and keep going repeating this again and again obtaining
\begin{align*}
   \pr(y_{\bigcup \set{G \in \mathcal{G} : G \ge H_1}})
   & =
   \prod_{\substack{G \in \mathcal{G} \\ G \ge H_1}}
   \pr(y_G \mid y_{q(G)})
   \\
   & =
   \pr(y_{H_1} \mid y_{q(H_1)})
   \prod_{\substack{G \in \mathcal{G} \\ G > H_1}}
   \pr(y_G \mid y_{q(G)})
\end{align*}
(if $G_1 = H_1$ we haven't done anything yet and this is just the same
factorization as \eqref{eq:factorize} in different notation).

Now we have to be careful with our notation.  Define
$$
   Q = \set{ q(H) : H \in \mathcal{H} }
$$
we need to not sum-integrate out any components of $y_Q$.

If $G_{i_1 + 1} \neq H_2$, then we want to sum-integrate out
$y_{G_{i_1 + 1} \setminus Q}$ obtaining
\begin{multline*}
   \pr(y_{H_1 \cup \{q(H_1)\} \cup \set{G \in \mathcal{G} : G > G_{i_1 + 1}}})
   \\
   =
   \pr(y_{H_1} \mid y_{q(H_1)})
   \pr(y_{G_{i_1 + 1} \cap Q} \mid y_{q(G_{i_1 + 1})})
   \prod_{j = i_1 + 2}^n
   \pr(y_{G_i} \mid y_{q(G_i)})
\end{multline*}
(this uses the discussion of $y_\emptyset$ preceding this proof,
since $G_{i_1 + 1} \cap Q$ may or may not be the empty set).

Continuing this process, we obtain
\begin{multline*}
   \pr(y_{H_1 \cup \{q(H_1)\} \cup \{ G_{i_2}, \ldots, G_n \}})
   \\
   =
   \pr(y_{H_1} \mid y_{q(H_1)})
   \prod_{j = i_1 + 1}^{i_2 - 1}
   \pr(y_{G_j \cap Q} \mid y_{q(G_j)})
   \prod_{j = i_2}^n
   \pr(y_{G_i} \mid y_{q(G_i)}).
\end{multline*}
And we can now see how this process continues
$$
   \pr(y_{H_1 \cup H_2 \cup \cdots \cup H_m \cup Q})
   =
   \prod_{H \in \mathcal{H}}
   \pr(y_H \mid y_{q(H)})
   \prod_{G \in \mathcal{G} \setminus \mathcal{H}}
   \pr(y_{G \cap Q} \mid y_{q(G)})
$$
And now integrating out $y_{H \setminus Q}$ in order gives
\begin{align*}
   \pr(y_Q)
   & =
   \prod_{H \in \mathcal{H}}
   \pr(y_{H \cap Q} \mid y_{q(H)})
   \prod_{G \in \mathcal{G} \setminus \mathcal{H}}
   \pr(y_{G \cap Q} \mid y_{q(G)})
\end{align*}
(Note that every $y_j$ for $j \in Q$ appears ``in front of the bar'' in
exactly one of these conditional probabilities because $\mathcal{G}$
is a partition.)
So
\begin{align*}
   \pr(y_{H_1 \cup H_2 \cup \cdots \cup H_m} \mid y_Q)
   & =
   \prod_{H \in \mathcal{H}}
   \frac{ \pr(y_H \mid y_{q(H)}) }{ \pr(y_{H \cap Q} \mid y_{q(H)}) }
   \\
   & =
   \prod_{H \in \mathcal{H}}
   \frac{ \pr(y_{H \cup \{q(H)\}}) }{ \pr(y_{(H \cap Q) \cup \{q(H)\}}) }
   \\
   & =
   \prod_{H \in \mathcal{H}}
   \pr( y_{H \setminus Q} \mid y_{Q} )
   \\
   & =
   \prod_{H \in \mathcal{H}}
   \pr( y_H \mid y_{Q} )
\end{align*}
the last step being \eqref{eq:before-and-after}.
\end{proof}

If $\mathcal{G}$ and $\mathcal{H}$ are partitions of a set $J$,
then we say that $\mathcal{G}$ is \emph{finer} than $\mathcal{H}$
if every element of $\mathcal{G}$ is contained in some element
of $\mathcal{H}$.  We also say that $\mathcal{H}$ is \emph{coarser}
than $\mathcal{G}$ to indicate the same concept.

Clearly, every element of $\mathcal{H}$ is the union of elements
of $\mathcal{G}$ it contains (because $\mathcal{G}$ is a partition).

\begin{theorem} \label{th:markov}
Suppose $\mathcal{G}$ and $q$ are as in \eqref{eq:factorize}
and Theorem~\ref{th:factorize}, and suppose $\mathcal{H}$ is a coarser
partition than $\mathcal{G}$.  Define
\begin{equation} \label{eq:q-markov}
   Q = \set{ q(G) : (G, H) \in \mathcal{G} \times \mathcal{H}
   \opand G \subset H \opand q(G) \notin H }
\end{equation}
then the random vectors $y_H$, $H \in \mathcal{H}$, are conditionally
independent given the random vector $y_Q$.
\end{theorem}

Note that \eqref{eq:before-and-after} is being used in this theorem too.
Some $y_j$ may appear in some $y_H$ and also in $y_Q$.

Also we repeat the comment following the lemma.  The theorem does not
assert conditional independence of the components of $y_H$ for any $H$.
The components of $y_G$ being dependent given $y_{q(G)}$ is the whole
point of dependence groups.

\begin{proof}
We prove this by induction. The induction variable is the partition
$\mathcal{H}$. We start with $\mathcal{H} = \mathcal{G}$.
Then we change $\mathcal{H}$ to coarser and coarser
partitions until we get to the $\mathcal{H}$ in the theorem statement.

The base of the induction is the case $\mathcal{H} = \mathcal{G}$
in which case the lemma and the theorem say the same thing.
So that establishes the base of the induction.

In each induction step we decrease the cardinality of $\mathcal{H}$ by one.
This means we take two elements $H'$ and $H''$ of $\mathcal{H}$ and merge
them to make one element of $\mathcal{H}$ after the induction step,
and all other elements of $\mathcal{H}$ remain
unchanged (in this particular induction step). We need to show that if the
assertion of the theorem is true before the induction step, then it is true
after the induction step, when $\mathcal{H}$ is changed as described.

Let $\mathcal{H}_\text{before}$ denote $\mathcal{H}$ before the induction step
and $\mathcal{H}_\text{after}$ denote $\mathcal{H}$ after
the induction step, so all elements of $\mathcal{H}_\text{before}$
and $\mathcal{H}_\text{after}$ are the same except
\begin{itemize}
\item $\mathcal{H}_\text{before}$ has elements $H'$ and $H''$ which are not
    in $\mathcal{H}_\text{after}$ and
\item $\mathcal{H}_\text{after}$ has the element $H' \cup H''$ which is not
    in $\mathcal{H}_\text{before}$.
\end{itemize}
Let $Q_\text{before}$ denote $Q$ before the induction step and $Q_\text{after}$
denote $Q$ after the induction step, so all of the elements
of $Q_\text{before}$ and $Q_\text{after}$ are the same except
\begin{itemize}
\item $q(G)$, $G \in \mathcal{G}$ such that $G \subset H' \cup H''$
    and $q(G) \in H' \cup H''$ are not in $Q_\text{after}$.
    (Some of these may not have been in $Q_\text{before}$ either.)
\end{itemize}

In case $Q_\text{before} = Q_\text{after}$ there is nothing to prove.
Conditional independence of $y_H$, $H \in \mathcal{H}_\text{before}$
given $y_{Q_\text{before}}$ clearly implies conditional independence
of $y_H$, $H \in \mathcal{H}_\text{after}$
given $y_{Q_\text{after}}$ in this case
where $Q_\text{before} = Q_\text{after}$.
The latter statement just forgets part of the assertion of the former.
(It forgets about conditional independence of $y_{H'}$ and $y_{H''}$.)

In case $Q_\text{before} \neq Q_\text{after}$ there is more work to be done.
The induction hypothesis says
$$
   \pr(y \mid y_{Q_\text{before}})
   =
   \prod_{H \in \mathcal{H}_\text{before}}
   \pr(y_H \mid y_{Q_\text{before}}).
$$
First we notice
\begin{equation} \label{eq:notice-too-before-after}
   Q_\text{before} \setminus Q_\text{after} \subset H' \cup H''
\end{equation}
so by the induction hypothesis
\begin{equation} \label{eq:notice-before-after}
   \pr(y_H \mid y_{Q_\text{before}})
   =
   \pr(y_H \mid y_{Q_\text{after}}),
   \qquad
   H \in \mathcal{H}_\text{before} \cap \mathcal{H}_\text{after}.
\end{equation}
Now
\begin{multline*}
   \pr(y_{H'} \mid y_{Q_\text{before}})
   \pr(y_{H''} \mid y_{Q_\text{before}})
   \pr(y_{Q_\text{before}})
   \\
   =
   \pr(y_{H' \cup H''} \mid y_{Q_\text{before}})
   \pr(y_{Q_\text{before}})
   \\
   =
   \pr(y_{H' \cup H'' \cup Q_\text{before}})
   \\
   =
   \pr(y_{H' \cup H'' \cup Q_\text{after}})
\end{multline*}
the first equality being the conditional independence asserted
by the induction hypothesis and the last equality being
\eqref{eq:notice-too-before-after}.  So
\begin{align*}
   \pr(y_{H' \cup H''} \mid  y_{Q_\text{after}})
   & =
   \frac{\pr(y_{H' \cup H'' \cup Q_\text{after}})}{\pr(y_{Q_\text{after}})}
   \\
   & =
   \pr(y_{H'} \mid y_{Q_\text{before}})
   \pr(y_{H''} \mid y_{Q_\text{before}})
   \frac{\pr(y_{Q_\text{before}})}{\pr(y_{Q_\text{after}})}
   \\
   & =
   \pr(y_{H'} \mid y_{Q_\text{before}})
   \pr(y_{H''} \mid y_{Q_\text{before}})
   \pr(y_{Q_\text{before}} \mid y_{Q_\text{after}})
\end{align*}
By a similar argument we have
\begin{align*}
   \pr(y \mid  y_{Q_\text{after}})
   & =
   \pr(y_{Q_\text{before}} \mid y_{Q_\text{after}})
   \prod_{H \in \mathcal{H}_\text{before}}
   \pr(y_H \mid y_{Q_\text{before}})
   \\
   & =
   \pr(y_{H'} \mid y_{Q_\text{before}})
   \pr(y_{H''} \mid y_{Q_\text{before}})
   \pr(y_{Q_\text{before}} \mid y_{Q_\text{after}})
   \\
   & \qquad
   \times
   \prod_{H \in \mathcal{H}_\text{before} \cap \mathcal{H}_\text{after}}
   \pr(y_H \mid y_{Q_\text{before}})
   \\
   & =
   \pr(y_{H' \cup H''} \mid  y_{Q_\text{after}})
   \prod_{H \in \mathcal{H}_\text{before} \cap \mathcal{H}_\text{after}}
   \pr(y_H \mid y_{Q_\text{before}})
   \\
   & =
   \pr(y_{H' \cup H''} \mid  y_{Q_\text{after}})
   \prod_{H \in \mathcal{H}_\text{before} \cap \mathcal{H}_\text{after}}
   \pr(y_H \mid y_{Q_\text{after}})
   \\
   & =
   \prod_{H \in \mathcal{H}_\text{after}}
   \pr(y_H \mid y_{Q_\text{after}})
\end{align*}
where the next-to-last step is \eqref{eq:notice-before-after}.
And reading from end to end gives the assertion that the induction step
must prove.  Hence we are done.
\end{proof}

Let
$$
   \mathcal{G}_\text{initial}
   =
   \set{ G \in \mathcal{G} : q(G) \notin J }
$$
(the notation is perhaps a bit misleading, this is the subset of $\mathcal{G}$
whose elements have predecessors that are initial nodes).
Now for $G \in \mathcal{G}_\text{initial}$, let
$$
   H_G = \set{ j \in J : (\exists k \in G)(j \succeq k) }
$$
where, as usual $\succeq$ denotes the reflexive transitive closure of the
predecessor relation.  These $H_G$ are the node sets for what are called
aster graphs for ``individuals'' (in scare quotes)
\index{aster graph!for ``individual''}
in Section~\ref{sec:scare-quotes} above.
Let
\begin{equation} \label{eq:individuals}
   \mathcal{H} = \set{ H_G : G \in \mathcal{G}_\text{initial} }.
\end{equation}

\begin{corollary} \label{cor:markov}
The random vectors $y_H$, $H \in \mathcal{H}$, with $\mathcal{H}$
defined by \eqref{eq:individuals} are (unconditionally) independent.
\end{corollary}
\begin{proof}
Immediate from the theorem because the $Q$ corresponding to this $\mathcal{H}$
consists of initial nodes only so $y_Q$ is a constant random vector, and
conditioning on a constant has no effect.  Any things conditionally independent
given $y_Q$ are unconditionally independent given $y_Q$.
\end{proof}





\chapter{Regularity}
\label{app:regular}

\index{exponential family!full}
\index{exponential family!regular}
As mentioned in Section~\ref{sec:aster-expfam} where the exponential
family assumption for aster models was introduced, the cumulant function
for the degenerate family concentrated at zero is the zero function
that is everywhere equal to zero.  The family consisting of this distribution
only is a regular full exponential family because $c_G$ is everywhere finite.
So the full canonical parameter space of this family is $\real^G$.

\begin{theorem} \label{th:regular}
If $y_{q(G)} = 0$ almost surely for any dependence group $G$, replace
the family for this dependence group by the degenerate family concentrated
at zero so $c_G$ is the zero function.
Then, if families for every dependence group of the aster model are regular full
exponential families, then so is the (joint) distribution of the aster model.
The full (unconditional) canonical parameter space of the aster model
is the range of the aster transform.  The cumulant function of the
aster model is given by \eqref{eq:aster-cumfun} for parameter values where
it is finite.
\end{theorem}

Let $\Theta_G$ denote the full canonical parameter space of the exponential
family for dependence group $G$, which is the set of points where $c_G$
is finite.  The the set of all $\theta$ values that correspond to possible
distributions in the aster model is
\begin{equation} \label{eq:reg-prod}
   \Theta = \prod_{G \in \mathcal{G}} \Theta_G
\end{equation}
where the product denotes Cartesian product: this is the set of all $\theta$
such that $\theta_G \in \Theta_G$ for all $G$.
If we temporarily give the aster transform a letter $f$, then
the range of this function is denoted $\Phi = f(\Theta)$.
This is the set of all vectors $\varphi$ that correspond to vectors $\theta$
that parameterize distributions in the aster model.

Note that we don't have an explicit description of $\Phi$.
We don't even have a closed-form expression for $f$, only the recursive
definition \eqref{eq:aster-transform}.  But we know that $f$ is a function
having domain $\Theta$ and range $\Phi$, and the inverse aster transform
is a function having domain $\Phi$ and range $\Theta$.

One assertion of the theorem is that when we calculate the cumulant
function of the (joint) distribution of the aster model
using \eqref{eq:cumfun-expfam} the result is finite if and only if
$\varphi \in \Phi$ and when it is finite the result agrees with
\eqref{eq:aster-cumfun}.
Another assertion of the theorem is that $\Phi$ is an open subset of
the vector space where $\varphi$ takes values.

\begin{proof}
From \eqref{eq:cumfun-expfam}
$$
   c(\varphi)
   =
   c(\varphi^*) + \log\left\{
   E_{\varphi^*}\bigl( e^{\inner{y, \varphi - \varphi^*}} \bigr) \right\}
$$
or
\begin{equation} \label{eq:cumfun-reg-one}
   e^{c(\varphi) - c(\varphi^*)}
   =
   E_{\varphi^*}\bigl( e^{\inner{y, \varphi - \varphi^*}} \bigr)
\end{equation}
(what were $\theta$ and $\psi$ in \eqref{eq:cumfun-expfam} have
become $\varphi$ and $\varphi^*$, respectively,
here because we want to emphasize that they are both possible values
of the unconditional canonical parameter vector).
Let $\theta$ and $\theta^*$ denote the conditional canonical parameter
vectors corresponding to $\varphi$ and $\varphi^*$, respectively.

We also note that \eqref{eq:cumfun-expfam} holds for each dependence group
\begin{equation} \label{eq:cumfun-reg-too}
   e^{c_G(\theta_G) - c(\theta_G^*)}
   =
   E_{\varphi^*}\left(
   e^{\inner{y_G, \theta_G - \theta_G^*}} \middle| y_{q(G)} = 1 \right)
\end{equation}
and since the cumulant function for sample size $n$ is $n$ times the cumulant
function for sample size one
\begin{equation} \label{eq:cumfun-reg-too-too}
   e^{y_{q(G)} [c_G(\theta_G) - c(\theta_G^*)]}
   =
   E_{\varphi^*}\left(
   e^{\inner{y_G, \theta_G - \theta_G^*}} \middle| y_{q(G)} \right)
\end{equation}

Use the total order on $\mathcal{G}$ guaranteed
to exist by Theorem~\ref{th:factorize}
to enumerate $\mathcal{G}$ as $G_1 < G_2 < \cdots < G_n$,
and for $k = 0$, $\ldots,$ $n$ define
$$
   \mathcal{G}_k = \set{ G_1, \ldots, G_k }
$$
where the notation is intended to mean that $\mathcal{G}_0$ is another
notation for the empty set.  We claim
\begin{multline} \label{eq:reg-induction}
   E_{\varphi^*}\bigl( e^{\inner{y, \varphi - \varphi^*}} \bigr)
   \\
   =
   E_{\varphi^*}\left(
   \prod_{\substack{G \in \mathcal{G}_k \\ q(G) \notin \bigcup \mathcal{G}_k}}
   e^{y_{q(G)} [c_G(\theta_G) - c_G(\theta_G^*)]}
   \prod_{G \in \mathcal{G} \setminus \mathcal{G}_k}
   e^{\inner{y_G, \varphi_G - \varphi_G^*}}
   \right)
\end{multline}
hold for $k = 0$, $\ldots,$ $n$ and we prove this by induction.

The base of the induction is the case $k = 0$ in which case the
first product is empty and by convention equal to one.
Then \eqref{eq:reg-induction} is obviously equivalent
to \eqref{eq:cumfun-reg-one}.

To prove the induction step we assume \eqref{eq:reg-induction}
and prove \eqref{eq:reg-induction} with $k$ replaced by $k + 1$.
Note that \eqref{eq:aster-transform} says
\begin{equation} \label{eq:reg-aster-transform}
   \theta_j - \theta_j^*
   =
   \varphi_j - \varphi_j^*
   +
   \sum_{\substack{G \in \mathcal{G} \\ q(G) = j}}
   [ c_G(\theta_G) - c_G(\theta_G^*) ]
\end{equation}
so
\begin{multline*}
   E_{\varphi^*}\left(
   \prod_{\substack{G \in \mathcal{G}_k \\ q(G) \notin \bigcup \mathcal{G}_k}}
   e^{y_{q(G)} [c_G(\theta_G) - c_G(\theta_G^*)]}
   \prod_{G \in \mathcal{G} \setminus \mathcal{G}_k}
   e^{\inner{y_G, \varphi_G - \varphi_G^*}}
   \right)
   =
   \\
   E_{\varphi^*}\left(
   \prod_{\substack{G \in \mathcal{G}_k \\
       q(G) \notin \bigcup \mathcal{G}_{k + 1}}}
   e^{y_{q(G)} [c_G(\theta_G) - c_G(\theta_G^*)]}
   \prod_{G \in \mathcal{G} \setminus \mathcal{G}_{k + 1}}
   e^{\inner{y_G, \varphi_G - \varphi_G^*}}
   e^{\inner{y_{G_k}, \theta_{G_k} - \theta_{G_k}^*}}
   \right)
   \\
   =
   E_{\varphi^*}\left(
   \prod_{\substack{G \in \mathcal{G}_k \\
       q(G) \notin \bigcup \mathcal{G}_{k + 1}}}
   e^{y_{q(G)} [c_G(\theta_G) - c_G(\theta_G^*)]}
   \prod_{G \in \mathcal{G} \setminus \mathcal{G}_{k + 1}}
   e^{\inner{y_G, \varphi_G - \varphi_G^*}}
   \right.
   \\
   \times
   \left.
   \vphantom{\prod_{\substack{G \in \mathcal{G}_{k + 1} \\
       q(G) \notin \bigcup \mathcal{G}_{k + 1}}}}
   E_{\varphi^*}\left\{
   e^{\inner{y_{G_k}, \theta_{G_k} - \theta_{G_k}^*}}
   \middle| y_{\bigcup (\mathcal{G} \setminus \mathcal{G}_k)}
   \right\}
   \right)
\end{multline*}
and this is equal to \eqref{eq:reg-induction} with $k$ replaced by $k + 1$
by the Markov property
$$
   E_{\varphi^*}\left\{
   e^{\inner{y_{G_k}, \theta_{G_k} - \theta_{G_k}^*}}
   \middle| y_{\bigcup (\mathcal{G} \setminus \mathcal{G}_k)}
   \right\}
   =
   E_{\varphi^*}\left\{
   e^{\inner{y_{G_k}, \theta_{G_k} - \theta_{G_k}^*}}
   \middle| y_{q(G_k)}
   \right\}
$$
and \eqref{eq:cumfun-reg-too-too}.
That finishes the proof of the induction claim.

The $k = n$ case of \eqref{eq:reg-induction} is
$$
   E_{\varphi^*}\bigl( e^{\inner{y, \varphi - \varphi^*}} \bigr)
   =
   E_{\varphi^*}\left(
   \prod_{\substack{G \in \mathcal{G} \\ q(G) \notin J}}
   e^{y_{q(G)} [c_G(\theta_G) - c_G(\theta_G^*)]}
   \right)
$$
and because every $y_{q(G)}$ appearing in the expectation is a constant
random variable at an initial node, the expectation does nothing, so
$$
   E_{\varphi^*}\bigl( e^{\inner{y, \varphi - \varphi^*}} \bigr)
   =
   \prod_{\substack{G \in \mathcal{G} \\ q(G) \notin J}}
   e^{y_{q(G)} [c_G(\theta_G) - c_G(\theta_G^*)]}
$$
so by \eqref{eq:cumfun-reg-one}
$$
   c(\varphi) = c(\varphi^*)
   + \sum_{\substack{G \in \mathcal{G} \\ q(G) \notin J}}
   y_{q(G)} [c_G(\theta_G) - c_G(\theta_G^*)]
$$
Now \eqref{eq:cumfun-expfam} only determines the cumulant function up
to an arbitrary constant, and here all of the starred parameters are
constant, so this does agree with \eqref{eq:aster-cumfun} up to an
arbitrary constant (which is all it can do).

We have now established that \eqref{eq:aster-cumfun} gives the
cumulant function of the (unconditional, joint) distribution of the
aster model when the parameter vectors in that formula are in the
parameter space.

To prove the assertion of the theorem about when the cumulant function
of the (unconditional, joint) distribution of the aster model is infinite,
we need all the cases of \eqref{eq:reg-induction} for $k = 1$, $\ldots,$ $n$.
Since $\theta^*$ and $\varphi^*$ must be valid parameter vectors,
$c_G(\theta_G^*)$ is always finite.
In \eqref{eq:reg-induction} it is unclear which $G$ the first product
runs over (it depends on the graph, or, alternatively, on the predecessor
function), but we always know that $G_k \notin \bigcup \mathcal{G}_k$
because that is the way the total order on $\mathcal{G}$ works:
$q(G_k)$ must either be an initial node or must be in some $G_m$ with $k < m$.
Thus case $k$ of \eqref{eq:reg-induction} tells us the expression is
infinite if $c_{G_k}(\theta_{G_k}) = \infty$ and $y_{q(G_k)}$ is not
zero almost surely.  But the former implies the latter cannot happen
by the first sentence of the theorem statement: if $y_{q(G_k)} = 0$
almost surely, then $c_{G_k}$ is the zero function.

Putting these statements together for all $k$, we see that
\eqref{eq:reg-induction} is infinite whenever $\theta \notin \Theta$,
where $\Theta$ is given by \eqref{eq:reg-prod}.
Putting together everything we have proved so far,
the cumulant function for the (unconditional, joint) distribution of
the aster model is finite and given by \eqref{eq:aster-cumfun} for
$\theta$ in \eqref{eq:reg-prod} and is infinite for
$\theta$ not in \eqref{eq:reg-prod}.

Now letting $f$ denote the aster transform as in the comments immediately
preceding the theorem statement, we have also shown that
the cumulant function for the (unconditional, joint) distribution of
the aster model is finite and given by \eqref{eq:aster-cumfun} for
$\varphi$ in $\Phi = f(\Theta)$ and is infinite for
$\varphi$ not in $\Phi = f(\Theta)$.

By assumption every $\Theta_G$ is an open set in the vector space
containing it.  Hence, a Cartesian product of open sets being an open set,
$\Theta$ is an open set in the vector space containing it.
We know that the aster transform and its inverse are (infinitely)
differentiable hence continuous.  Hence for any $\varphi \in \Phi$
the point $\theta = f^{-1}(\varphi)$ is in the interior of $\Theta$
(because $\Theta$ is an open set), hence there is a neighborhood $W$
of $\theta$ contained in $\Theta$, but then $f(W)$ is a neighborhood of
$\varphi$ contained in $\Phi$.  Thus $\Phi$ is a neighborhood of each
of its points, hence an open subset of the vector space containing it.

Thus the (unconditional, joint) distribution of the aster model
is a regular full exponential family.
\end{proof}

The first sentence of the theorem statement is for limiting conditional
models.
%%%%%%%%%% NEED CROSS REFERENCE to limiting conditional model %%%%%%%%%%
An aster model in which no family is degenerate and no initial node is zero
does not need this first sentence: we never have $y_j = 0$ almost surely
for any $j$.  This is easily proved by induction, but we won't bother
because limiting conditional models are a thing, so the theorem needs
to be stated the way it is.



\chapter{Families}

\section{Bernoulli}
\label{sec:bernoulli}

A random variable is \emph{Bernoulli} if its possible values are zero and one.
In other words, every Bernoulli random variable is zero-or-one-valued,
and vice versa.

This is the \emph{rationale} for the distribution, any dichotomous (two-valued)
random variable can be coded as Bernoulli.

This is a \emph{discrete} random variable.

This is a special case of the binomial distribution, which we do next.

\section{Binomial}
\label{sec:binomial}

A random variable is \emph{binomial} if it is the sum of IID Bernoulli
random variables.  Hence the Bernoulli distribution is the binomial
distribution for sample size one (for one term in the sum).

The \emph{probability mass function} is
\begin{equation} \label{eq:binomial-pmf}
   f(y) = \binom{n}{y} p^y (1 - p)^{n - y}, \qquad y = 0, 1, \ldots, n,
\end{equation}
where $p$ is the \emph{usual parameter}, the probability that any of the
$n$ Bernoulli random variables in the sum is equal to one.

The \emph{mean} and \emph{variance} are
\begin{align*}
   E(y) & = n p
   \\
   \var(y) & = n p (1 - p)
\end{align*}

This is an \emph{exponential family}.  From \eqref{eq:binomial-pmf}
the log likelihood is
$$
   l(\theta) = y \log(p) + (n - y) \log(1 - p)
   = y \cdot \log\left(\frac{p}{1 - p}\right) + n \log(1 - p)
$$
from which we see that we have an exponential family with
\emph{canonical statistic} $y$ and \emph{canonical parameter}
$$
   \theta = \log\left(\frac{p}{1 - p}\right)
$$
The right-hand side is so important that it is given a name.  The
$\logit$ function (pronounced low-jit) is given by
$$
   \logit(p) = \log\left(\frac{p}{1 - p}\right), \qquad 0 < p < 1.
$$
Its inverse function is
$$
   \logit^{-1}(\theta) = \frac{e^\theta}{1 + e^\theta},
   \qquad - \infty < \theta < \infty.
$$

The \emph{cumulant function} is
\begin{align*}
   c(\theta)
   & =
   - n \log(1 - p)
   \\
   & =
   - n \log\left(1 - \frac{e^\theta}{1 + e^\theta}\right)
   \\
   & =
   - n \log\left(\frac{1}{1 + e^\theta}\right)
   \\
   & =
   n \log\left(1 + e^\theta\right)
\end{align*}
Note that, as required for any sum of IID random variables, the cumulant
function for sample size $n$ is $n$ times the cumulant function for
sample size one (Section~\ref{sec:iid} above).

We check that this has the correct derivatives
$$
   c'(\theta) = \frac{n e^\theta}{1 + e^\theta} = n p
$$
and
$$
   c''(\theta)
   =
   \frac{n e^\theta}{1 + e^\theta}
   - \frac{n e^\theta e^\theta}{(1 + e^\theta)^2}
   =
   \frac{n e^\theta}{1 + e^\theta}
   \left[ 1 - \frac{e^\theta}{1 + e^\theta} \right]
   =
   n p (1 - p)
$$

The \emph{mean value parameter} is $\xi = n p$.

The \emph{canonical parameter space} is the range of the $\logit$ function,
which is the whole real line, $- \infty < \theta < \infty$.

The \emph{mean value parameter space} is $n$ times the domain
of the $\logit$ function $0 < \xi < n$.

Theorem~\ref{th:completion-fundamental} says limiting conditional models
are conditioned on the boundary of the closed convex support.
The closed convex support is the closed interval $[0, n]$, and its
boundary consists of two points $0$ and $n$.

Thus there are two limiting conditional models, one of which contains only
the distribution concentrated at zero and one of which contains only
the distribution concentrated at $n$.

In one-dimensional space there are only two directions.  Every positive
vector points in the same direction and gives the same LCM.  Every negative
vector points in the same direction and gives the same LCM.  (And, of course,
the zero vector points in no direction and gives the original model back
as the LCM corresponding to it.)

As discussed in Theorem~\ref{th:cumfun-lcm} above and its following comments,
it is important that we use \eqref{eq:cumfun-lcm} to determine the cumulant
function for the LCM.

So
\begin{align*}
   c_{- 1}(\theta)
   & =
   c(\theta) + \log \Pr\nolimits_\theta(Y = 0)
   \\
   & =
   n \log\left(1 + e^\theta\right)
   +
   n \log (1 - p)
   \\
   & =
   n \log\left(1 + e^\theta\right)
   +
   n \log \left(\frac{1}{1 + e^\theta}\right)
   \\
   & =
   0
\end{align*}
and
\begin{align*}
   c_{+ 1}(\theta)
   & =
   c(\theta) + \log \Pr\nolimits_\theta(Y = n)
   \\
   & =
   n \log\left(1 + e^\theta\right)
   +
   n \log (p)
   \\
   & =
   n \log\left(1 + e^\theta\right)
   +
   n \log \left(\frac{e^\theta}{1 + e^\theta}\right)
   \\
   & =
   n \theta
\end{align*}

\emph{Addition rule:}
the sum of $m$ independent and identically distributed
binomial random variables with sample size $n$ and usual parameter $p$
has the binomial distribution with sample size $m n$ and usual parameter $p$.

Hence if $y_{p(j)} \longrightarrow y_j$ is a binomial arrow for sample size $n$
the conditional distribution of $y_j$ given $y_{p(j)}$ is binomial
for sample size $n y_{p(j)}$.

This family is not implemented in either R package \code{aster} or
R package \code{aster2}.  Only the $n = 1$ special case, the Bernoulli
family is implemented.

\section{Poisson}
\label{sec:poisson}

A random variable is \emph{Poisson} if it has
the \emph{probability mass function}
\begin{equation} \label{eq:poisson-pmf}
   f(y) = \frac{\xi^y}{y !} e^{- \xi}, \qquad y = 0, 1, 2, \ldots,
\end{equation}
where $\xi$ is the \emph{usual parameter}, which turns out to be the
mean and variance of the distribution, hence also the mean-value parameter.

This is a \emph{discrete} random variable.

There are two rationales for this distribution, both so closely related
that they are almost one rationale.  First, the Poisson distribution is
an approximation to the $\text{binomial}(n, p)$ distribution when $n$
is very large and $p$ is very small and the mean $n p$ is moderate sized.
An example is a lottery.  Every week millions, sometimes hundreds of millions
of tickets are sold (that's $n$), the probability of any one ticket winning
is very small --- for example, for the Powerball lottery, the probability is
one over 292,201,338 (as we write this, the rules change from time to time) ---
(that's $p$), and $n p$ is moderate sized.  In weeks where the jackpot is small
and few tickets are sold, there are still tens of millions of tickets sold,
so $n p$ is less than one but not very small.  In weeks where the jackpot
is large, there may be many hundreds of millions
of tickets sold, so $n p$ is greater
than one and multiple winners are expected (they split the jackpot among them).
But regardless, the distribution of the number of winners is well approximated
by the $\text{Poisson}(n p)$ distribution.

Before we can discuss the second rationale, we discuss the \emph{addition rule:}
the sum of independent Poisson random variables is again Poisson.
It is not required that the independent Poisson random variables be
identically distributed.  Since the expectation of a sum is the sum of the
expectations, the sum of independent Poisson random variables having
means $\xi_1$, $\ldots,$ $\xi_n$ has
the $\text{Poisson}(\xi_1 + \cdots + \xi_n)$ distribution.

It follows (not obviously, but the derivation can be found in books
about spatial point processes) that the sum of $n$ independent Bernoulli
random variables is well approximated by a Poisson distribution provided
$n$ is very large and the means of all of the Bernoulli random variables
are very much smaller than the mean of the Poisson random variable.
Again, if the means of the Bernoulli random variables
are $\xi_1$, $\ldots,$ $\xi_n$, then the mean of the Poisson random variable
is $\xi_1 + \cdots + \xi_n$.  So we are assuming that each $\xi_i$ is very
much smaller than the sum.  To return to our lottery example, it does not
matter that each player is playing the same game.  So long as the expectation
of any one ticket winning is negligible compared to the expected number of
winners (for all tickets), the distribution of the number of winners will
be approximately Poisson.

Let's take a biological example.  Suppose we are counting ants, and we
have divided up the region in which we are counting ants with a very fine
grid.  If our grid is fine enough, the probability of counting more than one
ant in a grid cell will be negligible, perhaps impossible (if our grid cells
are so small that more than one ant could not fit).  Then the number of
ants in any one cell is a Bernoulli (zero-or-one-valued) random variable,
and the number of ants in any region that contains a very large number
of grid cells is very well approximated by the Poisson distribution.
If we take the limit as the size of the grid cells goes to zero we
get exact Poisson distributions.  Except that we forgot to mention
independence.  This assumes the Bernoulli random variables are independent,
that where one ant is has nothing whatsoever to do with where any other ant
is.  If we can accept this independence assumption, then the count of
ants in any region of any size large enough to have a moderate sized
expected number of ants can be assumed Poisson.

Now we abstract away from ants to be counting any things in regions of
any dimension.  The number of stars visible to the naked eye in a region
of sky, the number of raisins in slice of carrot cake, the number of white
blood cells in a drop of blood on a microscope slide, the number of ants
in a square meter region of your back yard, the number of leaves on a tree,
the number of calls arriving at a call center in a specified time interval,
and many other things can be assumed Poisson.

The independence assumption is crucial.  Pheromone trails and perhaps other
phenomena may make our counts of ants noticeably non-Poisson.  But if it
can be plausibly asserted that the probability of any one thing being counted
is independent of all the other things counted or not counted, then the
distribution of the total count is Poisson.

And even if the distribution of a count random variable fails to be exactly
Poisson due to some failure of the independence assumption, the Poisson
distribution may still may be a pretty good approximation (or may fail badly
if the independence assumption is grossly wrong).

As stated above, the \emph{mean} and \emph{variance} are
\begin{align*}
   E(y) & = \xi
   \\
   \var(y) & = \xi
\end{align*}

This is an \emph{exponential family}.  From \eqref{eq:poisson-pmf}
the log likelihood is
$$
   l(\theta) = y \log(\xi) - \xi
$$
(the term $\log(y !)$ can be dropped because it does not contain the
parameter), from which we see that we have an exponential family with
\emph{canonical statistic} $y$ and \emph{canonical parameter}
$$
   \theta = \log(\xi),
$$
so
$$
   \xi = e^{\theta}.
$$

The \emph{cumulant function} is
$$
   c(\theta) = \xi = e^\theta
$$
We check that this has the correct derivatives (and this is trivial)
$$
   c'(\theta) = e^\theta = \xi
$$
and
$$
   c''(\theta) = e^\theta = \xi
$$

The \emph{mean value parameter} is also the usual parameter $\xi$.

The \emph{canonical parameter space} is the range of the $\log$ function,
which is the whole real line, $- \infty < \theta < \infty$.

\begin{sloppypar}
The \emph{mean value parameter space} is the domain of the log function
\mbox{$0 < \xi < \infty$}.
\end{sloppypar}

\begin{sloppypar}
\emph{Thinning rule:} in the following graph
$$
\begin{CD}
   y_1 @>\text{Poi}>> y_2 @>\text{Ber}>> y_3
\end{CD}
$$
the conditional distribution of $y_3$ given $y_1$ (both arrows combined)
is $\text{Poisson}(\xi_3 \xi_2)$.  A thinned Poisson process is another
Poisson process, where ``thinning'' means we take each ``point'' counted
and accept or reject it independently with the same probability.
\end{sloppypar}

As discussed at the end of the preceding section, LCM
are conditioned on the boundary of the closed convex support.
The closed convex support is the closed interval $[0, \infty)$, and its
boundary consists of the single point $0$.

Thus there is one limiting conditional model, which contains only
the distribution concentrated at zero.

Also as discussed at the end of the preceding section,
it is important that we use \eqref{eq:cumfun-lcm} to determine the cumulant
function for the LCM.  So
\begin{align*}
   c_{- 1}(\theta)
   & =
   c(\theta) + \log \Pr\nolimits_\theta(Y = 0)
   \\
   & =
   e^\theta 
   +
   \log (e^{- \xi})
   \\
   & =
   e^\theta 
   - \xi
   \\
   & =
   0
\end{align*}
So, again as in the preceding section, the cumulant function for the LCM
concentrated at zero is the zero function.

As mentioned in Section~\ref{sec:infinitely-divisible} above,
the Poisson distribution is infinitely divisible.
This is easily verified from its cumulant function.
For any positive real number $r$
$$
   r c(\theta) = r e^{\theta} = e^{\theta + \log(r)}
$$
is a cumulant function.  In fact, it is a cumulant function for the Poisson
family.  One log likelihood for the Poisson family is
$$
   l(\theta) = y \theta - e^\theta
$$
but if we make the substitution $\theta = \psi + \log(r)$ we get
$$
   l(\psi) = y \psi + y \log(r) - e^{\psi + \log(r)}
$$
and we can drop the term that does not contain the new parameter $\psi$
obtaining
$$
   l(\psi) = y \psi - e^{\psi + \log(r)}
$$
and we see this has exponential family form with canonical statistic $y$,
canonical parameter $\psi$,
and cumulant function $c(\psi) = e^{\psi + \log(r)}$.

This is just a special case of the fact,
noted without proof in Section~\ref{sec:define-expfam},
that adding a constant to a canonical parameter gives
another canonical parameter.

Another way of thinking about this fact is that our new parameterization
just puts an offset $\log(r)$ in the exponential family.
But we know from Section~\ref{sec:canonical-affine-submodel} above
that canonical affine submodels of full exponential families are again
exponential families.

Note that in going from Section~\ref{sec:bernoulli}
to Section~\ref{sec:binomial} we just went from the family having cumulant
function $c(\theta) = 1 + e^\theta$ to the family
having cumulant function $n c(\theta)$,
something we know from Section~\ref{sec:iid} above is always valid.
So we might think that we would need another section to go from the
family having cumulant function $c(\theta) = e^\theta$ to the family
having cumulant function $r c(\theta)$, which is valid only when the
family is infinitely divisible.  But we have just found that that does
not give us a new family, but rather the same old Poisson family
(with an offset), so we do not need a new section for a new family.

\section{Zero-Truncated Poisson}
\label{sec:zero-truncated-poisson}

The \emph{zero-truncated Poisson} distribution is the Poisson distribution
conditioned on being nonzero.

The \emph{rationale} is that it can be used to incorporate zero-inflated
Poisson random variables into aster models.

This is a \emph{discrete} distribution.

If $f$ is the PMF of the Poisson distribution, then the PMF of
the zero-truncated Poisson distribution is
\begin{equation} \label{eq:zero-truncated-poisson-pmf-in-terms-of-poisson}
   g(y) = \frac{f(y)}{1 - f(0)}, 
   \qquad y = 1, 2, \ldots,
\end{equation}
that is, if $m$ is the mean of the untruncated Poisson distribution, then
the PDF of the zero-truncated Poisson distribution is
\begin{equation} \label{eq:zero-truncated-poisson-pmf}
   g(y) = \frac{m^y e^{- m}}{y ! (1 - e^{- m})}, 
   \qquad y = 1, 2, \ldots.
\end{equation}

Since this is not a ``brand name distribution'' the mean and variance
cannot just be looked up.  In aid of this calculation we prove a rather
trivial general theorem.
\begin{theorem} \label{th:truncated-mean-variance}
Suppose $X$ is a nonnegative-integer-valued random variable,
and $Y$ is the corresponding zero-truncated random variable.  Then
$$
   E(Y^k) = E(X^k) / \Pr(X > 0)
$$
for any positive integer $k$.
\end{theorem}
\begin{proof}
For this proof let $f$ denote the PMF of $X$ and $g$ the PMF of $Y$, so
the relationship between the two is given
by \eqref{eq:zero-truncated-poisson-pmf-in-terms-of-poisson}
even though we are no longer assuming $X$ is Poisson.  Then
\begin{align*}
   E(Y^k)
   & =
   \sum_{y = 1}^\infty y^k g(y)
   \\
   & =
   \frac{1}{1 - f(0)} \sum_{x = 1}^\infty x^k f(x)
   \\
   & =
   \frac{1}{1 - f(0)} \sum_{x = 0}^\infty x^k f(x)
   \\
   & =
   \frac{E(X)}{1 - f(0)}
   \\
   & =
   \frac{E(X)}{\Pr(X > 0)}
\end{align*}
where the third equality is the fact that the $x = 0$ term in the sum
is equal to zero.
\end{proof}
Together with
\begin{align*}
   \var(Y) & = E(Y^2) - E(Y)^2
   \\
   E(Y^2) & = \var(Y) + E(Y)^2
\end{align*}
which are well known from elementary probability theory, we can use the
theorem to calculate the mean and variance of zero-truncated random variables.

For the Poisson distribution, we have $E(X) = \var(X) = m$
so $E(X^2) = m + m^2$, so
\begin{equation} \label{eq:mean-of-zero-truncated-Poisson}
   E(Y)
   =
   \frac{E(X)}{\Pr(X > 0)}
   =
   \frac{m}{1 - e^{- m}}
\end{equation}
and
\begin{equation} \label{eq:variance-of-zero-truncated-Poisson}
\begin{split}
   \var(Y)
   & =
   E(Y^2) - E(Y)^2
   \\
   & =
   \frac{E(X^2)}{\Pr(X > 0)} - \left(\frac{E(X)}{\Pr(X > 0)}\right)^2
   \\
   & =
   \frac{m + m^2}{1 - e^{- m}} - \left( \frac{m}{1 - e^{- m}} \right)^2
\end{split}
\end{equation}
% Mathematica
% dist = PoissonDistribution[m]
% f[y_] = PDF[dist, y]
% g[y_] = f[y] / (1 - f[0])
% Sum[ g[y], {y, 1, Infinity} ]
% moo = Sum[ y g[y], {y, 1, Infinity} ]
% moo - m / (1 - f[0])
% Simplify[%]
% voo = Sum[ (y - moo)^2 g[y], {y, 1, Infinity} ]
% voo - ((m + m^2) / (1 - f[0]) - (m / (1 - f[0]))^2)
% Simplify[%]

This is an \emph{exponential family}.
From \eqref{eq:zero-truncated-poisson-pmf} the log likelihood is
$$
   l(\theta) = y \log(m) - m - \log(1 - e^{- m})
$$
(the term $\log(y !)$ can be dropped because it does not contain the
parameter), from which we see that we have an exponential family with
\emph{canonical statistic} $y$ and \emph{canonical parameter}
$$
   \theta = \log(m),
$$
so
$$
   m = e^{\theta},
$$
the relation between $\theta$ and $m$ being the same as for the Poisson
distribution.

But the \emph{usual parameter} $m$ is not the \emph{mean value parameter},
which is
\begin{equation} \label{eq:zero-truncated-poisson-theta-to-xi}
   \xi = \frac{m}{1 - f(0)} = \frac{m}{1 - e^{- m}}
   = \frac{\exp(\theta)}{1 - \exp(- \exp(\theta))}
\end{equation}
as we know from general exponential family theory,
%%%%%%%%%% NEED BACKWARD REFERENCE to mean value parameterization %%%%%%%%%%
the mapping $\theta \mapsto \xi$ given by the formula above is
strictly increasing and invertible and both it and its inverse mapping
$\xi \to \theta$ are infinitely differentiable.  But in this case
the inverse mapping $\xi \to \theta$ seems to have no closed-form expression.
% can Mathematica find closed-form expression?
% using stuff from above
% foo[m_] = moo
% bar[theta_] = foo[Exp[theta]]
% Solve[ bar[theta] == xi, theta ]
% Nope!  It has no clue.
The map $\xi \to \theta$ is what is called a \emph{link function} in the
terminology of generalized linear models (GLM).  The failure of some
families to have link functions in useful form is one reason why aster
model theory and practice never mentions link functions.  They make
sense for some families but not others.

The \emph{cumulant function} is
\begin{equation} \label{eq:zero-truncated-poisson-cumfun}
   c(\theta) = m + \log(1 - e^{- m})
   = e^\theta + \log(1 - \exp(- \exp(\theta)))
\end{equation}
We check that this has the correct derivatives
\begin{align*}
   c'(\theta)
   & =
   e^\theta 
   +
   \frac{\exp(- \exp(\theta)) \exp(\theta)}{1 - \exp(- \exp(\theta))}
   \\
   & =
   m + \frac{m e^{- m}}{1 - e^{- m}}
   \\
   & =
   \frac{m}{1 - e^{- m}}
\end{align*}
and
\begin{align*}
   c''(\theta)
   & =
   e^\theta 
   +
   \frac{\exp(- \exp(\theta)) \exp(\theta)}{1 - \exp(- \exp(\theta))}
   -
   \frac{\exp(- \exp(\theta)) \exp(\theta)^2}{1 - \exp(- \exp(\theta))}
   \\
   & \quad
   -
   \frac{\exp(- \exp(\theta))^2 \exp(\theta)^2}{(1 - \exp(- \exp(\theta)))^2}
   \\
   & =
   m + \frac{m e^{- m}}{1 - e^{- m}}
   - \frac{m^2 e^{- m}}{1 - e^{- m}}
   - \frac{m^2 e^{- 2 m}}{(1 - e^{- m})^2}
\end{align*}
% Mathematica
% voo - (m + m Exp[- m] / (1 - Exp[- m]) - m^2 Exp[- m] / (1 - Exp[- m]) -
%     m^2 Exp[- m]^2 / (1 - Exp[- m])^2)
% Simplify[%]
and this does simplify to be equal to our other expression for variance.

Two other formulas for the variance are also useful \citep{geyer-3701}.
\begin{subequations}
\begin{align}
   \var(y) & = \xi (1 + m - \xi)
   \label{eq:first-convenient-variance-formula}
   \\
   & = \xi (1 - \xi e^{- m})
   \label{eq:second-convenient-variance-formula}
\end{align}
\end{subequations}
% Mathematica
% voo - moo (1 + m - moo)
% Simplify[%]
% voo - moo (1 - moo Exp[- m])
% Simplify[%]
As $\theta \to - \infty$ and $m \to 0$ the mean value parameter $\xi$ converges
(using L'Hospital's rule) to
$$
   \lim_{m \to 0} \frac{m}{1 - e^{- m}} = \lim_{m \to 0} \frac{1}{e^{m}} = 1
$$
and \eqref{eq:first-convenient-variance-formula} shows the variance converges
to zero as $m \to 0$ and $\xi \to 1$.
As $\theta \to \infty$ and $m \to \infty$ the mean value parameter $\xi$
is approximately equal to $m$ because $f(0) = e^{- m}$ is approximately zero.
Then $\xi e^{- m}$ is small compared to one, and
and \eqref{eq:second-convenient-variance-formula} shows the variance is
also approximately equal to $\xi \approx m$.

As we said above, the \emph{mean value parameter} $\xi$ is not
the usual parameter $m$.

As can be seen from that fact that \eqref{eq:zero-truncated-poisson-cumfun}
is finite for all $\theta$,
the \emph{canonical parameter space} is
the whole real line, $- \infty < \theta < \infty$.

As we saw when discussing variance formulas, $\xi \to 1$ as $m \to 0$.
Thus the lower end of the mean value parameter space is one.
And from $m$ being the mean of a Poisson distribution so $m$ has
no upper bound, and from $m \approx \xi$ when either is large, we see that
$\xi$ also has no upper bound.  Thus
the \emph{mean value parameter space} is $1 < \xi < \infty$.

As discussed at the end of the two preceding sections, LCM
are conditioned on the boundary of the closed convex support.
The closed convex support is the closed interval $[1, \infty)$, and its
boundary consists of the single point $1$.

Thus there is one limiting conditional model, which contains only
the distribution concentrated at one.

Also as discussed at the end of the two preceding sections,
it is important that we use \eqref{eq:cumfun-lcm} to determine the cumulant
function for the LCM.  So
\begin{align*}
   c_{- 1}(\theta)
   & =
   c(\theta) + \log \Pr\nolimits_\theta(Y = 1)
   \\
   & =
   m + \log(1 - e^{- m})
   +
   \log \left(\frac{m e^{- m}}{1 - e^{- m}}\right)
   \\
   & =
   \log(m)
   \\
   & =
   \theta
\end{align*}
So the cumulant function of the distribution concentrated at one is the
identity function.

This just happens to agree with the $n = 1$ case for the binomial
distribution (Section~\ref{sec:binomial} above), but it need not have.
It all depends on how we defined the cumulant functions for these families
in the first place.  We could have added different arbitrary constants
to the cumulant functions of these families and they would still be
cumulant functions.

\section{Normal Location}

\begin{sloppypar}
The univariate normal distribution has \emph{probability density function}
(PDF)
\begin{equation} \label{eq:normal-pdf}
   f(y) = \frac{1}{\sqrt{2 \pi} \sigma} e^{- \frac{(y - \xi)^2}{2 \sigma^2}},
   \qquad - \infty < y < \infty.
\end{equation}
\end{sloppypar}

This is a \emph{continuous} random variable;
except when incorporated into an aster model, it is a mixture of discrete
and continuous.  For a normal-location arrow, when the predecessor is zero
the conditional distribution of the successor is the degenerate random variable
concentrated at zero, which is discrete, and when the predecessor is greater
than zero, the conditional distribution of the successor is continuous.

The \emph{rationale} is the celebrated central limit theorem,
or more precisely, theorems, because there are many variants.
In non-technical terms these theorems say that a random variable that is the sum
of a large number of random variables that are not too dependent,
not too heavy tailed, and not too unequal in size will be well approximated
by a normal distribution.  (If the random variable in question is the
sum of a large number of independent random variables, then Lindeberg's
central limit theorem using Lindeberg's condition specifies what
``not too heavy tailed, and not too unequal in size'' means.
If the random variable in question is the
sum of the components of a dependent stochastic process, then various
stationary process central limit theorems, Markov chain central limit theorems,
and the martingale central limit theorem, give various notions of what
``not too dependent'' means.)
\begin{quotation}
Everybody believes in the law of errors, the experimenters because they
think it is a mathematical theorem, the mathematicians because they
think it is an experimental fact.
\\
\hspace*{\fill} --- Lippman, quoted by Poincar\'{e}, quoted by \citet{cramer}
\end{quotation}
``The law of errors'' is an old name for the normal distribution.
It has also been named after de Moivre, Laplace, and Gauss.
The term ``normal distribution'' was popularized by K. Pearson in the early
twentieth century.  Like the term ``law of errors'' it builds into the name
the idea that it is the main, principle, or only distribution for random data.
Also note the Lippman quote is sarcastic.  Justification
for this belief was always known to be shaky.
(Harald Cram\'{e}r and Henri Poincar\'{e} are, of course, famous.  It is
unclear who the Monsieur Lippman was that Poincar\'{e} attributed this to.)

Since the nonparametrics revolution \citep{hollander-wolfe-chicken},
the exploratory data analysis revolution \citep{tukey},
the bootstrap revolution \citep{efron-tibshirani,davison-hinkley},
and the robustness revolution \citep{huber-ronchetti,hampel-et-al}
no user of statistics aware of these developments wants to blindly
assume normality, especially when it can be demonstrated to be grossly
incorrect using any of these tools.  But the normal distribution may
fit data well, so it continues to be used.  It just is no longer considered
the only distribution for data, as it was before 1950 (mostly, there was
the chi-square test for contingency tables).

The other rationale for this distribution (which has nothing to do with
aster models) is that the usual assumption of homoscedastic normal errors
for linear models makes the distribution of point estimates exactly normal
and the distribution of various test statistics exactly $t$ or exactly $F$.
This rationale is often attributed to Gauss and is why the normal distribution
is sometimes called Gaussian, because Gauss independently co-invented the
method of least squares and more-or-less gave this rationale (more-or-less
because his discussion was Bayesian rather than frequentist), but of course
this was a century before the $t$ and $F$ distributions were invented.

The \emph{mean} and \emph{variance} are
\begin{align*}
   E(y) & = \xi
   \\
   \var(y) & = \sigma^2
\end{align*}
When used in R package \code{aster} every family must be a one-parameter
exponential family of distributions, so when we consider this as such a family
we must pick one parameter to be treated as unknown
and the other parameter to be treated as known.
Because the location parameter $\xi$ is the mean value parameter,
we pick this to be the unknown parameter.

With this understanding, the log likelihood is
$$
   l(\theta) = - \frac{(y - \xi)^2}{2 \sigma^2}
$$
(the term $\sqrt{2 \pi} \sigma$ can be dropped because it does not contain the
unknown parameter $\xi$.  If we expand the quadratic, we get
$$
   l(\theta)
   =
   - \frac{y^2}{2 \sigma^2}
   + \frac{y \xi}{\sigma^2}
   - \frac{\xi^2}{2 \sigma^2}
$$
and can now drop another term not containing $\xi$ obtaining
$$
   l(\theta)
   =
   \frac{y \xi}{\sigma^2} - \frac{\xi^2}{2 \sigma^2}
$$
from which we see that we have an \emph{exponential family} with
\emph{canonical statistic} $y$ and \emph{canonical parameter}
$$
   \theta = \frac{\xi}{\sigma^2}
$$
so the \emph{mean value parameter} is
$$
   \xi = \sigma^2 \theta
$$
The \emph{cumulant function} is
$$
   c(\theta) = \frac{\xi^2}{2 \sigma^2}
   = \frac{\sigma^2 \theta^2}{2}
$$
We check that this has the correct derivatives
$$
   c'(\theta) = \sigma^2 \theta = \xi
$$
and
$$
   c''(\theta) = \sigma^2
$$

\emph{Addition rule:} the sum of $n$ independent and identically distributed
normal random variables with mean $\xi$ and variance $\sigma^2$ has the normal
distribution with mean $n \xi$ and variance $n \sigma^2$.

\emph{General Addition rule:} any sum of independent
normal random variables is again normal
(identically distributed is not required), but this has no application
in aster model theory.

There are no limit degenerate distributions.
This is because the boundary of the closed convex support,
which is the interval $(- \infty, + \infty)$ is empty.
We can never observe data on the boundary.

\section{Negative Binomial}
\label{sec:negative-binomial}

\subsection{Basics}
\label{sec:negative-binomial-basics}

According to the \code{help("NegBinomial")} in R, the negative binomial
distribution has \emph{probability mass function}
\begin{equation} \label{eq:negative-binomial-pmf}
   f(y) = \frac{\Gamma(\alpha + y)}{\Gamma(\alpha) \, y!} p^\alpha (1-p)^y,
   \qquad y = 0, 1, 2, \ldots.
\end{equation}
where $\alpha > 0$ is the shape parameter and $0 < p \le 1$ is the
usual parameter (success probability).  The case $\alpha = 1$ is the
geometric distribution.

The \emph{first rationale} for this distribution is inverse sampling,
and for this rationale $\alpha$ must be a positive integer.
If one has an infinite
sequence of IID Bernoulli random variables with usual parameter $p$,
then the distribution of the number of observed zero outcomes before the
$\alpha$-th nonzero outcome is negative binomial with shape parameter $\alpha$
and usual parameter $p$, that is, if one observes $y$ successes in $n$ trials,
then the distribution of $y$ is binomial if $n$ was fixed and the distribution
of $n - y$ is negative binomial if $y$ was fixed.
But this rationale has nothing to do with aster models.

The \emph{second rationale} for this distribution is overdispersed Poisson.
This distribution arises as a mixture of Poisson distributions,
as is discussed below (Section~\ref{sec:mixture}).
This is the reason it is implemented in R package
\code{aster}.  For this rationale $\alpha$ can be any positive real number.

The \emph{mean} and \emph{variance} in terms of these parameters are
\begin{align*}
   E(y) & = \frac{\alpha (1 - p)}{p}
   \\
   \var(y) & = \frac{\alpha (1 - p)}{p^2}
\end{align*}

From \eqref{eq:negative-binomial-pmf} the log likelihood is
$$
   l(\theta)
   =
   \log \Gamma(\alpha + y) - \log \Gamma(\alpha) - \log(y!)
   + \alpha \log(p) + y \log(1-p).
$$
from which we can see that if $\alpha$ is considered an unknown parameter,
this is \emph{not} an exponential family, so we consider $\alpha$ known,
which means we can drop terms not containing $p$ obtaining
$$
   l(\theta)
   =
   y \log(1 - p) + \alpha \log(p) 
$$
from which we see that we have an \emph{exponential family} with
\emph{canonical statistic} $y$ and \emph{canonical parameter}
$$
   \theta = \log(1 - p)
$$
and solving for $p$ gives
$$
   1 - p = e^\theta
$$
and
$$
   p = 1 - e^\theta
$$

The \emph{cumulant function} is
$$
   c(\theta)
   =
   - \alpha \log(p) 
   =
   - \alpha \log(1 - e^\theta)
$$
As $p$ goes from zero to one, $\theta$ goes from zero to $- \infty$ so
the formula above does not define the cumulant function on the whole real
line and equation (5) in \citet{geyer-gdor}, which is \eqref{eq:cumfun-expfam}
in this book, must be used
\begin{equation*}
   c(\theta) = c(\psi) +
   \log E_{\psi}\bigl( e^{y (\theta - \psi)} \bigr)
\end{equation*}
where $\psi$ is a fixed canonical parameter value, $\theta$ varies
over the whole real line, and the cumulant function has the value $\infty$
where the expectation does not exist.

Evaluating this we get, using the theorem associated with the negative
binomial distribution \citep{brand-name-distributions},
\begin{equation} \label{eq:negative-binomial-cumfun-derivation}
\begin{split}
   c(\theta)
   & =
   c(\psi) +
   \log \left( \sum_{y = 0}^\infty
   e^{y (\theta - \psi)} \cdot
   \frac{\Gamma(\alpha + y)}{\Gamma(\alpha) \, y!} p^\alpha (1-p)^y
   \right)
   \\
   & =
   c(\psi) +
   \log \left(
   p^\alpha
   \sum_{y = 0}^\infty
   \frac{\Gamma(\alpha + y)}{\Gamma(\alpha) \, y!}
   \left[ (1-p) e^{\theta - \psi} \right]^y
   \right)
   \\
   & =
   c(\psi) +
   \log \left(
   p^\alpha
   \left[ 1 - (1-p) e^{\theta - \psi} \right]^{- \alpha}
   \right)
\end{split}
\end{equation}
where $p$ is the usual parameter value corresponding to canonical parameter
value $\psi$, that is, $\psi = \log(1 - p)$ and the formula is only valid
when the infinite sequence converges, which it does if and only if
$-1 < (1-p) e^{\theta - \psi} < +1$.

Now $1 - p = e^\psi$, so we can simplify $(1-p) e^{\theta - \psi} = e^\theta$.
So the convergence criterion is $e^\theta < 1$ or $\theta < 0$, and the
formula simplifies to
$$
   c(\theta)
   =
   c(\psi)
   +
   \alpha \log(p)
   -
   \alpha \log \left( 1 - e^\theta \right)
$$
but the formula only determines the cumulant function up to an arbitrary
constant (which does not matter) so we can take the cumulant function to be
\begin{equation} \label{eq:cumfun-negative-binomial}
   c(\theta)
   =
   \begin{cases}
   - \alpha \log(1 - e^\theta), & \theta < 0
   \\
   \infty, & \theta \ge 0
   \end{cases}
\end{equation}
So the full canonical parameter space is, as we guessed before,
\begin{equation} \label{eq:canonical-parameter-space-negative-binomial}
   \Theta = \set{ \theta \in \real : \theta < 0 }
\end{equation}
and \eqref{eq:cumfun-negative-binomial} agrees with what we derived just
from looking at the log likelihood wherever the function is finite.

Let's check that this cumulant function gives the correct mean and variance.
\begin{align*}
   c'(\theta)
   & =
   \frac{\alpha e^\theta}{1 - e^\theta}
   \\
   & =
   \frac{\alpha (1 - p)}{p}
   \\
   c''(\theta)
   & =
   \frac{d}{d \theta}
   \frac{\alpha e^\theta}{1 - e^\theta}
   \\
   & =
   \frac{\alpha e^\theta}{1 - e^\theta}
   -
   \frac{\alpha e^{2 \theta}}{(1 - e^\theta)^2}
   \\
   & =
   \frac{\alpha (1 - p)}{p}
   \left(
   1
   -
   \frac{1 - p}{p}
   \right)
   \\
   & =
   \frac{\alpha (1 - p)}{p^2}
\end{align*}
as we had already been told but now have derived
from exponential family theory.

The \emph{mean value parameter}
\begin{equation} \label{eq:negative-binomial-mean-value}
   \xi = \frac{\alpha (1 - p)}{p}
\end{equation}
is not the usual parameter $p$.  Solving for $p$ gives
\begin{equation} \label{eq:negative-binomial-usual}
   p = \frac{\alpha}{\alpha + \xi}
\end{equation}
% Mathematica
% Solve[ xi == alpha (1 - p) / p, p ]

The \emph{canonical parameter space}
is \eqref{eq:canonical-parameter-space-negative-binomial}
which is not the whole real line.

The \emph{mean value parameter space} is the range of the derivative of
the cumulant function $0 < \xi < \infty$.

As discussed at the end of the Sections~\ref{sec:binomial},
\ref{sec:poisson}, and~\ref{sec:zero-truncated-poisson}, LCM
are conditioned on the boundary of the closed convex support.
The closed convex support is the closed interval $[0, \infty)$, and its
boundary consists of the single point $0$.

Thus there is one limiting conditional model, which contains only
the distribution concentrated at zero.

Also as discussed at the end of the those sections,
it is important that we use \eqref{eq:cumfun-lcm} to determine the cumulant
function for the LCM.  So
\begin{align*}
   c_{- 1}(\theta)
   & =
   c(\theta) + \log \Pr\nolimits_\theta(Y = 0)
   \\
   & =
   - \alpha \log(1 - e^\theta)
   +
   \log \left(p^\alpha\right)
   \\
   & =
   0
\end{align*}
So the cumulant function of the family concentrated at zero is the zero
function, as we also found in Sections~\ref{sec:binomial} and~\ref{sec:poisson}
above.  But as mentioned at the end of Section~\ref{sec:zero-truncated-poisson}
above, this agreement just happened because of arbitrary choices of arbitrary
constants in cumulant functions.

\subsection{Negative Binomial as Mixture of Poisson}
\label{sec:mixture}

As stated above, one rationale for the negative binomial distribution is
that it is a mixture of Poisson distributions.  Let the conditional
distribution of $Y$ given $\mu$ be Poisson with mean $\mu$,
and let the marginal distribution of $\mu$ be $\text{Gamma}(\alpha, \lambda)$.
Then the marginal distribution of $Y$ is given by
\begin{align*}
   f(y)
   & =
   \int f(y \mid \mu) g(\mu) \, d \mu
   \\
   & =
   \int_0^\infty \frac{\mu^y e^{- \mu}}{y!} \cdot
   \frac{\lambda^\alpha}{\Gamma(\alpha)} \mu^{\alpha - 1} e^{- \lambda \mu}
   \, d \mu
   \\
   & =
   \frac{1}{y!}
   \cdot
   \frac{\lambda^\alpha}{\Gamma(\alpha)}
   \int_0^\infty \mu^{y + \alpha - 1} e^{- (1 + \lambda) \mu}
   \, d \mu
   \\
   & =
   \frac{1}{y!}
   \cdot
   \frac{\lambda^\alpha}{\Gamma(\alpha)}
   \frac{\Gamma(y + \alpha)}{(1 + \lambda)^{y + \alpha}}
\end{align*}
using the theorem associated with the gamma distribution
\citep{brand-name-distributions}.

For this to be equal to \eqref{eq:negative-binomial-pmf} we need
$$
   \frac{1}{y!}
   \cdot
   \frac{\lambda^\alpha}{\Gamma(\alpha)}
   \frac{\Gamma(y + \alpha)}{(1 + \lambda)^{y + \alpha}}
   =
   \frac{\Gamma(\alpha + y)}{\Gamma(\alpha) \, y!} p^\alpha (1-p)^y
$$
that is
$$
   \frac{\lambda^\alpha}{(1 + \lambda)^{y + \alpha}}
   =
   p^\alpha (1-p)^y
$$
or
$$
   \left(\frac{\lambda}{1 + \lambda}\right)^\alpha
   \left(\frac{1}{1 + \lambda}\right)^y
   =
   p^\alpha (1-p)^y
$$
which happens if and only if $p = \lambda / (1 + \lambda)$ and
$1 - p = 1 / (1 + \lambda)$ so $\lambda = p / (1 - p)$.

\subsection{Poisson as Limit of Negative Binomial}

Reparameterize the negative binomial distribution so the parameters
are $\alpha$ and $\xi$ so the usual parameter is
\eqref{eq:negative-binomial-usual}
and the PMF \eqref{eq:negative-binomial-pmf} becomes
\begin{align*}
   f(y)
   & =
   \frac{\Gamma(\alpha + y)}{\Gamma(\alpha) \, y!}
   \left(\frac{\alpha}{\alpha + \xi}\right)^\alpha
   \left(\frac{\xi}{\alpha + \xi}\right)^y
   \\
   & =
   \frac{1}{y!}
   \left(\frac{\alpha}{\alpha + \xi}\right)^\alpha
   \left(\frac{\xi}{\alpha + \xi}\right)^y
   \prod_{k = 1}^y (\alpha + k - 1)
   \\
   & =
   \frac{\xi^y}{y!}
   \left(1 - \frac{\xi}{\alpha + \xi}\right)^\alpha
   \prod_{k = 1}^y \frac{\alpha + k - 1}{\alpha + \xi}
   \\
   & \to
   \frac{\xi^y}{y!} e^{- \xi}
\end{align*}
so as $\alpha \to \infty$ with $\xi$ fixed, we recover the Poisson
distribution.  But this does not happen if we let $\alpha \to \infty$
with some other parameter, such as $p$ or $\theta$, fixed.
(The limit $\left(1 - \frac{\xi}{\alpha + \xi}\right)^\alpha \to e^{- \xi}$
as $\alpha \to \infty$, which was used in the derivation above, follows
from $(1 + x / n)^n \to e^x$ as $n \to \infty$, which can be found
in any calculus book.)

The upshot of this section is that if the shape parameter $\alpha$
of a negative binomial distribution is large, then it is well approximated
by a Poisson distribution.  One only needs the negative binomial family
when the shape parameter is small.

The limit in this section is not like the limits in other sections of
this appendix.  In those other sections we took limits as the canonical
parameter of the exponential family went to plus or minus infinity.
Since the canonical parameter is considered unknown, this kind of
limit can arise in the process of maximum likelihood.
%%%%%%%%%% NEED BACKWARD REFERENCE to limiting conditional models %%%%%%%%%%
In this section we took a limit as the shape parameter $\alpha$ went
to infinity.
Since this parameter is considered known, this kind of
limit cannot arise in the process of maximum likelihood.

\section{Zero-Truncated Negative Binomial}

This section is just like Section~\ref{sec:zero-truncated-poisson} above
\emph{mutatis mutandis}.

The \emph{zero-truncated negative binomial} distribution is
the negative binomial distribution conditioned on being nonzero.

The \emph{rationale} is that it can be used to incorporate zero-inflated
negative binomial random variables into aster models.

This is a \emph{discrete} distribution.

\begin{sloppypar}
If $f$ is the PMF of the negative binomial distribution, then the PMF of
the zero-truncated negative binomial distribution is
\begin{equation}
\label{eq:zero-truncated-negative-binomial-pmf-in-terms-of-negative-binomial}
   g(y) = \frac{f(y)}{1 - f(0)},
   \qquad y = 1, 2, \ldots,
\end{equation}
that is, if $\alpha$ is the shape parameter and $p$ is the usual parameter
of the untruncated negative binomial distribution, then
the PDF of the zero-truncated negative binomial distribution is
\begin{equation} \label{eq:zero-truncated-negative-binomial-pmf}
   g(y) =
   \frac{\Gamma(\alpha + y) p^\alpha (1-p)^y}
   {\Gamma(\alpha) \, y! \, (1 - p^\alpha)},
   \qquad y = 1, 2, \ldots.
\end{equation}
\end{sloppypar}

Since this is not a ``brand name distribution'' the mean and variance
cannot just be looked up.
We still use Theorem~\ref{th:truncated-mean-variance}
and the comment following it to calculate the mean and the variance but now
\begin{align*}
   E(X) & = \frac{\alpha (1 - p)}{p}
   \\
   \var(X) & = \frac{\alpha (1 - p)}{p^2}
   \\
   E(X^2) & =
   \frac{\alpha (1 - p)}{p^2} + \left( \frac{\alpha (1 - p)}{p} \right)^2
   \\
   \Pr(X > 0) & = 1 - p^\alpha
\end{align*}
so
$$
   E(Y) = \frac{\alpha (1 - p)}{p (1 - p^\alpha)}
$$
and
\begin{align*}
   \var(Y)
   & = 
   E(Y^2) - E(Y)^2
   \\
   & = 
   \frac{E(X^2)}{\Pr(X > 0)} - \left(\frac{E(X)}{\Pr(X > 0)}\right)^2
   \\
   & =
   \frac{1}{1 - p^\alpha}
   \left[ \frac{\alpha (1 - p)}{p^2} + \left( \frac{\alpha (1 - p)}{p}
   \right)^2 \right]
   -
   \left( \frac{\alpha (1 - p)}{p (1 - p^\alpha)} \right)^2
\end{align*}
% Mathematica
% dist = NegativeBinomialDistribution[alpha,p]
% f[y_] = PDF[dist, y]
% g[y_] = f[y] / (1 - f[0])
% Sum[ g[y], {y, 1, Infinity} ]
% moo = Sum[ y g[y], {y, 1, Infinity} ]
% moo - alpha (1 - p) / (p (1 - p^alpha))
% Simplify[%]
% voo = Sum[ (y - moo)^2 g[y], {y, 1, Infinity} ]
% voo - (1 / (1 - p^alpha) (alpha (1 - p) / p^2 + (alpha (1 - p) / p)^2) -
%     (alpha (1 - p) / (p (1 - p^alpha)))^2)
% Simplify[%]

With the assumptions of Section~\ref{sec:negative-binomial} above
($\alpha$ known and $p$ unknown) this is an exponential family.
From \eqref{eq:zero-truncated-negative-binomial-pmf} the log likelihood is
$$
   l(\theta)
   =
   \log \Gamma(\alpha + y) + \alpha \log(p) + y \log(1-p)
   - \log \Gamma(\alpha) - \log(y!) - \log(1 - p^\alpha)
$$
and we may drop terms that do not contain the unknown parameter $p$ obtaining
$$
   l(\theta)
   =
   y \log(1-p) + \alpha \log(p) - \log(1 - p^\alpha)
$$
from which we see that we have an \emph{exponential family} with
\emph{canonical statistic} $y$ and \emph{canonical parameter}
$$
   \theta = \log(1 - p)
$$
(the same as for the untruncated negative binomial distribution)
and solving for $p$ gives
$$
   p = 1 - e^\theta
$$
(the same as for the untruncated negative binomial distribution).

The \emph{cumulant function} is
$$
   c(\theta)
   =
   - \alpha \log(p) + \log(1 - p^\alpha)
   =
   - \alpha \log(1 - e^\theta) + \log(1 - (1 - e^\theta)^\alpha)
$$
As in Section~\ref{sec:negative-binomial} this does not define the
cumulant function on the whole real line so we use
\begin{align*}
   c(\theta)
   & =
   c(\psi) +
   \log \left( \sum_{y = 1}^\infty
   e^{y (\theta - \psi)} \cdot
   \frac{\Gamma(\alpha + y) p^\alpha (1-p)^y}
   {\Gamma(\alpha) \, y! \, (1 - p^\alpha)}
   \right)
   \\
   & =
   c(\psi) +
   \log \left(
   \frac{p^\alpha}{1 - p^\alpha}
   \sum_{y = 1}^\infty
   \frac{\Gamma(\alpha + y)}{\Gamma(\alpha) \, y!}
   \left[ (1-p) e^{\theta - \psi} \right]^y
   \right)
   \\
   & =
   c(\psi) +
   \log \left(
   \frac{p^\alpha}{1 - p^\alpha}
   \left\{
   - 1 +
   \sum_{y = 0}^\infty
   \frac{\Gamma(\alpha + y)}{\Gamma(\alpha) \, y!}
   \left[ (1-p) e^{\theta - \psi} \right]^y
   \right\}
   \right)
   \\
   & =
   c(\psi) +
   \log \left(
   \frac{p^\alpha}{1 - p^\alpha}
   \left\{
   - 1 +
   \bigl[ 1 - (1-p) e^{\theta - \psi} \bigr]^{- \alpha}
   \right\}
   \right)
\end{align*}
where the last equality is the theorem associated with the negative
binomial distribution \citep{brand-name-distributions}
and where, as in \eqref{eq:negative-binomial-cumfun-derivation}, $p$ is
the usual parameter that goes with $\psi$ not $\theta$ so $p$ is a known
constant and $(1-p) e^{\theta - \psi} = e^\theta$ and the infinite series
converges if and only if $\theta < 0$.
Thus our formula simplifies to
$$
   c(\theta) = c(\psi) + \log\left( \frac{p^\alpha}{1 - p^\alpha} \right)
   +
   \log\left(- 1 + (1 - e^\theta)^{- \alpha} \right)
$$
and we may drop the terms that do not contain $\theta$ obtaining
$$
   c(\theta)
   =
   \begin{cases}
   \log\left((1 - e^\theta)^{- \alpha} - 1\right), & \theta < 0 \\
   \infty, & \theta \ge 0
   \end{cases}
$$
With some rearrangement, this agrees with what we deduced from looking
at the log likelihood.
% Mathematica
% ( - alpha Log[1 - Exp[theta]] + Log[1 - (1 - Exp[theta])^alpha] ) -
%     Log[(1 - Exp[theta])^(- alpha) - 1]
% Simplify[%]
% FullSimplify[%]
% PowerExpand[%]
%
% check
%
% dist = NegativeBinomialDistribution[alpha, 1 - E^theta]
% f[y_] = PDF[dist, y]
% g[y_] = f[y] / (1 - f[0])
% m[t_] = Sum[ Exp[y t] g[y], {y, 1, Infinity} ]
% k[t_] = PowerExpand[Log[m[t]]]
%
% c[theta_] = Log[- 1 + (1 - E^theta)^(- alpha) ]

Let's check that this cumulant function gives the correct mean and variance.
\begin{align*}
   c'(\theta)
   & =
   \frac{- \alpha (1 - e^\theta)^{- \alpha - 1} (- e^\theta)}
   {(1 - e^\theta)^{- \alpha} - 1}
   \\
   & =
   \frac{\alpha (1 - e^\theta)^{- \alpha - 1} e^\theta}
   {(1 - e^\theta)^{- \alpha} - 1}
   \\
   & =
   \frac{\alpha e^\theta}{(1 - e^\theta) [1 - (1 - e^\theta)^\alpha]}
   \\
   & =
   \frac{\alpha (1 - p)}{p (1 - p^\alpha)}
   \\
   c''(\theta)
   & =
   \frac{d}{d \theta}
   \frac{\alpha e^\theta}{(1 - e^\theta) [1 - (1 - e^\theta)^\alpha]}
   \\
   & =
   \frac{\alpha e^\theta}{(1 - e^\theta) [1 - (1 - e^\theta)^\alpha]}
   +
   \frac{\alpha e^\theta e^\theta}{(1 - e^\theta)^2 [1 - (1 - e^\theta)^\alpha]}
   \\
   & \qquad
   - \frac{\alpha^2 e^\theta e^\theta (1 - e^\theta)^{\alpha - 1}
   }{(1 - e^\theta) [1 - (1 - e^\theta)^\alpha]^2}
   \\
   & =
   \frac{\alpha (1 - p)}{p (1 - p^\alpha)}
   +
   \frac{\alpha (1 - p)^2}{p^2 (1 - p^\alpha)}
   -
   \frac{\alpha^2 (1 - p)^2 p^{\alpha - 2}}{(1 - p^\alpha)^2}
\end{align*}
% Mathematica
% (1 / 1 - p^alpha (alpha (1 - p) / p^2 + (alpha (1 - p) / p)^2) -
%     (alpha (1 - p) / (p (1 - p^alpha)))^2) -
%     (alpha (1 - p) / (p (1 - p^alpha)) +
%     alpha (1 - p)^2 / (p^2 (1 - p^alpha)) -
%     alpha^2 p^alpha (1 - p)^2 / (p^2 (1 - p^alpha)^2))
% Simplify[%]
% FullSimplify[%]
% PowerExpand[%]
%
% Oops!
%
% c[theta_] = Log[(1 - Exp[theta])^(- alpha) - 1]
% cp[theta_] = D[c[theta], theta]
% cp[theta_] = Simplify[cp[theta]]
% cpp[theta_] = D[cp[theta], theta]
% cpp[Log[1 - p]]
%
% alpha (1 - p) / (p (1 - p^alpha)) + alpha (1 - p)^2 / (p^2 (1 - p^alpha)) -
%     alpha^2 (1 - p)^2 p^(alpha - 2) / (1 - p^alpha)^2 - cpp[Log[1 - p]]
% Simplify[%]
% alpha (1 - p) / (p (1 - p^alpha)) + alpha (1 - p)^2 / (p^2 (1 - p^alpha)) -
%     alpha^2 (1 - p)^2 p^(alpha - 2) / (1 - p^alpha)^2 - voo
% Simplify[%]
And after some rearrangement $c''(\theta)$ agrees with the variance
calculated above.

The \emph{mean value parameter}
\begin{equation} \label{eq:zero-truncated-negative-binomial-mean-value}
   \xi
   =
   \frac{\alpha (1 - p)}{p (1 - p^\alpha)}
   =
   \frac{\alpha e^\theta}{(1 - e^\theta) [1 - (1 - e^\theta)^\alpha]}
\end{equation}
is not the usual parameter $p$.  As with zero-truncated Poisson,
we find that the inverse mapping $\xi \to \theta$ has no closed-form
expression.
% Mathematica
% Solve[ xi == cpp[theta], theta ]
As stated in Section~\ref{sec:zero-truncated-poisson} above
we know from general exponential family theory
that this mapping $\xi \to \theta$ is
%%%%%%%%%% NEED BACKWARD REFERENCE to mean value parameterization %%%%%%%%%%
strictly increasing, invertible, and infinitely differentiable.
But in this case the inverse mapping $\xi \to \theta$ seems to have
no closed-form expression.
Also as stated in Section~\ref{sec:zero-truncated-poisson} above,
this means that this family does not have what GLM theory calls a link function
in any useful form.

The full canonical parameter space is
\eqref{eq:canonical-parameter-space-negative-binomial}
as it was for the negative binomial.

Taking the limit in
\eqref{eq:zero-truncated-negative-binomial-mean-value}
as $\theta \to - \infty$ and $p \to 1$ we see that
$$
   \lim_{p \uparrow 1} \frac{\alpha (1 - p)}{p (1 - p^\alpha)}
   =
   \lim_{p \uparrow 1} \frac{- \alpha}{1 - p^\alpha - p \alpha p^{\alpha - 1}}
   =
   1
$$
(using L'Hospital's rule).
Taking the limit in
\eqref{eq:zero-truncated-negative-binomial-mean-value}
as $\theta \to \infty$ and $p \to 0$ we see that $\xi \to \infty$ in
this case.  And since we know from general exponential family theory
that $c'(\theta)$ is a continuous increasing function, it follows
that the \emph{mean value parameter space} is $1 < \xi < \infty$.

As discussed at the end of the Sections~\ref{sec:binomial},
\ref{sec:poisson}, \ref{sec:zero-truncated-poisson},
and~\ref{sec:negative-binomial-basics},
LCM are conditioned on the boundary of the closed convex support.
The closed convex support is the closed interval $[1, \infty)$, and its
boundary consists of the single point $1$.

Thus there is one limiting conditional model, which contains only
the distribution concentrated at one.

Also as discussed at the end of the those sections,
it is important that we use \eqref{eq:cumfun-lcm} to determine the cumulant
function for the LCM.  So
\begin{align*}
   c_{- 1}(\theta)
   & =
   c(\theta) + \log \Pr\nolimits_\theta(Y = 1)
   \\
   & =
   \log\left((1 - e^\theta)^{- \alpha} - 1\right)
   +
   \log\left(\frac{\alpha p^\alpha (1 - p)}{1 - p^\alpha}\right)
   \\
   & =
   \log\left(p^{- \alpha} - 1\right)
   +
   \log\left(\frac{\alpha p^\alpha (1 - p)}{1 - p^\alpha}\right)
   \\
   & =
   \log\left(\frac{1 - p^\alpha}{p^\alpha}\right)
   +
   \log\left(\frac{\alpha p^\alpha (1 - p)}{1 - p^\alpha}\right)
   \\
   & =
   \log(\alpha) + \log(1 - p)
   \\
   & =
   \log(\alpha) + \theta
\end{align*}
% Mathematica
% dist = NegativeBinomialDistribution[alpha,p]
% f[y_] = PDF[dist, y]
% g[y_] = f[y] / (1 - f[0])
% Sum[ g[y], {y, 1, Infinity} ]
%
% c[theta_] = Log[(1 - Exp[theta])^(- alpha) - 1]
% cm1[theta_] = c[theta] + Log[g[1]]
% p = 1 - E^theta
%
% $Assumptions = alpha > 0 && theta < 0
% cm1[theta]
% PowerExpand[%]
% FullSimplify[%]
%
% Limit[theta + s - c[theta + s], s -> -Infinity]

Thus we see that unlike in Sections~\ref{sec:binomial}
and~\ref{sec:zero-truncated-poisson}, the cumulant function for
the family concentrated at one is not the identity function but
rather the identity function plus a constant function.

\section{Multivariate Bernoulli}
\label{sec:multivariate-bernoulli}

A random vector is \emph{multivariate Bernoulli} if it always has exactly
one component equal to one and the rest of its components are equal to zero.

This is the \emph{rationale} for the distribution.
In categorical data analysis, where we have IID individuals that can take
values in a finite number of categories, the result for each individual
can be coded as multivariate Bernoulli.  The category corresponding to the
component of the random vector that is equal to one says which category
the individual is in.

This is a \emph{discrete} random vector.

This is a special case of the multinomial distribution, which we do next.

This family is implemented in R package \code{aster2}.
When incorporated in an aster model, this family is a \emph{dependence group}.
Because it is
multivariate, it cannot be implemented in R package \code{aster},
which allows only univariate families.  Although R package \code{aster2}
calls this family ``multinomial'' created by the family function
\code{fam.multinomial}, it is not the general multinomial described in
the following section, but the multivariate Bernoulli described in this
section, which is the $n = 1$ special case of the general multinomial.

\section{Multinomial}
\label{app:sec:multinomial}

This family is multi-dimensional and hence cannot be implemented in R package
\code{aster}.

When IID individuals (a simple random sample) are classified into mutually
exclusive and exhaustive categories (every individual falls in exactly one
category), the vector of category counts has
the \emph{multinomial distribution}.
This is one \emph{rationale} for this distribution.

The other \emph{rationale} is that when the sample size (predecessor)
is equal to one, this family serves as a $k$-way switch if there are $k$
categories
(and this $n = 1$ special case is the multivariate Bernoulli family
described in the preceding section).
%%%%%%%%%% NEED BACKWARD REFERENCE to k-way switch %%%%%%%%%%

This is the distribution of a random vector, not a random variable.

The Bernoulli and binomial distributions are related to the multinomial
distribution, but the multinomial distribution with two categories is
still a two-dimensional random vector, so it is not Bernoulli or binomial,
which are one-dimensional distributions (of random variables).
If $Y$ is a Bernoulli random variable, then
$(Y, 1 - Y)$ is a multivariate Bernoulli random vector.
If $Y$ is a binomial random variable with sample size $n$, then
$(Y, n - Y)$ is a multinomial random vector with sample size $n$.

This is a \emph{discrete} random vector.

The \emph{probability mass function} is
\begin{equation} \label{eq:multinomial-pmf}
   f(y) = \frac{n !}{\prod_{i \in I} y_i!} \prod_{i \in I} p_i^{y_i},
   \qquad y \in S,
\end{equation}
where $p$ is the \emph{usual parameter vector}, which is a probability
vector satisfying
\begin{gather*}
   p_i > 0
   \\
   \sum_{i \in I} p_i = 1
\end{gather*}
where $y$ is the the vector of counts (nonnegative integers) satisfying
\begin{subequations}
\begin{gather}
   y_i \ge 0
   \label{eq:multinomial-canonical-statistic-constraint-positivity}
   \\
   \sum_{i \in I} y_i = n
   \label{eq:multinomial-canonical-statistic-constraint-sum-to-n}
\end{gather}
\end{subequations}
where $n$ is the sample size (the number of IID individuals classified),
where $S$ is the sample space for $y$, the set
\begin{equation} \label{eq:multinomial-canonical-sample-space}
   S = \bigset{ y \in \nats^I : \sum_{i \in I} y_i = n},
\end{equation}
where $\nats$ denotes the \emph{natural numbers} $\{0, 1, 2, 3, \ldots\}$,
and where we have chosen the index set of $y$ and $\theta$ to be an
arbitrary finite set $I$ rather than $\{1, \ldots, k\}$ for some $k$
to fit in with the conventions of aster models (in an aster model $I$
would be a subset of nodes of the aster graph comprising a multinomial
dependence group, see Section~\ref{sec:subvector} for vectors and subvectors).

The \emph{mean vector} and \emph{variance matrix} have components
\begin{subequations}
\begin{alignat}{2}
   E(Y_i) & = n p_i
   \label{eq:mean-vector-components}
   \\
   \var(Y_i) & = n p_i (1 - p_i)
   \label{eq:variance-matrix-components-diagonal}
   \\
   \cov(Y_i, Y_j) & = - n p_i p_j, & \qquad & i \neq j
   \label{eq:variance-matrix-components-off-diagonal}
\end{alignat}
\end{subequations}
The mean of a random vector $Y$ is the vector whose components are the
means of the components of $Y$.  Here $E(Y) = \xi$ and $\xi_i = n p_i$.
The variance of a random vector $Y$ is the matrix whose components are the
covariances of the components of $Y$.  Here $\var(Y) = M$ has components
$m_{i j}$ which are given by \eqref{eq:variance-matrix-components-diagonal}
when $i = j$ and by \eqref{eq:variance-matrix-components-off-diagonal} when
$i \neq j$.

This is an \emph{exponential family}.  From \eqref{eq:multinomial-pmf}
the log likelihood is
\begin{equation} \label{eq:multinomial-logl-try-one}
   l(\theta) = \sum_{i \in I} y_i \log(p_i)
\end{equation}
from which we see that we have an exponential family with
\emph{canonical statistic vector} $y$ and \emph{canonical parameter vector}
$\theta$ having components
\begin{equation} \label{eq:multinomial-canonical-try-one}
   \theta_i = \log(p_i), \qquad i \in I.
\end{equation}

Trying to read the cumulant function off of \eqref{eq:multinomial-logl-try-one}
seems to say $c(\theta)$ is the constant function everywhere equal to zero,
and this is correct because the $p_i$ must sum to one, and as we shall see
the cumulant function does have the value zero
when
\begin{equation} \label{eq:multinomial-canonical-constraint-try-one}
   \sum_{i \in I} e^{\theta_i} = 1.
\end{equation}

But we want the cumulant function defined on the whole vector space where
$\theta$ lives, so we must use equation (5) in \citet{geyer-gdor},
which is \eqref{eq:cumfun-expfam} in this book,
\begin{align*}
   c(\theta)
   & =
   c(\psi)
   +
   \log E_\psi\left\{ e^{\sum_{i \in I} y_i (\theta_i - \psi_i)} \right\}
   \\
   & =
   c(\psi) + \log \sum_{y \in S}
   e^{\sum_{i \in I} y_i (\theta_i - \psi_i)}
   \cdot
   \frac{n !}{\prod_{i \in I} y_i!} \prod_{i \in I} p_i^{y_i}
   \\
   & =
   c(\psi) + \log \sum_{y \in S}
   \frac{n !}{\prod_{i \in I} y_i!}
   \prod_{i \in I} (p_i e^{\theta_i - \psi_i})^{y_i}
   \\
   & =
   c(\psi) + \log \left( \sum_{i \in I} p_i e^{\theta_i - \psi_i} \right)^n
   \\
   & =
   c(\psi) + n \log \left( \sum_{i \in I} p_i e^{\theta_i - \psi_i} \right)
\end{align*}
where the last equality is the multinomial theorem also called the theorem
associated with the multinomial distribution \citep{brand-name-distributions}.
Here $\psi$ is a possible value of the canonical parameter vector (held fixed)
and $p$ is the usual parameter vector corresponding to it so $p_i = e^{\psi_i}$
and $p_i e^{\theta_i - \psi_i} = e^{\theta_i}$.  So dropping $c(\psi)$,
which is an arbitrary constant, we obtain
\begin{equation} \label{eq:multinomial-cumfun}
   c(\theta) = n \log \left( \sum_{i \in I} e^{\theta_i} \right).
\end{equation}
Since this is finite for all vectors $\theta$, the full canonical parameter
space is the whole vector space $\real^I$ where $\theta$ lives.

This gives a log likelihood valid on the whole vector space where
$\theta$ lives
\begin{equation} \label{eq:multinomial-logl}
\begin{split}
   l(\theta) & = \inner{y, \theta} - c(\theta)
   \\
   & = \left( \sum_{i \in I} y_i \theta_i \right)
   - n \log \left( \sum_{i \in I} e^{\theta_i} \right)
\end{split}
\end{equation}

We check that \eqref{eq:multinomial-cumfun} has the correct derivatives
\begin{subequations}
\begin{align}
   \frac{\partial c(\theta)}{\partial \theta_i}
   & = 
   \frac{n e^{\theta_i}}{\sum_{j \in I} e^{\theta_j}}
   \label{eq:multinomial-cumfun-first-derivatives}
   \\
   \frac{\partial^2 c(\theta)}{\partial \theta_i^2}
   & = 
   \frac{n e^{\theta_i}}{\sum_{j \in I} e^{\theta_j}}
   -
   \frac{n e^{\theta_i} e^{\theta_i}}
   {\left( \sum_{j \in I} e^{\theta_j} \right)^2}
   \label{eq:multinomial-cumfun-second-derivatives-diagonal}
   \\
   \frac{\partial^2 c(\theta)}{\partial \theta_i \theta_j}
   & = 
   -
   \frac{n e^{\theta_i} e^{\theta_j}}
   {\left( \sum_{k \in I} e^{\theta_k} \right)^2}
   \label{eq:multinomial-cumfun-second-derivatives-off-diagonal}
\end{align}
\end{subequations}
But here we have a problem that this does not even make sense with our
previous notion of canonical parameters, which, recall, was defined
by \eqref{eq:multinomial-canonical-try-one} but only subject to the
constraint \eqref{eq:multinomial-canonical-constraint-try-one}.
So we do not have a notion of what the map $\theta \to p$ should be
for values of $\theta$ that do not satisfy the
constraint \eqref{eq:multinomial-canonical-constraint-try-one}.

We solve this problem by using the fundamental relationship between
cumulant functions and means and variances
(Section~\ref{sec:mean-variance-cumulant} above), which says the
derivatives above have to give means and covariances.
Thus \eqref{eq:multinomial-cumfun-first-derivatives} must
give the correct mean values
$$
   \frac{n e^{\theta_i}}{\sum_{j \in I} e^{\theta_j}} = n p_i
$$
so
\begin{equation} \label{eq:multinomial-canonical-to-usual}
   p_i = \frac{e^{\theta_i}}{\sum_{j \in I} e^{\theta_j}}, \qquad i \in I,
\end{equation}
gives the correct mapping between our new canonical parameters (now
$\theta$ can be any vector in $\real^I$) and the usual parameters.

This function is not invertible.  If one adds the same constant
to all of the $\theta_i$, then the value of $p_i$ does not change
$$
   \frac{e^{\theta_i + c}}{\sum_{j \in I} e^{\theta_j + c}}
   =
   \frac{e^c e^{\theta_i}}{e^c \sum_{j \in I} e^{\theta_j}}
   =
   \frac{e^{\theta_i}}{\sum_{j \in I} e^{\theta_j}}
$$

This illustrates another thing wrong with the concept of the link function:
it forces the canonical parameterization to be identifiable even when this
is inadvisable.  Using our choice of parameterization here there can be
no link function because the map $\theta \to \xi$ is not one-to-one, so
its inverse mapping does not exist (an that inverse mapping is supposed to
be the ``link'' function).

We also clear up a mystery left hanging and check
that \eqref{eq:multinomial-cumfun} does indeed evaluate to zero when the
constraint \eqref{eq:multinomial-canonical-constraint-try-one} holds.

Having made the identification \eqref{eq:multinomial-canonical-to-usual}
we see that \eqref{eq:multinomial-cumfun-second-derivatives-diagonal}
and \eqref{eq:multinomial-cumfun-second-derivatives-off-diagonal}
do give the correct variances and covariances
\begin{align*}
   \frac{\partial^2 c(\theta)}{\partial \theta_i^2}
   & =
   \var(y_i) = n p_i (1 - p_i)
   \\
   \frac{\partial^2 c(\theta)}{\partial \theta_i \partial \theta_j}
   & =
   \cov(y_i, y_j) = - n p_i p_j
\end{align*}

The canonical parameter space is the whole vector space where $\theta$ lives.

The mean value parameter vector, as stated above, is the vector $\xi$ having
components $\xi_i = n p_i$.  In case $n = 1$, the mean value parameter vector
is the usual parameter vector.  Otherwise, not.

The mean value parameter space is the relative interior of the
convex hull of $S$
\begin{equation} \label{eq:multinomial-mean-value-parameter-space}
   \Xi
   =
   \bigset{ \xi \in \real^I : \xi_i > 0, i \in I \opand
   \sum_{j \in I} \xi_j = n }
\end{equation}
To see this we note that $\theta_i = \log(\xi_i)$ always defines a point
in the canonical sample space so long as the $\xi_i$ are strictly positive.
And that point maps via \eqref{eq:multinomial-canonical-to-usual} to
$$
   p_i = \frac{e^{\theta_i}}{\sum_{j \in I} e^{\theta_j}}
   =
   \frac{\xi_i}{\sum_{j \in I} \xi_j} = \frac{\xi_i}{n}
$$
which makes the $p_i$ sum to one.
So every point in $\Xi$ is a mean value parameter vector value,
and, conversely, every point $\theta \in \real^I$ maps to a value $\xi$
that satisfies $\xi_i > 0$ for all $i$ and $\sum_{i \in I} \xi_i = n$.

\begin{theorem} \label{th:multinomial-lcm}
The closed convex support of the multinomial family is
$$
   C = \bigset{ \xi \in \real^I : \xi_i \ge 0, i \in I \opand
   \sum_{j \in I} \xi_j = n }
$$
the closure \eqref{eq:multinomial-mean-value-parameter-space}.
The support function is given by
\begin{equation} \label{eq:multinomial-support-function}
   \sigma_C(\delta) = n \max(\delta),
\end{equation}
where
$$
   \max(\delta) = \max_{i \in I} \delta_i.
$$
Define the hyperplane
$$
   H_\delta = \set{ y \in \real^I : \inner{y, \delta} = \sigma_C(\delta) },
$$
which is \eqref{eq:complete-fundamental-hyperplane} with the index set $I$
instead of $J$.
Then the cumulant function for the LCM conditioned on $H_\delta$ is
\begin{equation} \label{eq:multinomial-cumfun-lcm}
   c_\delta(\theta)
   =
   n \log \left(
   \sum_{\substack{i \in I \\ \delta_i = \max(\delta)}} e^{\theta_i} \right).
\end{equation}
If we define
$$
   G = \set{ i \in I : \delta_i = \max(\delta) },
$$
then under the LCM conditioned on $H_\delta$, the family of distributions
for $y_G$ is multinomial with sample size $n$
and $y_{I \setminus G} = 0$ almost surely.
\end{theorem}
\begin{proof}
For any $i \in I$ define the vector $v^{(i)}$ (we use the temporary notation
of superscript in parenthesis to denote a sequence of vectors)
having $v^{(i)}_i = n$ and $v^{(i)}_j = 0$ for $j \in I \setminus \{ i \}$.
Clearly, each such $v^{(i)}$ is in
the sample space \eqref{eq:multinomial-canonical-sample-space} and
has probability $p_i^n > 0$.  Thus it must be in any support.
Now the formula
$$
   C = \bigset{ \sum_{i \in I} p_i v^{(i)} : p_i \ge 0, i \in I \opand
   \sum_{j \in I} p_j = 1 }
$$
shows that $C$ is the convex hull of the vectors $v^{(i)}$.  Thus $C$ must
be contained in the closed convex support.  On the other hand, $C$ contains
$S$, so $C$ is the smallest closed convex set containing $S$.  Hence
$C$ is the closed convex support.

We now claim that any vector $v$ having $v_j = 0$ when
$\delta_j < \max(\delta)$ maximizes $\inner{\fatdot, \delta}$ over $C$.
We have
$$
   \inner{v, \delta}
   =
   \sum_{\substack{i \in I \\ \delta_i = \max(\delta)}} v_i \delta_i
   =
   \max(\delta)
   \sum_{\substack{i \in I \\ \delta_i = \max(\delta)}} v_i
   =
   \max(\delta)
   \sum_{i \in I} v_i
   =
   n \max(\delta)
$$
and for any $x \in C$ we have
$$
   \inner{v, \delta}
   =
   \sum_{i \in I} v_i \delta_i
   \le
   \max(\delta)
   \sum_{i \in I} v_i
   =
   n \max(\delta)
$$
Thus \eqref{eq:multinomial-support-function} is correct.

Now if $y \in C$ such that
\begin{equation} \label{eq:multinomial-lcm-foo}
   \inner{y, \delta} = \sigma_C(\delta) = n \max(\delta)
\end{equation}
we must have, as we have already seen, $y_j = 0$ if $\delta_j < \max(\delta)$.
Conversely, if $y \in C$ and $y_j = 0$ if $\delta_j < \max(\delta)$,
then \eqref{eq:multinomial-lcm-foo} holds.
Hence conditioning the original model on $Y \in H_\delta$ is the same as
conditioning on $Y_{I \setminus G} = 0$.  Then the usual formulas
for conditionals for multinomials show that $Y_G$ is multinomial
with sample size $n$, as asserted.

It only remains to calculate the cumulant function for the LCM using
\eqref{eq:cumfun-lcm}.
\begin{align*}
   c_\delta(\theta)
   & =
   c(\theta) + \log \Pr\nolimits_\theta(Y \in H_\delta)
   \\
   & =
   n \log \left( \sum_{i \in I} e^{\theta_i} \right)
   +
   \log \left( \sum_{i \in G} p_i \right)^n
   \\
   & =
   n \log \left( \sum_{i \in I} e^{\theta_i} \right)
   +
   n \log \left( \sum_{i \in G} p_i \right)
   \\
   & =
   n \log \left( \sum_{i \in I} e^{\theta_i} \right)
   +
   n \log \left(
   \frac{\sum_{i \in G} e^{\theta_i}}{\sum_{i \in I} e^{\theta_i}} \right)
   \\
   & =
   n \log \left( \sum_{i \in G} e^{\theta_i} \right)
\end{align*}
\end{proof}

\section{Normal Location-Scale}

The univariate normal distribution is curious in that it remains an exponential
family even if we consider both parameters unknown, but then
the dimensions of the canonical statistic vector and the canonical parameter
vector must match.  So if the canonical parameter vector is going to be
two-dimensional so must be the canonical statistic vector.

Let's see how that happens.  We already have the probability density
function \eqref{eq:normal-pdf} of the normal distribution.
Now if we write down the log likelihood not dropping any terms that contain
either parameter we get
\begin{equation} \label{eq:logl-normal-location-scale}
   l(\theta)
   =
   - \log(\sigma) - \frac{(x - \nu)^2}{2 \sigma^2}
   =
   - \log(\sigma) - \frac{x^2}{2 \sigma^2}
   + \frac{x \nu}{\sigma^2}
   - \frac{\nu^2}{2 \sigma^2}
\end{equation}
where for reasons to be discussed presently we have changed the notation
for the random variable from $y$ to $x$ and the notation
for the mean from $\xi$ to $\nu$.

As was discussed in Section~\ref{sec:define-expfam} in the main text,
there is some freedom in choosing the canonical statistic vector and
the canonical parameter vector we must have the terms containing both
data and parameters in exponential family form, that is,
$$
   - \frac{x^2}{2 \sigma^2} + \frac{x \nu}{\sigma^2}
   =
   y_1 \theta_1 + y_2 \theta_2
$$
but that still allows lots of choices.
We could, for example, choose $y_1 = x$ or $y_1 = x^2$ or $y_1 = - x^2 / 2$.
And each such choice forces a different choice of the corresponding
canonical parameter.

The choice made in the implementation in R package \code{aster2} is
\begin{align*}
   y_1 & = x,
   \\
   y_2 & = x^2,
   \\
   \theta_1 & = \frac{\nu}{\sigma^2}
   \\
   \theta_2 & = - \frac{1}{2 \sigma^2}
\end{align*}
We have had many examples where the usual parameters are not the canonical
parameters.  Here is our first example where the usual statistics are not
the canonical statistics (the usual statistic is one canonical statistic
but not the other).

In terms of the canonical parameters, the usual parameters are
\begin{align*}
   \sigma^2 & = - \frac{1}{2 \theta_2}
   \\
   \nu & = \theta_1 \sigma^2 = - \frac{\theta_1}{2 \theta_2}
\end{align*}

The first canonical parameter is unrestricted $- \infty < \theta_1 < \infty$
but the second canonical parameter is restricted $- \infty < \theta_2 < 0$.
Thus our guess at the cumulant function from looking at the log likelihood
\begin{equation} \label{eq:cumfun-normal-location-scale}
\begin{split}
   c(\theta)
   & =
   \log(\sigma) + \frac{\nu^2}{2 \sigma^2}
   \\
   & =
   \frac{1}{2} \log\left(- \frac{1}{2 \theta_2}\right)
   +
   \left( - \frac{\theta_1}{2 \theta_2} \right)^2 \cdot \frac{1}{2}
   \cdot (- 2 \theta_2)
   \\
   & =
   \frac{1}{2} \log\left(- \frac{1}{2 \theta_2}\right)
   -
   \frac{\theta_1^2}{4 \theta_2} 
\end{split}
\end{equation}
(this agrees with the cumulant function for this family
in R package \code{aster2}).

Notice that in going from the PDF to the log likelihood
\eqref{eq:logl-normal-location-scale}, on a factor of $1 / \sqrt{2 \pi}$
was dropped, so the PDF of the family (with respect to Lebesgue measure)
can now be written
$$
   f_\theta(x)
   =
   \frac{1}{\sqrt{2 \pi}} e^{x \theta_1 + x^2 \theta_2 - c(\theta)}
$$
and the fact that PDF integrate to one gives
$$
   \int e^{x \theta_1 + x^2 \theta_2} \, d x
   =
   \sqrt{2 \pi} e^{c(\theta)}
$$

Because our guess at the cumulant function is not defined on a whole vector
space we use equation (5) in \citet{geyer-gdor},
which is \eqref{eq:cumfun-expfam}
in this book, as we have done several times before in this appendix
\begin{align*}
   c(\theta)
   & =
   c(\psi) + \log E_\psi
   \bigl( e^{y_1 (\theta_1 - \psi_1) + y_2 (\theta_2 - \psi_2)} \bigr)
   \\
   & =
   c(\psi) + \log E_\psi
   \bigl( e^{y_1 (\theta_1 - \psi_1) + y_2 (\theta_2 - \psi_2)} \bigr)
   \\
   & =
   c(\psi)
   +
   \log \left( \frac{1}{\sqrt{2 \pi}} \int
   e^{x (\theta_1 - \psi_1) + x^2 (\theta_2 - \psi_2)}
   e^{x \psi_1 + x^2 \psi_2) - c(\psi)} \, d x \right)
   \\
   & =
   \log \left( \frac{1}{\sqrt{2 \pi}} \int
   e^{x \theta_1 + x^2 \theta_2} \, d x \right)
\end{align*}
and, as we have already seen, the last expression is equal to $c(\theta)$ for
$\theta$ such that $\theta_2 < 0$.
In case $\theta_2 \ge 0$ the integrand is either constant or goes to infinity
as $x \to \infty$ or $x \to - \infty$ or both.  In any case the integral
does not exist.  Thus the canonical parameter space is indeed
\begin{equation} \label{eq:canonical-parameter-space-normal-location-scale}
   \Theta = \set{ \theta \in \real^2 : \theta_2 < 0 }.
\end{equation}

The canonical sample space for sample size one is
$$
   S = \set{ y \in \real^2 : y_1^2 = y_2 }
$$
which is a curve.  Thus any intersection with a tangent line consists of
a single point.  And, because this is a continuous distribution, single points
have probability zero.  Thus in the limiting conditional model theorem
(Theorem~\ref{th:completion-fundamental} above) we are in the case where
the probability of $Y \in H_\delta$ is equal to zero
(regardless of what $\delta$ is).  So there are no LCM for this family.

For sample size one, the data are on the curve $S$ with probability one.
So the MLE does not exist with probability one.  This is no surprise.
One cannot estimate two parameters from one variable $x$.

For sample size greater than one, the data are in the interior of the convex
hull of the curve $S$ with probability one.
So the MLE exists with probability one.  This is no surprise either.
One can estimate the two parameters of the normal distribution from
data $x_1$, $x_2$, $\ldots$, $x_n$ for $n \ge 2$.

So we don't need the theory of limiting conditional models for this family.
That theory is only needed for discrete families.

% Mathematica
%
% dist = NormalDistribution[moo, Sqrt[voo]]
% f[x_] = PDF[dist, x]
% Integrate[f[x], {x, -Infinity, Infinity}, Assumptions -> voo > 0]
% Integrate[x f[x], {x, -Infinity, Infinity}, Assumptions -> voo > 0]
% Integrate[(x - moo)^2 f[x], {x, -Infinity, Infinity}, Assumptions -> voo > 0]
%
% cumfun[theta1_, theta2_] = Integrate[Exp[x theta1 + x^2 theta2] f[x],
%     {x, -Infinity, Infinity}, Assumptions -> voo > 0]

We forgot to check that \eqref{eq:cumfun-normal-location-scale} has
the correct derivatives.  Let's do that now.
\begin{align*}
   \frac{\partial c(\theta)}{\partial \theta_1}
   & =
   -
   \frac{\theta_1}{2 \theta_2}
   \\
   \frac{\partial c(\theta)}{\partial \theta_2}
   & =
   -
   \frac{1}{2 \theta_2}
   +
   \frac{\theta_1^2}{4 \theta_2^2}
   \\
   \frac{\partial^2 c(\theta)}{\partial \theta_1^2}
   & =
   - \frac{1}{2 \theta_2}
   \\
   \frac{\partial^2 c(\theta)}{\partial \theta_1 \partial \theta_2}
   & =
   \frac{\theta_1}{2 \theta_2^2}
   \\
   \frac{\partial^2 c(\theta)}{\partial \theta_2^2}
   & =
   \frac{1}{2 \theta_2^2}
   -
   \frac{\theta_1^2}{2 \theta_2^3}
\end{align*}
Translating these back to functions of $x$ and the original parameters,
they say
\begin{align*}
   E(X) & = \nu
   \\
   E(X^2) & = \sigma^2 + \nu^2
   \\
   \var(X) & = \sigma^2
   \\
   \cov(X, X^2) & = 2 \nu \sigma^2
   \\
   \var(X^2) & = 4 \nu^2 \sigma^2 + 2 \sigma^4
\end{align*}
The first three of these are well known.  The other two do check
(in Mathematica).
% Mathematica
% c[theta1_, theta2_] = (1/2) Log[- 1 / (2 theta2)] - theta1^2 / (4 theta2) 
% c1[theta1_, theta2_] = D[c[theta1, theta2], theta1]
% c2[theta1_, theta2_] = D[c[theta1, theta2], theta2]
% c11[theta1_, theta2_] = D[c1[theta1, theta2], theta1]
% c12[theta1_, theta2_] = D[c1[theta1, theta2], theta2]
% c22[theta1_, theta2_] = D[c2[theta1, theta2], theta2]
% c1[nu / sigma^2, - 1 / (2 sigma^2)]
% c2[nu / sigma^2, - 1 / (2 sigma^2)]
% c11[nu / sigma^2, - 1 / (2 sigma^2)]
% c12[nu / sigma^2, - 1 / (2 sigma^2)]
% c22[nu / sigma^2, - 1 / (2 sigma^2)]
%
% dist = NormalDistribution[nu, sigma]
% f[x_] = PDF[dist, x]
% $Assumptions = sigma > 0 && Element[nu, Reals]
% Integrate[f[x], {x, -Infinity, Infinity}]
% Integrate[(x - nu)(x^2 - nu^2 - sigma^2)f[x], {x, -Infinity, Infinity}]
% Integrate[(x^2 - nu^2 - sigma^2)^2 f[x], {x, -Infinity, Infinity}]

\section{Gamma Rate}

The gamma distribution has PDF
\begin{equation} \label{eq:gamma-pdf}
   f(y) = \frac{\lambda^\alpha}{\Gamma(\alpha)} y^{\alpha - 1} e^{- \lambda y},
   \qquad 0 < y < \infty,
\end{equation}
where $\alpha > 0$ and $\lambda > 0$ are parameters.

This is a \emph{continuous} random variable;
except when incorporated into an aster model, it is a mixture of discrete
and continuous.  For a gamma-rate arrow, when the predecessor is zero
the conditional distribution of the successor is the degenerate random variable
concentrated at zero, which is discrete, and when the predecessor is greater
than zero, the conditional distribution of the successor is continuous.

This is the most well known continuous distribution after
the normal distribution, and it has many rationales, but these rationales
do not seem to justify its inclusion in aster models, which is why
R packages \code{aster} and \code{aster2} do not include it.
This family includes the exponential distribution (case $\alpha = 1$)
and the chi-square distributions (case $\alpha = n / 2$ and $\lambda = 1 / 2$)
as special cases.

The main rationale is vague.  This is the most well-known distribution
with support $(0, \infty)$.  So it is a TTT (thing to do) when one wants such.

The \emph{mean} and \emph{variance} are
\begin{subequations}
\begin{align}
   E(y) & = \frac{\alpha}{\lambda}
   \label{eq:gamma-mean}
   \\
   \var(y) & = \frac{\alpha}{\lambda^2}
   \label{eq:gamma-var}
\end{align}
\end{subequations}

In this section, we treat this as a one-parameter family
with $\alpha$ known and $\lambda$ unknown.

With this understanding, the log likelihood is
$$
   l(\theta)
   = 
   \alpha \log(\lambda) - \lambda y
$$
(we have dropped terms that do not contain $\lambda$).
From this we see that we have an \emph{exponential family} with
\emph{canonical statistic} $y$,
\emph{canonical parameter} $\theta = - \lambda$, and
\emph{cumulant function}
\begin{equation} \label{eq:cumfun-gamma-rate}
   c(\theta) = - \alpha \log(\lambda) = - \alpha \log(- \theta).
\end{equation}

We check that this has the correct derivatives
$$
   c'(\theta) = \frac{\alpha}{- \theta} = \frac{\alpha}{\lambda}
$$
and
$$
   c''(\theta) = \frac{\alpha}{\theta^2} = \frac{\alpha}{\lambda^2}.
$$

Because our guess at the cumulant function is not defined on a whole vector
space we use equation (5) in \citet{geyer-gdor},
which is \eqref{eq:cumfun-expfam}
in this book, as we have done several times before in this appendix
\begin{align*}
   c(\theta)
   & =
   c(\psi) + \log E_\psi \bigl( e^{y (\theta - \psi)} \bigr)
   \\
   & =
   c(\psi) + \log \int e^{y (\theta - \psi)} f_\psi(y) \, d y
   \\
   & =
   c(\psi) + \log \int e^{y (\theta - \psi)}
   \frac{\lambda_\psi^\alpha}{\Gamma(\alpha)} y^{\alpha - 1}
   e^{- \lambda_\psi y} \, d y
\end{align*}
where $\lambda_\psi = - \psi$ so
\begin{align*}
   c(\theta)
   & =
   \log \int
   \frac{1}{\Gamma(\alpha)} y^{\alpha - 1} e^{y \theta} \, d y
\end{align*}
and this integral clearly exists if and only if $\theta < 0$ in which
case, by \eqref{eq:gamma-pdf} integrating to one,
we get \eqref{eq:cumfun-gamma-rate}, as we must.

Hence the full canonical parameter space of this family is
$$
   \Theta = \set{ \theta \in \real : \theta < 0 }.
$$

The \emph{mean value parameter} is
$$
   \xi = \frac{\alpha}{- \theta}
$$

\emph{Addition rule:} the sum of $n$ independent and identically distributed
gamma random variables with shape parameter $\alpha$ and rate parameter
$\lambda$ has the gamma distribution
with shape parameter $n \alpha$ and rate parameter $\lambda$.

\emph{General addition rule:} the sum of $n$ independent
gamma random variables with shape parameters $\alpha_1$, $\ldots,$ $\alpha_n$
and rate parameter $\lambda$ (same for all) has the gamma distribution
with shape parameter $\alpha_1 + \cdots + \alpha_n$
and rate parameter $\lambda$.

There are no limit degenerate distributions.
This is because the boundary of the closed convex support,
which is the interval $[0, \infty)$ is the point zero.
We can never observe data on the boundary, because continuous
distributions put probability zero at any point.

\section{Gamma Shape-Rate}

If we take the gamma distribution with PDF \eqref{eq:gamma-pdf}
to have both parameters unknown, then the log likelihood is
$$
   l(\theta)
   =
   \alpha \log(\lambda) - \log \Gamma(\alpha) + \alpha \log(x) - \lambda x
$$
(we have dropped terms that do not contain $\alpha$ or $\lambda$ and have
changed the data variable from $y$ to $x$).  This has the form of
a two-dimensional exponential family with \emph{canonical statistics}
$y_1 = x$ and $y_2 = \log x$, \emph{canonical parameters}
$\theta_1 = - \lambda$ and $\theta_2 = \alpha$, and
\emph{cumulant function}
\begin{equation} \label{eq:cumfun-gamma-shape-rate}
   c(\theta) = \log \Gamma(\alpha) - \alpha \log(\lambda)
   = \log \Gamma(\theta_2) - \theta_2 \log(- \theta_1)
\end{equation}
With this definition of canonical parameters and statistics, we can rewrite
the PDF as
\begin{equation} \label{eq:gamma-pdf-too}
\begin{split}
   f_\theta(y)
   & =
   \frac{\lambda^\alpha}{\Gamma(\alpha)} x^{\alpha - 1} e^{- \lambda x}
   \\
   & =
   x^{- 1} e^{\inner{y, \theta} - c(\theta)}
   \\
   & =
   y_1^{- 1} e^{\inner{y, \theta} - c(\theta)}
\end{split}
\end{equation}
Because our guess at the cumulant function is not defined on a whole vector
space we use equation (5) in \citet{geyer-gdor},
which is \eqref{eq:cumfun-expfam}
in this book, as we have done several times before in this appendix
\begin{align*}
   c(\theta)
   & =
   c(\psi) + \log E_\psi \bigl( e^{\inner{y, \theta - \psi}} \bigr)
   \\
   & =
   c(\psi) + \log \int e^{\inner{y, \theta - \psi}} f_\psi(y) \, d y
   \\
   & =
   c(\psi) + \log \int e^{\inner{y, \theta - \psi}}
   y_1^{- 1} e^{\inner{y, \psi} - c(\psi)} \, d x
   \\
   & =
   \log \int y_1^{- 1} e^{\inner{y, \theta}} \, d x
\end{align*}
We know that \eqref{eq:gamma-pdf-too} integrates if and only if
$\alpha > 0$ and $\lambda > 0$ \citep[Slides~27--30]{slides-5101-deck-6}.
It follows that the full canonical parameter space is
$$
   \Theta = \set{ \theta \in \real^2 : \theta_1 < 0 \opand \theta_2 > 0 }
$$
and in order that \eqref{eq:gamma-pdf-too} integrate to one when the integral
exists, the last integral above must be $c(\theta)$.

Let us check that \eqref{eq:cumfun-gamma-shape-rate} has the correct
derivatives
\begin{align*}
   \frac{\partial c(\theta)}{\partial \theta_1}
   & =
   - \frac{\theta_2}{\theta_1}
   \\
   \frac{\partial c(\theta)}{\partial \theta_2}
   & =
   \mydigamma(\theta_2) - \log(- \theta_1)
   \\
   \frac{\partial^2 c(\theta)}{\partial \theta_1^2}
   & =
   \frac{\theta_2}{\theta_1^2}
   \\
   \frac{\partial^2 c(\theta)}{\partial \theta_1 \partial \theta_2}
   & =
   - \frac{1}{\theta_1}
   \\
   \frac{\partial^2 c(\theta)}{\partial \theta_2^2}
   & =
   \trigamma(\theta_2)
\end{align*}
Translating these back to functions of $x$ and the original parameters,
they say
\begin{align*}
   E(X) & = \frac{\alpha}{\lambda}
   \\
   E\{\log(X)\} & = \mydigamma(\alpha) - \log(\lambda)
   \\
   \var(X) & = \frac{\alpha}{\lambda^2}
   \\
   \cov\{X, \log(X)\} & = \frac{1}{\lambda}
   \\
   \var\{\log(X)\} & = \trigamma(\alpha)
\end{align*}
where the digamma function is the first derivative of $\log \Gamma(\fatdot)$,
and the trigamma function is the second derivative of this function.

Two of these agree with the known formulas for the mean and variance
of a gamma random variable \eqref{eq:gamma-mean} and \eqref{eq:gamma-var}.
The others involve integrals we don't know how to do other than by
the method of cumulant functions (what we just did).
(But Mathematica knows how to do these integrals.)
% Mathematica
% dist = GammaDistribution[alpha, 1 / lambda]
% f[x_] = PDF[dist, x]
% Integrate[ f[x], {x, 0, Infinity}, Assumptions -> alpha > 0 && lambda > 0 ]
% moo = Integrate[ x f[x], {x, 0, Infinity},
%     Assumptions -> alpha > 0 && lambda > 0 ]
% voo = Integrate[ (x - moo)^2 f[x], {x, 0, Infinity},
%     Assumptions -> alpha > 0 && lambda > 0 ]
% loo = Integrate[ Log[x] f[x], {x, 0, Infinity},
%     Assumptions -> alpha > 0 && lambda > 0 ]
% uoo = Integrate[ (Log[x] - loo)^2 f[x], {x, 0, Infinity},
%     Assumptions -> alpha > 0 && lambda > 0 ]
% woo = Integrate[ (Log[x] - loo) (x - moo) f[x], {x, 0, Infinity},
%     Assumptions -> alpha > 0 && lambda > 0 ]

The mean value parameter is the two-dimensional vector $\xi$ having components
\begin{align*}
   \xi_1
   & =
   - \frac{\theta_2}{\theta_1}
   \\
   \xi_2
   & =
   \mydigamma(\theta_2) - \log(- \theta_1)
\end{align*}

The addition rules for the gamma distribution were given in the preceding
section (they do not depend on which parameters are considered known or
unknown).

The canonical sample space for sample size one is
$$
   S = \set{ y \in \real^2 : y_1 = e^{y_2} }
$$
which is a curve.  Thus any intersection with a tangent line consists of
a single point.  And, because this is a continuous distribution, single points
have probability zero.  Thus in the limiting conditional model theorem
(Theorem~\ref{th:completion-fundamental} above) we are in the case where
the probability of $Y \in H_\delta$ is equal to zero
(regardless of what $\delta$ is).  So there are no LCM for this family.

For sample size one, the data are on the curve $S$ with probability one.
So the MLE does not exist with probability one.  This is no surprise.
One cannot estimate two parameters from one variable $x$.

For sample size greater than one, the data are in the interior of the convex
hull of the curve $S$ with probability one.
So the MLE exists with probability one.  This is no surprise either.
One can estimate the two parameters of the normal distribution from
data $x_1$, $x_2$, $\ldots$, $x_n$ for $n \ge 2$.

So we don't need the theory of limiting conditional models for this family.
That theory is only needed for discrete families.

The last three paragraphs almost exactly repeat what was said about
the normal-location-scale family.  The reasoning applies to any continuous
distribution.  The details distinguishing one of these families from the
other do not matter in this argument.

\section{K-Truncated Families}

R package \code{aster} implements $k$-truncated Poisson and $k$-truncated
negative binomial for any nonnegative integer $k$.  Example~{3} in
\citet{aster2} used two-truncated negative binomial.  We no longer think
%%%%%%%%%% NEED BACKWARD REFERENCE to subsampling %%%%%%%%%%
using this family is the best way to do this example.  Moreover we know
of no other examples in life history analysis that need these families
(except for the zero-truncated ones already discussed).  Therefore
we omit further discussion of them.



\bibliography{tpa}


\chapter*{Index of Notation}

\noindent
\begin{raggedright}
\noindent
\begin{longtable}{lp{4.4in}}
$\real$ & the real number system \\
$\nats$ & the natural number system, which starts at zero \\
$N$ & the set of nodes of the full aster graph, including initial nodes \\
$J$ & the set of non-initial nodes of the full aster graph \\
$\mathcal{G}$ & the family of dependence groups, a partition of $J$ \\
$q$ & the set-to-index predecessor function, which maps $\mathcal{G} \to N$ \\
$p$ & the index-to-index predecessor function, which maps $J \to N$ \\
$\real^A$ & the set of all functions $A \to \real$ considered as
    a vector space \\
$y_j$ & a component of the vector $y$: if $y \in \real^A$ and $j \in A$,
    then $y_j$ is the value of the function $y$ at argument $j$ \\
$y_A$ & a subvector of the vector $y$: if $y \in \real^B$ and $A \subset B$,
    then $y_A$ is the restriction of the function $y$ to the set $A$ \\
$\pr(y)$ & unconditional distribution of the random vector $y$ described
    somehow (Section~\ref{sec:factorization}) \\
$\pr(y_A \mid y_B)$ & conditional distribution of the random vector $y_A$
    described somehow (Section~\ref{sec:factorization}
    and equation~\eqref{eq:before-and-after} and the surrounding discussion) \\
$\theta$ & conditional canonical parameter vector of the saturated model \\
$\varphi$ & unconditional canonical parameter vector of the saturated model \\
$\xi$ & conditional mean value parameter vector of the saturated model \\
$\mu$ & unconditional mean value parameter vector of the saturated model \\
$\beta$ & unconditional canonical parameter vector of a canonical
    affine submodel \\
$\tau$ & unconditional mean value parameter vector of a canonical
    affine submodel \\
$a$ & offset vector \\
$M$ & model matrix \\
\end{longtable}
\end{raggedright}


\printindex

\end{document}

