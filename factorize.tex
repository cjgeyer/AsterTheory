
\chapter{The Factorization Theorem}
\label{app:factorize}

\begin{proof}[Proof of Theorem~\ref{th:factorize}]
A valid factorization factors joint equals conditional times marginal
$$
   \pr(y) = \pr(y_{G_1} \mid y_{N \setminus G_1}) \pr(y_{N \setminus G_1})
$$
The marginal on the right-hand side can then be considered a joint to be
factored further
$$
   \pr(y)
   =
   \pr(y_{G_1} \mid y_{N \setminus G_1})
   \pr(y_{G_2} \mid y_{N \setminus (G_1 \cup G_2)})
   \pr(y_{N \setminus (G_1 \cup G_2)})
$$
and again and again giving
\begin{equation} \label{eq:factorize-general}
   \pr(y)
   =
   \pr(y_{N \setminus \bigcup_{j = 1}^k G_j})
   \prod_{i = 1}^k
   \pr(y_{G_i} \mid y_{N \setminus \bigcup_{j = 1}^i G_j})
\end{equation}
and the only condition that is required to make \eqref{eq:factorize-general}
valid is that the index sets $G_i$ are disjoint.  This is the only operation
in classical (non-measure-theoretic) probability theory that factorizes
probability distributions.  A factorization is valid if and only if it
has the form \eqref{eq:factorize-general}.

When we match up \eqref{eq:factorize} and \eqref{eq:factorize-general}
we see that the $G_i$ must be the elements of $\mathcal{G}$ so the two
products are the same.  For the conditional distributions to
match up we must have $\pr(y_{G_i} \mid y_{N \setminus \bigcup_{j = 1}^i G_j})$
in \eqref{eq:factorize-general} can actually be written as
$\pr(y_{G_i} \mid y_{q(G_i)})$, that is,
\begin{itemize}
\item this conditional distribution actually depends only on the single
    variable $y_{q(G_i)}$ not on the rest of the variables that are components
    of $y_{N \setminus \bigcup_{j = 1}^i G_j}$ and
\item $q(G_i) \in N \setminus \bigcup_{j = 1}^i G_j$, that is, either
    $q(G_i) \in G_j$ for some $j > i$ or $q(G_i)$ is an initial node
    ($q(G_i) \notin G_j$ for any $j$).  In either case,
    $q(G_i) \in G_j$ implies $i < j$.  Thus we have the condition of
    the theorem: $G_i < G_j$ if and only if $i < j$.
\end{itemize}
Finally, we must match up the marginal term on the right-hand side of
\eqref{eq:factorize-general}.  It matches nothing in \eqref{eq:factorize},
which is the same as saying it must be equal to one, which is they same as
saying $y_{N \setminus J}$ where $J = \bigcup \mathcal{G}$ is a constant
random vector.
\end{proof}

